\section{भाग–९ संघीय व्यवस्थापन कार्यविधि}

(२) अर्थ विधेयक, नेपाली सेना, नेपाल प्रहरी वा सशस्त्र प्रहरी बल, नेपाल लगायत सुरक्षा निकायसँग सम्बन्धित विधेयक सरकारी विधेयकको रूपमा मात्र प्रस्तुत गरिनेछ ।

(३) “अर्थ विधेयक” भन्नाले देहायमा उल्लिखित सबै वा कुनै विषयसँग सम्बन्ध राख्ने विधेयकलाई जनाउँछः–

(क) कर लगाउने, उठाउने, खारेज गर्ने, छूट दिने, परिवर्तन गर्ने वा कर प्रणालीलाई व्यवस्थित गर्ने विषय,
(ख) संघीय सञ्चित कोष वा अन्य कुनै संघीय सरकारी कोषको संरक्षण गर्ने, त्यस्तो कोषमा रकम जम्मा गर्ने वा त्यस्तो कोषबाट कुनै रकम विनियोजन वा खर्च गर्ने वा विनियोजन वा खर्च गर्न खोजिएको रकम घटाउने, बढाउने वा खारेज गर्ने विषय,
(ग) नेपाल सरकारले ऋण प्राप्त गर्ने वा जमानत दिने विषय व्यवस्थित गर्ने वा नेपाल सरकारले लिएको वा लिने आर्थिक दायित्व सम्बन्धी कानून संशोधन गर्ने विषय,
(घ) संघीय सरकारी कोषमा प्राप्त हुने सबै प्रकारको राजस्व, ऋण असुलीबाट प्राप्त रकम र अनुदानको रकम जिम्मा राख्ने, लगानी गर्ने वा नेपाल सरकारको लेखा वा लेखापरीक्षण गर्ने विषय, र
(ङ) खण्ड (क), (ख), (ग) वा (घ) सँग प्रत्यक्ष सम्बन्ध भएका अन्य प्रासंगिक विषयहरू ।

तर कुनै अनुमतिपत्र दस्तुर, निवेदन दस्तुर, नवीकरण दस्तुर जस्ता दस्तुर, शुल्क वा महसूल लगाउने वा कुनै जरिवाना वा कैद हुने व्यवस्था भएको कारणले मात्र कुनै विधेयक अर्थ विधेयक मानिने छैन ।
(४) कुनै विधेयक अर्थ विधेयक हो होइन भन्ने प्रश्न उठेमा सभामुखको निर्णय अन्तिम हुनेछ ।

\textbf{१११. विधेयक पारित गर्ने विधिः}

(१) संघीय संसदको एउटा सदनले पारित गरेको विधेयक यथाशीघ्र अर्को सदनमा पठाइनेछ र सो सदनले पारित गरेपछि प्रमाणीकरणका लागि राष्ट्रपति समक्ष पेश गरिनेछ ।

(२) प्रतिनिधि सभाले पारित गरेको अर्थ विधेयक राष्ट्रिय सभामा पठाइनेछ । राष्ट्रिय सभाले सो विधेयकमा छलफल गरी विधेयक प्राप्त गरेको पन्ध्र दिनभित्र कुनै सुझाव भए सुझाव सहित प्रतिनिधि सभामा फिर्ता पठाउनु पर्नेछ ।

(३) उपधारा (२) बमोजिम सुझाव सहित फिर्ता आएको विधेयकमा प्रतिनिधि सभाले छलफल गरी उचित देखेको सुझाव समावेश गरी प्रमाणीकरणका लागि राष्ट्रपति समक्ष पेश गर्नेछ ।

(४) उपधारा (२) बमोजिम अर्थ विधेयक प्राप्त गरेको पन्ध्र दिनसम्ममा राष्ट्रिय सभाले सो विधेयक फिर्ता नगरेमा प्रतिनिधि सभाले प्रमाणीकरणका लागि राष्ट्रपति समक्ष पेश गर्न सक्नेछ ।

(५) प्रतिनिधि सभाले पारित गरी राष्ट्रिय सभामा पठाएको अर्थ विधेयक बाहेक अन्य विधेयक राष्ट्रिय सभाले आफू समक्ष प्राप्त भएको दुई
महीनाभित्र पारित गरी वा सुझाव सहित फिर्ता पठाउनु पर्नेछ । त्यस्तो समयावधिभित्र राष्ट्रिय सभाले सो विधेयक फिर्ता नगरेमा प्रतिनिधि सभाले तत्काल कायम रहेको सदस्य संख्याको बहुमत सदस्यहरूको निर्णयबाट सो विधेयक प्रमाणीकरणका लागि राष्ट्रपति समक्ष पेश गर्न सक्नेछ ।

(६) अर्थ विधेयक बाहेक कुनै सदनले पारित गरेको अन्य विधेयक अर्को सदनले अस्वीकृत गरेमा वा संशोधन सहित पारित गरेमा सो विधेयक उत्पत्ति भएको सदनमा फिर्ता पठाउनु पर्नेछ ।

(७) उपधारा (६) बमोजिम राष्ट्रिय सभाबाट अस्वीकृत भई वा संशोधन सहित प्रतिनिधि सभामा फिर्ता आएको विधेयक उपर विचार गरी प्रतिनिधि सभाको तत्काल कायम रहेको सदस्य संख्याको बहुमत सदस्यहरूले प्रस्तुत रूपमा वा संशोधन सहित पुनः पारित गरेमा सो विधेयक प्रमाणीकरणका लागि राष्ट्रपति समक्ष पेश गरिनेछ ।छड

(८) उपधारा (६) बमोजिम प्रतिनिधि सभाबाट संशोधन सहित राष्ट्रिय सभामा फिर्ता आएको विधेयक राष्ट्रिय सभाले पनि तत्काल कायम रहेको सदस्य संख्याको बहुमत सदस्यले त्यस्तो संशोधन सहित पुनः पारित गरेमा प्रमाणीकरणका लागि राष्ट्रपति समक्ष पेश गरिनेछ ।

(९) देहाय बमोजिमको विधेयक दुवै सदनको संयुक्त बैठकमा प्रस्तुत गरिनेछ र संयुक्त बैठकले विधेयकलाई प्रस्तुत रूपमा वा संशोधनसहित पारित गरेमा विधेयक उत्पत्ति भएको सदनले प्रमाणीकरणका लागि राष्ट्रपति समक्ष पेश गर्नेछः–

(क) राष्ट्रिय सभाले पारित गरेको तर प्रतिनिधि सभाले अस्वीकार गरेको, वा
(ख) प्रतिनिधि सभाले संशोधन सहित राष्ट्रिय सभामा फिर्ता पठाएको तर राष्ट्रिय सभा त्यस्तो संशोधनमा सहमत हुन नसकेको ।

(१०) कुनै विधेयक विचाराधीन रहेको अवस्थामा सदनको अधिवेशनको अन्त्य भए पनि त्यस्तो विधेयकमाथि आगामी अधिवेशनमा कारबाही हुन सक्नेछ ।

तर कुनै विधेयक प्रतिनिधि सभामा प्रस्तुत भई विचाराधीन रहेको वा प्रतिनिधि सभामा पारित भई राष्ट्रिय सभामा विचाराधीन रहेको अवस्थामा प्रतिनिधि सभा विघटन भएमा वा त्यसको कार्यकाल समाप्त भएमा त्यस्तो विधेयक निष्क्रिय हुनेछ ।

\textbf{११२. विधेयक फिर्ता लिनेः} विधेयक प्रस्तुतकर्ताले सदनको स्वीकृति लिई विधेयक फिर्ता लिन सक्नेछ ।

\textbf{११३. विधेयकमा प्रमाणीकरणः} (१) धारा १११ बमोजिम प्रमाणीकरणका लागि राष्ट्रपति समक्ष पेश गरिने विधेयक उत्पत्ति भएको सदनको सभामुख वा अध्यक्षले प्रमाणित गरी पेश गर्नु पर्नेछ । तर अर्थ विधेयकका हकमा अर्थ विधेयक हो भनी सभामुखले प्रमाणित गर्नु पर्नेछ ।

(२) यस धारा बमोजिम प्रमाणीकरणका लागि राष्ट्रपति समक्ष पेश भएको विधेयक पन्ध्र दिनभित्र प्रमाणीकरण गरी त्यसको सूचना यथासम्भव चाँडो दुवै सदनलाई दिनु पर्नेछ ।

(३) प्रमाणीकरणका लागि पेश भएको अर्थ विधेयक बाहेक अन्य विधेयकमा पुनर्विचार हुनु आवश्यक छ भन्ने राष्ट्रपतिलाई लागेमा त्यस्तोछढ विधेयक पेश भएको पन्ध्र दिनभित्र निजले सन्देश सहित विधेयक उत्पत्ति भएको सदनमा फिर्ता पठाउनेछ ।

(४) राष्ट्रपतिले कुनै विधेयक सन्देश सहित फिर्ता गरेमा त्यस्तो विधेयकमाथि दुवै सदनले पुनर्विचार गरी त्यस्तोे विधेयक प्रस्तुत रूपमा वा
संशोधन सहित पारित गरी पुनः पेश गरेमा त्यसरी पेश भएको पन्ध्र दिनभित्र राष्ट्रपतिले प्रमाणीकरण गर्नेछ ।

(५) राष्ट्रपतिबाट प्रमाणीकरण भएपछि विधेयक ऐन बन्नेछ ।

\textbf{११४. अध्यादेशः} (१) संघीय संसदको दुवै सदनको अधिवेशन चलिरहेको अवस्थामा बाहेक अन्य अवस्थामा तत्काल केही गर्न आवश्यक परेमा मन्त्रिपरिषदको सिफारिसमा राष्ट्रपतिले अध्यादेश जारी गर्न सक्नेछ । (२) उपधारा (१) बमोजिम जारी भएको अध्यादेश ऐन सरह मान्य हुनेछ ।

तर त्यस्तो प्रत्येक अध्यादेश,–

(क) जारी भएपछि बसेको संघीय संसदको दुवै सदनमा पेश गरिनेछ र दुवै सदनले स्वीकार नगरेमा स्वतः निष्क्रिय हुनेछ,
(ख) राष्ट्रपतिबाट जुनसुकै बखत खारेज हुन सक्नेछ, र
(ग) खण्ड (क) वा (ख) बमोजिम निष्क्रिय वा खारेज नभएमा दुवै सदनको बैठक बसेको साठी दिन पछि स्वतः निष्क्रिय हुनेछ ।

स्पष्टीकरणः यस उपधाराको प्रयोजनका लागि “दुवै सदनको बैठक बसेको दिन” भन्नाले संघीय संसदका दुवै सदनको अधिवेशन प्रारम्भ भएको वा बैठक बसेको दिन सम्झनु पर्छ र सो शब्दले संघीय संसदका सदनहरूको बैठक अघिपछि गरी बसेको अवस्थामा जुन सदनको बैठक पछि बसेको छ सोही दिनलाई जनाउँछ ।