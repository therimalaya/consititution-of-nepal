\section{भाग–२५ राष्ट्रिय मानव अधिकार आयोग}

(४) उपधारा (२) बमोजिम नियुक्त अध्यक्ष तथा सदस्यको पुनः नियुक्ति हुन सक्ने छैन ।
तर सदस्यलाई अध्यक्षको पदमा नियुक्ति गर्न सकिनेछ र त्यस्तो सदस्य अध्यक्षको पदमा नियुक्ति भएमा निजको पदावधि गणना गर्दा सदस्य भएको अवधिलाई समेत जोडी गणना गरिनेछ ।

(५) उपधारा (३) मा जुनसुकै कुरा लेखिएको भए तापनि देहायको कुनै अवस्थामा राष्ट्रिय मानव अधिकार आयोगका अध्यक्ष वा सदस्यको पद रिक्त हुनेछः–

(क) निजले राष्ट्रपति समक्ष लिखित राजीनामा दिएमा,

(ख) निजको विरुद्ध धारा १०१ बमोजिम महाभियोगको प्रस्ताव पारित भएमा,

(ग) शारीरिक वा मानसिक अस्वस्थताको कारण सेवामा रही कार्य सम्पादन गर्न असमर्थ रहेको भनी संवैधानिक परिषदको
सिफारिसमा राष्ट्रपतिले पदमुक्त गरेमा,

(घ) निजको मृत्यु भएमा ।

(६) देहायको योग्यता भएको व्यक्ति राष्ट्रिय मानव अधिकार आयोगको अध्यक्ष वा सदस्य पदमा नियुक्तिका लागि योग्य हुनेछः–

(क) अध्यक्षको हकमा मानव अधिकारको संरक्षण र संवर्धनको क्षेत्रमा विशिष्ट योगदान पुर्‍याएका प्रधान न्यायाधीश वा सर्वाेच्च अदालतको न्यायाधीशको पदबाट सेवानिवृत्त व्यक्ति वा मानव अधिकारको संरक्षण र संवर्धन वा राष्ट्रिय जीवनका विभिन्न क्षेत्रमा कम्तीमा बीस वर्ष क्रियाशील रही विशिष्ट योगदान पुर्‍याई ख्यातिप्राप्त गरेको,

(ख) सदस्यको हकमा मानव अधिकारको संरक्षण र संवर्धन, बालबालिकाको हकहितको क्षेत्रमा कार्यरत वा राष्ट्रिय जीवनका विभिन्न क्षेत्रमा कम्तीमा बीस वर्ष क्रियाशील रही विशिष्ट योगदान पुर्‍याई ख्यातिप्राप्त गरेको,

(ग) मान्यताप्राप्त विश्वविद्यालयबाट स्नातक उपाधि हासिल गरेको,

(घ) पैंतालिस वर्ष उमेर पूरा गरेको,

(ङ) नियुक्ति हुँदाका बखत राजनीतिक दलको सदस्य नरहेको,

(च) उच्च नैतिक चरित्र भएको ।

(७) राष्ट्रिय मानव अधिकार आयोगका अध्यक्ष र सदस्यको पारिश्रमिक र सेवाका शर्त संघीय कानून बमोजिम हुनेछ । राष्ट्रिय मानव अधिकार आयोगका अध्यक्ष र सदस्य आफ्नो पदमा बहाल रहेसम्म निजहरूलाई मर्का पर्ने गरी पारिश्रमिक र सेवाका शर्त परिवर्तन गरिने छैन ।
तर चरम आर्थिक विश्रृंखलताका कारण संकटकाल घोषणा भएको अवस्थामा यो व्यवस्था लागू हुने छैन ।

(८) राष्ट्रिय मानव अधिकार आयोगको अध्यक्ष वा सदस्य भइसकेको व्यक्ति अन्य सरकारी सेवामा नियुक्तिका लागि ग्राह्य हुने छैन ।
तर कुनै राजनीतिक पदमा वा कुनै विषयको अनुसन्धान, जाँचबुझ वा छानबीन गर्ने वा कुनै विषयको अध्ययन वा अन्वेषण गरी राय, मन्तव्य वा सिफारिस पेश गर्ने कुनै पदमा नियुक्त भई काम गर्न यस उपधारामा लेखिएको कुनै कुराले बाधा पुर्‍याएको मानिने छैन ।

\textbf{२४९. राष्ट्रिय मानव अधिकार आयोगको काम, कर्तव्य र अधिकारः} (१) मानव अधिकारको सम्मान, संरक्षण र संवर्धन तथा त्यसको प्रभावकारी कार्यान्वयनलाई सुनिश्चित गर्नु राष्ट्रिय मानव अधिकार आयोगको कर्तव्य हुनेछ ।

(२) उपधारा (१) मा उल्लिखित कर्तव्य पूरा गर्न लागि राष्ट्रिय मानव अधिकार आयोगले देहाय बमोजिमका काम गर्नेछः–

(क) कुनै व्यक्ति वा समूहको मानव अधिकार उल्लंघन वा त्यसको दुरुत्साहन भएकोमा पीडित आफैं वा निजको तर्फबाट कसैले
आयोग समक्ष प्रस्तुत वा प्रेषित गरेको निवेदन वा उजूरी वा कुनै स्रोतबाट आयोगलाई प्राप्त भएको वा आयोगको जानकारीमा आएको विषयमा छानबिन तथा अनुसन्धान गरी दोषी उपर कारबाही गर्न सिफारिस गर्ने,

(ख) मानव अधिकारको उल्लंघन हुनबाट रोक्ने जिम्मेवारी वा कर्तव्य भएको पदाधिकारीले आफ्नो जिम्मेवारी पूरा नगरेमा वा कर्तव्य पालन नगरेमा वा जिम्मेवारी पूरा गर्न वा कर्तव्य पालन गर्न उदासीनता देखाएमा त्यस्तो पदाधिकारी उपर विभागीय कारबाही गर्न सम्बन्धित अधिकारी समक्ष सिफारिस गर्ने,

(ग) मानव अधिकार उल्लंघन गर्ने व्यक्ति वा संस्थाका विरुद्ध मुद्दा चलाउनु पर्ने आवश्यकता भएमा कानून बमोजिम अदालतमा मुद्दा दायर गर्न सिफारिस गर्ने,

(घ) मानव अधिकारको चेतना अभिवृद्धि गर्न नागरिक समाजसँग समन्वय र सहकार्य गर्ने,

(ङ) मानव अधिकारको उल्लंघनकर्तालाई विभागीय कारबाही तथा सजाय गर्न कारण र आधार खुलाई सम्बन्धित निकाय समक्ष
सिफारिस गर्ने,

(च) मानव अधिकारसँग सम्बन्धित कानूनको आवधिक रूपमा पुनरावलोकन गर्ने तथा त्यसमा गर्नु पर्ने सुधार तथा संशोधनका सम्बन्धमा नेपाल सरकार समक्ष सिफारिस गर्ने,

(छ) मानव अधिकारसँग सम्बन्धित अन्तर्राष्ट्रिय सन्धि वा सम्झौताको नेपाल पक्ष बन्नु पर्ने भएमा त्यसको कारणसहित नेपाल सरकारलाई सिफारिस गर्ने र नेपाल पक्ष बनिसकेका सन्धि वा सम्झौताको कार्यान्वयन भए वा नभएको अनुगमन गरी कार्यान्वयन नभएको पाइएमा त्यसको कार्यान्वयन गर्न नेपाल सरकार समक्ष सिफारिस गर्ने,

(ज) मानव अधिकारको उल्लंघनका सम्बन्धमा राष्ट्रिय मानव अधिकार आयोगले गरेको सिफारिस वा निर्देशन पालन वा कार्यान्वयन नगर्ने पदाधिकारी, व्यक्ति वा निकायको नाम कानून बमोजिम सार्वजनिक गरी मानव अधिकार उल्लंघनकर्ताको रूपमा अभिलेख राख्ने ।

(३) मानव अधिकार आयोगले आफ्नो कार्य सम्पादन गर्दा वा कर्तव्य पालन गर्दा देहाय बमोजिमको अधिकार प्रयोग गर्न सक्नेछः–
(क) कुनै व्यक्तिलाई आयोग समक्ष उपस्थित गराई जानकारी वा बयान लिने वा बकपत्र गराउने, प्रमाण बुझ्ने, दशी प्रमाण दाखिला गर्न लगाउने सम्बन्धमा अदालतलाई भए सरहको अधिकार प्रयोग गर्ने,

(ख) मानव अधिकारको गम्भीर उल्लंघन हुन लागेको वा भइसकेको सूचना आयोगले कुनै किसिमबाट प्राप्त गरेमा कुनै व्यक्ति वा
निजको आवास वा कार्यालयमा विना सूचना प्रवेश गर्ने, खानतलासी लिने तथा त्यसरी खानतलासी लिंदा मानव अधिकारको उल्लंघनसँग सम्बन्धित लिखत, प्रमाण वा सबुत कब्जामा लिने,

(ग) कुनै व्यक्तिको मानव अधिकार उल्लंघन भइरहेको कुरा जानकारी भई तत्काल कारबाही गर्नु पर्ने आवश्यक देखिएमा विना सूचना सरकारी कार्यालय वा अन्य ठाउँमा प्रवेश गर्ने र उद्धार गर्ने,

(घ) मानव अधिकारको उल्लंघनबाट पीडितलाई कानून बमोजिम क्षतिपूर्ति दिन आदेश दिने ।

(४) राष्ट्रिय मानव अधिकार आयोगले आफ्नो काम, कर्तव्य र अधिकार मध्ये कुनै काम, कर्तव्य र अधिकार सो आयोगको अध्यक्ष, कुनै
सदस्य वा नेपाल सरकारको कर्मचारीलाई तोकिएको शर्तको अधीनमा रही प्रयोग तथा पालन गर्ने गरी प्रत्यायोजन गर्न सक्नेछ ।

(५) राष्ट्रिय मानव अधिकार आयोगको अन्य काम, कर्तव्य र अधिकार तथा कार्यविधि संघीय कानून बमोजिम हुनेछ ।