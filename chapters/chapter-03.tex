\section{भाग–३ मौलिक हक र कर्तव्य}

(१) कानून बमोजिम बाहेक कुनै पनि व्यक्तिलाई वैयक्तिक स्वतन्त्रताबाट वञ्चित गरिने छैन ।

(२) प्रत्येक नागरिकलाई देहायको स्वतन्त्रता हुनेछः–

(क) विचार र अभिव्यक्तिको स्वतन्त्रता,
(ख) विना हातहतियार शान्तिपूर्वक भेला हुने स्वतन्त्रता,
(ग) राजनीतिक दल खोल्ने स्वतन्त्रता,
(घ) संघ र संस्था खोल्ने स्वतन्त्रता,
(ङ) नेपालको कुनै पनि भागमा आवतजावत र बसोबास गर्ने स्वतन्त्रता,
(च) नेपालको कुनै पनि भागमा पेशा, रोजगार गर्ने र उद्योग, व्यापार तथा व्यवसायको स्थापना र सञ्चालन गर्ने स्वतन्त्रता ।

तर

(१) खण्ड (क) को कुनै कुराले नेपालको सार्वभौमसत्ता, भौगोलिक अखण्डता, राष्ट्रियता र स्वाधीनतामा वा संघीय इकाइ वा विभिन्न जात, जाति, धर्म, सम्प्रदायबीचको सु–सम्बन्धमा खलल पर्नेे, जातीय भेदभाव वा छुवाछूतलाई दुरुत्साहन गर्ने, श्रमप्रति अवहेलना गर्ने, गाली बेइज्जती, अदालतको अवहेलना हुने, अपराध गर्न दुरुत्साहन गर्ने वा सार्वजनिक शिष्टाचार वा नैतिकताको प्रतिकूल हुने कार्यमा मनासिब प्रतिबन्ध लगाउने गरी ऐन बनाउन रोक लगाएको मानिने छैन ।

(२) खण्ड (ख) को कुनै कुराले नेपालको सार्वभौमसत्ता, भौगोलिक अखण्डता, राष्ट्रियता र स्वाधीनता, संघीयड इकाइबीचको सु–सम्बन्ध वा सार्वजनिक शान्ति र व्यवस्थामा खलल पर्ने कार्यमा मनासिब प्रतिबन्ध लगाउने गरी ऐन बनाउन रोक लगाएको मानिने छैन ।

(३) खण्ड (ग) को कुनै कुराले नेपालको सार्वभौमसत्ता, भौगोलिक अखण्डता, राष्ट्रियता र स्वाधीनतामा खलल पर्ने, राष्ट्रको विरुद्ध जासूसी गर्ने, राष्ट्रिय गोपनीयता भंग गर्ने वा नेपालको सुरक्षामा आँच पुर्‍याउने गरी कुनै विदेशी राज्य, संगठन वा प्रतिनिधिलाई सहयोग गर्ने वा राज्यद्रोह गर्ने वा संघीय इकाइबीचको सु–सम्बन्धमा खलल पर्ने वा जातीय वा साम्प्रदायिक विद्वेष फैलाउने वा विभिन्न जात, जाति, धर्म र सम्प्रदायबीचको सु–सम्बन्धमा खलल पर्ने वा केवल जाति, भाषा, धर्म, सम्प्रदाय वा लिंगको आधारमा कुनै राजनीतिक दलको सदस्यता प्राप्त गर्ने वा बन्देज लगाउने वा नागरिकहरूबीच विभेद गर्ने गरी राजनीतिक दल गठन गर्ने, हिंसात्मक कार्य गर्न दुरुत्साहन गर्ने वा सार्वजनिक नैतिकताको प्रतिकूल हुने कार्र्यमा मनासिब प्रतिबन्ध लगाउने गरी ऐन बनाउन रोक लगाएको मानिने छैन ।

(४) खण्ड (घ) को कुनै कुराले नेपालको सार्वभौमसत्ता, भौगोलिक अखण्डता, राष्ट्रियता र स्वाधीनतामा खलल पर्ने, राष्ट्रको विरुद्ध जासूसी गर्ने, राष्ट्रिय गोपनीयता भंग गर्ने वा नेपालको सुरक्षामा आँच पुर्‍याउने गरी कुनै विदेशी राज्य, संगठन वा प्रतिनिधिलाई सहयोग गर्ने, राज्यद्रोह गर्ने वा संघीय इकाइबीचको सु–सम्बन्धमा खलल पर्ने वा जातीय वा साम्प्रदायिक विद्वेष फैलाउने वा विभिन्न जात, जाति, धर्म र सम्प्रदायबीचको सु–सम्बन्धमा खलल पर्ने वा हिंसात्मक कार्य गर्न दुरुत्साहन गर्ने वा सार्वजनिक नैतिकताको प्रतिकूल हुने कार्यमा मनासिब प्रतिबन्ध लगाउने गरी ऐन बनाउन रोक लगाएको मानिने छैन ।

(५) खण्ड (ङ) को कुनै कुराले सर्वसाधारण जनताको हित वा संघीय इकाइबीचको सु–सम्बन्ध वा विभिन्न जात, जाति, धर्म वा सम्प्रदायहरूका बीचको सु–सम्बन्धमा खलल पर्ने वा हिंसात्मक कार्य गर्ने वा त्यस्तो कार्य गर्न दुरुत्साहन गर्ने कार्यमा मनासिब प्रतिबन्ध लगाउने गरी ऐन बनाउन रोक लगाएको मानिने छैन ।

(६) खण्ड (च) को कुनै कुराले संघीय इकाइबीचको सु– सम्बन्धमा खलल पुर्‍याउने कार्य वा सर्वसाधारण जनताको सार्वजनिक स्वास्थ्य, शिष्टाचार वा नैतिकताको प्रतिकूल हुने कार्यमा रोक लगाउने वा कुनै खास उद्योग, व्यापार वा सेवा राज्यले मात्र सञ्चालन गर्न पाउने वा कुनै पेशा, रोजगार, उद्योग, व्यापार वा व्यवसाय गर्नका लागि कुनै शर्त वा योग्यता तोक्ने गरी ऐन बनाउन रोक लगाएको मानिने छैन ।

\textbf{१८. समानताको हकः}

(१) सबै नागरिक कानूनको दृष्टिमा समान हुनेछन् । कसैलाई पनि कानूनको समान संरक्षणबाट वञ्चित गरिने छैन ।

(२) सामान्य कानूनको प्रयोगमा उत्पत्ति, धर्म, वर्ण, जात, जाति, लिंग, शारीरिक अवस्था, अपांगता, स्वास्थ्य स्थिति, वैवाहिक स्थिति, गर्भावस्था, आर्थिक अवस्था, भाषा वा क्षेत्र, वैचारिक आस्था वा यस्तै अन्य कुनै आधारमा भेदभाव गरिने छैन ।

(३) राज्यले नागरिकहरूका बीच उत्पत्ति, धर्म, वर्ण, जात, जाति, लिंग,आर्थिक अवस्था, भाषा, क्षेत्र, वैचारिक आस्था वा यस्तै अन्य कुनै आधारमा भेदभाव गर्ने छैन ।

तर सामाजिक वा सांस्कृतिक दृष्टिले पिछडिएका महिला, दलित, आदिवासी, आदिवासी जनजाति, मधेशी, थारू, मुस्लिम, उत्पीडित वर्ग, पिछडा वर्ग, अल्पसंख्यक, सीमान्तीकृत, किसान, श्रमिक, युवा, बालबालिका, ज्येष्ठ नागरिक, लैंगिक तथा यौनिक अल्पसंख्यक, अपांगता भएका व्यक्ति, गर्भावस्थाका व्यक्ति, अशक्त वा असहाय, पिछडिएको क्षेत्र र आर्थिक रूपले  विपन्न खस आर्य लगायत नागरिकको संरक्षण, सशक्तीकरण वा विकासका लागि कानून बमोजिम विशेष व्यवस्था गर्न रोक लगाएको मानिने छैन ।

स्पष्टीकरणः यस भाग र भाग ४ को प्रयोजनका लागि “आर्थिक रूपले विपन्न” भन्नाले संघीय कानूनमा तोकिएको आयभन्दा कम आय भएको व्यक्ति सम्झनु पर्छ ।

(४) समान कामका लागि लैंगिक आधारमा पारिश्रमिक तथा सामाजिक सुरक्षामा कुनै भेदभाव गरिने छैन ।

(५) पैतृक सम्पत्तिमा लैंगिक भेदभाव विना सबै सन्तानको समान हक हुनेछ ।

\textbf{१९. सञ्चारको हकः}

(१) विद्युतीय प्रकाशन, प्रसारण तथा छापा लगायतका जुनसुकै माध्यमबाट कुनै समाचार, सम्पादकीय, लेख, रचना वा अन्य कुनै पाठ्य, श्रव्य, श्रव्यदृश्य सामग्रीको प्रकाशन तथा प्रसारण गर्न वा सूचना प्रवाह गर्न वा छाप्न पूर्व प्रतिबन्ध लगाइने छैन ।

तर नेपालको सार्वभौमसत्ता, भौगोलिक अखण्डता, राष्ट्रियता वा संघीय इकाइबीचको सु–सम्बन्ध वा विभिन्न जात, जाति, धर्म वा सम्प्रदाय बीचको सु–सम्बन्धमा खलल पर्ने, राज्यद्रोह, गाली बेइज्जती वा अदालतको अवहेलना हुने वा अपराध गर्न दुरुत्साहन गर्ने वा सार्वजनिक शिष्टाचार, नैतिकताको प्रतिकूल कार्य गर्ने, श्रमप्रति अवहेलना गर्ने र जातीय छुवाछूत एवं लैंगिक भेदभावलाई दुरुत्साहन गर्ने कार्यमा मनासिब प्रतिबन्ध लगाउने गरी ऐन बनाउन रोक लगाएको मानिने छैन ।

(२) कुनै श्रव्य, श्रव्यदृश्य वा विद्युतीय उपकरणको माध्यम वा छापाखानाबाट कुनै समाचार, लेख, सम्पादकीय, रचना, सूचना वा अन्य कुनै सामग्री मुद्रण वा प्रकाशन, प्रसारण गरे वा छापे बापत त्यस्तो सामग्री प्रकाशन, प्रसारण गर्ने वा छाप्ने रेडियो, टेलिभिजन, अनलाइन वा अन्य कुनै किसिमको डिजिटल वा विद्युतीय उपकरण, छापा वा अन्य सञ्चार माध्यमलाई बन्द, जफत वा दर्ता खारेज वा त्यस्तो सामग्री जफत गरिने छैन ।

तर यस उपधारामा लेखिएको कुनै कुराले रेडियो, टेलिभिजन, अनलाइन वा अन्य कुनै किसिमको डिजिटल वा विद्युतीय उपकरण, छापाखाना वा अन्य सञ्चार माध्यमको नियमन गर्न ऐन बनाउन बन्देज लगाएको मानिने छैन ।

(३) कानून बमोजिम बाहेक कुनै छापा, विद्युतीय प्रसारण तथा टेलिफोन लगायतका सञ्चार साधनलाई अवरुद्ध गरिने छैन ।

\textbf{२०. न्याय सम्बन्धी हकः}

(१) कुनै पनि व्यक्तिलाई पक्राउ भएको कारण सहितको सूचना नदिई थुनामा राखिने छैन ।

(२) पक्राउमा परेका व्यक्तिलाई पक्राउ परेको समयदेखि नै आफूले रोजेको कानून व्यवसायीसँग सल्लाह लिन पाउने तथा कानून व्यवसायीद्वारा पुर्पक्ष गर्ने हक हुनेछ । त्यस्तो व्यक्तिले आफ्नो कानून व्यवसायीसँग गरेको परामर्श र निजले दिएको सल्लाह गोप्य रहनेछ ।

तर शत्रु देशको नागरिकको हकमा यो उपधारा लागू हुने छैन ।

स्पष्टीकरणः यस उपधाराको प्रयोजनका लागि “कानून व्यवसायी” भन्नाले कुनै अड्डा अदालतमा कुनै व्यक्तिको प्रतिनिधित्व गर्न कानूनले अधिकार दिएको व्यक्ति सम्झनु पर्छ ।

(३) पक्राउ गरिएको व्यक्तिलाई पक्राउ भएको समय तथा स्थानबाट बाटोको म्याद बाहेक चौबीस घण्टाभित्र मुद्दा हेर्ने अधिकारी समक्ष उपस्थित गराउनु पर्नेछ र त्यस्तो अधिकारीबाट आदेश भएमा बाहेक पक्राउ भएको व्यक्तिलाई थुनामा राखिने छैन । तर निवारक नजरबन्दमा राखिएका व्यक्ति र शत्रु देशको नागरिकको हकमा यो उपधारा लागू हुने छैन ।

(४) तत्काल प्रचलित कानूनले सजाय नहुने कुनै काम गरे बापत कुनै व्यक्ति सजायभागी हुने छैन र कुनै पनि व्यक्तिलाई कसूर गर्दाको अवस्थामा कानूनमा तोकिएभन्दा बढी सजाय दिइने छैन ।

(५) कुनै अभियोग लागेको व्यक्तिलाई निजले गरेको कसूर प्रमाणित नभएसम्म कसूरदार मानिने छैन ।

(६) कुनै पनि व्यक्ति विरुद्ध अदालतमा एकै कसूरमा एक पटकभन्दा बढी मुद्दा चलाइने र सजाय दिइने छैन ।

(७) कुनै कसूरको अभियोग लागेको व्यक्तिलाई आफ्नो विरुद्ध साक्षी  हुन बाध्य पारिने छैन ।

(८) प्रत्येक व्यक्तिलाई निज विरुद्ध गरिएको कारबाहीको जानकारी पाउने हक हुनेछ ।
(९) प्रत्येक व्यक्तिलाई स्वतन्त्र, निष्पक्ष र सक्षम अदालत वा न्यायिक निकायबाट स्वच्छ सुनुवाइको हक हुनेछ ।

(१०) असमर्थ पक्षलाई कानून बमोजिम निःशुल्क कानूनी सहायता पाउने हक हुनेछ ।

\textbf{२१. अपराध पीडितको हकः}

(१) अपराध पीडितलाई आफू पीडित भएको मुद्दाको अनुसन्धान तथा कारबाही सम्बन्धी जानकारी पाउने हक हुनेछ ।
(२) अपराध पीडितलाई कानून बमोजिम सामाजिक पुनःस्थापना र क्षतिपूर्ति सहितको न्याय पाउने हक हुनेछ ।

\textbf{२२.यातना विरुद्धको हकः}

(१) पक्राउ परेको वा थुनामा रहेको व्यक्तिलाई शारीरिक वा मानसिक यातना दिइने वा निजसँग निर्मम, अमानवीय वा अपमानजनक व्यवहार गरिने छैन ।

(२) उपधारा (१) बमोजिमको कार्य कानून बमोजिम दण्डनीय हुनेछ र त्यस्तो व्यवहारबाट पीडित व्यक्तिलाई कानून बमोजिम क्षतिपूर्ति पाउने हक हुनेछ ।

\textbf{२३. निवारक नजरबन्द विरुद्धको हकः}

(१) नेपालकोे सार्वभौमसत्ता, भौगोलिक अखण्डता वा सार्वजनिक शान्ति र व्यवस्थामा तत्काल खलल पर्ने पर्याप्त आधार नभई कसैलाई पनि निवारक नजरबन्दमा राखिने छैन ।

(२) उपधारा (१) बमोजिम निवारक नजरबन्दमा रहेको व्यक्तिका स्थितिको बारेमा निजको परिवारका सदस्य वा नजिकको नातेदारलाई कानून बमोजिम तत्काल जानकारी दिनु पर्नेछ ।

तर शत्रु देशको नागरिकका हकमा यो उपधारा लागू हुने छैन ।

(३) निवारक नजरबन्दमा राख्ने अधिकारीले कानून विपरीत वा बदनियतपूर्वक कुनै व्यक्तिलाई नजरबन्दमा राखेमा त्यस्तो व्यक्तिलाई कानून बमोजिम क्षतिपूर्ति पाउने हक हुनेछ ।

\textbf{२४. छुवाछूत तथा भेदभाव विरुद्धको हकः}

(१) कुनै पनि व्यक्तिलाई निजको उत्पत्ति, जात, जाति, समुदाय, पेशा, व्यवसाय वा शारीरिक अवस्थाको आधारमा कुनै पनि निजी तथा सार्वजनिक स्थानमा कुनै प्रकारको छुवाछूत वा भेदभाव गरिने छैन ।

(२) कुनै वस्तु, सेवा वा सुविधा उत्पादन वा वितरण गर्दा त्यस्तो वस्तु, सेवा वा सुविधा कुनै खास जात वा जातिको व्यक्तिलाई खरीद वा प्राप्त गर्नबाट रोक लगाइने वा त्यस्तो वस्तु, सेवा वा सुविधा कुनै खास जात वा जातिको व्यक्तिलाई मात्र बिक्री वितरण वा प्रदान गरिने छैन ।

(३) उत्पत्ति, जात, जाति वा शारीरिक अवस्थाको आधारमा कुनै व्यक्ति वा समुदायलाई उच्च वा नीच दर्शाउने, जात, जाति वा छुवाछूतको आधारमा सामाजिक भेदभावलाई न्यायोचित ठान्ने वा छुवाछूत तथा जातीय उच्चता वा घृणामा आधारित विचारको प्रचार प्रसार गर्न वा जातीय विभेदलाई कुनै पनि किसिमले प्रोत्साहन गर्न पाइने छैन ।

(४) जातीय आधारमा छुवाछूत गरी वा नगरी कार्यस्थलमा कुनै प्रकारको भेदभाव गर्न पाइने छैन ।

(५) यस धाराको प्रतिकूल हुने गरी भएका सबै प्रकारका छुवाछूत तथा भेदभावजन्य कार्य गम्भीर सामाजिक अपराधका रूपमा कानून बमोजिम दण्डनीय हुनेछन् र त्यस्तो कार्यबाट पीडित व्यक्तिलाई कानून बमोजिम क्षतिपूर्ति पाउने हक हुनेछ ।

\textbf{२५. सम्पत्तिको हकः}

(१) प्रत्येक नागरिकलाई कानूनको अधीनमा रही सम्पत्ति आर्जन गर्ने, भोग गर्ने, बेचबिखन गर्ने, व्यावसायिक लाभ प्राप्त गर्ने र सम्पत्तिको अन्य कारोबार गर्ने हक हुनेछ ।

तर राज्यले व्यक्तिको सम्पत्तिमा कर लगाउन र प्रगतिशील करको मान्यता अनुरूप व्यक्तिको आयमा कर लगाउन सक्नेछ ।

स्पष्टीकरणः यस धाराको प्रयोजनका लागि “सम्पत्ति” भन्नाले चल अचल लगायत सबै प्रकारको सम्पत्ति सम्झनु पर्छ र सो शब्दले बौद्धिक सम्पत्ति समेतलाई जनाउँछ ।

(२) सार्वजनिक हितका लागि बाहेक राज्यले कुनै व्यक्तिको सम्पत्ति अधिग्रहण गर्ने, प्राप्त गर्ने वा त्यस्तो सम्पत्ति उपर अरु कुनै प्रकारले कुनै अधिकारको सिर्जना गर्ने छैन । तर कुनै पनि व्यक्तिले गैरकानूनी रूपले आर्जन गरेको सम्पत्तिको हकमा यो उपधारा लागू हुने छैन ।

(३) उपधारा (२) बमोजिम सार्वजनिक हितका लागि राज्यले कुनै पनि व्यक्तिको सम्पत्ति अधिग्रहण गर्दा क्षतिपूर्तिको आधार र कार्यप्रणाली ऐन बमोजिम हुनेछ ।

(४) उपधारा (२) र (३) को व्यवस्थाले भूमिको उत्पादन र उत्पादकत्व वृद्धि गर्न, कृषिको आधुनिकीकरण र व्यवसायीकरण, वातावरण संरक्षण, व्यवस्थित आवास तथा शहरी विकास गर्ने प्रयोजनका लागि राज्यले कानून बमोजिम भूमि सुधार, व्यवस्थापन र नियमन गर्न बाधा पर्ने छैन ।

(५) उपधारा (३) बमोजिम राज्यले सार्वजनिक हितका लागि कुनै व्यक्तिको सम्पत्ति अधिग्रहण गरेकोमा त्यस्तो सार्वजनिक हितको सटृा अर्काे कुनै सार्वजनिक हितका लागि त्यस्तो सम्पत्ति प्रयोग गर्न बाधा पर्ने छैन ।

\textbf{२६. धार्मिक स्वतन्त्रताको हकः} (१) धर्ममा आस्था राख्ने प्रत्येक व्यक्तिलाई आफ्नोे आस्था अनुसार धर्मको अवलम्बन, अभ्यास र संरक्षण गर्ने स्वतन्त्रता हुनेछ ।

(२) प्रत्येक धार्मिक सम्प्रदायलाई धार्मिक स्थल तथा धार्मिक गुठी सञ्चालन र संरक्षण गर्ने हक हुनेछ । तर धार्मिक स्थल तथा धार्मिक गुठीको सञ्चालन र संरक्षण गर्न तथा गुठी सम्पत्ति तथा जग्गाको व्यवस्थापनका लागि कानून बनाई नियमित गर्न बाधा पुगेको मानिने छैन ।

(३) यस धाराद्वारा प्रदत्त हकको प्रयोग गर्दा कसैले पनि सार्वजनिक स्वास्थ्य, शिष्टाचार र नैतिकताको प्रतिकूल हुने वा सार्वजनिक शान्ति भंग गर्ने क्रियाकलाप गर्न, गराउन वा कसैको धर्म परिवर्तन गराउने वा अर्काको धर्ममा खलल पर्ने काम वा व्यवहार गर्न वा गराउन हुँदैन त्यस्तो कार्य कानून बमोजिम दण्डनीय हुनेछ ।

\textbf{२७. सूचनाको हकः}

प्रत्येक नागरिकलाई आफ्नो वा सार्वजनिक सरोकारको कुनै पनि विषयको सूचना माग्ने र पाउने हक हुनेछ ।

तर कानून बमोजिम गोप्य राख्नु पर्ने सूचनाको जानकारी दिन कसैलाई बाध्य पारिने छैन ।

\textbf{२८. गोपनीयताको हकः}

कुनै पनि व्यक्तिको जीउ, आवास, सम्पत्ति, लिखत, तथ्यांक, पत्राचार र चरित्र सम्बन्धी विषयको गोपनीयता कानून बमोजिम बाहेक अनतिक्रम्य हुनेछ ।

\textbf{२९. शोषण विरुद्धको हकः}

(१) प्रत्येक व्यक्तिलाई शोषण विरुद्धको हक हुनेछ ।
(२) धर्म, प्रथा, परम्परा, संस्कार, प्रचलन वा अन्य कुनै आधारमा कुनै पनि व्यक्तिलाई कुनै किसिमले शोषण गर्न पाइने छैन ।
(३) कसैलाई पनि बेचबिखन गर्न, दास वा बाँधा बनाउन पाइने छैन ।
(४) कसैलाई पनि निजको इच्छा विरुद्ध काममा लगाउन पाइने छैन । तर सार्वजनिक प्रयोजनका लागि नागरिकलाई राज्यले अनिवार्य
सेवामा लगाउन सक्ने गरी कानून बनाउन रोक लगाएको मानिने छैन ।
(५) उपधारा (३) र (४) विपरीतको कार्य कानून बमोजिम दण्डनीय हुनेछ र पीडितलाई पीडकबाट कानून बमोजिम क्षतिपूर्ति पाउने हक हुनेछ ।
\textbf{३०. स्वच्छ वातावरणको हकः}

(१) प्रत्येक नागरिकलाई स्वच्छ र स्वस्थ वातावरणमा बाँच्न पाउने हक हुनेछ ।
(२) वातावरणीय प्रदूषण वा ह्रासबाट हुने क्षतिबापत पीडितलाई प्रदूषकबाट कानून बमोजिम क्षतिपूर्ति पाउने हक हुुनेछ ।
(३) राष्ट्रको विकास सम्बन्धी कार्यमा वातावरण र विकासबीच समुचित सन्तुलनका लागि आवश्यक कानूनी व्यवस्था गर्न यस धाराले बाधा
पुर्‍याएको मानिने छैन ।

\textbf{३१. शिक्षा सम्बन्धी हकः}

(१) प्रत्येक नागरिकलाई आधारभूत शिक्षामा पहुँचको हक हुनेछ ।
(२) प्रत्येक नागरिकलाई राज्यबाट आधारभूत तहसम्मको शिक्षा अनिवार्य र निःशुल्क तथा माध्यमिक तहसम्मको शिक्षा निःशुल्क पाउने हक हुनेछ ।
(३) अपांगता भएका र आर्थिक रूपले विपन्न नागरिकलाई कानून बमोजिम निःशुल्क उच्च शिक्षा पाउने हक हुनेछ ।
(४) दृष्टिविहीन नागरिकलाई ब्रेललिपि तथा बहिरा र स्वर वा बोलाइ सम्बन्धी अपांगता भएका नागरिकलाई सांकेतिक भाषाको माध्यमबाट कानून बमोजिम निःशुल्क शिक्षा पाउने हक हुनेछ ।
(५) नेपालमा बसोबास गर्ने प्रत्येक नेपाली समुदायलाई कानून बमोजिम आफ्नो मातृभाषामा शिक्षा पाउने र त्यसका लागि विद्यालय तथा
शैक्षिक संस्था खोल्ने र सञ्चालन गर्ने हक हुनेछ ।

\textbf{३२. भाषा तथा संस्कृतिको हकः}

(१) प्रत्येक व्यक्ति र समुदायलाई आफ्नो भाषा प्रयोग गर्ने हक हुनेछ ।
(२) प्रत्येक व्यक्ति र समुदायलाई आफ्नो समुदायको सांस्कृतिक जीवनमा सहभागी हुन पाउने हक हुनेछ ।
(३) नेपालमा बसोबास गर्ने प्रत्येक नेपाली समुदायलाई आफ्नो भाषा, लिपि, संस्कृति, सांस्कृतिक सभ्यता र सम्पदाको संवर्धन र संरक्षण गर्ने हक हुनेछ ।

\textbf{३३. रोजगारीको हकः}

(१) प्रत्येक नागरिकलाई रोजगारीको हक हुनेछ । रोजगारीको शर्त, अवस्था र बेरोजगार सहायता संघीय कानून बमोजिम हुनेछ ।
(२) प्रत्येक नागरिकलाई रोजगारीको छनौट गर्न पाउने हक हुनेछ ।

\textbf{३४. श्रमको हकः}

(१) प्रत्येक श्रमिकलाई उचित श्रम अभ्यासको हक हुनेछ ।

स्पष्टीकरणः यस धाराको प्रयोजनका लागि “श्रमिक” भन्नाले पारिश्रमिक लिई रोजगारदाताका लागि शारीरिक वा बौद्धिक कार्य गर्ने कामदार वा मजदूर सम्झनु पर्छ ।

(२) प्रत्येक श्रमिकलाई उचित पारिश्रमिक, सुविधा तथा योगदानमा आधारित सामाजिक सुरक्षाको हक हुनेछ ।

(३) प्रत्येक श्रमिकलाई कानून बमोजिम ट्रेड युनियन खोल्ने, त्यसमा सहभागी हुने तथा सामूहिक सौदाबाजी गर्न पाउने हक हुनेछ ।

\textbf{३५. स्वास्थ्य सम्बन्धी हकः}

(१) प्रत्येक नागरिकलाई राज्यबाट आधारभूत स्वास्थ्य सेवा निःशुल्क प्राप्त गर्ने हक हुनेछ र कसैलाई पनि आकस्मिक स्वास्थ्य
सेवाबाट वञ्चित गरिने छैन ।
(२) प्रत्येक व्यक्तिलाई आफ्नो स्वास्थ्य उपचारको सम्बन्धमा जानकारी पाउने हक हुनेछ ।
(३) प्रत्येक नागरिकलाई स्वास्थ्य सेवामा समान पहुँचको हक हुनेछ ।
(४) प्रत्येक नागरिकलाई स्वच्छ खानेपानी तथा सरसफाइमा पहुँचको हक हुनेछ ।

\textbf{३६. खाद्य सम्बन्धी हकः }

(१) प्रत्येक नागरिकलाई खाद्य सम्बन्धी हक हुनेछ ।
(२) प्रत्येक नागरिकलाई खाद्यवस्तुको अभावमा जीवन जोखिममा पर्ने  अवस्थाबाट सुरक्षित हुने हक हुनेछ ।
(३) प्रत्येक नागरिकलाई कानून बमोजिम खाद्य सम्प्रभुताको हक हुनेछ ।

\textbf{३७. आवासको हकः}

(१) प्रत्येक नागरिकलाई उपयुक्त आवासको हक हुनेछ ।
(२) कानून बमोजिम बाहेक कुनै पनि नागरिकलाई निजको स्वामित्वमा रहेको वासस्थानबाट हटाइने वा अतिक्रमण गरिने छैन ।

\textbf{३८. महिलाको हकः}

(१) प्रत्येक महिलालाई लैंगिक भेदभाव विना समान वंशीय हक हुनेछ ।
(२) प्रत्येक महिलालाई सुरक्षित मातृत्व र प्रजनन स्वास्थ्य सम्बन्धी हक हुनेछ ।
(३) महिला विरुद्व धार्मिक, सामाजिक, सांस्कृतिक परम्परा, प्रचलन वा अन्य कुनै आधारमा शारीरिक, मानसिक, यौनजन्य, मनोवैज्ञानिक वा अन्य कुनै किसिमको हिंसाजन्य कार्य वा शोषण गरिने छैन । त्यस्तो कार्य कानून बमोजिम दण्डनीय हुनेछ र पीडितलाई कानून बमोजिम क्षतिपूर्ति पाउने हक हुनेछ ।
(४) राज्यका सबै निकायमा महिलालाई समानुपातिक समावेशी सिद्धान्तको आधारमा सहभागी हुने हक हुनेछ ।
(५) महिलालाई शिक्षा, स्वास्थ्य, रोजगारी र सामाजिक सुरक्षामा सकारात्मक विभेदका आधारमा विशेष अवसर प्राप्त गर्ने हक हुनेछ ।
(६) सम्पत्ति तथा पारिवारिक मामिलामा दम्पतीको समान हक हुनेछ ।

\textbf{३९. बालबालिकाको हकः}

(१) प्रत्येक बालबालिकालाई आफ्नो पहिचान सहित नामकरण र जन्मदर्ताको हक हुनेछ ।
(२) प्रत्येक बालबालिकालाई परिवार तथा राज्यबाट शिक्षा, स्वास्थ्य, पालन पोषण, उचित स्याहार, खेलकूद, मनोरञ्जन तथा सर्वांगीण व्यक्तित्व विकासको हक हुनेछ ।
(३) प्रत्येक बालबालिकालाई प्रारम्भिक बाल विकास तथा बाल सहभागिताको हक हुनेछ ।
(४) कुनै पनि बालबालिकालाई कलकारखाना, खानी वा यस्तै अन्य जोखिमपूर्ण काममा लगाउन पाइने छैन ।
(५) कुनै पनि बालबालिकालाई बाल विवाह, गैरकानूनी ओसारपसार र अपहरण गर्न वा बन्धक राख्न पाइने छैन ।
(६) कुनै पनि बालबालिकालाई सेना, प्रहरी वा सशस्त्र समूहमा भर्ना वा प्रयोग गर्न वा सांस्कृतिक वा धार्मिक प्रचलनका नाममा कुनै पनि माध्यम वा प्रकारले दुव्र्यवहार, उपेक्षा वा शारीरिक, मानसिक, यौनजन्य वा अन्य कुनै प्रकारको शोषण गर्न वा अनुचित प्रयोग गर्न पाइने छैन ।
(७) कुनै पनि बालबालिकालाई घर, विद्यालय वा अन्य जुनसुकै स्थान र अवस्थामा शारीरिक, मानसिक वा अन्य कुनै किसिमको यातना दिन पाइने छैन ।
(८) प्रत्येक बालबालिकालाई बाल अनुकूल न्यायको हक हुनेछ ।
(९) असहाय, अनाथ, अपांगता भएका, द्वन्द्वपीडित, विस्थापित एवं जोखिममा रहेका बालबालिकालाई राज्यबाट विशेष संरक्षण र सुविधा पाउने हक हुनेछ ।

(१०) उपधारा (४), (५), (६) र (७) विपरीतका कार्य कानून बमोजिम दण्डनीय हुनेछन् र त्यस्तो कार्यबाट पीडित बालबालिकालाई पीडकबाट कानून बमोजिम क्षतिपूर्ति पाउने हक हुनेछ ।

\textbf{४०. दलितको हकः}

(१) राज्यका सबै निकायमा दलितलाई समानुपातिक समावेशी सिद्धान्तको आधारमा सहभागी हुने हक हुनेछ । सार्वजनिक सेवा लगायतका रोजगारीका अन्य क्षेत्रमा दलित समुदायको सशक्तीकरण, प्रतिनिधित्व र सहभागिताका लागि कानून बमोजिम विशेष व्यवस्था गरिनेछ ।

(२) दलित विद्यार्थीलाई प्राथमिकदेखि उच्च शिक्षासम्म कानून बमोजिम छात्रवृत्ति सहित निःशुल्क शिक्षाको व्यवस्था गरिनेछ । प्राविधिक र व्यावसायिक उच्च शिक्षामा दलितका लागि कानून बमोजिम विशेष व्यवस्था गरिनेछ ।

(३) दलित समुदायलाई स्वास्थ्य र सामाजिक सुरक्षा प्रदान गर्न कानून बमोजिम विशेष व्यवस्था गरिनेछ ।

(४) दलित समुदायलाई आफ्नो परम्परागत पेशा, ज्ञान, सीप र प्रविधिको प्रयोग, संरक्षण र विकास गर्ने हक हुनेछ । राज्यले दलित समुदायका परम्परागत पेशासँग सम्बन्धित आधुनिक व्यवसायमा उनीहरूलाई प्राथमिकता दिई त्यसका लागि आवश्यक पर्ने सीप र स्रोत उपलब्ध गराउनेछ ।

(५) राज्यले भूमिहीन दलितलाई कानून बमोजिम एक पटक जमीन उपलब्ध गराउनु पर्नेछ ।
(६) राज्यले आवासविहीन दलितलाई कानून बमोजिम बसोबासको व्यवस्था गर्नेछ ।
(७) दलित समुदायलाई यस धाराद्वारा प्रदत्त सुविधा दलित महिला, पुरुष र सबै समुदायमा रहेका दलितले समानुपातिक रूपमा प्राप्त गर्ने गरी न्यायोचित वितरण गर्नु पर्नेछ ।

\textbf{४१. ज्येष्ठ नागरिकको हकः} ज्येष्ठ नागरिकलाई राज्यबाट विशेष संरक्षण तथा सामाजिक सुरक्षाको हक हुनेछ ।

\textbf{४२. सामाजिक न्यायको हकः}

(१) आर्थिक, सामाजिक वा शैक्षिक दृष्टिले पछाडि परेका महिला, दलित, आदिवासी जनजाति, मधेशी, थारू, मुस्लिम, पिछडा वर्ग, अल्पसंख्यक, सीमान्तीकृत, अपांगता भएका व्यक्ति, लैंगिक तथा यौनिक अल्पसंख्यक, किसान, श्रमिक, उत्पीडित वा पिछडिएको क्षेत्रका नागरिक तथा आर्थिक रूपले विपन्न खस आर्यलाई समानुपातिक समावेशी सिद्धान्तका आधारमा राज्यका निकायमा सहभागिताको हक हुनेछ । पहिलो संशोधनद्वारा संशोधित ।

(२) आर्थिक रूपले विपन्न तथा लोपोन्मुख समुदायका नागरिकको संरक्षण, उत्थान, सशक्तीकरण र विकासका लागि शिक्षा, स्वास्थ्य, आवास, रोजगारी, खाद्यान्न र सामाजिक सुरक्षामा विशेष अवसर तथा लाभ पाउने हक हुनेछ ।

(३) अपांगता भएका नागरिकलाई विविधताको पहिचान सहित मर्यादा र आत्मसम्मानपूर्वक जीवनयापन गर्न पाउने र सार्वजनिक सेवा तथा
सुविधामा समान पहुँचको हक हुनेछ ।

(४) प्रत्येक किसानलाई कानून बमोजिम कृषि कार्यका लागि भूमिमा पहुँच, परम्परागत रूपमा प्रयोग र अवलम्बन गरिएको स्थानीय बीउ बिजन र कृषि प्रजातिको छनौट र संरक्षणको हक हुनेछ ।

(५) नेपालमा अग्रगामी लोकतान्त्रिक परिवर्तनकोे लागि भएका सबै जन आन्दोलन, सशस्त्र संघर्ष र क्रान्तिका क्रममा जीवन उत्सर्ग गर्ने
शहीदका परिवार, बेपत्ता पारिएका व्यक्तिका परिवार, लोकतन्त्रका योद्धा द्वन्द्वपीडित र विस्थापित, अपांगता भएका व्यक्ति, घाइते तथा पीडितलाई न्याय एवं उचित सम्मान सहित शिक्षा, स्वास्थ्य, रोजगारी, आवास र सामाजिक सुरक्षामा कानून बमोजिम प्राथमिकताका साथ अवसर पाउने हक हुनेछ ।

\textbf{४३. सामाजिक सुरक्षाको हकः}

आर्थिक रूपले विपन्न, अशक्त र असहाय अवस्थामा रहेका, असहाय एकल महिला, अपांगता भएका, बालबालिका, आफ्नो हेरचाह आफैं गर्न नसक्ने तथा लोपोन्मुख जातिका नागरिकलाई कानून बमोजिम सामाजिक सुरक्षाको हक हुनेछ ।

\textbf{४४. उपभोक्ताको हकः}

(१) प्रत्येक उपभोक्तालाई गुणस्तरीय वस्तु तथा सेवा प्राप्त गर्ने हक हुनेछ ।

(२) गुणस्तरहीन वस्तु वा सेवाबाट क्षति पुगेको व्यक्तिलाई कानून बमोजिम क्षतिपूर्ति पाउने हक हुनेछ ।

\textbf{४५. देश निकाला विरुद्धको हकः} कुनै पनि नागरिकलाई देश निकाला गरिने छैन ।

\textbf{४६. संवैधानिक उपचारको हकः} यस भागद्वारा प्रदत्त हकको प्रचलनका लागि धारा १३३ वा १४४ मा लेखिए बमोजिम संवैधानिक उपचार पाउने हक हुनेछ ।

\textbf{४७. मौलिक हकको कार्यान्वयनः} यस भागद्वारा प्रदत्त हकहरूको कार्यान्वयनका लागि आवश्यकता अनुसार राज्यले यो संविधान प्रारम्भ भएको तीन वर्षभित्र कानूनी व्यवस्था गर्नेछ ।

\textbf{४८. नागरिकका कर्तव्यः }प्रत्येक नागरिकका कर्तव्य देहाय बमोजिम हुनेछन्ः–

(क) राष्ट्रप्रति निष्ठावान हुँदै नेपालको राष्ट्रियता, सार्वभौमसत्ता र अखण्डताको रक्षा गर्नु,
(ख) संविधान र कानूनको पालना गर्नु,
(ग) राज्यले चाहेका बखत अनिवार्य सेवा गर्नु,
(घ) सार्वजनिक सम्पत्तिको सुरक्षा र संरक्षण गर्नु