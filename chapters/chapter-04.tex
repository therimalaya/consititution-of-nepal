\section{भाग–४ राज्यका निर्देशक सिद्धान्त, नीति तथा दायित्व}

(१) नेपालको स्वतन्त्रता, सार्वभौमसत्ता, भौगोलिक अखण्डता र स्वाधीनतालाई सर्वोपरि राख्दै नागरिकको जीउ, धन, समानता र स्वतन्त्रताको संरक्षण गरी कानूनको शासन, मौलिक हक तथा मानव अधिकारका मूल्य र मान्यता, लैंगिक समानता, समानुपातिक समावेशीकरण, सहभागिता र सामाजिक न्यायको माध्यमबाट राष्ट्रिय जीवनका सबै क्षेत्रमा न्यायपूर्ण व्यवस्था कायम गर्दै लोककल्याणकारी राज्यव्यवस्थाको स्थापना गर्ने तथा परस्पर सहयोगमा आधारित संघीयताका आधारमा संघीय इकाइहरूबीचको सम्बन्ध सञ्चालन गर्दै स्थानीय स्वायत्तता र विकेन्द्रीकरणको आधारमा शासन व्यवस्थामा समानुपातिक सिद्धान्तलाई आत्मसात् गर्दै लोकतान्त्रिक अधिकारको उपभोग गर्न पाउने अवस्था सुनिश्चित गर्न संघीय लोकतान्त्रिक गणतन्त्रात्मक व्यवस्था सुदृढ गर्ने राज्यको राजनीतिक उद्देश्य हुनेछ ।

(२) धर्म, संस्कृति, संस्कार, प्रथा, परम्परा, प्रचलन वा अन्य कुनै पनि आधारमा हुने सबै प्रकारका विभेद, शोषण र अन्यायको अन्त्य गरी सभ्य र समतामूलक समाजको निर्माण गर्ने एवं राष्ट्रिय गौरव, लोकतन्त्र, जनपक्षीयता, श्रमको सम्मान, उद्यमशीलता, अनुशासन, मर्यादा र सहिष्णुतामा आधारित सामाजिक सांस्कृतिक मूल्यहरूको विकास गर्ने तथा सांस्कृतिक विविधताको सम्मान गर्दै सामाजिक सद्भाव, ऐक्यबद्धता र सामञ्जस्य कायम गरी राष्ट्रिय एकता सुदृढ गर्ने राज्यको सामाजिक र सांस्कृतिक उद्देश्य हुनेछ ।

(३) सार्वजनिक, निजी र सहकारी क्षेत्रको सहभागिता तथा विकास मार्फत उपलब्ध साधन र स्रोतको अधिकतम परिचालनद्वारा तीव्र आर्थिक वृद्धि हासिल गर्दै दिगो आर्थिक विकास गर्ने तथा प्राप्त उपलब्धिहरूको न्यायोचित वितरण गरी आर्थिक असमानताको अन्त्य गर्दै शोषणरहित समाजको निर्माण गर्न राष्ट्रिय अर्थतन्त्रलाई आत्मनिर्भर, स्वतन्त्र तथा उन्नतिशील बनाउँदै समाजवाद उन्मुख स्वतन्त्र र समृद्ध अर्थतन्त्रको विकास गर्ने राज्यको आर्थिक उद्देश्य हुनेछ ।

(४) नेपालको स्वतन्त्रता, सार्वभौमसत्ता, भौगोलिक अखण्डता, स्वाधीनता र राष्ट्रिय हितको रक्षा गर्दै सार्वभौमिक समानताका आधारमा अन्तर्राष्ट्रिय सम्बन्ध कायम गरी विश्व समुदायमा राष्ट्रिय सम्मानको अभिवृद्धि गर्नेतर्फ राज्यको अन्तर्राष्ट्रिय सम्बन्ध निर्देशित हुनेछ ।

\textbf{५१. राज्यका नीतिहरूः}

राज्यले देहायका नीतिहरू अवलम्बन गर्नेछः–

\textbf{(क) राष्ट्रिय एकता र राष्ट्रिय सुरक्षा सम्बन्धी नीतिः}
(१) नेपालको स्वतन्त्रता, सार्वभौमसत्ता, भौगोलिक अखण्डता र स्वाधीनताको संरक्षण गर्दै राष्ट्रिय एकता अक्षुण्ण राख्ने,

(२) विभिन्न जात, जाति, धर्म, भाषा, संस्कृति र सम्प्रदायबीच पारस्परिक सद्भाव, सहिष्णुता र ऐक्यबद्धता कायम गरी संघीय इकाइबीच परस्परमा सहयोगात्मक सम्बन्ध विकास गर्दै राष्ट्रिय एकता प्रवर्धन गर्ने,

(३) राष्ट्रिय सुरक्षा प्रणालीको विकास गरी शान्ति सुरक्षाको व्यवस्था गर्ने,

(४) सर्वांगीण मानवीय सुरक्षाको प्रत्याभूति गर्ने,

(५) राष्ट्रिय सुरक्षा नीतिका आधारमा नेपाली सेना, नेपाल प्रहरी, सशस्त्र प्रहरी, बल नेपाल लगायत सबै सुरक्षा निकायलाई सबल, सुदृढ, व्यावसायिक, समावेशी र जनउत्तरदायी बनाउने,

(६) राष्ट्रिय आवश्यकता अनुरूप नागरिकलाई राष्ट्रको सेवा गर्न तत्पर र सक्षम बनाउने,

(७) पूर्व कर्मचारी, सैनिक र प्रहरी लगायतका पूर्व राष्ट्रसेवकहरूमा रहेको ज्ञान, सीप र अनुभवलाई राष्ट्र हितमा समुचित उपयोग गर्ने ।

\textbf{(ख) राजनीतिक तथा शासन व्यवस्था सम्बन्धी नीतिः}

(१) राजनीतिक उपलब्धिको रक्षा, सुदृढीकरण र विकास गर्दै आर्थिक, सामाजिक तथा सांस्कृतिक रूपान्तरणका माध्यमबाट जनताको सर्वोत्तम हित र समुन्नति प्रत्याभूत गर्ने,

(२) मानव अधिकारको संरक्षण र संवर्धन गर्दै विधिको शासन कायम राख्ने,

(३) नेपाल पक्ष भएका अन्तर्राष्ट्रिय सन्धि सम्झौताहरूको कार्यान्वयन गर्ने,

(४) सार्वजनिक प्रशासनलाई स्वच्छ, सक्षम, निष्पक्ष, पारदर्शी, भ्रष्टाचारमुक्त, जनउत्तरदायी र सहभागितामूलक बनाउँदै राज्यबाट प्राप्त हुने सेवा सुविधामा जनताको समान र सहज पहुँच सुनिश्चित गरी सुशासनको प्रत्याभूति गर्ने,

(५) आमसञ्चारलाई स्वच्छ, स्वस्थ, निष्पक्ष, मर्यादित, जिम्मेवार र व्यावसायिक बनाउन आवश्यक व्यवस्था गर्ने,

(६) संघीय इकाइबीच जिम्मेवारी, स्रोत साधन र प्रशासनको साझेदारी गर्दै सुमधुर र सहयोगात्मक सम्बन्धको विकास र विस्तार गर्ने ।

\textbf{(ग) सामाजिक र सांस्कृतिक रूपान्तरण सम्बन्धी नीतिः}

(१) स्वस्थ र सभ्य संस्कृतिको विकास गरी सामाजिक सुसम्बन्धमा आधारित समाजको निर्माण गर्ने,

(२) ऐतिहासिक, पुरातात्विक तथा सांस्कृतिक सम्पदाको संरक्षण, संवर्धन र विकासका लागि अध्ययन, अनुसन्धान, उत्खनन तथा प्रचार प्रसार गर्ने,

(३) सामाजिक, सांस्कृतिक तथा सेवामूलक कार्यमा स्थानीय समुदायको सिर्जनशीलताको प्रवर्धन र परिचालन गरी स्थानीय जनसहभागिता अभिवृद्धि गर्दै सामुदायिक विकास गर्ने,

(४) राष्ट्रिय सम्पदाको रूपमा रहेका कला, साहित्य र सङ्गीतको विकासमा जोड दिने,

(५) समाजमा विद्यमान धर्म, प्रथा, परम्परा, रीति तथा संस्कारका नाममा हुने सबै प्रकारका विभेद, असमानता, शोषण र
अन्यायको अन्त गर्ने,

(६) देशको सांस्कृतिक विविधता कायम राख्दै समानता एवं सहअस्तित्वका आधारमा विभिन्न जातजाति र समुदायको भाषा, लिपि, संस्कृति, साहित्य, कला, चलचित्र र सम्पदाको संरक्षण र विकास गर्ने,

(७) बहुभाषिक नीति अवलम्बन गर्ने ।

\textbf{(घ) अर्थ, उद्योग र वाणिज्य सम्बन्धी नीतिः}

(१) सार्वजनिक, निजी र सहकारी क्षेत्रको सहभागिता र स्वतन्त्र विकास मार्फत राष्ट्रिय अर्थतन्त्र सुदृढ गर्ने,

(२) अर्थतन्त्रमा निजी क्षेत्रको भूमिकालाई महत्व दिदै उपलब्ध साधन र स्रोतको अधिकतम परिचालन गरी आर्थिक समृद्धि हासिल गर्ने,

(३) सहकारी क्षेत्रलाई प्रवर्धन गर्दै राष्ट्रिय विकासमा अत्यधिक परिचालन गर्ने,

(४) आर्थिक क्षेत्रका सबै गतिविधिमा स्वच्छता, जवाफदेही र प्रतिस्पर्धा कायम गर्न नियमनको व्यवस्था गर्दै सर्वांगीण राष्ट्रिय विकासमा प्रोत्साहन र परिचालन गर्ने,

(५) उपलब्ध साधन, स्रोत तथा आर्थिक विकासको प्रतिफलको न्यायोचित वितरण गर्ने,

(६) तुलनात्मक लाभका क्षेत्रको पहिचान गरी उद्योगको विकास र विस्तारद्वारा निर्यात प्रवर्धन गर्दै वस्तु तथा सेवाको बजार विविधीकरण र विस्तार गर्ने,

(७) कालाबजारी, एकाधिकार, कृत्रिम अभाव सिर्जना गर्ने र प्रतिस्पर्धा नियन्त्रण जस्ता कार्यको अन्त्य गर्दै राष्ट्रिय अर्थतन्त्रलाई प्रतिस्पर्धी बनाई व्यापारिक स्वच्छता र अनुशासन कायम गरी उपभोक्ताको हित संरक्षण गर्ने,

(८) राष्ट्रिय अर्थतन्त्रको विकासका लागि राष्ट्रिय उद्योगधन्दा र साधन स्रोतको संरक्षण र प्रवर्धन गरी नेपाली श्रम, सीप र कच्चा पदार्थमा आधारित स्वदेशी लगानीलाई प्राथमिकता दिने,

(९) राष्ट्रिय अर्थतन्त्रको विकासका लागि स्वदेशी लगानीलाई प्राथमिकता दिने,

(१०) राष्ट्रिय हित अनुकूल आयात प्रतिस्थापन, निर्यात प्रवर्धनका क्षेत्रमा वैदेशिक पूँजी तथा प्रविधिको लगानीलाई आकर्षित गर्दै पूर्वाधार विकासमा प्रोत्साहन एवं परिचालन गर्ने,

(११) वैदेशिक सहायता लिंदा राष्ट्रिय आवश्यकता र प्राथमिकतालाई आधार बनाउँदै यसलाई पारदर्शी बनाउने र वैदेशिक सहायताबाट प्राप्त रकम राष्ट्रिय बजेटमा समाहित गर्ने,

(१२) गैरआवासीय नेपालीहरूको ज्ञान, सीप, प्रविधि र पूँजीलाई राष्ट्रिय विकासमा उपयोग गर्ने,

(१३) औद्योगिक करिडोर, विशेष आर्थिक क्षेत्र, राष्ट्रिय परियोजना, विदेशी लगानीका परियोजनाको सन्दर्भमा अन्तर प्रदेश तथा प्रदेश र संघ बीच समन्वय स्थापित गराई आर्थिक विकासलाई गतिशीलता प्रदान गर्ने ।

\textbf{(ङ) कृषि र भूमिसुधार सम्बन्धी नीतिः}

(१) भूमिमा रहेको दोहोरो स्वामित्व अन्त्य गर्दै किसानको हितलाई ध्यानमा राखी वैज्ञानिक भूमिसुधार गर्ने,

(२) अनुपस्थित भू–स्वामित्वलाई निरुत्साहित गर्दै जग्गाको चक्लाबन्दी गरी उत्पादन र उत्पादकत्व वृद्धि गर्ने,

(३) किसानको हक हित संरक्षण र संवर्धन गर्दै कृषिको उत्पादन र उत्पादकत्व बढाउन भूउपयोग नीतिको अवलम्बन गरी भूमिको व्यवस्थापन र कृषिको व्यवसायीकरण, औद्योगिकीकरण, विविधीकरण र आधुनिकीकरण गर्ने,

(४) भूमिको उत्पादनशीलता, प्रकृति तथा वातावरणीय सन्तुलन समेतका आधारमा नियमन र व्यवस्थापन गर्दै त्यसको समुचित उपयोग गर्ने,

(५) कृषकका लागि कृषि सामग्री, कृषि उपजको उचित मूल्य र बजारमा पहुँचको व्यवस्था गर्ने ।

\textbf{(च) विकास सम्बन्धी नीतिः}

(१) क्षेत्रीय सन्तुलन सहितको समावेशी आर्थिक विकासका लागि क्षेत्रीय विकासको योजना अन्तर्गत दिगो सामाजिक आर्थिक विकासका रणनीति र कार्यक्रमहरू तर्जुमा गरी समन्वयात्मक तवरले कार्यान्वयन गर्ने,

(२) विकासका दृष्टिले पछाडि परेका क्षेत्रलाई प्राथमिकता दिंदै सन्तुलित, वातावरण अनुकूल, गुणस्तरीय तथा दिगो रूपमा भौतिक पूर्वाधारको विकास गर्ने

(३) विकास निर्माणको प्रक्रियामा स्थानीय जनसहभागिता अभिवृद्धि गर्ने,

(४) वैज्ञानिक अध्ययन अनुसन्धान एवं विज्ञान र प्रविधिको  आविष्कार, उन्नयन र विकासमा लगानी अभिवृद्धि गर्नेे तथा वैज्ञानिक, प्राविधिक, बौद्धिक र विशिष्ट प्रतिभाहरूको संरक्षण गर्ने,

(५) राष्ट्रिय आवश्यकता अनुसार सूचना प्रविधिको विकास र विस्तार गरी त्यसमा सर्वसाधारण जनताको सहज र सरल पहुँच सुनिश्चित गर्ने तथा राष्ट्रिय विकासमा सूचना प्रविधिको उच्चतम उपयोग गर्ने,

(६) विकासको प्रतिफल वितरणमा विपन्न नागरिकलाई प्राथमिकता दिंदै आम जनताले न्यायोचित रूपमा पाउने व्यवस्था गर्नेे,

(७) एकीकृत राष्ट्रिय परिचय व्यवस्थापन सूचना प्रणाली विकास गरी नागरिकका सबै प्रकारका सूचना र विवरणहरू एकीकृत रूपमा व्यवस्थापन गर्ने तथा यसलाई राज्यबाट उपलब्ध हुने सेवा सुविधा र राष्ट्रिय विकास योजनासँग आबद्ध गर्ने,

(८) जनसांख्यिक तथ्यांकलाई अद्यावधिक गर्दै राष्ट्रिय विकास योजनासँग आबद्ध गर्ने ।

\textbf{(छ) प्राकृतिक साधन स्रोतको संरक्षण, संवर्धन र उपयोग सम्बन्धी नीतिः}

(१) राष्ट्रिय हित अनुकूल तथा अन्तरपुस्ता समन्यायको मान्यतालाई आत्मसात् गर्दै देशमा उपलब्ध प्राकृतिक स्रोत साधनको संरक्षण, संवर्धन र वातावरण अनुकूल दिगो रूपमा उपयोग गर्ने र स्थानीय समुदायलाई प्राथमिकता र अग्राधिकार दिंदै प्राप्त प्रतिफलहरूको न्यायोचित वितरण गर्ने,

(२) जनसहभागितामा आधारित स्वदेशी लगानीलाई प्राथमिकता दिंदै जलस्रोतको बहुउपयोगी विकास गर्ने,

(३) नवीकरणीय ऊर्जाको उत्पादन तथा विकास गर्दै नागरिकका आधारभूत आवश्यकता परिपूर्तिका लागि सुपथ र सुलभ रूपमा भरपर्दो ऊर्जाको आपूर्ति सुनिश्चित गर्ने तथा ऊर्जाको समुचित प्रयोग गर्ने,द्दठ

(४) जलउत्पन्न प्रकोप नियन्त्रण र नदीको व्यवस्थापन गर्दै दिगो र भरपर्दो सिंचाइको विकास गर्ने,

(५) जनसाधारणमा वातावरणीय स्वच्छता सम्बन्धी चेतना बढाई औद्योगिक एवं भौतिक विकासबाट वातावरणमा पर्न सक्ने जोखिमलाई न्यूनीकरण गर्दै वन, वन्यजन्तु, पक्षी, वनस्पति तथा जैविक विविधताको संरक्षण, संवर्धन र दिगो उपयोग गर्ने,

(६) वातावरणीय सन्तुलनका लागि आवश्यक भूभागमा वन क्षेत्र कायम राख्ने,

(७) प्रकृति, वातावरण वा जैविक विविधतामाथि नकारात्मक असर परेको वा पर्न सक्ने अवस्थामा नकारात्मक वातावरणीय प्रभाव निर्मूल वा न्यून गर्न उपयुक्त उपायहरू अवलम्बन गर्ने,

(८) वातावरण प्रदूषण गर्नेले सो बापत दायित्व ब्यहोर्नुपर्ने तथा वातावरण संरक्षणमा पूर्वसावधानी र पूर्वसूचित सहमति जस्ता पर्यावरणीय दिगो विकासका सिद्धान्त अवलम्बन गर्ने,

(९) प्राकृतिक प्रकोपबाट हुने जोखिम न्यूनीकरण गर्न पूर्व सूचना, तयारी, उद्धार, राहत एवं पुनस्र्थापना गर्ने ।

\textbf{(ज) नागरिकका आधारभूत आवश्यकता सम्बन्धी नीतिः}

(१) शिक्षालाई वैज्ञानिक, प्राविधिक, व्यावसायिक, सीपमूलक, रोजगारमूलक एवं जनमुखी बनाउँदै सक्षम, प्रतिस्पर्धी, नैतिक एवं राष्ट्रिय हितप्रति समर्पित जनशक्ति तयार गर्ने,

(२) शिक्षा क्षेत्रमा राज्यको लगानी अभिवृद्धि गर्दै शिक्षामा भएको निजी क्षेत्रको लगानीलाई नियमन र व्यवस्थापन गरी सेवामूलक बनाउने,

(३) उच्च शिक्षालाई सहज, गुणस्तरीय र पहुँच योग्य बनाई क्रमशः निःशुल्क बनाउँदै लैजाने,

(४) नागरिकको व्यक्तित्व विकासका लागि सामुदायिक सूचना केन्द्र र पुस्तकालयको स्थापना र प्रवर्धन गर्ने,

(५) नागरिकलाई स्वस्थ बनाउन राज्यले जनस्वास्थ्यको क्षेत्रमा आवश्यक लगानी अभिवृद्धि गर्दै जाने,द्दड

(६) गुणस्तरीय स्वास्थ्य सेवामा सबैको सहज, सुलभ र समान पहँुच सुनिश्चित गर्र्ने,

(७) नेपालको परम्परागत चिकित्सा पद्धतिको रूपमा रहेको आयुर्वेदिक, प्राकृतिक चिकित्सा र होमियोपेथिक लगायत स्वास्थ्य पद्धतिको संरक्षण र प्रवर्धन गर्ने,

(८) स्वास्थ्य क्षेत्रमा राज्यको लगानी अभिवृद्धि गर्दै यस क्षेत्रमा भएको निजी लगानीलाई नियमन र व्यवस्थापन गरी सेवामूलक बनाउने,

(९) स्वास्थ्य सेवालाई सर्वसुलभ र गुणस्तरीय बनाउन स्वास्थ्य  अनुसन्धानमा जोड दिंदै स्वास्थ्य संस्था र स्वास्थ्यकर्मीको संख्या वृद्धि गर्दै जाने,

(१०) नेपालको क्षमता र आवश्यकताका आधारमा जनसंख्या व्यवस्थापनका लागि परिवार नियोजनलाई प्रोत्साहित गर्दै मातृ शिशु मृत्युदर घटाई औसत आयु बढाउने,

(११) अव्यवस्थित बसोबासलाई व्यवस्थापन गर्ने तथा योजनाबद्ध र व्यवस्थित बस्ती विकास गर्ने,

(१२) कृषि क्षेत्रमा लगानी अभिवृद्धि गर्दै खाद्य सम्प्रभुताको मान्यता अनुरूप जलवायु र माटो अनुकूलको खाद्यान्न उत्पादनलाई प्रोत्साहन गरी खाद्यान्नको दिगो उत्पादन, आपूर्ति, सञ्चय, सुरक्षा र सुलभ तथा प्रभावकारी वितरणको व्यवस्था गर्ने,

(१३) आधारभूत वस्तु तथा सेवामा सबै नागरिकहरूको समान पहँुच सुनिश्चित गर्दै दुर्गम र पछाडि पारिएको क्षेत्रलाई विशेष प्राथमिकता दिई योजनाबद्ध आपूर्तिको व्यवस्था गर्ने,

(१४) यातायात सुविधामा नागरिकहरूको सरल, सहज र समान पहँुच सुनिश्चित गर्दै यातायात क्षेत्रमा लगानी अभिवृद्धि गर्ने र वातावरण अनुकूल प्रविधिलाई प्राथमिकता दिंदै सार्वजनिक यातायातलाई प्रोत्साहन र निजी यातायातलाई नियमन गरी यातायात क्षेत्रलाई सुरक्षित, व्यवस्थित र अपांगता भएका व्यक्ति अनुकूल बनाउने,

(१५) नागरिकको स्वास्थ्य बीमा सुनिश्चित गर्दै स्वास्थ्य उपचारमा पहुँचको व्यवस्था मिलाउने ।

\textbf{(झ) श्रम र रोजगार सम्बन्धी नीति :}

(१) सबैले काम गर्न पाउने अवस्था सुनिश्चित गर्दै देशको मुख्य सामाजिक आर्थिक शक्तिको रूपमा रहेको श्रमशक्तिलाई दक्ष र व्यावसायिक बनाउने र स्वदेशमा नै रोजगारी अभिवृद्धि गर्ने,

(२) मर्यादित श्रमको अवधारणा अनुरूप सबै श्रमिकको आधारभूत अधिकार सुनिश्चित गर्दै सामाजिक सुरक्षा प्रत्याभूत गर्ने,

(३) बालश्रम लगायत श्रम शोषणका सबै रूपको अन्त्य गर्ने,

(४) श्रमिक र उद्यमी व्यवसायीबीच सुसम्बन्ध कायम गर्दैै व्यवस्थापनमा श्रमिकको सहभागिता प्रोत्साहन गर्ने,

(५) वैदेशिक रोजगारीलाई शोषणमुक्त, सुरक्षित र व्यवस्थित गर्न तथा श्रमिकको रोजगारी र अधिकारको प्रत्याभूति गर्न यस क्षेत्रको नियमन र व्यवस्थापन गर्ने,

(६) वैदेशिक रोजगारीबाट आर्जन भएको पूँजी, सीप, प्रविधि र अनुभवलाई स्वदेशमा उत्पादनमूलक क्षेत्रमा लगाउन
प्रोत्साहन गर्ने ।

\textbf{(ञ) सामाजिक न्याय र समावेशीकरण सम्बन्धी नीतिः}

(१) असहाय अवस्थामा रहेका एकल महिलालाई सीप, क्षमता र योग्यताको आधारमा रोजगारीमा प्राथमिकता दिंदै जीविकोपार्जनका लागि समुचित व्यवस्था गर्दै जाने,

(२) जोखिममा परेका, सामाजिक र पारिवारिक बहिष्करणमा परेका तथा हिंसा पीडित महिलालाई पुनःस्थापना, संरक्षण, सशक्तीकरण गरी स्वावलम्बी बनाउने,

(३) प्रजनन अवस्थामा आवश्यक सेवा सुविधा उपभोगको सुनिश्चितता गर्ने,

(४) बालबच्चाको पालन पोषण, परिवारको हेरचाह जस्ता काम र योगदानलाई आर्थिक रूपमा मूल्यांकन गर्र्नेे,

(५) बालबालिकाको सर्वाेत्तम हितलाई प्राथमिक रूपमा ध्यान दिने,

(६) मुक्त कमैया, कम्हलरी, हरवा, चरवा, हलिया, भूमिहीन, सुकुम्बासीहरूको पहिचान गरी बसोबासका लागि घर घडेरीघण् तथा जीविकोपार्जनका लागि कृषियोग्य जमीन वा रोजगारीको व्यवस्था गर्दै पुनःस्थापना गर्ने,

(७) राष्ट्रिय विकासमा युवा सहभागिता अभिवृद्धि गर्दै राजनीतिक, आर्थिक, सामाजिक र सांस्कृतिक अधिकारहरूको पूर्ण उपयोगको वातावरण सिर्जना गर्ने, युवाको सशक्तीकरण र विकासका लागि शिक्षा, स्वास्थ्य, रोजगारी लगायतका क्षेत्रमा विशेष अवसर प्रदान गर्दै व्यक्तित्व विकास गर्ने तथा राज्यको सर्वांगीण विकासमा योगदानका लागि उपयुक्त अवसर प्रदान गर्ने,

(८) आदिवासी जनजातिको पहिचान सहित सम्मानपूर्वक बाँच्न पाउने अधिकार सुनिश्चित गर्न अवसर तथा लाभका लागि विशेष व्यवस्था गर्दै यस समुदायसँग सरोकार राख्ने निर्णयहरूमा सहभागी गराउने तथा आदिवासी जनजाति र स्थानीय समुदायको परम्परागत ज्ञान, सीप, संस्कृति, सामाजिक परम्परा र अनुभवलाई संरक्षण र संवर्धन गर्र्नेे,

(९) अल्पसंख्यक समुदायलाई आफ्नो पहिचान कायम राखी सामाजिक र सांस्कृतिक अधिकार प्रयोगको अवसर तथा लाभका लागि विशेष व्यवस्था गर्ने,

(१०) मधेशी समुदाय, मुस्लिम र पिछडा वर्गलाई आर्थिक,सामाजिक तथा सांस्कृतिक अवसर र लाभको समान वितरण तथा त्यस्ता समुदायभित्रका विपन्न नागरिकको संरक्षण, उत्थान, सशक्तीकरण र विकासका अवसर तथा लाभका लागि विशेष व्यवस्था गर्ने,

(११) उत्पीडित तथा पिछडिएको क्षेत्रका नागरिकको संरक्षण, उत्थान, सशक्तीकरण, विकास र आधारभूत आवश्यकता परिपूर्तिका अवसर तथा लाभका लागि विशेष व्यवस्था गर्ने,

(१२) सामाजिक सुरक्षा र सामाजिक न्याय प्रदान गर्दा सबै लिंग, क्षेत्र र समुदायभित्रका आर्थिक रूपले विपन्नलाई प्राथमिकता प्रदान गर्ने,

(१३) स्वस्थ, सक्षम र अनुशासित नागरिक तयार गर्न खेलकूद तथा खेलाडीमा योजनाबद्ध लगानी गर्ने र खेलकूदलाई राष्ट्रिय एकता सुदृढ गर्ने एवं अन्तर्राष्ट्रिय क्षेत्रमा राष्ट्रिय सम्मान अभिवृद्धि गर्र्ने माध्यमको रूपमा विकास गर्ने,घज्ञ

(१४) सामुदायिक तथा राष्ट्रिय वा अन्तर्राष्ट्रिय गैरसरकारी संघ संस्थाको लगानी र भूमिकालाई जवाफदेही र पारदर्शी बनाउँदै त्यस्ता संस्थाहरूको स्थापना, स्वीकृति, सञ्चालन, नियमन र व्यवस्थापनका लागि एकद्वार प्रणाली अपनाउने र राष्ट्रिय आवश्यकता र प्राथमिकताका क्षेत्रमा मात्र त्यस्ता संघ संस्थाहरूलाई संलग्न गराउने ।

\textbf{(ट) न्याय र दण्ड व्यवस्था सम्बन्धी नीतिः}

(१) न्याय प्रशासनलाई छिटो छरितो, सर्वसुलभ, मितव्ययी, निष्पक्ष, प्रभावकारी र जनउत्तरदायी बनाउने,

(२) सामान्य प्रकृतिका विवाद समाधानका लागि मेलमिलाप, मध्यस्थता जस्ता वैकल्पिक उपायहरू अवलम्बन गर्ने,

(३) राजनीतिक, प्रशासनिक, न्यायिक, सामाजिक लगायत सबै क्षेत्रको भ्रष्टाचार र अनियमितता नियन्त्रणका लागि प्रभावकारी उपाय अवलम्बन गर्ने ।

\textbf{(ठ) पर्यटन सम्बन्धी नीतिः}

नेपालका ऐतिहासिक, सांस्कृतिक, धार्मिक, पुरातात्विक र प्राकृतिक सम्पदाहरूको पहिचान, संरक्षण, प्रवर्धन एवं प्रचार प्रसार मार्फत राष्ट्रिय अर्थतन्त्रको महत्वपूर्ण आधारको रूपमापर्यावरण अनुकूल पर्यटन उद्योगको विकास गर्ने, पर्यटन संस्कृतिको विकास गर्न आवश्यक वातावरण एवं नीति निर्माण गर्ने तथा पर्यटन उद्योगको लाभ वितरणमा स्थानीय जनतालाई प्राथमिकता दिने ।

\textbf{(ड) अन्तर्राष्ट्रिय सम्बन्ध सम्बन्धी नीतिः}

(१) नेपालको सार्वभौमसत्ता, भौगोलिक अखण्डता, स्वाधीनता र राष्ट्रिय हितको रक्षा गर्न क्रियाशील रहँदै संयुक्त राष्ट्रसंघको बडापत्र, असंलग्नता, पञ्चशीलको सिद्धान्त, अन्तर्राष्ट्रिय कानून र विश्वशान्तिको मान्यताका आधारमा राष्ट्रको सर्वोपरि हितलाई ध्यानमा राखी स्वतन्त्र परराष्ट्र नीति सञ्चालन गर्ने,

(२) विगतमा भएका सन्धिहरूको पुनरावलोकन गर्दै समानता र पारस्परिक हितको आधारमा सन्धि सम्झौताहरू गर्ने ।

\textbf{५२. राज्यको दायित्वः}

नेपालको स्वतन्त्रता, सार्वभौमसत्ता, भौगोलिक अखण्डता र स्वाधीनतालाई अक्षुण्ण राख्दै मौलिक हक तथा मानव अधिकारको संरक्षण र
संवर्र्धन, राज्यका निर्देशक सिद्धान्तहरूको अनुसरण तथा राज्यका नीतिहरूको क्रमशः कार्यान्वयन गर्दै नेपाललाई समृद्ध तथा समुन्नत बनाउने राज्यको दायित्व हुनेछ ।

\textbf{५३. प्रतिवेदन पेश गर्नेः}

यस भागमा उल्लिखित राज्यका निर्देशक सिद्धान्त, नीति र दायित्व कार्यान्वयनका सम्बन्धमा गरेका काम र प्राप्त उपलब्धि सहितको वार्षिक प्रतिवेदन नेपाल सरकारले राष्ट्रपति समक्ष पेश गर्नेछ र राष्ट्रपतिले त्यस्तो प्रतिवेदन प्रधानमन्त्री मार्फत संघीय संसद समक्ष पेश गर्ने व्यवस्था गर्नेछ ।

\textbf{५४. अनुगमन सम्बन्धी व्यवस्थाः} यस भागमा उल्लिखित राज्यका निर्देशक सिद्धान्त, नीति र दायित्वको प्रगतिशील कार्यान्वयन भए नभएको अनुगमन र मूल्यांकन गर्न संघीय संसदमा कानून बमोजिम एक समिति रहनेछ ।

\textbf{५५. अदालतमा प्रश्न उठाउन नसकिनेः} यस भागमा लेखिएका कुनै विषय कार्यान्वयन भए वा नभएको सम्बन्धमा कुनै अदालतमा प्रश्न उठाउन सकिने छैन ।