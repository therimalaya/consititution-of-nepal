\section{भाग–८ संघीय व्यवस्थापिका}

(१) प्रतिनिधि सभामा देहाय बमोजिमका दुई सय पचहत्तर सदस्य रहनेछन्ः–

(क) नेपाललाई जनसंख्या र भौगोलिक अनुकुलता तथा विशिष्टताका आधारमा एक सय पैंसठ्ठी निर्वाचन क्षेत्र कायम गरी प्रत्येक निर्वाचन क्षेत्रबाट एक जना रहने गरी पहिलो हुने निर्वाचित हुने निर्वाचन प्रणाली बमोजिम निर्वाचित हुने एक सय पैंसठ्ठी सदस्य,

(ख) सम्पूर्ण देशलाई एक निर्वाचन क्षेत्र मानी राजनीतिक दललाई मत दिने समानुपातिक निर्वाचन प्रणाली बमोजिम निर्वाचित
हुने एक सय दश सदस्य ।

(२) समानुपातिक निर्वाचन प्रणाली बमोजिम हुने प्रतिनिधि सभाको निर्वाचनका लागि राजनीतिक दलले उम्मेदवारी दिंदा जनसंख्याको आधारमा महिला, दलित, आदिवासी जनजाति, खस आर्य, मधेसी, थारू, मुस्लिम, पिछडिएको क्षेत्र समेतबाट बन्द सूचीका आधारमा प्रतिनिधित्व गराउने व्यवस्था संघीय कानून बमोजिम हुनेछ । त्यसरी उम्मेदवारी दिंदा भूगोल र प्रादेशिक सन्तुलनलाई समेत ध्यान दिनु पर्नेछ ।

स्पष्टीकरणः यस उपधाराको प्रयोजनका लागि “खस आर्य” भन्नाले क्षेत्री, ब्राम्हण, ठकुरी, संन्यासी (दशनामी) समुदाय सम्झनु पर्छ ।

(३) उपधारा (२) बमोजिम राजनीतिक दलले उम्मेदवारी दिंदा अपांगता भएको व्यक्तिको समेत प्रतिनिधित्व हुने व्यवस्था गर्नु पर्नेछ ।

(४) उपधारा (१) बमोजिम हुने प्रतिनिधि सभा सदस्यको निर्वाचन कानून बमोजिम गोप्य मतदानद्वारा हुनेछ ।

(५) अठार वर्ष उमेर पूरा भएको प्रत्येक नेपाली नागरिकलाई संघीय कानून बमोजिम कुनै एक निर्वाचन क्षेत्रमा मतदान गर्ने अधिकार हुनेछ ।

(६) प्रतिनिधि सभाका सदस्यका लागि हुने निर्वाचनमा मतदान गर्न अधिकार पाएको धारा ८७ बमोजिम योग्यता पुगेको व्यक्ति संघीय कानूनको अधीनमा रही कुनै पनि निर्वाचन क्षेत्रबाट उम्मेदवार हुन पाउनेछ । तर एउटै व्यक्ति एकभन्दा बढी निर्वाचन क्षेत्रमा एकै पटक उम्मेदवार हुन पाउने छैन ।

(७) प्रतिनिधि सभाको कार्यकाल छ महीनाभन्दा बढी अवधि बाँकी छँदै कुनै सदस्यको स्थान रिक्त भएमा त्यस्तो स्थान जुन निर्वाचन प्रणालीबाट पूर्ति भएको थियो सोही प्रक्रियाद्वारा पूर्ति गरिनेछ ।

(८) यस भागमा अन्यत्र जुनसुकै कुरा लेखिएको भए तापनि संघीय संसदमा प्रतिनिधित्व गर्ने प्रत्येक राजनीतिक दलबाट निर्वाचित कुल सदस्य संख्याको कम्तीमा एक तिहाइ सदस्य महिला हुनु पर्नेछ । त्यसरी निर्वाचित गर्दा उपधारा (१) को खण्ड (क) र धारा ८६ को उपधारा (२) को खण्ड (क) बमोजिम निर्वाचित सदस्यहरू मध्ये कुनै राजनीतिक दलको एक तिहाइ सदस्य महिला निर्वाचित हुन नसकेमा त्यस्तो राजनीतिक दलले उपधारा (१) को खण्ड (ख) बमोजिम सदस्य निर्वाचित गर्दा आफ्नो दलबाट संघीय संसदमा निर्वाचित हुने कुल सदस्यको कम्तीमा एक तिहाइ महिला सदस्य हुने गरी निर्वाचित गर्नु पर्नेछ ।

(९) प्रतिनिधि सभाको निर्वाचन र तत्सम्बन्धी अन्य विषय संघीय कानून बमोजिम हुनेछ ।

\textbf{८५. प्रतिनिधि सभाको कार्यकालः} (१) यस संविधान बमोजिम अगावै विघटन भएकोमा बाहेक प्रतिनिधि सभाको कार्यकाल पाँच वर्षको हुनेछ ।

(२) उपधारा (१) मा जुनसुकै कुरा लेखिएको भए तापनि संकटकालीन अवस्थाको घोषणा वा आदेश लागू रहेको अवस्थामा संघीय ऐन बमोजिम प्रतिनिधि सभाको कार्यकाल एक वर्षमा नबढ्ने गरी थप गर्न सकिनेछ ।

(३) उपधारा (२) बमोजिम थप गरिएको प्रतिनिधि सभाको कार्यकाल संकटकालीन अवस्थाको घोषणा वा आदेश खारेज भएको मितिले छ महीना पुगेपछि स्वतः समाप्त हुनेछ ।

\textbf{८६. राष्ट्रिय सभाको गठन र सदस्यहरूको पदावधिः}

(१) राष्ट्रिय सभा एक स्थायी सदन हुनेछ ।

(२) राष्ट्रिय सभामा देहाय बमोजिमका उनान्साठी सदस्य रहनेछन्ः–

(क) प्रदेश सभाका सदस्य, गाउँपालिकाका अध्यक्ष र उपाध्यक्ष तथा नगरपालिकाका प्रमुख र उपप्रमुख रहेको निर्वाचकद्धट मण्डलद्वारा संघीय कानून बमोजिम प्रदेश सभाका सदस्य, गाउँपालिकाका अध्यक्ष र उपाध्यक्ष तथा नगरपालिकाका प्रमुख र उपप्रमुखको मतको भार फरक हुने गरी प्रत्येक प्रदेशबाट कम्तीमा तीन जना महिला, एक जना दलित र एक जना अपांगता भएका व्यक्ति वा अल्पसंख्यक सहित
आठ जना गरीे निर्वाचित छपन्न जना,

(ख) नेपाल सरकारको सिफारिसमा राष्ट्रपतिबाट मनोनीत कम्तीमा एक जना महिला सहित तीन जना ।

(३) राष्ट्रिय सभाका सदस्यहरूको पदावधि छ वर्षको हुनेछ । राष्ट्रिय सभाका एक तिहाइ सदस्यको पदावधि प्रत्येक दुई वर्षमा समाप्त हुनेछ ।
तर यो संविधान प्रारम्भ भएपछि पहिलो पटक सदस्यको पदावधि कायम गर्दा गोला प्रथाद्वारा एक तिहाइको दुई वर्ष, अर्को एक तिहाइको चार वर्ष र बाँकी एक तिहाइको छ वर्षको हुने गरी पदावधि कायम गरिनेछ ।

(४) यो संविधान प्रारम्भ भएपछि राष्ट्रिय सभाका सदस्यको पहिलो पटक पदावधि गणना गर्दा राष्ट्रिय सभाको प्रथम बैठक बसेको दिनबाट
सम्पूर्ण सदस्यहरूको पदावधि प्रारम्भ भएको मानिनेछ ।

(५) राष्ट्रिय सभाको रिक्त हुन आउने स्थानको पूर्ति त्यस्तो स्थान रिक्त गर्ने सदस्यको निर्वाचन वा मनोनयन जुन तरीकाले भएको थियो सोह तरीकाले बाँकी अवधिका लागि गरिनेछ ।

(६) राष्ट्रिय सभाका सदस्यको निर्वाचन सम्बन्धी अन्य व्यवस्था संघीय कानून बमोजिम हुनेछ ।

\textbf{८७. सदस्यका लागि योग्यताः}

(१) देहायको योग्यता भएको व्यक्ति संघीय संसदको सदस्य हुन योग्य हुनेछः–

(क) नेपालको नागरिक,
(ख) प्रतिनिधि सभाका लागि पच्चीस वर्ष र राष्ट्रिय सभाका लागि पैंतीस वर्ष उमेर पूरा भएको,
(ग) नैतिक पतन देखिने फौजदारी कसूरमा सजाय नपाएको,
(घ) कुनै संघीय कानूनले अयोग्य नभएको, र
(ङ) कुनै लाभको पदमा बहाल नरहेको ।

स्पष्टीकरणः यस खण्डको प्रयोजनका लागि “लाभको पद” भन्नाले निर्वाचन वा मनोनयनद्वारा पूर्ति गरिने राजनीतिक पद बाहेक सरकारी कोषबाट पारिश्रमिक वा आर्थिक सुविधा पाउने अन्य पद सम्झनु पर्छ ।

(२) कुनै पनि व्यक्ति एकै पटक दुवै सदनको सदस्य हुन सक्ने छैन ।

(३) निर्वाचन, मनोनयन वा नियुक्ति हुने राजनीतिक पदमा बहाल रहेकोे व्यक्ति यस भाग बमोजिम संघीय संसदको सदस्य पदमा निर्वाचित वा मनोनीत भएमा संघीय संसदको सदस्य पदको शपथ ग्रहण गरेको दिनदेखि निजको त्यस्तो पद स्वतः रिक्त हुनेछ ।

\textbf{८८. शपथः} संघीय संसदको प्रत्येक सदनका सदस्यले सदन वा त्यसको कुनै समितिको बैठकमा पहिलो पटक भाग लिनु अघि संघीय कानून बमोजिम शपथ लिनु पर्नेछ ।

\textbf{८९. स्थानको रिक्तताः} देहायको कुनै अवस्थामा संघीय संसदको सदस्यको स्थान रिक्त हुनेछः–

(क) निजले सभामुख वा अध्यक्ष समक्ष लिखित राजीनामा दिएमा,
(ख) निजको धारा ८७ बमोजिमको योग्यता नभएमा वा नरहेमा,
(ग) प्रतिनिधि सभाको कार्यकाल वा राष्ट्रिय सभा सदस्यको पदावधि समाप्त भएमा,
(घ) निज सम्बन्धित सदनलाई सूचना नदिई लगातार दशवटा बैठकमा अनुपस्थित रहेमा,
(ङ) जुन दलको उम्मेदवार भई सदस्य निर्वाचित भएको हो त्यस्तो दलले संघीय कानून बमोजिम निजले दल त्याग गरेको कुरा सूचित गरेमा,
(च) निजको मृत्यु भएमा ।

\textbf{९०. सदस्यका लागि अयोग्यता सम्बन्धी निर्णयः} संघीय संसदको कुनै सदस्य धारा ८७ बमोजिम अयोग्य छ वा हुन गएको छ भन्ने प्रश्न उठेमा त्यसको अन्तिम निर्णय सर्वाेच्च अदालतको संवैधानिक इजलासले गर्नेछ ।

\textbf{९१. प्रतिनिधि सभाको सभामुख र उपसभामुखः} (१) प्रतिनिधि सभाको पहिलो बैठक प्रारम्भ भएको मितिले पन्ध्र दिनभित्र प्रतिनिधि सभाका सदस्यहरूले आफूमध्येबाट प्रतिनिधि सभाको सभामुख र उपसभामुखको निर्वाचन गर्नेछन् ।

(२) उपधारा (१) बमोजिम निर्वाचन गर्दा प्रतिनिधि सभाको सभामुख र उपसभामुख मध्ये एक जना महिला हुने गरी गर्नु पर्नेछ र प्रतिनिधि सभाको सभामुख र उपसभामुख फरक फरक दलको प्रतिनिधि हुनु पर्नेछ । तर प्रतिनिधि सभामा एकभन्दा बढी दलको प्रतिनिधित्व नभएको वा प्रतिनिधित्व भएर पनि उम्मेदवारी नदिएको अवस्थामा एकै दलको सदस्य प्रतिनिधि सभाको सभामुख र उपसभामुख हुन बाधा पर्ने छैन ।
(३) प्रतिनिधि सभाको सभामुख वा उपसभामुखको पद रिक्त भएमा प्रतिनिधि सभाका सदस्यहरूले आफूमध्येबाट प्रतिनिधि सभाको सभामुख र उपसभामुखको निर्वाचन गरी रिक्त स्थानको पूर्ति गर्ने छन् ।

(४) प्रतिनिधि सभाको सभामुखको अनुपस्थितिमा उपसभामुखले प्रतिनिधि सभाको अध्यक्षता गर्नेछ ।

(५) प्रतिनिधि सभाको सभामुख र उपसभामुखको निर्वाचन नभएको वा दुवै पद रिक्त भएको अवस्थामा प्रतिनिधि सभाको बैठकको अध्यक्षता उपस्थित सदस्य मध्ये उमेरको हिसाबले ज्येष्ठ सदस्यले गर्नेछ ।

(६) देहायको कुनै अवस्थामा प्रतिनिधि सभाको सभामुख वा  उपसभामुखको पद रिक्त हुनेछः–

(क) निज प्रतिनिधि सभाको सदस्य नरहेमा, तर प्रतिनिधि सभा विघटन भएको अवस्थामा आफ्नो पदमा बहाल रहेका प्रतिनिधि सभाका सभामुख र उपसभामुख प्रतिनिधि सभाका लागि हुने अर्को निर्वाचनको उम्मेदवारी दाखिल गर्ने अघिल्लो दिनसम्म आफ्नो पदमा बहाल रहनेछन् ।
(ख) निजले लिखित राजीनामा दिएमा,
(ग) निजले पद अनुकूल आचरण नगरेको भन्ने प्रस्ताव प्रतिनिधि सभाको तत्काल कायम रहेको सम्पूर्ण सदस्य संख्याको दुई तिहाइ बहुमतबाट पारित भएमा ।
(७) प्रतिनिधि सभाको सभामुखले पद अनुकूलको आचरण नगरेको भन्ने प्रस्ताव उपर छलफल हुने बैठकको अध्यक्षता प्रतिनिधि सभाको उपसभामुखले गर्नेछ । त्यस्तो प्रस्तावको छलफलमा प्रतिनिधि सभाको सभामुखले भाग लिन र मत दिन पाउनेछ ।द्धढ

\textbf{९२. राष्ट्रिय सभाको अध्यक्ष र उपाध्यक्षः} (१) राष्ट्रिय सभाको पहिलो बैठक प्रारम्भ भएको मितिले पन्ध्र दिनभित्र राष्ट्रिय सभाका सदस्यहरूले आफूमध्येबाट राष्ट्रिय सभाको अध्यक्ष र उपाध्यक्षको निर्वाचन गर्नेछन् ।

(२) उपधारा (१) बमोजिम निर्वाचन गर्दा राष्ट्रिय सभाको अध्यक्ष र उपाध्यक्ष मध्ये एक जना महिला हुने गरी गर्नु पर्नेछ ।

(३) राष्ट्रिय सभाको अध्यक्ष वा उपाध्यक्षको पद रिक्त भएमा राष्ट्रिय सभाका सदस्यहरूले आफूमध्येबाट राष्ट्रिय सभाको अध्यक्ष वा उपाध्यक्षको निर्वाचन गरी रिक्त स्थानको पूर्ति गर्ने छन् ।

(४) राष्ट्रिय सभाको अध्यक्षकोे अनुपस्थितिमा राष्ट्रिय सभाको उपाध्यक्षले राष्ट्रिय सभाको अध्यक्षता गर्नेछ ।

(५) राष्ट्रिय सभाको अध्यक्ष र उपाध्यक्षको निर्वाचन नभएको वा पद रिक्त भएको अवस्थामा राष्ट्रिय सभाको बैठकको अध्यक्षता उपस्थित
सदस्यमध्ये उमेरको हिसाबले ज्येष्ठ सदस्यले गर्नेछ ।

(६) देहायको कुनै अवस्थामा राष्ट्रिय सभाको अध्यक्ष वा उपाध्यक्षको पद रिक्त हुनेछः–
(क) निज राष्ट्रिय सभाको सदस्य नरहेमा,
(ख) निजले लिखित राजीनामा दिएमा,
(ग) निजले पद अनुकूल आचरण नगरेको भन्ने प्रस्ताव राष्ट्रिय सभाको तत्काल कायम रहेको सम्पूर्ण सदस्य संख्याको दुई तिहाइ बहुमतबाट पारित भएमा ।

(७) राष्ट्रिय सभाको अध्यक्षले पद अनुकूलको आचरण नगरेको भन्ने प्रस्ताव उपर छलफल हुने बैठकको अध्यक्षता राष्ट्रिय सभाको उपाध्यक्षले गर्नेछ । त्यस्तो प्रस्तावको छलफलमा राष्ट्रिय सभाको अध्यक्षले भाग लिन र मत दिन पाउनेछ ।

\textbf{९३. अधिवेशनको आव्हान र अन्त्यः} (१) राष्ट्रपतिले प्रतिनिधि सभाका लागि भएको निर्वाचनको अन्तिम परिणाम घोषणा भएको मितिले तीस दिनभित्र संघीय संसदको अधिवेशन आव्हान गर्नेछ । त्यसपछि यस संविधान बमोजिम राष्ट्रपतिले समय समयमा दुवै वा कुनै सदनको अधिवेशन आव्हान गर्नेछ । तर एउटा अधिवेशनको समाप्ति र अर्को अधिवेशनको प्रारम्भका बीचको अवधि छ महीनाभन्दा बढी हुने छैन ।छण्

(२) राष्ट्रपतिले संघीय संसदको दुवै वा कुनै सदनको अधिवेशनको अन्त्य गर्न सक्नेछ ।

(३) राष्ट्रपतिले प्रतिनिधि सभाको अधिवेशन चालू नरहेको वा बैठक स्थगित भएको अवस्थामा अधिवेशन वा बैठक बोलाउन वाञ्छनीय छ भनी प्रतिनिधि सभाको सम्पूर्ण सदस्य संख्याको एक चौथाइ सदस्यहरूले लिखित अनुरोध गरेमा त्यस्तो अधिवेशन वा बैठक बस्ने मिति र समय तोक्नेछ । त्यसरी तोकिएको मिति र समयमा प्रतिनिधि सभाको अधिवेशन प्रारम्भ हुने वा बैठक बस्नेछ ।

\textbf{९४. गणपूरक संख्याः} यस संविधानमा अन्यथा लेखिएकोमा बाहेक संघीय संसदको कुनै पनि सदनको बैठकमा सम्पूर्ण सदस्य संख्याको एक चौथाइ सदस्य उपस्थित नभएसम्म कुनै प्रश्न वा प्रस्ताव निर्णयका लागि प्रस्तुत हुने छैन ।

\textbf{९५. राष्ट्रपतिबाट सम्बोधनः} (१) राष्ट्रपतिले संघीय संसदको कुनै सदनको बैठक वा दुवै सदनको संयुक्त बैठकलाई सम्बोधन गर्न र त्यसका लागि सदस्यको उपस्थिति आह्वान गर्न सक्नेछ ।

(२) राष्ट्रपतिले प्रतिनिधि सभाको निर्वाचन पछिको पहिलो अधिवेशन र प्रत्येक वर्षको पहिलो अधिवेशन प्रारम्भ भएपछि संघीय संसदको दुवै सदनको संयुक्त बैठकलाई सम्बोधन गर्नेछ ।

\textbf{९६. उपप्रधानमन्त्री, मन्त्री, राज्य मन्त्री र सहायक मन्त्रीले दुवै सदनको बैठकमा भाग लिन पाउनेः}

उपप्रधानमन्त्री, मन्त्री, राज्य मन्त्री र सहायक मन्त्रीले संघीय संसदको कुनै सदन वा त्यसको समितिमा उपस्थित हुन र कारबाही
तथा छलफलमा भाग लिन पाउने छन् । तर आफू सदस्य नभएको सदन वा त्यसको समितिमा मतदान गर्न पाउने छैनन् ।

\textbf{९७. समितिको गठनः} (१) प्रतिनिधि सभा र राष्ट्रिय सभाले संघीय कानून बमोजिम समितिहरू गठन गर्न सक्नेछन् ।
(२) संघीय संसदको दुई सदनका बीचको कार्यप्रणालीलाई व्यवस्थित गर्न, कुनै विधेयकमा रहेको मतभिन्नता अन्त्य गर्न वा अन्य कुनै खास
कार्यका लागि दुवै सदनको संयुक्त समिति गठन गरियोस् भनी कुनै सदनले प्रस्ताव पारित गरेमा संयुक्त समितिको गठन गरिनेछ । त्यस्तो संयुक्त समितिमा प्रतिनिधि सभाका सदस्य पाँच जना र राष्ट्रिय सभाका सदस्य एक जनाको अनुपातमा समावेशिताको आधारमा बढीमा पच्चीस जना सदस्य रहनेछन् ।छज्ञ

\textbf{९८. सदस्यको स्थान रिक्त रहेको अवस्थामा सदनको कार्य सञ्चालनः}

संघीय संसदकोे कुनै सदस्यको स्थान रिक्त भए पनि सदनले आफ्नो कार्य सञ्चालन गर्न सक्नेछ । संघीय संसदको कुनै सदनको कारबाहीमा भाग लिन नपाउने कुनै व्यक्तिले भाग लिएको कुरा पछि पत्ता लाग्यो भने पनि भइसकेको कार्य अमान्य हुने छैन ।

\textbf{९९. मतदानः}

यस संविधानमा अन्यथा व्यवस्था गरिएकोमा बाहेक संघीय संसदको कुनै सदनमा निर्णयका लागि प्रस्तुत गरिएको जुनसुकै प्रस्तावको
निर्णय उपस्थित भई मतदान गर्ने सदस्यहरूको बहुमतबाट हुनेछ । अध्यक्षता गर्ने व्यक्तिलाई मत दिने अधिकार हुने छैन ।

तर मत बराबर भएमा निजले आफ्नो निर्णायक मत दिनेछ ।

\textbf{१००. विश्वासको मत र अविश्वासको प्रस्ताव सम्बन्धी व्यवस्थाः}

(१) प्रधानमन्त्रीले कुनै पनि बखत आफूमाथि प्रतिनिधि सभाको विश्वास छ भन्ने कुरा स्पष्ट गर्न आवश्यक वा उपयुक्त ठानेमा विश्वासको मतका लागि प्रतिनिधि सभा समक्ष प्रस्ताव राख्न सक्नेछ ।

(२) प्रधानमन्त्रीले प्रतिनिधित्व गर्ने दल विभाजित भएमा वा सरकारमा सहभागी दलले आफ्नो समर्थन फिर्ता लिएमा तीस दिनभित्र प्रधानमन्त्रीले विश्वासको मतका लागि प्रतिनिधि सभा समक्ष प्रस्ताव राख्नु पर्नेछ ।

(३) उपधारा (१) र (२) बमोजिम पेश भएको प्रस्ताव प्रतिनिधि सभामा तत्काल कायम रहेका सम्पूर्ण सदस्य संख्याको बहुमतबाट पारित हुन नसकेमा प्रधानमन्त्री आफ्नो पदबाट मुक्त हुनेछ ।

(४) प्रतिनिधि सभामा तत्काल कायम रहेका सम्पूर्ण सदस्यहरू मध्ये एक चौथाइ सदस्यले प्रधानमन्त्रीमाथि सदनको विश्वास छैन भनी लिखित रूपमा अविश्वासको प्रस्ताव पेश गर्न सक्ने छन् ।

तर प्रधानमन्त्री नियुक्त भएको पहिलो दुई वर्षसम्म र एक पटक राखेको अविश्वासको प्रस्ताव असफल भएको एक वर्ष भित्र अविश्वासको
प्रस्ताव पेश गर्न सकिने छैन । (५) उपधारा (४) बमोजिम अविश्वासको प्रस्ताव पेश गर्दा प्रधानमन्त्रीका लागि प्रस्तावित सदस्यको नाम समेत उल्लेख गरेकोे हुनु पर्नेछ ।

(६) उपधारा (४) बमोजिम पेश भएको अविश्वासको प्रस्ताव प्रतिनिधि सभामा तत्काल कायम रहेका सम्पूर्ण सदस्य संख्याको बहुमतबाट पारित भएमा प्रधानमन्त्री पदमुक्त हुनेछ ।

(७) उपधारा (६) बमोजिम अविश्वासको प्रस्ताव पारित भई प्रधानमन्त्रीको पद रिक्त भएमा उपधारा (५) बमोजिम प्रस्ताव गरिएको प्रतिनिधि सभा सदस्यलाई राष्ट्रपतिले धारा ७६ बमोजिम प्रधानमन्त्री नियुक्त गर्नेछ ।

\textbf{१०१. महाभियोगः}

(१) यो संविधान र कानूनको गम्भीर उल्लंघन गरेको आधारमा प्रतिनिधि सभामा तत्काल कायम रहेको सम्पूर्ण सदस्य संख्याको एक चौथाइ सदस्यले राष्ट्रपति वा उपराष्ट्रपति विरुद्ध महाभियोगको प्रस्ताव पेश गर्न सक्नेछन् । त्यस्तो प्रस्ताव संघीय संसदको दुवै सदनको तत्काल कायम रहेका सम्पूर्ण सदस्य संख्याको कम्तीमा दुई तिहाइ बहुमतबाट पारित भएमा निज पदबाट मुक्त हुनेछ ।

(२) यो संविधान र कानूनको गंभीर उल्लंघन गरेको, कार्यक्षमताको अभाव वा खराब आचरण भएको वा इमानदारीपूर्वक आफ्नो पदीय कर्तव्यको पालन नगरेको वा आचार संहिताको गम्भीर उल्लंघन गरेको कारणले आफ्नो पदीय जिम्मेवारी पूरा गर्न नसकेको आधारमा प्रतिनिधि सभामा तत्काल कायम रहेको सम्पूर्ण सदस्य संख्याको एक चौथाइ सदस्यले नेपालको प्रधान न्यायाधीश वा सर्वोच्च अदालतका न्यायाधीश, न्याय परिषदका सदस्य, संवैधानिक निकायका प्रमुख वा पदाधिकारीका विरुद्ध महाभियोगको प्रस्ताव पेश गर्न सक्नेछन् । त्यस्तो प्रस्ताव प्रतिनिधि सभामा तत्काल कायम रहेका सम्पूर्ण सदस्य संख्याको कम्तीमा दुई तिहाइ बहुमतबाट पारित भएमा सम्बन्धित व्यक्ति पदबाट मुक्त हुनेछ ।

(३) उपधारा (२) बमोजिमको कुनै व्यक्तिको विरुद्ध महाभियोगको प्रस्ताव पेश गर्ने आधार र कारण विद्यमान भए नभएको छानबीन गरी सिफारिस गर्ने प्रयोजनका लागि प्रतिनिधि सभामा एक महाभियोग सिफारिस समिति रहनेछ ।

(४) उपधारा (३) बमोजिमको समितिमा प्रतिनिधि सभाका एघार जना सदस्य रहनेछन् ।

(५) उपधारा (२) बमोजिम महाभियोगबाट पदमुक्त हुने व्यक्तिले संविधानको गम्भीर उल्लंघन गरेको वा कार्यक्षमताको अभाव वा खराब आचरण वा पदीय दायित्वको पालन इमानदारीपूर्वक नगरेको वा आचार संहिताको गम्भीर उल्लंघन गरेको भन्ने आधारमा प्राप्त सूचना, जानकारी वा उजुरी ग्राह्य रहेको भनी प्रतिनिधि सभाका कम्तीमा तीन जना सदस्यले प्रमाणित गरी पेश गरेमा उपधारा (३) बमोजिमको समितिले त्यस्तो उजुरीमाथि संघीय कानून बमोजिम छानबिन गरी महाभियोग सम्बन्धीछघ कारबाहीका लागि प्रतिनिधि सभा समक्ष सिफारिस गरेमा उपधारा (२) बमोजिम महाभियोगको प्रस्ताव पेश हुन सक्नेछ ।

(६) उपधारा (२) बमोजिम महाभियोगको कारबाही प्रारम्भ भएपछि नेपालको प्रधान न्यायाधीश वा सर्वोच्च अदालतका न्यायाधीश, न्यायपरिषदका सदस्य, संवैधानिक निकायका प्रमुख वा पदाधिकारीले त्यस्तो कारबाहीको टुंगो नलागेसम्म आफ्नो पदको कार्य सम्पादन गर्न पाउने छैन ।

(७) उपधारा (१) वा (२) बमोजिम महाभियोगको आरोप लागेको व्यक्तिलाई सफाइ पेश गर्ने मनासिब मौका दिनु पर्नेछ ।

(८) यस धारा बमोजिम महाभियोगको प्रस्ताव पारित भई पदमुक्त भएका राष्ट्रपति वा उपराष्ट्रपति, नेपालको प्रधान न्यायाधीश वा सर्वोच्च अदालतका न्यायाधीश, न्यायपरिषदका सदस्य, संवैधानिक निकायका प्रमुख वा पदाधिकारीले पदमा रहँदा कुनै कसूर गरेको भए त्यस्तो कसूरमा संघीय कानून बमोजिम कारबाही गर्न बाधा पर्ने छैन ।

(९) उपधारा (१) वा (२) बमोजिम महाभियोगको प्रस्ताव पारित भई पदमुक्त भएकोे व्यक्तिले त्यस्तो पदबाट पाउने कुनै सुविधा लिन र भविष्यमा कुनै पनि सार्वजनिक पदमा नियुक्ति वा मनोनयन हुन सक्ने छैन ।

(१०) महाभियोग सम्बन्धी अन्य व्यवस्था संघीय कानून बमोजिम हुनेछ ।

\textbf{१०२. अनधिकार उपस्थित भएमा वा मतदान गरेमा सजायः} धारा ८८ बमोजिम शपथ नलिएको वा संघीय संसदको सदस्य नभएको कुनै व्यक्ति सदस्यको हैसियतले संघीय संसदको कुनै सदन वा त्यसको समितिको बैठकमा उपस्थित भएमा वा मतदान गरेमा निजलाई त्यस्तो बैठकको अध्यक्षता गर्ने व्यक्तिको आदेशले त्यसरी उपस्थित भएको वा मतदान गरेको प्रत्येक पटकका लागि पाँच हजार रुपैयाँ जरिबाना हुनेछ र त्यस्तो जरिबाना सरकारी बाँकी सरह असुल उपर गरिनेछ ।

\textbf{१०३. विशेषाधिकारः} (१) यस संविधानको अधीनमा रही संघीय संसदको दुवै सदनमा पूर्ण वाक् स्वतन्त्रता रहनेछ र सदनमा व्यक्त गरेको कुनै कुरा वा दिएको कुनै मतलाई लिएर कुनै पनि सदस्यलाई पक्राउ गरिने, थुनामा राखिने वा निज उपर कुनै अदालतमा कारबाही चलाइने छैन ।

(२) यस संविधानको अधीनमा रही संघीय संसदको प्रत्येक सदनलाई आफ्नो काम कारबाही र निर्णय गर्ने पूर्ण अधिकार रहनेछ र सदनको कुनै कारबाही नियमित छ वा छैन भनी निर्णय गर्ने अधिकार सम्बन्धित सदनलाई मात्र हुनेछ । यस सम्बन्धमा कुनै अदालतमा प्रश्न उठाइने छैन ।

(३) संघीय संसदको कुनै सदनको कुनै पनि कारबाहीमाथि त्यसको असल नियतबारे शंका उठाई कुनै टीका–टिप्पणी गरिने छैन र कुनै सदस्यले बोलेको कुनै कुराको सम्बन्धमा जानी–जानी गलत वा भ्रामक अर्थ लगाई कुनै प्रकारको प्रकाशन वा प्रसारण गर्न पाइने छैन ।

(४) उपधारा (१) र (३) को व्यवस्था संघीय संसदको सदस्य बाहेक सदनको बैठकमा भाग लिन पाउने अन्य व्यक्तिका हकमा पनि लागू हुनेछ ।

(५) संघीय संसदको कुनै सदनले दिएको अधिकार अन्तर्गत कुनै लिखत, प्रतिवेदन, मतदान वा कारबाही प्रकाशित गरेको विषयलाई लिएर कुनै व्यक्ति उपर अदालतमा कारबाही चल्ने छैन ।

स्पष्टीकरणः यो उपधारा र उपधारा (१), (२), (३) र (४) को प्रयोजनका लागि “सदन” भन्नाले प्रतिनिधि सभा वा राष्ट्रिय सभा सम्झनु पर्छ र सो शब्दले संघीय संसदको संयुक्त बैठक वा समिति वा संयुक्त समितिलाई समेत जनाउँछ ।

(६) संघीय संसदको कुनै पनि सदस्यलाई अधिवेशन बोलाइएको सूचना जारी भएपछि अधिवेशन अन्त्य नभएसम्मको अवधिभर पक्राउ गरिने छैन । तर कुनै फौजदारी अभियोगमा कुनै सदस्यलाई संघीय कानून बमोजिम पक्राउ गर्न यस उपधाराले बाधा पुर्‍याएको मानिने छैन । त्यसरी कुनै सदस्य पक्राउ गरिएमा पक्राउ गर्ने अधिकारीले त्यसको सूचना सम्बन्धित सदनको अध्यक्षता गर्ने व्यक्तिलाई तुरुन्त दिनु पर्नेछ ।

(७) विशेषाधिकारको हननलाई संघीय संसदको अवहेलना मानिनेछ र कुनै विशेषाधिकारको हनन भएको छ वा छैन भन्ने सम्बन्धमा निर्णय गर्ने अधिकार सम्बन्धित सदनलाई मात्र हुनेछ ।

(८) कसैले कुनै सदनको अवहेलना गरेमा सम्बन्धित सदनको अध्यक्षता गर्ने व्यक्तिले सदनको निर्णयबाट सो व्यक्तिलाई सचेत गराउन, नसीहत दिन वा तीन महीनामा नबढ्ने गरी कैद गर्न वा दश हजार रुपैयाँसम्म जरिबाना गर्न सक्नेछ र त्यस्तो जरिबाना सरकारी बाँकी सरह असुल उपर गरिनेछ । तर सम्बन्धित सदनलाई सन्तोष हुने गरी त्यस्तो व्यक्तिले क्षमायाचना गरेमा सदनले क्षमा प्रदान गर्न वा तोकिसकेको सजायलाई माफी गर्न वा घटाउन सक्नेछ ।छछ

(९) संघीय संसदको विशेषाधिकार सम्बन्धी अन्य व्यवस्था संघीय कानून बमोजिम हुनेछ ।

\textbf{१०४. कार्य सञ्चालन विधिः} (१) संघीय संसदको प्रत्येक सदनले आफ्नो कार्य सञ्चालन गर्न, बैठकको सुव्यवस्था कायम राख्न र समितिहरूको गठन, काम, कारबाही र कुनै सदन वा समितिको कार्यविधि नियमित गर्न नियमावली बनाउनेछ । त्यसरी नियमावली नबनेसम्म संघीय संसदले आफ्नो कार्यविधि आफै नियमित गर्नेछ ।

(२) संघीय संसदको संयुक्त बैठकको कार्य सञ्चालन र संघीय संसदको संयुक्त समितिको गठन र काम कारबाही संघीय संसदको दुवै सदनको संयुक्त बैठकले स्वीकृत गरेको नियमावली वा कार्यविधि बमोजिम हुनेछ ।

\textbf{१०५. बहसमा बन्देजः} नेपालको कुनै अदालतमा विचाराधीन मुद्दाहरूका सम्बन्धमा न्याय निरूपणमा प्रतिकूल असर पार्ने विषय तथा न्यायाधीशले कर्तव्य पालनको सिलसिलामा गरेको न्यायिक कार्यको सम्बन्धमा संघीय संसदको कुनै सदनमा छलफल गरिने छैन । तर महाभियोगको प्रस्तावमा छलफल गर्दा न्यायाधीशको आचरणको सम्बन्धमा कुनै कुरा व्यक्त गर्न यस धाराले बाधा पुर्‍याएको मानिने छैन ।

\textbf{१०६. संघीय संसदको महासचिव र सचिवः} (१) राष्ट्रपतिले प्रतिनिधि सभाको सभामुख र राष्ट्रिय सभाको अध्यक्ष दुवैको संयुक्त सिफारिसमा संघीय संसदको महासचिव, सभामुखको सिफारिसमा प्रतिनिधि सभाको सचिव र राष्ट्रिय सभाको अध्यक्षको सिफारिसमा राष्ट्रिय सभाको सचिव नियुक्त गर्नेछ ।

(२) संघीय संसदको महासचिव, प्रतिनिधि सभाको सचिव र राष्ट्रिय सभाको सचिवको योग्यता, काम, कर्तव्य, अधिकार तथा सेवाका अन्य शर्त संघीय कानून बमोजिम हुनेछ ।

\textbf{१०७. संघीय संसदको सचिवालयः} संघीय संसदको काम कारबाही सञ्चालन तथा व्यवस्थापन गर्न एक सचिवालय रहनेछ । त्यस्तो सचिवालयको स्थापना र तत्सम्बन्धी अन्य व्यवस्था संघीय कानून बमोजिम हुनेछ ।

\textbf{१०८. पारिश्रमिकः} प्रतिनिधि सभाको सभामुख र उपसभामुख, राष्ट्रिय सभाको अध्यक्ष र उपाध्यक्ष, समितिका सभापति तथा संघीय संसदका सदस्यहरूको पारिश्रमिक र सुविधा संघीय कानून बमोजिम हुनेछ । त्यस्तो कानून नबनेसम्म नेपाल सरकारले तोके बमोजिम हुनेछ ।

 