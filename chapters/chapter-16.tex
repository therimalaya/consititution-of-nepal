\section{भाग–१६ प्रदेश आर्थिक कार्यप्रणाली}

(क) प्रदेश सञ्चित कोषमाथि व्ययभार भएको रकम,

(ख) विनियोजन ऐन बमोजिम खर्च हुने रकम,

(ग) विनियोजन विधेयक विचाराधीन रहेको अवस्थामा पेश्कीको रूपमा ऐन बमोजिम खर्च हुने रकम, वा
(घ) विशेष अवस्थामा व्ययको विवरण मात्र भएको उधारो खर्च ऐनद्वारा व्यय हुने रकम ।

तर प्रदेश आकस्मिक कोषका हकमा धारा २१२ बमोजिम हुनेछ ।

\textbf{२०६. प्रदेश सञ्चित कोषमाथि व्ययभारः} देहायका विषयसँग सम्बन्धित खर्च प्रदेश सञ्चित कोषमाथि व्ययभार हुनेछ र त्यस्तो व्ययका लागि प्रदेश सभाको स्वीकृति आवश्यक पर्ने छैन :–

(क) प्रदेश सभामुख र प्रदेश उपसभामुखलाई दिइने पारिश्रमिक र सुविधाका रकम,
(ख) प्रदेश लोकसेवा आयोगका अध्यक्ष र सदस्यलाई दिईने पारिश्रमिक र सुविधाका रकम,
(ग) प्रदेश सरकारको दायित्वको ऋण सम्बन्धी व्ययभार,
(घ) प्रदेश सरकारको विरुद्ध अदालतबाट भएको फैसला वा आदेश अनुसार तिर्नु पर्ने रकम, र
(ङ) प्रदेश कानूनले प्रदेश सञ्चित कोषमाथि व्ययभार हुने भनी निर्धारण गरेको रकम ।

\textbf{२०७. राजस्व र व्ययको अनुमान :}

(१) प्रदेशको अर्थमन्त्रीले प्रत्येक आर्थिक वर्षको सम्बन्धमा प्रदेश सभा समक्ष देहायका कुरा समेत खुलाई वार्षिक अनुमान
पेश गर्न सक्नेछः–
(क) राजस्वको अनुमान,
(ख) प्रदेश सञ्चित कोषमाथि व्ययभार हुने आवश्यक रकमहरू, र
(ग) प्रदेश विनियोजन ऐन बमोजिम व्यय हुने आवश्यक रकमहरू ।

२) उपधारा (१) बमोजिम वार्षिक अनुमान पेश गर्दा अघिल्लो आर्थिक वर्षमा प्रत्येक मन्त्रालयलाई छुट्याइएको खर्चको रकम र खर्च अनुसारको लक्ष्य हासिल भयो वा भएन त्यसको विवरण पनि साथै पेश गर्नु पर्नेछ ।

\textbf{२०८. प्रदेश विनियोजन ऐनः} प्रदेश विनियोजन ऐन बमोजिम व्यय हुने रकम शीर्षकमा उल्लेख गरी विनियोजन विधेयकमा राखिनेछन् ।

\textbf{२०९. पूरक अनुमानः}

(१) कुनै आर्थिक वर्षमा देहायको अवस्था पर्न आएमा प्रदेशको अर्थमन्त्रीले प्रदेश सभा समक्ष पूरक अनुमान पेश गर्न सक्नेछः–

(क) चालू आर्थिक वर्षका लागि प्रदेश विनियोजन ऐन बमोजिम कुनै सेवाका लागि खर्च गर्न अख्तियारी दिइएको रकम अपर्याप्त भएमा वा त्यस वर्षका लागि प्रदेश विनियोजन ऐनले अधिकार नदिएको नयाँ सेवामा खर्च गर्न आवश्यक भएमा, वा

(ख) चालू आर्थिक वर्षमा प्रदेश विनियोजन ऐन बमोजिम अख्तियारी दिएको रकमभन्दा बढी खर्च हुन गएमा ।

(२) पूरक अनुमानमा राखिएको रकम सम्बन्धित शीर्षकमा उल्लेख गरी पूरक विनियोजन विधेयकमा राखिनेछ ।

\textbf{२१०. पेश्की खर्चः}

(१) यस भागमा अन्यत्र जुनसुकै कुरा लेखिएको भए तापनि प्रदेश विनियोजन विधेयक विचाराधीन रहेको अवस्थामा आर्थिक वर्षका लागि
अनुमान गरिएको व्ययको कुनै अंश पेश्कीका रूपमा प्रदेश ऐन बमोजिम खर्च गर्न सकिनेछ ।

(२) धारा २०७ बमोजिम राजस्व र व्ययको अनुमान पेश नगरिएसम्म पेश्की खर्च विधेयक प्रस्तुत गरिने छैन र पेश्कीको रकम आर्थिक वर्षको व्यय अनुमानको एक तिहाइ भन्दा बढी हुने छैन ।

(३) प्रदेश पेश्की खर्च ऐन बमोजिम खर्च भएको रकम प्रदेश विनियोजन विधेयकमा समावेश गरिनेछ ।

\textbf{२११. उधारो खर्चः} यस भागमा अन्यत्र जुनसुकै कुरा लेखिएको भए तापनि प्राकृतिक कारण वा अन्य कारणले गर्दा प्रदेशमा संकटको अवस्था परी धारा २०७ को उपधारा (१) बमोजिम चाहिने विवरण खुलाउन अव्यावहारिक वा प्रदेशको सुरक्षा वा हितको दृष्टिले अवाञ्छनीय देखिएमा प्रदेशको अर्थमन्त्रीले व्ययको विवरण मात्र भएको उधारो खर्च विधेयक प्रदेश सभा समक्ष पेश गर्न सक्नेछ ।

\textbf{२१२. प्रदेश आकस्मिक कोषः}

(१) प्रदेश ऐन बमोजिम प्रदेश आकस्मिक कोषको नामले एउटा कोष स्थापना गर्न सकिनेछ र त्यस्तो कोषमा समय समयमा प्रदेश ऐन बमोजिम निर्धारण भएको रकम जम्मा गरिनेछ ।

(२) उपधारा (१) बमोजिमको कोष प्रदेश सरकारको नियन्त्रणमा रहनेछ । प्रदेश सरकारले त्यस्तो कोषबाट आकस्मिक कार्यका लागि खर्च गर्न सक्नेछ ।

(३) उपधारा (२) बमोजिमको खर्चको रकम प्रदेश ऐन बमोजिम यथाशीध्र सोधभर्ना गरिनेछ ।

\textbf{२१३. आर्थिक कार्यविधि सम्बन्धी ऐनः} प्रदेश ऐन बमोजिम विनियोजित रकम एक शीर्षकबाट अर्को शीर्षकमा रकमान्तर गर्ने र आर्थिक कार्यविधि सम्बन्धी अन्य व्यवस्था प्रदेश ऐन बमोजिम हुनेछ ।