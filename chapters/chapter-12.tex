\section{भाग–१२ महान्यायाधिवक्ता}

(३) सर्वोच्च अदालतको न्यायाधीश हुने योग्यता भएको व्यक्ति महान्यायाधिवक्ताको पदमा नियुक्तिका लागि योग्य हुनेछ ।

(४) देहायको कुनै अवस्थामा महान्यायाधिवक्ताको पद रिक्त हुनेछः–

(क) निजले प्रधानमन्त्री मार्फत राष्ट्रपति समक्ष लिखित राजीनामा दिएमा,
(ख) निजलाई प्रधानमन्त्रीको सिफारिसमा राष्ट्रपतिबाट पदमुक्त गरिएमा,
(ग) निजको मृत्यु भएमा ।

(५) महान्यायाधिवक्ताको पारिश्रमिक तथा अन्य सुविधा सर्वोच्चअदालतको न्यायाधीश सरह हुनेछ । महान्यायाधिवक्ताको सेवाका अन्य शर्त कानून बमोजिम हुनेछन् ।

\textbf{१५८. महान्यायाधिवक्ताको काम, कर्तव्य र अधिकारः}

(१) महान्यायाधिवक्ता नेपाल सरकारको मुख्य कानूनी सल्लाहकार हुनेछ । संवैधानिक एवं कानूनी विषयमा नेपाल सरकार र नेपाल सरकारले तोकिदिएको अन्य अधिकारीलाई राय सल्लाह दिनु महान्यायाधिवक्ताको कर्तव्य हुनेछ ।

(२) नेपाल सरकारको हक, हित वा सरोकार निहित रहेको मुद्दामा महान्यायाधिवक्ता वा निजको मातहतका सरकारी वकीलबाट नेपाल
सरकारको प्रतिनिधित्व गरिनेछ । यस संविधानमा अन्यथा व्यवस्था भएकोमा बाहेक कुनै अदालत वा न्यायिक निकाय वा अधिकारी समक्ष नेपाल सरकारको तर्पmबाट मुद्दा चलाउने वा नचलाउने भन्ने कुराको अन्तिम निर्णय गर्ने अधिकार महान्यायाधिवक्तालाई हुनेछ ।

(३) नेपाल सरकारको तर्फबाट दायर भएको मुद्दा फिर्ता लिंदा महान्यायाधिवक्ताको राय लिनु पर्नेछ ।

(४) महान्यायाधिवक्ताले संघीय संसद वा त्यसको कुनै समितिलेडण् गरेको आमन्त्रण बमोजिम त्यस्तो बैठकमा उपस्थित भई कानूनी प्रश्नको सम्बन्धमा राय व्यक्त गर्न सक्नेछ ।

(५) आफ्नो पदीय कर्तव्यको पालना गर्दा महान्यायाधिवक्तालाई नेपालको जुनसुकै अदालत, कार्यालय र पदाधिकारी समक्ष उपस्थित हुने
अधिकार हुनेछ ।

(६) उपधारा (२) को अतिरिक्त महान्यायाधिवक्तालाई आफ्नो कर्तव्य पालन गर्दा देहायको काम गर्ने अधिकार हुनेछः–

(क) नेपाल सरकार वादी वा प्रतिवादी भई दायर भएका मुद्दा मामिलामा नेपाल सरकारको तर्फबाट प्रतिरक्षा गर्ने,
(ख) मुद्दा मामिलाका रोहमा सर्वोच्च अदालतले गरेको कानूनको व्याख्या वा प्रतिपादन गरेको कानूनी सिद्धान्तको कार्यान्वयन  भए वा नभएको अनुगमन गर्ने वा गराउने,
(ग) हिरासतमा रहेको व्यक्तिलाई यस संविधानको अधीनमा रही मानवोचित व्यवहार नगरेको वा त्यस्तो व्यक्तिलार्ई आफन्तसँग वा कानून व्यवसायी मार्फत भेटघाट गर्न नदिएको भन्ने उजुरी परेमा वा जानकारी हुन आएमा छानबीन गरी त्यस्तो हुनबाट रोक्न सम्बन्धित अधिकारीलाई आवश्यक निर्देशन दिने ।

(७) महान्यायाधिवक्ताले यो धारा बमोजिम आफ्नो काम, कर्तव्य र अधिकार तोकिएको शर्तको अधीनमा रही प्रयोग र पालन गर्ने गरी मातहतका सरकारी वकीललाई प्रत्यायोजन गर्न सक्नेछ ।

(८) यस धारामा लेखिएका काम, कर्तव्य र अधिकारको अतिरिक्त महान्यायाधिवक्ताको अन्य काम, कर्तव्य र अधिकार यो संविधान र संघीय कानून बमोजिम हुनेछ ।

\textbf{१५९. वार्षिक प्रतिवेदनः}

(१) महान्यायाधिवक्ताले प्रत्येक वर्ष यो संविधान र संघीय कानून बमोजिम आपूmले सम्पादन गरेको कामको वार्षिक प्रतिवेदन तयार गरी
राष्ट्रपति समक्ष पेश गर्नेछ र राष्ट्रपतिले प्रधानमन्त्री मार्फत त्यस्तो प्रतिवेदन संघीय संसद समक्ष पेश गर्न लगाउनेछ ।

(२) उपधारा (१) बमोजिम पेश गरिने प्रतिवेदनमा अन्य कुराको अतिरिक्त महान्यायाधिवक्ताले वर्षभरिमा संवैधानिक एवं कानूनी विषयमा दिएको राय सल्लाहको संक्षिप्त विवरण सहितको संख्या, सरकारवादी भई चलेका मुद्दा सम्बन्धी विवरण, नेपाल सरकार वादी वा प्रतिवादी भई दायर भएका मुद्दा मामिलामा प्रतिरक्षा गरेको विवरण, नेपाल सरकार वादी भई चल्ने मुद्दामा भविष्यमा गरिनुपर्ने सुधार तथा अपराधका प्रवृत्ति सम्बन्धी विवरणउल्लेख गर्नु पर्नेछ ।

\textbf{१६०. मुख्य न्यायाधिवक्ताः}

(१) महान्यायाधिवक्ताको मातहतमा रहने गरी प्रत्येक प्रदेशमा एक मुख्य न्यायाधिवक्ता रहनेछ ।
(२) मुख्य न्यायाधिवक्ताको नियुक्ति सम्बन्धित मुख्यमन्त्रीको सिफारिसमा प्रदेश प्रमुखबाट हुनेछ । मुख्यमन्त्रीले चाहेको अवधिसम्म मुख्य न्यायाधिवक्ता आफ्नो पदमा बहाल रहनेछ ।
(३) उच्च अदालतको न्यायाधीश हुने योग्यता भएको व्यक्ति मुख्य न्यायाधिवक्ताको पदमा नियुक्तिका लागि योग्य हुनेछ ।

(४) देहायको कुनै अवस्थामा मुख्य न्यायाधिवक्ताको पद रिक्त हुनेछः–

(क) निजले मुख्यमन्त्री मार्फत प्रदेश प्रमुख समक्ष लिखित राजीनामा दिएमा,
(ख) निजलाई मुख्यमन्त्रीको सिफारिसमा प्रदेश प्रमुखले पदमुक्त गरेमा,
(ग) निजको मृत्यु भएमा ।

(५) मुख्य न्यायाधिवक्ता प्रदेश सरकारको मुख्य कानूनी सल्लाहकार हुनेछ र संवैधानिक एवं कानूनी विषयमा प्रदेश सरकार वा प्रदेश सरकारले तोकिदिएको अन्य अधिकारीलाई राय सल्लाह दिनु मुख्य न्यायाधिवक्ताको कर्तव्य हुनेछ ।

(६) मुख्य न्यायाधिवक्ताको कार्यालय अन्तर्गतका कर्मचारीहरूको व्यवस्थापन महान्यायाधिवक्ताको कार्यालयले गर्नेछ ।

(७) मुख्य न्यायाधिवक्ताको पारिश्रमिक तथा अन्य सुविधा उच्च अदालतको न्यायाधीश सरह हुनेछ । मुख्य न्यायाधिवक्ताको काम, कर्तव्य र अधिकार तथा सेवाका अन्य शर्त प्रदेश कानून बमोजिम हुनेछ ।

\textbf{१६१. सेवाका शर्त र सुविधा सम्बन्धी व्यवस्थाः}

सरकारी वकील तथा महान्यायाधिवक्ताको मातहतमा रहने अन्य कर्मचारीहरूको पारिश्रमिक, सुविधा तथा सेवाका शर्त सम्बन्धी व्यवस्था संघीय ऐन बमोजिम हुनेछ ।