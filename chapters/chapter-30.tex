\section{भाग–३० संकटकालीन अधिकार}

(४) उपधारा (३) बमोजिम अनुमोदनका लागि पेश भएको घोषणा वा आदेश संघीय संसदका दुवै सदनमा तत्काल कायम रहेका सम्पूर्ण सदस्य संख्याको कम्तीमा दुई तिहाइ बहुमतले अनुमोदन गरेमा त्यस्तो घोषणा वा आदेश भएको मितिले तीन महीनासम्म कायम रहनेछ ।

(५) उपधारा (३) बमोजिम अनुमोदनका लागि पेश भएको घोषणा वा आदेश उपधारा (४) बमोजिम अनुमोदन नभएमा स्वतः निष्क्रिय हुनेछ ।

(६) यस धारामा अन्यत्र जुनसुकै कुरा लेखिएको भए तापनि उपधारा (१) बमोजिमको अवस्था अझै विद्यमान छ भनी उपधारा

(४) बमोजिमको अवधि समाप्त नहुँदै अर्काे एक पटक तीन महीनामा नबढाई संकटकालीन घोषणा वा आदेशको अवधि थप गर्ने प्रस्ताव संघीय संसदमा पेश गर्न सकिनेछ ।

(७) उपधारा (६) बमोजिमको प्रस्ताव संघीय संसदका दुवै सदनमा तत्काल कायम रहेका सम्पूर्ण सदस्य संख्याको कम्तीमा दुई तिहाइ बहुमतले पारित भएमा त्यस्तो प्रस्तावमा उल्लिखित अवधि सम्मका लागि संकटकालीन अवस्थाको घोषणा वा आदेश कायम रहनेछ ।

(८) प्रतिनिधि सभा विघटन भएको अवस्थामा उपधारा (३), (४), (६) र (७) बमोजिम संघीय संसदले प्रयोग गर्ने अधिकार राष्ट्रिय सभाले प्रयोग गर्नेछ ।

(९) उपधारा (१) बमोजिम संकटकालीन अवस्थाको घोषणा वा आदेश भएपछि त्यस्तो अवस्थाको निवारण गर्न राष्ट्रपतिले आवश्यक आदेश जारी गर्न सक्नेछ । त्यसरी जारी भएको आदेश संकटकालीन अवस्था बहाल रहेसम्म कानून सरह लागू हुनेछ ।

(१०) उपधारा (१) बमोजिम संकटकालीन अवस्थाको घोषणा वा आदेश जारी गर्दा त्यस्तो घोषणा वा आदेश बहाल रहेसम्मका लागि भाग–३ मा व्यवस्था भएका मौलिक हक निलम्बन गर्न सकिनेछ ।

तर धारा १६, धारा १७ को उपधारा (२) को खण्ड (ग) र (घ), धारा १८, धारा १९ को उपधारा (२), धारा २०, २१, २२, २४, धारा २६ को उपधारा (१), २९, ३०, ३१, ३२, ३५, ३६ को उपधारा (१) र (२), ३८, ३९, ४० को उपधारा (२) र (३), ४१, ४२, ४३, ४५ र धारा ४६ बमोजिमको संवैधानिक उपचारको हक र बन्दी प्रत्यक्षीकरणको उपचार प्राप्त गर्ने हक निलम्बन गरिने छैन ।

(११) उपधारा (१०) बमोजिम यस संविधानको कुनै धारा निलम्बन गरिएकोमा सो धाराद्वारा प्रदत्त मौलिक हकको प्रचलनका लागि कुनै अदालतमा निवेदन दिन वा त्यस सम्बन्धमा कुनै अदालतमा प्रश्न उठाउन सकिने छैन ।

(१२) यस धारा बमोजिमको घोषणा वा आदेश बहाल रहेको अवस्थामा कुनै पदाधिकारीले बदनियतसाथ कुनै काम गरेबाट कसैलाई कुनै प्रकारको क्षति भएको रहेछ भने पीडितले त्यस्तो घोषणा वा आदेश समाप्त भएको मितिले तीन महीनाभित्र आफूलाई परेको क्षति बापत क्षतिपूर्तिको दाबी गर्न सक्नेछ । त्यस्तो दाबी परेमा अदालतले संघीय कानून बमोजिम क्षतिपूर्ति भराइदिन र पीडकलाई सजाय गर्न सक्नेछ ।

(१३) यस धारा बमोजिमको घोषणा वा आदेश राष्ट्रपतिले जुनसुकै बखत फिर्ता लिन सक्नेछ ।