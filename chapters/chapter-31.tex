\section{भाग–३१ संविधान संशोधन}

(४) उपधारा (२) बमोजिम पेश भएको विधेयक कुनै प्रदेशको सीमाना परिवर्तन वा अनुसूची–६ मा उल्लिखित विषयसँग सम्बन्धित भएमा त्यस्तो विधेयक संघीय संसदमा प्रस्तुत भएको तीस दिनभित्र सम्बन्धित सदनको सभामुख वा अध्यक्षले सहमतिका लागि प्रदेश सभामा पठाउनु पर्नेछ ।

(५) उपधारा (४) बमोजिम पठाइएको विधेयक तीन महीनाभित्र सम्बन्धित प्रदेश सभाका तत्काल कायम रहेका सम्पूर्र्ण सदस्यहरूको बहुमतबाट स्वीकृत वा अस्वीकृत गरी त्यसको जानकारी संघीय संसदमा पठाउनु पर्नेछ ।
तर कुनै प्रदेश सभा कायम नरहेको अवस्थामा त्यस्तो प्रदेश सभा गठन भई त्यसको पहिलो बैठक बसेको मितिले तीन महीनाभित्र स्वीकृत वा अस्वीकृत गरी पठाउनु पर्नेछ ।

(६) उपधारा (५) बमोजिमको अवधिभित्र त्यस्तो विधेयक स्वीकृत वा अस्वीकृत भएको जानकारी नदिएमा संघीय संसदको विधेयक उत्पत्ति भएको सदनले त्यस्तो विधेयक उपरको कारबाही अगाडि बढाउन बाधा पर्ने छैन ।

(७) उपधारा (५) बमोजिमको अवधिभित्र बहुसंख्यक प्रदेश सभाले त्यस्तो विधेयक अस्वीकृत गरेको सूचना संघीय संसदको सम्बन्धित सदनलाई दिएमा त्यस्तो विधेयक निष्क्रिय हुनेछ ।

(८) प्रदेश सभाको सहमति आवश्यक नपर्ने वा उपधारा (५) बमोजिम बहुसंख्यक प्रदेश सभाबाट स्वीकृत भई आएको विधेयक संघीय संसदका दुबै सदनमा तत्काल कायम रहेका सम्पूर्ण सदस्य संख्याको कम्तीमा दुई तिहाइ बहुमतबाट पारित गर्नु पर्नेछ ।
(९) उपधारा (८) बमोजिम पारित भएको विधेयक प्रमाणीकरणका लागि राष्ट्रपति समक्ष पेश गरिनेछ ।

(१०) राष्ट्रपतिले उपधारा (९) बमोजिम पेश भएको विधेयक प्राप्त भएको पन्ध्रदिन भित्र प्रमाणीकरण गर्नेछ र प्रमाणीकरण भएको मिति देखि संविधान संशोधन हुनेछ ।