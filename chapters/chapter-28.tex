\section{भाग–२८ राष्ट्रिय सुरक्षा सम्बन्धी व्यवस्था}

(४) राष्ट्रिय सुरक्षा परिषद सम्बन्धी अन्य व्यवस्था संघीय कानून बमोजिम हुनेछ ।

\textbf{२६७. नेपाली सेना सम्बन्धी व्यवस्था :} (१) नेपालको स्वतन्त्रता, सार्वभौमसत्ता, भौगोलिक अखण्डता, स्वाधीनता र राष्ट्रिय एकताको रक्षाका लागि यस संविधानप्रति प्रतिबद्ध समावेशी नेपाली सेनाको एक संगठन रहनेछ ।

(२) राष्ट्रपति नेपाली सेनाको परमाधिपति हुनेछ ।

(३) नेपाली सेनामा महिला, दलित, आदिवासी, आदिवासी जनजाति, खस आर्य, मधेशी, थारू, मुस्लिम, पिछडा वर्ग तथा पिछडिएको क्षेत्रका नागरिकको प्रवेश समानता र समावेशी सिद्धान्तको आधारमा संघीय कानून बमोजिम सुनिश्चित गरिनेछ।

(४) नेपाल सरकारले नेपाली सेनालाई संघीय कानून बमोजिम विकास निर्माण र विपद व्यवस्थापन लगायतका अन्य कार्यमा समेत परिचालन गर्न सक्नेछ ।

(५) राष्ट्रपतिले मन्त्रिपरिषदको सिफारिसमा प्रधान सेनापतिको नियुक्ति र पदमुक्ति गर्नेछ ।

(६) नेपालको सार्वभौमसत्ता, भौगोलिक अखण्डता वा कुनै भागको सुरक्षामा युद्ध, बाह्य आक्रमण, सशस्त्र विद्रोह वा चरम आर्थिक विश्रृंखलताको कारणले गम्भीर संकट उत्पन्न भएमा राष्ट्रिय सुरक्षा परिषदको सिफारिशमा नेपाल सरकार, मन्त्रिपरिषदको निर्णय बमोजिम राष्ट्रपतिबाट नेपाली सेना परिचालनको घोषणा हुनेछ । नेपाली सेना परिचालनको घोषणा भएको मितिले एक महीनाभित्र प्रतिनिधि सभाबाट अनुमोदन हुनु पर्नेछ ।

(७) नेपाली सेना सम्बन्धी अन्य व्यवस्था कानून बमोजिम हुनेछ ।

\textbf{२६८. नेपाल प्रहरी, सशस्त्र प्रहरी बल, नेपाल र राष्ट्रिय अनुसन्धान संगठन सम्बन्धी व्यवस्था : }(१) संघमा नेपाल प्रहरी, सशस्त्र प्रहरी बल, नेपाल र राष्ट्रिय अनुसन्धान विभाग रहनेछन् ।

(२) प्रत्येक प्रदेशमा प्रदेश प्रहरी संगठन रहनेछ ।

(३) नेपाल प्रहरी र प्रदेश प्रहरीले सम्पादन गर्ने कार्यको सञ्चालन, सुपरिवेक्षण र समन्वय सम्बन्धी व्यवस्था संघीय कानून बमोजिम हुनेछ ।

(४) नेपाल प्रहरी, सशस्त्र प्रहरी बल, नेपाल र राष्ट्रिय अनुसन्धान विभाग सम्बन्धी अन्य व्यवस्था संघीय कानून बमोजिम हुनेछ ।