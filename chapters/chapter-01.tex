\section{भाग–१ प्रारम्भिक}

\textbf{२. सार्वभौमसत्ता र राजकीयसत्ताः} नेपालको सार्वभौमसत्ता र राजकीयसत्ता नेपाली जनतामा निहित रहेको छ । यसको प्रयोग यस संविधानमा व्यवस्था भए बमोजिम हुनेछ ।

\textbf{३. राष्ट्रः} बहुजातीय, बहुभाषिक, बहुधार्मिक, बहुसांस्कृतिक विशेषतायुक्त, भौगोलिक विविधतामा रहेका समान आकांक्षा र नेपालको राष्ट्रिय स्वतन्त्रता, भौगोलिक अखण्डता, राष्ट्रिय हित तथा समृद्धिप्रति आस्थावान रही एकताको सूत्रमा आबद्ध सबै नेपाली जनता समष्टिमा राष्ट्र हो ।

\textbf{४. नेपाल राज्यः} (१) नेपाल स्वतन्त्र, अविभाज्य, सार्वभौमसत्तासम्पन्न, धर्मनिरपेक्ष, समावेशी, लोकतन्त्रात्मक, समाजवाद उन्मुख, संघीय लोकतान्त्रिक गणतन्त्रात्मक राज्य हो ।

स्पष्टीकरणः यस धाराको प्रयोजनको लागि “धर्मनिरपेक्ष” भन्नाले सनातनदेखि चलिआएको धर्म संस्कृतिको संरक्षण लगायत धार्मिक, सांस्कृतिक स्वतन्त्रता सम्झनु पर्छ ।

(२) नेपालको क्षेत्र देहाय बमोजिम हुनेछः– (क) यो संविधान प्रारम्भ हुँदाका बखतको क्षेत्र, र

(ख) यो संविधान प्रारम्भ भएपछि प्राप्त हुने क्षेत्र ।

\textbf{५. राष्ट्रिय हितः} (१) नेपालको स्वतन्त्रता, सार्वभौमसत्ता, भौगोलिक अखण्डता, राष्ट्रियता, स्वाधीनता, स्वाभिमान, नेपालीको हक हितको रक्षा, सीमानाको सुरक्षा, आर्थिक समुन्नति र समृद्धि नेपालको राष्ट्रिय हितका आधारभूत विषय हुनेछन् ।

(२) राष्ट्र हित प्रतिकूलको आचरण र कार्य संघीय कानून बमोजिम दण्डनीय हुनेछ ।

\textbf{६. राष्ट्रभाषाः} नेपालमा बोलिने सबै मातृभाषाहरू राष्ट्रभाषा हुन् ।

\textbf{७. सरकारी कामकाजको भाषाः} (१) देवनागरी लिपिमा लेखिने नेपाली भाषा नेपालको सरकारी कामकाजको भाषा हुनेछ ।

(२) नेपाली भाषाका अतिरिक्त प्रदेशले आफ्नो प्रदेशभित्र बहुसंख्यक जनताले बोल्ने एक वा एकभन्दा बढी अन्य राष्ट्रभाषालाई प्रदेश कानून
बमोजिम प्रदेशको सरकारी कामकाजको भाषा निर्धारण गर्न सक्नेछ ।

(३) भाषा सम्बन्धी अन्य कुरा भाषा आयोगको सिफारिसमा नेपाल सरकारले निर्णय गरे बमोजिम हुनेछ ।

\textbf{८. राष्ट्रिय झण्डाः} (१) सिम्रिक रंगको भुइँ र गाढा नीलो रंगको किनारा भएको दुई त्रिकोण अलिकति जोडिएको, माथिल्लो भागमा खुर्पे चन्द्रको बीचमा सोह्रमा आठ कोण देखिने सेतो आकार र तल्लो भागमा बाह्र कोणयुक्त सूर्यको सेतो आकार अंकित भएको झण्डा नेपालको राष्ट्रिय झण्डा हो ।

(२) नेपालको राष्ट्रिय झण्डा, राष्ट्रिय झण्डा बनाउने तरीका र तत्सम्बन्धी अन्य विवरण अनुसूची–१ मा उल्लेख भए बमोजिम हुनेछ ।

\textbf{९. राष्ट्रिय गान इत्यादिः} (१) नेपालको राष्ट्रिय गान अनुसूची–२ मा उल्लेख भए बमोजिम हुनेछ ।

(२) नेपालको निशान छाप अनुसूची–३ मा उल्लेख भए बमोजिम हुनेछ ।

(३) नेपालको राष्ट्रिय फूल लालीगुराँस, राष्ट्रिय रंग सिम्रिक, राष्ट्रिय जनावर गाई र राष्ट्रिय पक्षी डाँफे हुनेछ ।