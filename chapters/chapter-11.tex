\section{भाग–११ न्यायपालिका}

(क) सर्वोच्च अदालत,
(ख) उच्च अदालत, र
(ग) जिल्ला अदालत ।

(२) उपधारा (१) मा लेखिएदेखि बाहेक कानून बमोजिम मुद्दा हेर्न स्थानीय स्तरमा न्यायिक निकाय वा विवाद समाधानका वैकल्पिक उपाय
अवलम्बन गर्न आवश्यकता अनुसार अन्य निकाय गठन गर्न सकिनेछ ।

\textbf{१२८. सर्वोच्च अदालतः}

(१) नेपालमा एक सर्वोच्च अदालत हुनेछ ।
(२) सर्वोच्च अदालत अभिलेख अदालत हुनेछ । यस संविधानमा अन्यथा व्यवस्था भएकोमा बाहेक सबै अदालत र न्यायिक निकायहरू सर्वोच्च अदालत मातहत रहनेछन् । संविधान र कानूनको व्याख्या गर्ने अन्तिम अधिकार सर्वाेच्च अदालतलाई हुनेछ ।
(३) सर्वोच्च अदालतले आफ्नो र आफ्नो अधिकार क्षेत्रभित्र पर्ने अदालत, विशिष्टीकृत अदालत वा अन्य न्यायिक निकायहरूको न्याय प्रशासन वा व्यवस्थापन सम्बन्धी विषयमा निरीक्षण, सुपरिवेक्षण गरी आवश्यक निर्देशन दिन सक्नेछ ।
(४) मुद्दा मामिलाका रोहमा सर्वोच्च अदालतले गरेको संविधान र कानूनको व्याख्या वा प्रतिपादन गरेको कानूनी सिद्धान्त सबैले पालन गर्नु पर्नेछ । सर्वाेच्च अदालतले आफ्नो वा मातहतको अदालतको न्यायसम्पादनको कार्यमा कसैले अवरोध गरेमा वा आदेश वा फैसलाको अवज्ञा गरेमा कानून बमोजिम अवहेलनामा कारबाही चलाई सजाय गर्न सक्नेछ ।

\textbf{१२९. नेपालको प्रधान न्यायाधीश तथा सर्वोच्च अदालतका न्यायाधीशको नियुक्ति र योग्यताः}

(१) सर्वाेच्च अदालतमा नेपालको प्रधान न्यायाधीशका अतिरिक्त बढीमा बीस जना न्यायाधीश रहनेछन् ।
(२) संवैधानिक परिषदको सिफारिसमा प्रधान न्यायाधीशको र न्याय परिषदको सिफारिसमा सर्वाेच्च अदालतका अन्य न्यायाधीशको नियुक्ति राष्ट्रपतिबाट हुनेछ ।
(३) सर्वोच्च अदालतको न्यायाधीश पदमा कम्तीमा तीन वर्ष काम गरेको व्यक्ति प्रधान न्यायाधीशको पदमा नियुक्ति हुन योग्य हुनेछ ।
(४) प्रधान न्यायाधीशको पदावधि छ वर्षको हुनेछ ।
(५) कानूनमा स्नातक उपाधि प्राप्त गरी उच्च अदालतको मुख्य न्यायाधीश वा न्यायाधीशको पदमा कम्तीमा पाँच वर्ष काम गरेको वा कानूनमा स्नातक उपाधि प्राप्त गरी वरिष्ठ अधिवक्ता वा अधिवक्ताको हैसियतमा कम्तीमा पन्ध्र वर्ष निरन्तर वकालत गरेको वा कम्तीमा पन्ध्र वर्षसम्म न्याय वा कानूनको क्षेत्रमा निरन्तर काम गरी विशिष्ट कानूनविदको रूपमा ख्याति प्राप्त गरेको वा न्याय सेवाको राजपत्रांकित प्रथम श्रेणी वा सोभन्दा माथिल्लो पदमा कम्तीमा बाह्र वर्ष काम गरेको नेपाली नागरिक सर्वोच्च अदालतको न्यायाधीशको पदमा नियुक्तिका लागि योग्य मानिनेछ ।

स्पष्टीकरणः यो संविधान लागू हुनुभन्दा पहिले पुनरावेदन अदालतमा मुख्य न्यायाधीश वा न्यायाधीश भई काम गरेको अवधिलाई यो उपधाराको प्रयोजनका लागि उच्च अदालतको मुख्य न्यायाधीश वा न्यायाधीशको हैसियतमा काम गरेको अवधि मानिनेछ ।

(६) प्रधान न्यायाधीशको पद रिक्त भएमा वा कुनै कारणले प्रधान न्यायाधीश आफ्नो पदको काम गर्न असमर्थ भएमा वा बिदा बसेको वा नेपालबाहिर गएको कारणले प्रधान न्यायाधीश सर्वोच्च अदालतमा उपस्थित नहुने अवस्था भएमा सर्वोच्च अदालतको वरिष्ठतम न्यायाधीशले कायममुकायम प्रधान न्यायाधीश भई काम गर्नेेछ ।

\textbf{१३०. प्रधान न्यायाधीश तथा न्यायाधीशको सेवाका शर्त तथा सुविधाः}

(१) प्रधान न्यायाधीश तथा सर्वोच्च अदालतको न्यायाधीशले कम्तीमा पाँच वर्ष काम गरी राजीनामा दिएमा वा अनिवार्य अवकाश प्राप्त गरेमा वा निजको मृत्यु भएमा संघीय कानून बमोजिम निवृत्तिभरण पाउनेछ ।
(२) यस संविधानमा अन्यथा व्यवस्था गरिएकोमा बाहेक प्रधान न्यायाधीश तथा सर्वोच्च अदालतका न्यायाधीशको पारिश्रमिक र सेवाका अन्य शर्त संघीय कानून बमोजिम हुनेछ ।
(३) उपधारा (१) र (२) मा जुनसुकै कुरा लेखिएको भए तापनिटट महाभियोगद्वारा पदमुक्त भएको वा नैतिक पतन देखिने फौजदारी कसूरमा अदालतबाट सजाय पाएको प्रधान न्यायाधीश वा सर्वोच्च अदालतको न्यायाधीशले उपदान वा निवृत्तिभरण पाउने छैन ।
(४) प्रधान न्यायाधीश वा सर्वोच्च अदालतको न्यायाधीशलाई मर्का पर्ने गरी निजको पारिश्रमिक र सेवाका अन्य शर्त परिवर्तन गरिने छैन । तर चरम आर्थिक विश्रृंखलताकोे कारणले संकटकालीन अवस्थाको घोषणा भएको अवस्थामा यो व्यवस्था लागू हुने छैन ।

\textbf{१३१. प्रधान न्यायाधीश वा सर्वाेच्च अदालतको न्यायाधीशको पद रिक्त हुने:}

देहायको कुनै अवस्थामा प्रधान न्यायाधीश वा सर्वोच्च अदालतको न्यायाधीशको पद रिक्त हुनेछः–
(क) निजले राष्ट्रपति समक्ष लिखित राजीनामा दिएमा,
(ख) निजको उमेर पैंसठ्ठी वर्ष पूरा भएमा,
(ग) निजको विरुद्ध धारा १०१ बमोजिम महाभियोगको प्रस्ताव पारित भएमा,
(घ) शारीरिक वा मानसिक अस्वस्थताको कारण सेवामा रही कार्य सम्पादन गर्न असमर्थ रहेको भनी प्रधान न्यायाधीशको हकमा संवैधानिक परिषद र सर्वाेच्च अदालतको न्यायाधीशको हकमा न्याय परिषदको सिफारिसमा राष्ट्रपतिले पदमुक्त गरेमा,
(ङ) निजले नैतिक पतन देखिने फौजदारी कसूरमा अदालतबाट सजाय पाएमा,
(च) निजको मृत्यु भएमा ।

\textbf{१३२. प्रधान न्यायाधीश तथा सर्वाेच्च अदालतको न्यायाधीशलाई अन्य कुनै काममा लगाउन नहुनेः}

(१) प्रधान न्यायाधीश वा सर्वोच्च अदालतको न्यायाधीशलाई न्यायाधीशको पदमा बाहेक अन्य कुनै पदमा काममा लगाइने वा काजमा
खटाइने छैन । तर नेपाल सरकारले न्याय परिषदसँग परामर्श गरी सर्वाेच्च अदालतको न्यायाधीशलाई न्यायिक जाँचबुझको काममा वा केही खास अवधिका लागि कानून वा न्याय सम्बन्धी अनुसन्धान वा अन्वेषणको कुनै काममा खटाउन सक्नेछ ।
(२) प्रधान न्यायाधीश वा सर्वोच्च अदालतको न्यायाधीश भइसकेको व्यक्ति यस संविधानमा अन्यथा उल्लेख भएकोमा बाहेक कुनै पनि सरकारी पदमा नियुक्तिका लागि ग्राह्य हुने छैन ।

\textbf{१३३. सर्वोच्च अदालतको अधिकार क्षेत्रः}

(१) यस संविधानद्वारा प्रदत्त मौलिक हक उपर अनुचित बन्देज लगाइएकोे वा अन्य कुनै कारणले कुनै कानून यो संविधानसँग बाझिएको हुँदा त्यस्तो कानून वा त्यसको कुनै भाग वा प्रदेश सभाले बनाएको कुनै कानून संघीय संसदले बनाएको कुनै कानूनसँग बाझिएको वा नगर सभा वा गाउँ सभाले बनाएको कुनै कानून संघीय संसद वा प्रदेश सभाले बनाएको कुनै कानूनसँग बाझिएको हँुदा त्यस्तो कानून वात्यसको कुनै भाग बदर घोषित गरी पाऊँ भनी कुनै पनि नेपाली नागरिकले सर्वोच्च अदालतमा निवेदन दिन सक्नेछ र सो अनुसार कुनै कानून बाझिएको देखिएमा सो कानूनलाई प्रारम्भदेखि नै वा निर्णय भएको मितिदेखि अमान्य र बदर घोषित गर्ने असाधारण अधिकार सर्वोच्च अदालतलाई हुनेछ ।

(२) यस संविधानद्वारा प्रदत्त मौलिक हकको प्रचलनका लागि वा अर्को उपचारको व्यवस्था नभएको वा अर्को उपचारको व्यवस्था भए पनि
त्यस्तो उपचार अपर्याप्त वा प्रभावहीन देखिएको अन्य कुनै कानूनी हकको प्रचलनका लागि वा सार्वजनिक हक वा सरोकारको कुनै विवादमा समावेश भएको कुनै संवैधानिक वा कानूनी प्रश्नको निरूपणका लागि आवश्यक र उपयुक्त आदेश जारी गर्ने, उचित उपचार प्रदान गर्ने, त्यस्तो हकको प्रचलन गराउने वा विवाद टुंगो लगाउने असाधारण अधिकार सर्वोच्च अदालतलाई हुनेछ ।

(३) उपधारा (२) बमोजिमको असाधारण अधिकार क्षेत्र अन्तर्गत सर्वोच्च अदालतले बन्दी प्रत्यक्षीकरण, परमादेश, उत्प्रेषण, प्रतिषेध,
अधिकारपृच्छा लगायत अन्य उपयुक्त आदेश जारी गर्न सक्नेछ । तर अधिकार क्षेत्रको अभाव भएकोमा बाहेक संघीय संसद वा प्रदेश
सभाको आन्तरिक काम कारबाही र संघीय संसद वा प्रदेश सभाले चलाएको विशेषाधिकारको कारबाही र तत्सम्बन्धमा तोकेको सजायमा यस उपधारा अन्तर्गत सर्वोच्च अदालतले हस्तक्षेप गर्ने छैन ।

(४) यस संविधानको अधीनमा रही सर्वोच्च अदालतलाई संघीय कानूनमा व्यवस्था भए बमोजिम मुद्दाको शुरु कारबाही र किनारा गर्ने,
पुनरावेदन सुन्ने, साधक जाँच्ने, मुद्दा दोहो¥याउने, निवेदन सुन्ने वा आफ्नो फैसला वा अन्तिम आदेशको पुनरावलोकन गर्ने अधिकार हुनेछ । त्यसरी पुनरावलोकन गर्दा पहिला फैसला गर्ने न्यायाधीश बाहेक अन्य न्यायाधीशले गर्ने छन् ।

(५) उच्च अदालतले शुरू कारबाही र किनारा गरेको मुद्दाको पुनरावेदन सुन्ने र संविधान र कानूनको व्याख्या सम्बन्धी प्रश्न समावेशटड
भएको सार्वजनिक महत्वको विषय वा सर्वोच्च अदालतबाट निर्णय हुनु उपयुक्त छ भनी उच्च अदालतले आफ्नो राय सहित सिफारिस गरेको मुद्दाको निरूपण गर्ने अधिकार सर्वोच्च अदालतलाई हुनेछ ।

(६) सर्वाेच्च अदालतको अन्य अधिकार र कार्यविधि संघीय कानून बमोजिम हुनेछ ।

\textbf{१३४. मुद्दा सार्न सक्नेः}

(१) सारभूत रूपमा समान प्रश्न समावेश भएको मुद्दा सर्वोच्च अदालत र उच्च अदालतहरूमा विचाराधीन रहेको अवस्थामा त्यस्तो प्रश्न सार्वजनिक महत्वको हो भन्ने सर्वोच्च अदालतलाई लागेमा वा महान्यायाधिवक्ता वा मुद्दाका पक्षको निवेदनबाट देखिएमा त्यस्ता मुद्दा
झिकाई साथै राखी फैसला गर्ने अधिकार सर्वोच्च अदालतलाई हुनेछ ।

(२) कुनै उच्च अदालतमा दायर भएको मुद्दामा सुनुवाई हुँदा न्यायिक निष्पक्षतामा प्रश्न उठ्ने विशेष परिस्थिति देखिएमा कारण र आधार खुलाई कानून बमोजिम एक उच्च अदालतबाट अर्को उच्च अदालतमा त्यस्तो मुद्दा सारी सुनुवाई गर्न सर्वोच्च अदालतले आदेश दिन सक्नेछ ।
\textbf{१३५ बहस पैरवी गर्न नपाउनेः}

प्रधान न्यायाधीश र सर्वोच्च अदालतको न्यायाधीशले सेवानिवृत्त भएपछि कुनै पनि अड्डा अदालतमा बहस पैरवी, मेलमिलाप वा मध्यस्थता सम्बन्धी कार्य गर्न पाउने छैन ।

\textbf{१३६. प्रधान न्यायाधीशको जिम्मेवारीः}

सर्वोच्च अदालत र मातहतका अदालत, विशिष्टीकृत अदालत वा अन्य न्यायिक निकायहरूको न्याय प्रशासनलाई प्रभावकारी बनाउने अन्तिम जिम्मेवारी प्रधान न्यायाधीशको हुनेछ ।

\textbf{१३७. संवैधानिक इजलासको गठनः}

(१) सर्वाेच्च अदालतमा एक संवैधानिक इजलास रहनेछ । त्यस्तो इजलासमा प्रधान न्यायाधीश र न्याय परिषदको सिफारिसमा प्रधान न्यायाधीशले तोकेका अन्य चार जना न्यायाधीश रहने छन् ।
(२) उपधारा (१) बमोजिमको इजलासले धारा १३३ को उपधारा (१) बमोजिम परेका निवेदनको अतिरिक्त देहायका मुद्दाको शुरू कारबाही र
किनारा गर्नेछः–

(क) संघ र प्रदेश, प्रदेश र प्रदेश, प्रदेश र स्थानीय तह तथा स्थानीय तहहरू बीचको अधिकार क्षेत्रको बारेमा भएको विवाद सम्बन्धी,
(ख) संघीय संसद वा प्रदेश सभा सदस्यको निर्वाचन सम्बन्धी विवाद र संघीय संसदका सदस्य वा प्रदेश सभाका सदस्यको अयोग्यता सम्बन्धी ।

(३) धारा १३३ मा जुनसुकै कुरा लेखिएको भए तापनि सर्वाेच्च अदालतमा विचाराधीन कुनै मुद्दामा गम्भीर संवैधानिक व्याख्याको प्रश्न समावेश भएको देखिएमा त्यस्तो मुद्दा उपधारा (१) बमोजिमको इजलासबाट हेर्ने गरी प्रधान न्यायाधीशले तोक्न सक्नेछ ।

(४) संवैधानिक इजलासको सञ्चालन सम्बन्धी अन्य व्यवस्था सर्वाेच्च अदालतले निर्धारण गरे बमोजिम हुनेछ ।

\textbf{१३८. वार्षिक प्रतिवेदनः}

(१) सर्वोच्च अदालत, न्याय परिषद र न्याय सेवा आयोगले प्रत्येक वर्ष आफ्नो वार्षिक प्रतिवेदन राष्ट्रपति समक्ष पेश गर्नेछ र राष्ट्रपतिले त्यस्तो प्रतिवेदन प्रधानमन्त्री मार्फत संघीय संसद समक्ष पेश गर्नेछ ।

(२) उपधारा (१) बमोजिम प्रस्तुत भएको वार्षिक प्रतिवेदन माथि छलफल हुँदा संघीय संसदले कुनै सुझाव दिन आवश्यक देखेमा नेपाल
सरकार, कानून तथा न्याय मन्त्रालय मार्फत सम्बन्धित निकायलाई दिन सक्नेछ ।

(३) उपधारा (१) बमोजिमको वार्षिक प्रतिवेदन सम्बन्धी अन्य व्यवस्था संघीय कानून बमोजिम हुनेछ ।

\textbf{१३९. उच्च अदालतः}

(१) प्रत्येक प्रदेशमा एक उच्च अदालत रहनेछ ।

(२) उच्च अदालतले आफ्नो र आफ्ना मातहतका अदालत वा न्यायिक निकायहरूबाट हुने न्याय सम्पादनको कार्यमा कसैले अवरोध गरेमा वा आदेश वा फैसलाको अवज्ञा गरेमा संघीय कानून बमोजिम अवहेलनामा कारबाही चलाई सजाय गर्न सक्नेछ ।
(३) प्रत्येक उच्च अदालतमा मुख्य न्यायाधीशका अतिरिक्त संघीय कानूनमा व्यवस्था भए बमोजिमको संख्यामा न्यायाधीशहरू रहनेछन् ।

\textbf{१४०. उच्च अदालतका मुख्य न्यायाधीश तथा न्यायाधीशको नियुक्ति र योग्यताः}

(१) प्रधान न्यायाधीशले न्याय परिषदको सिफारिसमा उच्च अदालतका मुख्य न्यायाधीश तथा न्यायाधीशको नियुक्ति गर्नेछ ।

(२) कानूनमा स्नातक उपाधि प्राप्त गरी जिल्ला न्यायाधीशको पदमा कम्तीमा पाँच वर्ष काम गरेको वा कानूनमा स्नातक उपाधि प्राप्त गरी वरिष्ठ अधिवक्ता वा अधिवक्ताको रूपमा कम्तीमा दश वर्ष निरन्तर वकालत गरेको वा कम्तीमा दश वर्ष कानूनको अध्यापन, अन्वेषण वा कानून वा न्याय सम्बन्धी अन्य कुनै क्षेत्रमा निरन्तर काम गरेको वा न्याय सेवाको कम्तीमा राजपत्रांकित प्रथम श्रेणीको पदमा कम्तीमा पाँच वर्ष काम गरेको नेपालीठण् नागरिक उच्च अदालतको मुख्य न्यायाधीश तथा न्यायाधीशको पदमा नियुक्तिका लागि योग्य मानिनेछ ।

(३) उच्च अदालतका मुख्य न्यायाधीश र न्यायाधीशको नियुक्ति गर्दा उपधारा (२) बमोजिम योग्यता पुुगेका व्यक्तिहरूमध्येबाट जिल्ला
न्यायाधीशको हकमा निजले वार्षिक रूपमा फैसला गरेको मुद्दाको अनुपात र माथिल्लो अदालतमा अन्तिम निर्णय हुँदा मुद्दा सदर, बदर वा उल्टी भएकोे मूल्यांकनको आधारमा, न्याय सेवाको कम्तीमा राजपत्रांकित प्रथम श्रेणीको पदमा कम्तीमा पाँच वर्ष काम गरेको व्यक्तिको हकमा ज्येष्ठता, योग्यता र कार्यसम्पादनकोे स्तरको मूल्यांकनको आधारमा र अन्यको हकमा वरिष्ठता, व्यावसायिक निरन्तरता, इमानदारी, पेशागत आचरण र न्याय र कानूनको क्षेत्रमा गरेको योगदानको मूल्यांकन गरी नियुक्ति गरिनेछ ।

(४) मुख्य न्यायाधीशको पद रिक्त भएमा वा अरु कुनै कारणले मुख्य न्यायाधीश आफ्नो पदको काम गर्न असमर्थ भएको वा बिदा बसेको वा प्रदेश बाहिर गएको कारणले मुख्य न्यायाधीश उच्च अदालतमा उपस्थित हुन सक्ने अवस्था नरहेमा उच्च अदालतको वरिष्ठतम न्यायाधीशले कायममुकायम मुख्य न्यायाधीश भई काम गर्नेेछ ।

\textbf{१४१. मुख्य न्यायाधीश तथा न्यायाधीशको सेवाका शर्त तथा सुविधाः}

(१) यस संविधानमा अन्यथा व्यवस्था गरिएकोमा बाहेक उच्च अदालतको मुख्य न्यायाधीश तथा न्यायाधीशको पारिश्रमिक र सेवाको अन्य शर्त संघीय कानून बमोजिम हुनेछ ।

(२) उपधारा (१) मा जुनसुकै कुरा लेखिएको भए तापनि न्याय परिषदबाट कारबाही भई पदमुक्त भएको वा नैतिक पतन देखिने फौजदारी
कसूरमा अदालतबाट सजाय पाई पदमुक्त भएको उच्च अदालतको मुख्य न्यायाधीश तथा न्यायाधीशले उपदान वा निवृित्तभरण पाउने छैन ।
तर शारीरिक वा मानसिक अस्वस्थताको कारण सेवामा रही कार्य सम्पादन गर्न असमर्थ रहेको भनी न्याय परिषदले पदमुक्त गरेको अवस्थामा यो व्यवस्था लागू हुने छैन ।

(३) उच्च अदालतको मुख्य न्यायाधीश वा न्यायाधीशलाई मर्का पर्ने गरी निजको पारिश्रमिक र सेवाका अन्य शर्त परिवर्तन गरिने छैन ।
तर चरम आर्थिक विश्रृंखलताका कारण संकटकाल घोषणा भएको अवस्थामा यो व्यवस्था लागू हुने छैन ।

\textbf{१४२. मुख्य न्यायाधीश वा न्यायाधीशको पद रिक्त हुनेः} (१) देहायको कुनै अवस्थामा उच्च अदालतको मुख्य न्यायाधीश वा न्यायाधीशको पद रिक्त हुनेछः–

(क) निजले प्रधान न्यायाधीश समक्ष लिखित राजीनामा दिएमा,
(ख) निजको उमेर त्रिसठ्ठी वर्ष पूरा भएमा,
(ग) निजको कार्यक्षमताको अभाव, खराब आचरण, इमानदारी पूर्वक आफ्नो कर्तव्यको पालन नगरेको, बदनियतपूर्वक कामकारबाही गरेको वा निजले पालन गर्नु पर्ने आचारसंहिताको गम्भीर उल्लंघन गरेको आधारमा न्याय परिषदको सिफारिसमा प्रधान न्यायाधीशले पदमुक्त गरेमा,
(घ) निज शारीरिक वा मानसिक अस्वस्थताको कारण सेवामा रही कार्य सम्पादन गर्न असमर्थ रहेको भनी न्यायपरिषदको सिफारिसमा प्रधान न्यायाधीशले पदमुक्त गरेमा,
(ङ) निजले नैतिक पतन देखिने फौजदारी कसूरमा अदालतबाट सजाय पाएमा,
(च) निजको मृत्यु भएमा ।

(२) उपधारा (१) को खण्ड (ग) बमोजिम पदमुक्त गर्नु अघि आरोप लागेको न्यायाधीशलाई आफनो सफाइ पेश गर्न मनासिब मौका दिनु पर्नेछ । त्यसरी कारबाही प्रारम्भ भएको न्यायाधीशले कारबाहीको टुंगो नलागेसम्म आफ्नो पदको कार्य सम्पादन गर्न पाउने छैन । (३) पदमुक्त भएको मुख्य न्यायाधीश वा न्यायाधीशले पदमा रहँदा गरेको कसूरमा संघीय कानून बमोजिम कारबाही गर्न बाधा पर्ने छैन ।

\textbf{१४३. मुख्य न्यायाधीश तथा न्यायाधीशलाई अन्य कुनै काममा लगाउन नहुने र सरुवा सम्बन्धी व्यवस्थाः}

(१) उच्च अदालतको मुख्य न्यायाधीश वा न्यायाधीशलाई न्यायाधीशको पदमा बाहेक अन्य कुनै पदमा काममा लगाइने वा काजमा खटाइने छैन । तर नेपाल सरकारले न्याय परिषदसँग परामर्श गरी उच्च अदालतको न्यायाधीशलाई न्यायिक जाँचबुझको काममा वा केही खास अवधिका लागि कानून वा न्याय सम्बन्धी अनुसन्धान, अन्वेषण वा राष्ट्रिय सरोकारको कुनै काममा खटाउन सक्नेछ ।
(२) न्याय परिषदको सिफारिसमा प्रधान न्यायाधीशले उच्च अदालतकाठद्द न्यायाधीशलाई एक उच्च अदालतको न्यायाधीशबाट अर्को उच्च अदालतको न्यायाधीशमा सरुवा गर्न सक्नेछ ।

\textbf{१४४. उच्च अदालतको अधिकार क्षेत्रः}

(१) यस संविधानद्वारा प्रदत्त मौलिक हकको प्रचलनका लागि वा अर्को उपचारको व्यवस्था नभएको वा अर्को उपचारको व्यवस्था भए पनि सो उपचार अपर्याप्त वा प्रभावहीन देखिएको अन्य कुनै कानूनी हकको प्रचलनका लागि वा सार्वजनिक हक वा सरोकारको कुनै विवादमा समावेश भएको कुनै कानूनी प्रश्नको निरूपणका लागि आवश्यक र उपयुक्त आदेश जारी गर्ने अधिकार उच्च अदालतलाई हुनेछ ।
(२) उपधारा (१) को प्रयोजनका लागि उच्च अदालतले बन्दीप्रत्यक्षीकरण, परमादेश, उत्प्रेषण, प्रतिषेध, अधिकारपृच्छा लगायत जुनसुकै उपयुक्त आदेश जारी गर्न सक्नेछ ।

तर अधिकार क्षेत्रको अभाव भएकोमा बाहेक संघीय संसद वा प्रदेश सभाको आन्तरिक काम कारबाही र संघीय संसद वा प्रदेश सभाले चलाएको विशेषाधिकारको कारबाही र तत्सम्बन्धमा तोकेको सजायमा यस उपधारा बमोजिम उच्च अदालतले हस्तक्षेप गर्ने छैन ।
(३) उच्च अदालतलाई संघीय कानून बमोजिम शुरू मुद्दा हेर्ने, पुनरावेदन सुन्ने र साधक जाँच्ने अधिकार हुनेछ ।
(४) उच्च अदालतको अन्य अधिकार तथा कार्यविधि संघीय कानून बमोजिम हुनेछ ।

\textbf{१४५. मुद्दा सार्न सक्नेः} (१) उच्च अदालतले आफ्नो क्षेत्राधिकारभित्रका मातहत अदालतमा विचाराधीन रहेका मुद्दामा प्रदेश कानून सम्बन्धी प्रश्न समावेश छ र उक्त मुद्दाको निर्णय गर्न सो प्रश्नको निराकरण हुनु अनिवार्य छ भन्ने लागेमा त्यस्ता मुद्दाहरू मातहत अदालतबाट झिकाई मुद्दाको पूरै निर्णय गर्न वा त्यस्तो प्रश्नमा मात्र निर्णय गरेर मुद्दा शुरू अदालतमा फिर्ता पठाउन सक्नेछ ।
(२) कुनै जिल्ला अदालतमा दायर भएको मुद्दामा सुनुवाई हुँदा न्यायिक निष्पक्षतामा प्रश्न उठ्ने विशेष परिस्थिति देखिएमा कारण र आधार खुलाई संघीय कानून बमोजिम आफ्नो मातहतका एक जिल्लाबाट अर्को जिल्ला अदालतमा त्यस्तो मुद्दा सारी सुनुवाई गर्न उच्च अदालतले आदेश दिन सक्नेछ ।

\textbf{१४६. बहस पैरवी गर्न पाउनेः} उच्च अदालतको न्यायाधीश भई सेवानिवृत्त भएको व्यक्तिले आपूmले न्यायाधीशको हैसियतमा सेवा गरेका उच्च अदालत र मातहतका अदालत बाहेक अन्य उच्च अदालत र सर्वोच्च अदालतमाठघ उपस्थित भई बहस पैरवी गर्न पाउनेछ ।

\textbf{१४७. मुख्य न्यायाधीशको जिम्मेवारीः} उच्च अदालत र त्यसको मातहतको अदालत वा अन्य न्यायिक निकायको न्याय प्रशासनलाई प्रभावकारी बनाउने जिम्मेवारी मुख्य न्यायाधीशको हुनेछ । सो प्रयोजनका लागि मुख्य न्यायाधीशले यस संविधान तथा संघीय कानूनको अधीनमा रही मातहतका अदालत तथा न्यायिक निकायलाई आवश्यक निर्देशन दिन सक्नेछ ।

\textbf{१४८. जिल्ला अदालतः} (१) प्रत्येक जिल्लामा एक जिल्ला अदालत हुनेछ ।
(२) प्रदेश कानून बमोजिम स्थापित स्थानीय स्तरका न्यायिक निकाय जिल्ला अदालतको मातहतमा रहनेछन् । जिल्ला अदालतले आफ्नो मातहतका न्यायिक निकायहरूको निरीक्षण एवं सुपरिवेक्षण गर्न र आवश्यक निर्देशन दिन सक्नेछ ।

\textbf{१४९. जिल्ला अदालतको न्यायाधीशको नियुक्ति, योग्यता तथा पारिश्रमिक र सेवाकाअन्य शर्तः} (१) जिल्ला अदालतका न्यायाधीशको नियुक्ति न्याय परिषदको सिफारिसमा प्रधान न्यायाधीशबाट हुनेछ ।
(२) जिल्ला अदालतमा रिक्त न्यायाधीशको पद देहाय बमोजिम पूर्ति गरिनेछः–

(क) रिक्त पदमध्ये बीस प्रतिशत पदमा कानूनमा स्नातक उपाधि प्राप्त गरी न्याय सेवाको राजपत्रांकित द्वितीय श्रेणीको पदमा कम्तीमा तीन वर्ष काम गरेका अधिकृतहरूमध्येबाट ज्येष्ठता, योग्यता र कार्यक्षमताको मूल्यांकनको आधारमा,

(ख) रिक्त पदमध्ये चालीस प्रतिशत पदमा कानूनमा स्नातक उपाधि प्राप्त गरी न्याय सेवाको राजपत्रांकित द्वितीय श्रेणीको पदमा कम्तीमा तीन वर्ष काम गरेका अधिकृतहरूमध्येबाट खुला प्रतियोगितात्मक परीक्षाको आधारमा,

(ग) रिक्त पदमध्ये बाँकी चालीस प्रतिशत पदमा कानूनमा स्नातक उपाधि प्राप्त गरी अधिवक्ताको रूपमा निरन्तर कम्तीमा आठ वर्ष वकालत गरेका, कानूनमा स्नातक उपाधि प्राप्त गरी न्याय सेवाको राजपत्रांकित पदमा कम्तीमा आठ वर्ष काम गरेका वा कानूनको अध्यापन, अन्वेषण वा कानून वा न्याय सम्बन्धी अन्य कुनै क्षेत्रमा निरन्तर कम्तीमा आठ वर्ष काम गरेका नेपालीठद्ध नागरिकमध्येबाट खुल्ला प्रतियोगितात्मक परीक्षाको आधारमा ।

(३) उपधारा (२) को खण्ड (ख) र (ग) बमोजिमको योग्यता भएका व्यक्तिहरूमध्येबाट संघीय कानून बमोजिम न्याय सेवा आयोगले लिखित र मौखिक प्रतियोगितात्मक परीक्षा लिई योग्यताक्रम बमोजिम जिल्ला न्यायाधीशमा नियुक्तिका लागि न्याय परिषदलाई सिफारिस गर्नेछ ।
(४) जिल्ला अदालतको न्यायाधीशको पारिश्रमिक र सेवाका अन्य शर्त संघीय कानून बमोजिम हुनेछ ।
(५) जिल्ला अदालतको न्यायाधीशलाई मर्का पर्ने गरी निजको पारिश्रमिक र सेवाका अन्य शर्त परिवर्तन गरिने छ्रैन ।

तर चरम आर्थिक विश्रृंखलताका कारण संकटकाल घोषणा भएको अवस्थामा यो व्यवस्था लागू हुने छैन ।

(६) देहायको कुनै अवस्थामा जिल्ला अदालतको न्यायाधीशको पद रिक्त हुनेछः–

(क) निजले प्रधान न्यायाधीश समक्ष लिखित राजीनामा दिएमा,
(ख) निजको उमेर त्रिसठ्ठी वर्ष पूरा भएमा,
(ग) निजको कार्यक्षमताको अभाव, खराब आचरण, इमानदारीपूर्वक आफ्नो कर्तव्यको पालन नगरेको, बदनियतपूर्वक कामकारबाही गरेको वा निजले पालन गर्नु पर्ने आचारसंहिताको गम्भीर उल्लंघन गरेको आधारमा न्याय परिषदको सिफारिसमा प्रधान न्यायाधीशले पदमुक्त
गरेमा,
(घ) निज शारीरिक वा मानसिक अस्वस्थताको कारण सेवामा रही कार्य सम्पादन गर्न असमर्थ रहेको भनी न्याय परिषदको सिफारिसमा प्रधान न्यायाधीशले पदमुक्त गरेमा,
(ङ) निजले नैतिक पतन देखिने फौजदारी कसूरमा अदालतबाट सजाय पाएमा,
(च) निजको मृत्यु भएमा ।

(७) उपधारा (६) को खण्ड (ग) बमोजिम पदमुक्त गर्नु अघि आरोप लागेको जिल्ला न्यायाधीशलाई आफ्नो सफाइ पेश गर्न मनासिब मौका दिनुठछ पर्नेछ । त्यसरी कारबाही प्रारम्भ भएको जिल्ला न्यायाधीशले कारबाहीको टुंगो नलागेसम्म आफ्नो पदको कार्यसम्पादन गर्न पाउने छैन ।

(८) पदमुक्त भएको जिल्ला न्यायाधीशले पदमा रहँदा गरेको कसूरमा संघीय कानून बमोजिम कारबाही गर्न बाधा पर्ने छैन ।

\textbf{१५०. जिल्ला न्यायाधीशलाई अन्य कुनै काममा लगाउन नहुने र सरुवा सम्बन्धी व्यवस्थाः} (१) जिल्ला न्यायाधीशलाई न्यायाधीशको पदमा बाहेक अन्य कुनै पदमा काममा लगाइने वा काजमा खटाइने छैन । तर नेपाल सरकारले न्याय परिषदसँग परामर्श गरी जिल्ला न्यायाधीशलाई न्यायिक जाँचबुझको काममा वा केही खास अवधिका लागि कानून वा न्याय सम्बन्धी अनुसन्धान वा अन्वेषण तथा निर्वाचन सम्बन्धी काममा खटाउन सक्नेछ ।

(२) न्याय परिषदको सिफारिसमा प्रधान न्यायाधीशले जिल्ला न्यायाधीशलाई एक जिल्ला अदालतबाट अर्को जिल्ला अदालतमा सरुवा गर्न
सक्नेछ ।

\textbf{१५१. जिल्ला अदालतको अधिकार क्षेत्रः}

(१) जिल्ला अदालतलाई संघीय कानूनमा अन्यथा व्यवस्था भएकोमा बाहेक आफ्नो अधिकारक्षेत्रभित्रका सम्पूर्ण मुद्दाको शुरू कारबाही र किनारा गर्ने, बन्दी प्रत्यक्षीकरण निषेधाज्ञा लगायत कानून बमोजिमका निवेदन हेर्ने, अर्धन्यायिक निकायले गरेको निर्णय उपर कानून
बमोजिम पुनरावेदन सुन्ने, प्रदेश कानून बमोजिम गठित स्थानीयस्तरका न्यायिक निकायले गरेको निर्णय उपर पुनरावेदन सुन्ने तथा आफू र आफ्नो मातहतका अदालतहरूको न्याय सम्पादनको कार्यमा कसैले अवरोध गरेमा वा आदेश वा फैसलाको अवज्ञा गरेमा संघीय कानून बमोजिम अवहेलनामा कारबाही चलाइ सजाय गर्न सक्ने अधिकार हुनेछ ।

(२) जिल्ला अदालतको अधिकार क्षेत्र र कार्यविधि सम्बन्धी अन्य व्यवस्था संघीय कानून बमोजिम हुनेछ ।

\textbf{१५२. विशिष्टीकृत अदालतः}

(१) धारा १२७ मा लेखिएदेखि बाहेक खास किसिम र प्रकृतिका मुद्दाहरूको कारबाही र किनारा गर्न संघीय कानून बमोजिम अन्य विशिष्टीकृत अदालत, न्यायिक निकाय वा न्यायाधिकरणको स्थापना र गठन गर्न सकिनेछ । तर कुनै खास मुद्दाका लागि विशिष्टीकृत अदालत, न्यायिक निकाय वा न्यायाधिकरणको गठन गरिने छैन ।

(२) एक वर्ष भन्दा बढी कैद सजाय हुने फौजदारी कसूर सम्बन्धीठट मुद्दा अदालत वा विशिष्टीकृत अदालत वा सैनिक अदालत वा न्यायिक
निकाय बाहेक अन्य निकायको अधिकार क्षेत्रमा पर्ने छैन ।

\textbf{१५३. न्याय परिषदः}

(१) यस संविधान बमोजिम न्यायाधीशको नियुक्ति, सरुवा, अनुशासन सम्बन्धी कारबाही, बर्खासी र न्याय प्रशासन सम्बन्धी अन्य
विषयको सिफारिस गर्न वा परामर्श दिन एउटा न्याय परिषद रहनेछ, जसमा देहाय बमोजिमका अध्यक्ष र सदस्यहरू रहनेछन्ः–

(क) प्रधान न्यायाधीश –अध्यक्ष
(ख) संघीय कानून तथा न्याय मन्त्री –सदस्य
(ग) सर्वोच्च अदालतका वरिष्ठतम न्यायाधीश एक जना – सदस्य
(घ) राष्ट्रपतिले प्रधानमन्त्रीको सिफारिसमा नियुक्त गरेको एक जना कानूनविद् –सदस्य
(ङ) नेपाल बार एसोसिएशनको सिफारिसमा राष्ट्रपतिद्वारा नियुक्त कम्तीमा बीस वर्षको अनुभव प्राप्त वरिष्ठ अधिवक्ता वा अधिवक्ता – सदस्य

(२) उपधारा (१) को खण्ड (घ) र (ङ) बमोजिमको सदस्यको पदावधि चार वर्षको हुनेछ र निजको पारिश्रमिक तथा सुविधा सर्वोच्च अदालतको न्यायाधीशको सरह हुनेछ ।

(३) उपधारा (१) को खण्ड (घ) र (ङ) बमोजिमको सदस्य सर्वोच्च अदालतको न्यायाधीश सरह समान आधारमा र समान तरीकाले पदमुक्त
हुनेछ ।

(४) न्याय परिषदको अध्यक्ष तथा सदस्यले कुनै न्यायाधीशको सम्बन्धमा पर्न आएको उजुरीसँग सम्बद्ध मुद्दाको अध्ययन गरी न्याय
परिषदमा त्यसको प्रतिवेदन दिन सक्नेछ ।

(५) कुनै न्यायाधीशको विषयमा पर्न आएको उजुरीको सम्बन्धमा प्रारम्भिक छानबीन गराउँदा विशेषज्ञबाट विस्तृत छानबीन गर्नुपर्ने देखिएमा न्याय परिषदले जाँचबुझ समिति गठन गर्न सक्नेछ ।

(६) यो संविधान बमोजिम महाभियोगको कारबाहीबाट पदमुक्त हुन सक्ने न्यायाधीश बाहेक अन्य न्यायाधीशले भ्रष्टाचार गरी अख्तियारको दुरुपयोग गरेकोमा न्यायपरिषदले अनुसन्धान गरी कानून बमोजिम मुद्दा चलाउन सक्नेछ ।
(७) न्याय परिषदले यस संविधान बमोजिम सर्वोच्च अदालतको प्रधान न्यायाधीश र न्यायाधीश, उच्च अदालतका मुख्य न्यायाधीश र न्यायाधीशको पदमा नियुक्त हुन योग्यता पुग्ने व्यक्तिहरूको अद्यावधिक अभिलेख तयार गर्नु  पर्नेछ ।
(८) न्याय परिषदको अन्य काम, कर्तव्य र अधिकार संघीय कानून बमोजिम हुनेछ ।

\textbf{१५४. न्याय सेवा आयोगः}

(१) नेपाल सरकारले कानून बमोजिम संघीय न्याय सेवाको राजपत्रांकित पदमा नियुक्ति, सरुवा, बढुवा गर्दा वा त्यस्तो पदमा बहाल रहेको कुनै कर्मचारीलाई विभागीय सजाय गर्दा न्याय सेवा आयोगको सिफारिस बमोजिम गर्नेछ । तर संघीय सरकारी सेवामा बहाल नरहेको व्यक्तिलाई संघीय न्याय सेवाको राजपत्रांकित पदमा नयाँ भर्नाद्वारा स्थायी नियुक्ति गर्दा वा संघीय न्याय सेवाको राजपत्र अनङ्कित पदबाट सोही सेवाको राजपत्रांकित पदमा बढुवा गर्दा नेपाल सरकारले लोक सेवा आयोगको सिफारिस बमोजिम गर्नु पर्नेछ ।

स्पष्टीकरणः यस धाराको प्रयोजनका लागि संघीय न्याय सेवाको राजपत्रांकित पदमा नियुक्ति गर्दा लिइने खुला र आन्तरिक प्रतियोगितात्मक परीक्षा लोक सेवा आयोगले लिनेछ ।

(२) न्याय सेवा आयोगमा देहाय बमोजिमका अध्यक्ष र सदस्यहरू हुनेछन्ः–

(क) प्रधान न्यायाधीश –अध्यक्ष
(ख) संघीय कानून तथा न्याय मन्त्री –सदस्य
(ग) सर्वोच्च अदालतको वरिष्ठतम न्यायाधीश –सदस्य
(घ) लोक सेवा आयोगको अध्यक्ष –सदस्य
(ङ) महान्यायाधिवक्ता –सदस्य

(३) न्याय सेवा आयोगको अन्य काम, कर्तव्य, अधिकार र कार्यविधि संघीय कानून बमोजिम हुनेछ ।

\textbf{१५५. सेवाका शर्त र सुविधा सम्बन्धी व्यवस्थाः} संघीय न्याय सेवाका कर्मचारीहरूको पारिश्रमिक, सुविधा तथा सेवाका शर्त सम्बन्धी व्यवस्था संघीय ऐन बमोजिम हुनेछ ।

\textbf{१५६. प्रदेश न्याय सेवा आयोग सम्बन्धी व्यवस्थाः} प्रदेश न्याय सेवा आयोगको गठन र प्रदेश न्याय सेवाका कर्मचारीहरूको पारिश्रमिक, सुविधा तथा सेवाका शर्त सम्बन्धी व्यवस्था संघीय कानून बमोजिम हुनेछ ।