\section{भाग–५ राज्यको संरचना र राज्यशक्तिको बाँडफाँड}

(३) यो संविधान प्रारम्भ हुँदाका बखत नेपालमा कायम रहेका अनुसूची–४ मा उल्लेख भए बमोजिमका जिल्लाहरू रहेका प्रदेश रहनेछन् ।

(४) स्थानीय तह अन्तर्गत गाउँँपालिका, नगरपालिका र जिल्ला सभा रहनेछन् । गाउँपालिका र नगरपालिकामा रहने वडाको संख्या संघीय कानून बमोजिम हुनेछ ।

(५) संघीय कानून बमोजिम सामाजिक सांस्कृतिक संरक्षण वा आर्थिक विकासका लागि विशेष, संरक्षित वा स्वायत्त क्षेत्र कायम गर्न सकिनेछ ।

(६) संघ, प्रदेश र स्थानीय तहले नेपालको स्वतन्त्रता, सार्वभौमसत्ता, भौगोलिक अखण्डता, स्वाधीनता, राष्ट्रिय हित, सर्वांगीण विकास, बहुदलीय प्रतिस्पर्धात्मक लोकतान्त्रिक गणतन्त्रात्मक संघीय शासन प्रणाली, मानव अधिकार तथा मौलिक हक, कानूनी राज्य, शक्ति पृथकीकरण र नियन्त्रण तथा सन्तुलन, बहुलता र समानतामा आधारित समतामूलक समाज, समावेशी प्रतिनिधित्व र पहिचानको संरक्षण गर्ने छन् ।

\textbf{५७. राज्यशक्तिको बाँडफाँडः}

(१) संघको अधिकार अनुसूची–५ मा उल्लिखित विषयमा निहित रहनेछ र त्यस्तो अधिकारको प्रयोग यो संविधान र संघीय कानून बमोजिम हुनेछ ।

(२) प्रदेशको अधिकार अनुसूची–६ मा उल्लिखित विषयमा निहित रहनेछ र त्यस्तो अधिकारको प्रयोग यो संविधान र प्रदेश कानून बमोजिम
हुनेछ ।

(३) संघ र प्रदेशको साझा अधिकार अनुसूची–७ मा उल्लिखित विषयमा निहित रहनेछ र त्यस्तो अधिकारको प्रयोग यो संविधान, संघीय
कानून र प्रदेश कानून बमोजिम हुनेछ ।

(४) स्थानीय तहको अधिकार अनुसूची–८ मा उल्लिखित विषयमा निहित रहनेछ र त्यस्तो अधिकारको प्रयोग यो संविधान र गाउँ सभा वा
नगर सभाले बनाएको कानून बमोजिम हुनेछ ।

(५) संघ, प्रदेश र स्थानीय तहको साझा अधिकार अनुसूची–९ मा उल्लिखित विषयमा निहित रहनेछ र त्यस्तो अधिकारको प्रयोग यो संविधान रघद्ध संघीय कानून, प्रदेश कानून र गाउँ सभा वा नगर सभाले बनाएको कानून बमोजिम हुनेछ ।

(६) उपधारा (३) वा (५) बमोजिम प्रदेश सभा, गाउँ सभा वा नगर सभाले कानून बनाउँदा संघीय कानूनसँग नबाझिने गरी बनाउनु पर्नेछ र प्रदेश सभा, गाउँ सभा वा नगर सभाले बनाएको त्यस्तो कानून संघीय कानूनसँग बाझिएमा बाझिएको हदसम्म आमान्य हुनेछ ।

(७) उपधारा (५) बमोजिम गाउँ सभा वा नगर सभाले कानून बनाउँदा प्रदेश कानूनसँग नबाझिने गरी बनाउनु पर्नेछ र गाउँ सभा वा
नगर सभाले बनाएको त्यस्तो कानून प्रदेश कानूनसँग बाझिएमा बाझिएको हदसम्म आमान्य हुनेछ ।

\textbf{५८. अवशिष्ट अधिकारः}

यस संविधान बमोजिम संघ, प्रदेश तथा स्थानीय तहको अधिकारको सूची वा साझा सूचीमा उल्लेख नभएको वा यो संविधानमा कुनै तहले प्रयोग गर्ने गरी नतोकिएको विषयमा संघको अधिकार हुनेछ ।

\textbf{५९. आर्थिक अधिकारको प्रयोगः}

(१) संघ, प्रदेश र स्थानीय तहले आफ्नो अधिकारभित्रको आर्थिक अधिकार सम्बन्धी विषयमा कानून बनाउने, वार्षिक बजेट बनाउने, निर्णय गर्ने, नीति तथा योजना तयार गर्ने र त्यसको कार्यान्वयन गर्ने छन् ।

(२) संघले साझा सूचीका विषयमा र आर्थिक अधिकारका अन्य क्षेत्रमा प्रदेशलाई समेत लागू हुने गरी आवश्यक नीति, मापदण्ड र कानून बनाउन सक्नेछ ।

(३) संघ, प्रदेश र स्थानीय तहले आ–आफ्नो तहको बजेट बनाउने छन् र प्रदेश र स्थानीय तहले बजेट पेश गर्ने समय संघीय कानून बमोजिम
हुनेछ ।

(४) संघ, प्रदेश र स्थानीय तहले प्राकृतिक स्रोतको प्रयोग वा विकासबाट प्राप्त लाभको समन्यायिक वितरणको व्यवस्था गर्नु पर्नेछ । त्यस्तो
लाभको निश्चित अंश रोयल्टी, सेवा वा वस्तुको रूपमा परियोजना प्रभावित क्षेत्र र स्थानीय समुदायलाई कानून बमोजिम वितरण गर्नु पर्नेछ ।

(५) संघ, प्रदेश र स्थानीय तहले प्राकृतिक स्रोतको उपयोग गर्दा स्थानीय समुदायले लगानी गर्न चाहेमा लगानीको प्रकृति र आकारको आधारमा कानून बमोजिमको अंश लगानी गर्न प्राथमिकता दिनु पर्नेछ ।घछ

(६) वैदेशिक सहायता र ऋण लिने अधिकार नेपाल सरकारको हुनेछ । त्यस्तो सहायता वा ऋण लिंदा देशको समष्टिगत आर्थिक स्थायित्व हुने गरी लिनु पर्नेछ ।

(७) संघ, प्रदेश र स्थानीय तहको बजेट घाटा व्यवस्थापन तथा अन्य वित्तीय अनुशासन सम्बन्धी व्यवस्था संघीय कानून बमोजिम हुनेछ ।

\textbf{६०. राजस्व स्रोतको बाँडफाँडः}

(१) संघ, प्रदेश र स्थानीय तहले आफ्नो आर्थिक अधिकारक्षेत्र भित्रको विषयमा कर लगाउन र ती स्रोतहरूबाट राजस्व उठाउन सक्नेछन् । तर साझा सूचीभित्रको विषयमा र कुनै पनि तहको सूचीमा नपरेका विषयमा कर लगाउने र राजस्व उठाउने व्यवस्था नेपाल सरकारले निर्धारण
गरे बमोजिम हुनेछ ।

(२) नेपाल सरकारले संकलन गरेको राजस्व संघ, प्रदेश र स्थानीय तहलाई न्यायोचित वितरण गर्ने व्यवस्था मिलाउनेछ ।

(३) प्रदेश र स्थानीय तहले प्राप्त गर्ने वित्तीय हस्तान्तरणको परिमाण राष्ट्रिय प्राकृतिक स्रोत तथा वित्त आयोगको सिफारिस बमोजिम हुनेछ ।

(४) नेपाल सरकारले प्रदेश र स्थानीय तहलाई खर्चको आवश्यकता र राजस्वको क्षमताको आधारमा वित्तीय समानीकरण अनुदान वितरण गर्नेछ ।

(५) प्रदेशले नेपाल सरकारबाट प्राप्त अनुदान र आफ्नो स्रोतबाट उठ्ने राजस्वलाई मातहतको स्थानीय तहको खर्चको आवश्यकता र राजस्व क्षमताको आधारमा प्रदेश कानून बमोजिम वित्तीय समानीकरण अनुदान वितरण गर्ने छन् ।

(६) नेपाल सरकारले संघीय सञ्चित कोषबाट प्रदान गर्ने सशर्त अनुदान, समपूरक अनुदान वा अन्य प्रयोजनका लागि दिने विशेष अनुदान
वितरण सम्बन्धी व्यवस्था संघीय कानून बमोजिम हुनेछ ।

(७) संघ, प्रदेश र स्थानीय तह बीच राजस्वको बाँडफाँड गर्दा सन्तुलित र पारदर्शी रूपमा गर्नु पर्नेछ ।

(८) राजस्व बाँडफाँड सम्बन्धी संघीय ऐन बनाउँदा राष्ट्रिय नीति, राष्ट्रिय आवश्यकता, प्रदेश र स्थानीय तहको स्वायत्तता, प्रदेश र स्थानीय
तहले जनतालाई पुर्‍याउनु पर्ने सेवा र उनीहरूलाई प्रदान गरिएको आर्थिक अधिकार, राजस्व उठाउन सक्ने क्षमता, राजस्वको सम्भाव्यता र उपयोग, विकास निर्माणमा गर्नुपर्ने सहयोग, क्षेत्रीय असन्तुलन, गरीबी र असमानताकोघट न्यूनीकरण, वञ्चितीकरणको अन्त्य, आकस्मिक कार्य र अस्थायी आवश्यकता पूरा गर्न सहयोग गर्नु पर्ने विषयहरूमा ध्यान दिनु पर्नेछ ।