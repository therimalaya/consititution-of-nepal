\section{अनुसूची–४ प्रदेश र सम्बन्धित प्रदेशमा रहने जिल्लाहरू}

\begin{multicols*}{2}
    
\textbf{प्रदेश नं. १}

ताप्लेजुङ\\
पाँचथर\\
इलाम\\
संखुवासभा\\
तेह्रथुम\\
धनकुटा\\
भोजपुर\\
खोटाङ\\
सोलुखुम्बु\\
ओखलढुंगा\\
उदयपुर\\
झापा\\
मोरङ\\
सुनसरी\\

\textbf{प्रदेश नं. २}

सप्तरी\\
सिराहा\\
धनुषा\\
महोत्तरी\\
सर्लाही’\\
रौतहट\\
बारा\\
पर्सा \\

\textbf{प्रदेश नं. ३}

दोलखा\\
रामेछाप\\
सिन्धुली\\
काभ्रेपलाञ्चोक\\
सिन्धुपाल्चोक\\
रसुवा\\
नुवाकोट\\
धादिङ\\
चितवन\\
मकवानपुर\\
भक्तपुर\\
ललितपुर\\
काठमाडौं\\

\textbf{प्रदेश नं. ४}

गोरखा\\
लमजुङ\\
तनहुँ\\
कास्की\\
मनाङ\\
मुस्ताङ\\
पर्वत\\
स्याङजा\\
म्याग्दी\\
बाग्लुङ\\
नवलपरासी (बर्दघाट सुस्ता पूर्व\\

\textbf{प्रदेश नं. ५}

नवलपरासी (बर्दघाट सुस्ता पश्चिम)\\
रूपन्देही\\
कपिलबस्तु\\
पाल्पा\\
अर्घाखाँची\\
गुल्मी\\
रुकुम (पूर्वी भाग)\\
रोल्पा\\
प्यूठान\\
दाङ\\
बाँके\\
बर्दिया\\

\textbf{प्रदेश नं. ६}

रुकुम (पश्चिम भाग)\\
सल्यान\\
डोल्पा\\
जुम्ला\\
मुगु\\
हुम्ला\\
कालिकोट\\
जाजरकोट\\
दैलेख\\
सुर्खेत\\

\textbf{प्रदेश नं. ७}

बाजुरा\\
बझाङ\\
डोटी\\
अछाम\\
दार्चुला\\
बैतडी\\
डडेल्धुरा\\
कञ्चनपुर\\
कैलाली\\

\end{multicols*}
