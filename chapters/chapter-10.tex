\section{भाग–१० संघीय आर्थिक कार्यप्रणाली}

\textbf{११७. संघीय सञ्चित कोष वा संघीय सरकारी कोषबाट व्ययः}

देहाय बमोजिमका रकम बाहेक संघीय सञ्चित कोष वा अन्य कुनै संघीय सरकारी कोषबाट कुनै रकम झिक्न सकिने छैनः–

(क) संघीय सञ्चित कोषमाथि व्ययभार भएको रकम,
(ख) संघीय विनियोजन ऐन बमोजिम खर्च हुने रकम,
(ग) विनियोजन विधेयक विचाराधीन रहेको अवस्थामा पेश्कीको रूपमा संघीय ऐन बमोजिम खर्च हुने रकम, वा
(घ) विशेष अवस्थामा व्ययको विवरण मात्र भएको संघीय उधारो खर्च ऐन बमोजिम व्यय हुने रकम । तर संघीय आकस्मिक कोषका हकमा धारा १२४ बमोजिम हुनेछ ।

\textbf{११८. संघीय सञ्चित कोषमाथि व्ययभारः}

देहायका विषयसँग सम्बन्धित खर्च संघीय सञ्चित कोषमाथि व्ययभार हुनेछ र त्यस्तो व्ययका लागि संघीय संसदको स्वीकृति आवश्यक पर्ने छैनः–

(क) राष्ट्रपति र उपराष्ट्रपतिको पारिश्रमिक तथा सुविधाको रकम,
(ख) नेपालको प्रधान न्यायाधीश, सर्वोच्च अदालतका न्यायाधीश र न्यायपरिषदका सदस्यलाई दिइने पारिश्रमिक तथा सुविधाको रकम,
(ग) प्रतिनिधि सभाका सभामुख र उपसभामुख, राष्ट्रिय सभाका अध्यक्ष र उपाध्यक्षलाई दिइने पारिश्रमिक तथा सुविधाकोे रकम,
(घ) संवैधानिक निकायका प्रमुख र पदाधिकारीलाई दिइने पारिश्रमिक तथा सुविधाको रकम,
(ङ) प्रदेश प्रमुखको पारिश्रमिक तथा सुविधाको रकम,
(च) राष्ट्रपति वा उपराष्ट्रपतिको कार्यालय, सर्वोच्च अदालत, न्यायपरिषद, संवैधानिक निकाय र प्रदेश प्रमुखको कार्यालयको प्रशासनिक व्यय,
(छ) नेपाल सरकारको दायित्वको ऋण सम्बन्धी व्ययभार,
(ज) नेपाल सरकारको विरुद्ध अदालतबाट भएको फैसला वा आदेश अनुसार तिर्नु पर्ने रकम, र
(झ) संघीय कानून बमोजिम संघीय सञ्चित कोषमाथि व्ययभार हुने रकम ।

\textbf{११९. राजस्व र व्ययको अनुमानः}

(१) नेपाल सरकारको अर्थमन्त्रीले प्रत्येक आर्थिक वर्षको सम्बन्धमा संघीय संसदका दुवै सदनको संयुक्त बैठकमा देहायका विषयहरू समेत खुलाई वार्षिक अनुमान पेश गर्नु पर्नेेछः–
(क) राजस्वको अनुमान,
(ख) संघीय सञ्चित कोषमाथि व्ययभार हुने आवश्यक रकमहरू, र
(ग) संघीय विनियोजन ऐन बमोजिम व्यय हुने आवश्यक रकमहरू ।

(२) उपधारा (१) बमोजिम वार्षिक अनुमान पेश गर्दा अघिल्लो आर्थिक वर्षमा प्रत्येक मन्त्रालयलाई छुट्याइएको खर्चको रकम र त्यस्तो खर्च
अनुसारको लक्ष्य हासिल भयो वा भएन त्यसको विवरण पनि साथै पेश गर्नु पर्नेछ ।

(३) नेपाल सरकारको अर्थमन्त्रीले उपधारा (१) बमोजिमको राजस्व र व्ययको अनुमान प्रत्येक वर्ष जेठ महीनाको पन्ध्र गते संघीय संसदमा पेश गर्नेछ ।

\textbf{१२०. विनियोजन ऐनः} विनियोजन ऐन बमोजिम व्यय हुने रकम सम्बन्धित शीर्षकमा उल्लेख गरी विनियोजन विधेयकमा राखिनेछ ।

\textbf{१२१. पूरक अनुमानः} (१) कुनै आर्थिक वर्षमा देहायको अवस्था पर्न आएमा नेपाल सरकारको अर्थमन्त्रीले प्रतिनिधि सभामा पूरक अनुमान पेश गर्न सक्नेछः–

(क) चालू आर्थिक वर्षका लागि विनियोजन ऐनले कुनै सेवाका लागि खर्च गर्न अख्तियारी दिएको रकम अपर्याप्त भएमा वा त्यस वर्षका लागि विनियोजन ऐनले अख्तियारी नदिएको नयाँ सेवामा खर्च गर्न आवश्यक भएमा, वा

(ख) चालू आर्थिक वर्षमा विनियोजन ऐनले अख्तियारी दिएको रकमभन्दा बढी खर्च हुन गएमा ।

(२) पूरक अनुमानमा राखिएको रकम सम्बन्धित शीर्षकमा उल्लेख गरी पूरक विनियोजन विधेयकमा राखिनेछ ।

\textbf{१२२. पेश्की खर्चः} (१) यस भागमा अन्यत्र जुनसुकै कुरा लेखिएको भए तापनि विनियोजन विधेयक विचाराधीन रहेको अवस्थामा आर्थिक वर्षका लागि अनुमान गरिएको व्ययको कुनै अंश पेश्कीका रूपमा संघीय ऐन बमोजिम खर्च गर्न सकिनेछ ।

(२) धारा ११९ बमोजिम राजस्व र व्ययको अनुमान पेश नगरिएसम्म पेश्की खर्च विधेयक प्रस्तुत गरिने छैन र पेश्कीको रकम आर्थिक वर्षको व्यय अनुमानको एक तिहाइ भन्दा बढी हुने छैन ।

(३) संघीय पेश्की खर्च ऐन बमोेजिम खर्च भएको रकम विनियोजन विधेयकमा समावेश गरिनेछ ।

\textbf{१२३. उधारो खर्चः} यस भागमा अन्यत्र जुनसुकै कुरा लेखिएको भए तापनि प्राकृतिक कारण वा बाह्य आक्रमणको आशंका वा आन्तरिक विघ्न वा अन्य कारणले संकटको अवस्था परी धारा ११९ को उपधारा (१) अन्तर्गत चाहिने विवरण खुलाउन अव्यावहारिक वा राज्यको सुरक्षा वा हितका दृष्टिले अवाञ्छनीय देखिएमा नेपाल सरकारको अर्थमन्त्रीले व्ययको विवरण मात्र भएको उधारो खर्च विधेयक प्रतिनिधि सभामा पेश गर्न सक्नेछ ।

\textbf{१२४. संघीय आकस्मिक कोषः} (१) संघीय ऐन बमोजिम आकस्मिक कोषका नामले एउटा कोष स्थापना गर्न सकिनेछ र त्यस्तो कोषमा समय समयमा संघीय ऐन बमोजिम निर्धारण भएको रकम जम्मा गरिनेछ ।
(२) उपधारा (१) बमोजिमको कोष नेपाल सरकारको नियन्त्रणमा रहनेछ र नेपाल सरकारले त्यस्तो कोषबाट आकस्मिक कार्यका लागि खर्च
गर्न सक्नेछ ।
(३) उपधारा (२) बमोजिमको खर्चको रकम संघीय ऐन बमोजिम यथाशीध्र सोधभर्ना गरिनेछ ।

\textbf{१२५. आर्थिक कार्यविधि सम्बन्धी ऐनः}  संघीय ऐन बमोजिम विनियोजित रकम एक शीर्षकबाट अर्को शीर्षकमा रकमान्तर गर्ने र आर्थिक कार्यविधि सम्बन्धी अन्य व्यवस्था संघीय ऐन बमोजिम हुनेछ ।