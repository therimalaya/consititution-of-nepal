\section{भाग–१७ स्थानीय कार्यपालिका}

(३) यो संविधान र अन्य कानूनको अधीनमा रही गाउँपालिका र नगरपालिकाको शासन व्यवस्थाको सामान्य निर्देशन, नियन्त्रण र सञ्चालन गर्ने अभिभारा गाउँ कार्यपालिका र नगर कार्यपालिकाको हुनेछ ।

(४) गाउँपालिका र नगरपालिकाका कार्यकारिणी कार्य गाउँ कार्यपालिका र नगर कार्यपालिकाको नाममा हुनेछ ।

(५) उपधारा (४) बमोजिम गाउँ कार्यपालिका र नगर कार्यपालिकाको नाममा हुने निर्णय वा आदेश र तत्सम्बन्धी अधिकारपत्रको प्रमाणीकरण स्थानीय कानून बमोजिम हुनेछ ।

\textbf{२१५. गाउँ कार्यपालिका अध्यक्ष र उपाध्यक्ष सम्बन्धी व्यवस्थाः}

(१) प्रत्येक गाउँपालिकामा एक जना गाउँ कार्यपालिका अध्यक्ष रहनेछ । निजको अध्यक्षतामा गाउँ कार्यपालिका गठन हुनेछ ।

(२) उपधारा (१) बमोजिमको गाउँ कार्यपालिकामा एक जना उपाध्यक्ष, प्रत्येक वडाबाट निर्वाचित वडा अध्यक्ष र उपधारा (४) बमोजिम
निर्वाचित सदस्य रहनेछन् ।

(३) अध्यक्ष र उपाध्यक्षको निर्वाचन सम्बन्धित गाउँपालिका क्षेत्रभित्रका मतदाताले एक व्यक्ति एक मतको आधारमा गोप्य मतदानद्वारा पहिलो हुने निर्वाचित हुने निर्वाचन प्रणाली बमोजिम गर्नेछन् ।

स्पष्टीकरणः यस धाराको प्रयोजनका लागि “अध्यक्ष र उपाध्यक्ष” भन्नाले गाउँ कार्यपालिकाको अध्यक्ष र उपाध्यक्ष सम्झनु पर्छ ।

(४) धारा २२२ बमोजिमको गाउँ सभाको निर्वाचनको अन्तिम परिणाम प्राप्त भएको मितिले पन्ध्र दिनभित्र गाउँ सभाका सदस्यहरूले आफूमध्येबाट निर्वाचित गरेका चार जना महिला सदस्य र उपधारा (५) बमोजिमको योग्यता भएका दलित वा अल्पसंख्यक समुदायबाट गाउँ सभाले निर्वाचित गरेका दुई जना सदस्य समेत गाउँ कार्यपालिकाको सदस्य हुनेछन् ।

(५) देहायको योग्यता भएको व्यक्ति अध्यक्ष, उपाध्यक्ष, वडा अध्यक्ष र सदस्यको पदमा निर्वाचित हुन योग्य हुनेछः–

(क) नेपाली नागरिक,

(ख) एक्काइस वर्ष उमेर पूरा भएको,

(ग) गाउँपालिकाको मतदाता नामावलीमा नाम समावेश भएको, र

(घ) कुनै कानूनले अयोग्य नभएको ।

(६) अध्यक्ष, उपाध्यक्ष, वडा अध्यक्ष र सदस्यको पदावधि निर्वाचित भएको मितिले पाँच वर्षको हुनेछ ।

(७) अध्यक्षको पदमा दुई कार्यकाल निर्वाचित भएको व्यक्ति गाउँपालिकाको निर्वाचनमा उम्मेदवार हुन पाउने छैन ।

(८) देहायको कुनै अवस्थामा अध्यक्ष, उपाध्यक्ष, वडा अध्यक्ष र सदस्यको पद रिक्त हुनेछः–

(क) अध्यक्षले उपाध्यक्ष समक्ष र उपाध्यक्षले अध्यक्षसमक्ष लिखित राजीनामा दिएमा,

(ख) निजको पदावधि समाप्त भएमा,

(ग) निजको मृत्यु भएमा ।

(९) अध्यक्ष वा उपाध्यक्षको एक वर्षभन्दा बढी पदावधि बाँकी रहेको अवस्थामा उपधारा (७) बमोजिम पद रिक्त हुन गएमा बाँकी अवधिका लागि रिक्त पदको पूर्ति उपनिर्वाचनद्वारा हुनेछ ।

\textbf{२१६. नगर कार्यपालिका प्रमुख र उपप्रमुख सम्बन्धी व्यवस्थाः}

(१) प्रत्येक नगरपालिकामा एक जना नगर कार्यपालिका प्रमुख रहनेछ । निजको अध्यक्षतामा नगर कार्यपालिका गठन हुनेछ ।
(२) उपधारा (१) बमोजिमको नगर कार्यपालिकामा एक जना उपप्रमुख, प्रत्येक वडाबाट निर्वाचित वडा अध्यक्ष र उपधारा (४) बमोजिम निर्वाचित सदस्य रहनेछन् ।

(३) प्रमुख र उपप्रमुखको निर्वाचन सम्बन्धित नगरपालिका क्षेत्रभित्रका मतदाताले एक व्यक्ति एक मतको आधारमा गोप्य मतदानद्वारा पहिलो हुने निर्वाचित हुने निर्वाचन प्रणाली बमोजिम गर्ने छन् ।

स्पष्टीकरणः यस धाराको प्रयोजनका लागि “प्रमुख र उपप्रमुख” भन्नाले नगर कार्यपालिकाको प्रमुख र उपप्रमुख सम्झनु पर्छ ।

(४) धारा २२३ बमोजिमको नगर सभाको निर्वाचनको अन्तिम परिणाम प्राप्त भएको मितिले पन्ध्र दिनभित्र नगर सभाका सदस्यहरूले आफूमध्येबाट निर्वाचित गरेका पाँच जना महिला सदस्य र उपधारा (५) बमोजिमको योग्यता भएका दलित वा अल्पसंख्यक समुदायबाट नगर सभाले निर्वाचित गरेका तीन जना सदस्य समेत नगर कार्यपालिकाको सदस्य हुने छन् ।

(५) देहायको योग्यता भएको व्यक्ति प्रमुख, उपप्रमुख, वडा अध्यक्ष र सदस्यको पदमा निर्वाचित हुन योग्य हुनेछः–

(क) नेपाली नागरिक,

(ख) एक्काइस वर्ष उमेर पूरा भएको,

(ग) नगरपालिकाको मतदाता नामावलीमा नाम समावेश भएको, र

(घ) कुनै कानूनले अयोग्य नभएको ।

(६) प्रमुख, उपप्रमुख, वडा अध्यक्ष र सदस्यको पदावधि निर्वाचित भएको मितिले पाँच वर्षको हुनेछ ।

(७) प्रमुखको पदमा दुई कार्यकाल निर्वाचित भएको व्यक्ति नगरपालिकाको निर्वाचनमा उम्मेदवार हुन पाउने छैन ।

(८) देहायको कुनै अवस्थामा प्रमुख, उपप्रमुख, वडा अध्यक्ष र सदस्यको पद रिक्त हुनेछः–

(क) प्रमुखले उपप्रमुख समक्ष र उपप्रमुखले प्रमुख समक्ष लिखित राजीनामा दिएमा,

(ख) निजको पदावधि समाप्त भएमा,

(ग) निजको मृत्यु भएमा ।

(९) प्रमुख वा उपप्रमुखको एक वर्षभन्दा बढी पदावधि बाँकी रहेको अवस्थामा उपधारा (८) बमोजिम पद रिक्त हुन गएमा बाँकी अवधिका लागि रिक्त पदको पूर्ति उपनिर्वाचनद्वारा हुनेछ ।

\textbf{२१७. न्यायिक समितिः}

(१) कानून बमोजिम आफ्नो अधिकारक्षेत्र भित्रका विवाद निरूपण गर्न गाउँपालिका वा नगरपालिकाले प्रत्येक गाउँपालिकामा उपाध्यक्षको संयोजकत्वमा र प्रत्येक नगरपालिकामा उपप्रमुखको संयोजकत्वमा तीन सदस्यीय एक न्यायिक समिति रहनेछ ।

(२) उपधारा (१) बमोजिमको न्यायिक समितिमा गाउँ सभा वा नगर सभाबाट आफूमध्येबाट निर्वाचित गरेका दुई जना सदस्यहरू रहनेछन् ।

\textbf{२१८. गाउँ कार्यपालिका र नगर कार्यपालिकाको कार्य सञ्चालनः} गाउँ कार्यपालिका र नगर कार्यपालिकाबाट स्वीकृत नियमावली बमोजिम गाउँ कार्यपालिका वा नगर कार्यपालिकाको कार्य विभाजन र कार्य सम्पादन हुनेछ ।

\textbf{२१९. स्थानीय तहको कार्यकारिणी सम्बन्धी अन्य व्यवस्थाः} यस भागमा लेखिए देखि बाहेक स्थानीय तहको कार्यकारिणी सम्बन्धी अन्य व्यवस्था यस संविधानको अधीनमा रही संघीय कानून बमोजिम हुनेछ ।

\textbf{२२०. जिल्ला सभा र जिल्ला समन्वय समितिः}

(१) जिल्ला भित्रका गाउँपालिका र नगरपालिकाहरूबीच समन्वय गर्न एक जिल्ला सभा रहनेछ ।

(२) जिल्ला सभामा जिल्ला भित्रका गाउँ कार्यपालिका अध्यक्ष र उपाध्यक्ष तथा नगर कार्यपालिका प्रमुख र उपप्रमुख सदस्य रहनेछन् । गाउँ
सभा र नगर सभाको निर्वाचनको अन्तिम परिणाम प्राप्त भएको मितिले तीस दिन भित्र जिल्ला सभाको पहिलो बैठक बस्नेछ ।

(३) जिल्ला सभाले एक जना प्रमुख, एक जना उपप्रमुख, कम्तीमा तीन जना महिला र कम्तीमा एक जना दलित वा अल्पसंख्यक सहित बढीमा नौ जना सदस्य रहेको जिल्ला समन्वय समितिको निर्वाचन गर्नेछ । जिल्ला समन्वय समितिले जिल्ला सभाको तर्फबाट गर्नु पर्ने सम्पूर्ण कार्य सम्पादन गर्नेछ ।

(४) सम्बन्धित जिल्लाभित्रको गाउँ सभा वा नगर सभाको सदस्य जिल्ला समन्वय समितिको प्रमुख, उपप्रमुख वा सदस्य पदको उम्मेदवार हुन योग्य हुनेछ । जिल्ला समन्वय समितिको प्रमुख उपप्रमुख वा सदस्यको पदमा निर्वाचित भएमा त्यस्तो व्यक्तिको गाउँ सभा वा नगर सभाको सदस्य पद स्वतः रिक्त हुनेछ ।

(५) जिल्ला समन्वय समितिको प्रमुख, उपप्रमुख र सदस्यको पदावधि निर्वाचित भएको मितिले पाँच वर्षको हुनेछ ।

(६) देहायको कुनै अवस्थामा जिल्ला समन्वय समितिको प्रमुख, उपप्रमुख वा सदस्यको पद रिक्त हुनेछः–

(क) प्रमुखले उपप्रमुख समक्ष र उपप्रमुख वा सदस्यले प्रमुख समक्ष लिखित राजीनामा दिएमा,

(ख) निजको पदावधि समाप्त भएमा,

(ग) निजको मृत्यु भएमा ।

(७) जिल्ला सभाको काम, कर्तव्य र अधिकार देहाय बमोजिम  हुनेछः –

(क) जिल्लाभित्रका गाउँपालिका र नगरपालिका बीच समन्वय गर्ने,

(ख) विकास तथा निर्माण सम्बन्धी कार्यमा सन्तुलन कायम गर्न सोको अनुगमन गर्ने,

(ग) जिल्लामा रहने संघीय र प्रदेश सरकारी कार्यालय र गाउँपालिका र नगरपालिका बीच समन्वय गर्ने,

(घ) प्रदेश कानून बमोजिमका अन्य कार्यहरू गर्ने ।

(८) जिल्ला सभाको सञ्चालन, जिल्ला समन्वय समितिका सदस्यले पाउने सुविधा तथा जिल्ला सभा सम्बन्धी अन्य व्यवस्था प्रदेश कानून
बमोजिम हुनेछ ।