\section{अनुसूची–५ संघको अधिकारको सूची}

१.रक्षा र सेना सम्बन्धी\\
(क) राष्ट्रिय एकता र भौगोलिक अखण्डताको संरक्षण\\
(ख) राष्ट्रिय सुरक्षा सम्बन्धी\\
२. युद्ध र प्रतिरक्षा\\
३. हातहतियार, खरखजाना कारखाना तथा उत्पादन सम्बन्धी\\
४. केन्द्रीय प्रहरी, सशस्त्र प्रहरी बल, राष्ट्रिय गुप्तचर तथा अनुसन्धान, शान्ति सुरक्षा\\
५. केन्द्रीय योजना, केन्द्रीय बैंक, वित्तीय नीति, मुद्रा र बैंकिङ्ग, मौद्रिक नीति, विदेशी अनुदान, सहयोग र ऋण\\
६. परराष्ट्र तथा कूटनीतिक मामिला, अन्तर्राष्ट्रिय सम्बन्ध र संयुक्त राष्ट्रसंघ सम्बन्धी\\
७. अन्तर्राष्ट्रिय सन्धि वा सम्झौता, सुपुर्दगी, पारस्परिक कानूनी सहायता र अन्तर्राष्ट्रिय सीमा, अन्तर्राष्ट्रिय सीमा नदी,\\
८. दूरसञ्चार, रेडियो फ्रिक्वेन्सीको बाँडफाँड, रेडियो, टेलिभिजन र हुलाक\\
९. भन्सार, अन्तःशुल्क, मूल्य अभिवृद्धि कर, संस्थागत आयकर, व्यक्तिगत आयकर, पारिश्रमिक कर, राहदानी शुल्क, भिसा शुल्क, पर्यटन दस्तुर, सेवा शुल्क दस्तुर, दण्ड जरिबाना\\
१०. संघीय निजामती सेवा, न्याय सेवा र अन्य सरकारी सेवा\\
११. जलस्रोतको संरक्षण र बहुआयामिक उपयोग सम्बन्धी नीति र मापदण्ड\\
१२. अन्तरदेशीय तथा अन्तरप्रदेश विद्युत प्रसारण लाइन\\
१३. केन्द्रीय तथ्यांक (राष्ट्रिय र अन्तर्राष्ट्रिय मानक र गुणस्तर)\\
१४. केन्द्रीय स्तरका ठूला विद्युत, सिंचाइ र अन्य आयोजना तथा परियोजना\\
१५. केन्द्रीय विश्वविद्यालय, केन्द्रीयस्तरका प्रज्ञा प्रतिष्ठान, विश्वविद्यालय मापदण्ड र नियमन, केन्द्रीय पुस्तकालय\\
१६. स्वास्थ्य नीति, स्वास्थ्य सेवा, स्वास्थ्य मापदण्ड, गुणस्तर र अनुगमन, राष्ट्रिय वा विशिष्ट सेवा प्रदायक अस्पताल, परम्परागत उपचार सेवा, सरुवा रोग नियन्त्रण\\
१७. संघीय संसद, संघीय कार्यपालिका, स्थानीय तह सम्बन्धी मामिला, विशेष संरचना\\
१८. अन्तर्राष्ट्रिय व्यापार, विनिमय, बन्दरगाह, क्वारेन्टाइन\\
१९. हवाई उड्डयन, अन्तर्राष्ट्रिय विमानस्थल\\
२०. राष्ट्रिय यातायात नीति, रेल तथा राष्ट्रिय लोकमार्गको व्यवस्थापन\\
२१. सर्वोच्च अदालत, उच्च अदालत, जिल्ला अदालत तथा न्याय प्रशासन सम्बन्धी कानून\\
२२. नागरिकता, राहदानी, भिसा, अध्यागमन\\
२३. आणविक ऊर्जा, वायुमण्डल र अन्तरिक्ष सम्बन्धी\\
२४. बौद्धिक सम्पत्ति (पेटेन्ट, डिजाइन, टे«डमार्क र प्रतिलिपि अधिकार समेत)\\
२५. नाप–तौल\\
२६. खानी उत्खनन\\
२७. राष्ट्रिय तथा अन्तर्राष्ट्रिय वातावरण व्यवस्थापन, राष्ट्रिय निकुञ्ज, वन्यजन्तु आरक्ष तथा सिमसार क्षेत्र, राष्ट्रिय वन नीति, कार्बन सेवा\\
२८. बीमा नीति, धितोपत्र, सहकारी नियमन\\
२९. भूउपयोग नीति, बस्ती विकास नीति, पर्यटन नीति, वातावरण अनुकूलन\\
३०. फौजदारी, देवानी कानूनको निर्माण\\
३१. सुरक्षित छापाखाना\\
३२. सामाजिक सुरक्षा र गरीबी निवारण\\
३३. संवैधानिक निकायहरू, राष्ट्रिय महत्वका आयोगहरू\\
३४. पुरातात्विक महत्वका स्थान र प्राचीन स्मारक\\
३५. संघ, प्रदेश र स्थानीय तहको अधिकारको सूचीमा वा साझा सूचीमा उल्लेख नभएको कुनै विषय तथा यो संविधान र संघीय कानूनमा नतोकिएको विषय\\