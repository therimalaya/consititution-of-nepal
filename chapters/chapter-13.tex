\section{भाग–१३ प्रदेश कार्यपालिका}

(३) प्रदेशको कार्यकारिणी कार्यहरू प्रदेश सरकारका नाममा हुने छन् ।

(४) यो संविधानको अधीनमा रही प्रदेशको कार्यकारिणी अधिकार अनुसूची–६, अनुसूची–७ र अनुसूची–९ बमोजिमको सूचीमा उल्लेख भए
बमोजिम हुनेछ । तर संघ र प्रदेशको साझा अधिकारको विषयका सम्बन्धमा यो संविधान र संघीय कानूनमा स्पष्ट उल्लेख भएकोमा बाहेक प्रदेश मन्त्रिपरिषदले कार्यकारिणी अधिकारको प्रयोग गर्दा नेपाल सरकारसँग समन्वय गरी गर्नु पर्नेछ ।

(५) उपधारा (३) बमोजिम प्रदेश सरकारको नाममा हुने निर्णय वा आदेश र तत्सम्बन्धी अधिकारपत्रको प्रमाणीकरण प्रदेश कानून बमोजिम
हुनेछ ।

\textbf{१६३. प्रदेश प्रमुख सम्बन्धी व्यवस्थाः}

(१) प्रत्येक प्रदेशमा नेपाल सरकारको प्रतिनिधिको रूपमा प्रदेश प्रमुख रहनेछ ।
(२) राष्ट्रपतिले प्रत्येक प्रदेशका लागि एक प्रदेश प्रमुख नियुक्ति गर्नेछ ।
(३) राष्ट्रपतिले पदावधि समाप्त हुनुभन्दा अगावै निजलाई पदमुक्त गरेमा बाहेक प्रदेश प्रमुखको पदावधि पाँच वर्षको हुनेछ ।
(४) एउटै व्यक्ति एक पटकभन्दा बढी एकै प्रदेशमा प्रदेश प्रमुख हुन सक्ने छैन ।

\textbf{१६४. प्रदेश प्रमुखको योग्यताः} देहायको योग्यता भएको व्यक्ति प्रदेश प्रमुख हुनका लागि योग्य हुनेछ ः–
(क) संघीय संसदको सदस्य हुन योग्य भएकोे,
(ख) पैंतीस वर्ष उमेर पूरा भएको, र
(ग) कुनै कानूनले अयोग्य नभएको ।

\textbf{१६५. प्रदेश प्रमुखको पद रिक्त हुने अवस्थाः} (१) देहायको कुनै अवस्थामा प्रदेश प्रमुखको पद रिक्त हुनेछः–

(क) निजले राष्ट्रपति समक्ष लिखित राजीनामा दिएमा,
(ख) निजको पदावधि समाप्त भएमा वा सो अगावै राष्ट्रपतिले निजलाई पदमुक्त गरेमा,
(ग) निजको मृत्यु भएमा ।

(२) कुनै प्रदेशको प्रदेश प्रमुखको पद रिक्त भएको अवस्थामा प्रदेश प्रमुखको नियुक्ति नभएसम्मका लागि राष्ट्रपतिले अर्को कुनै प्रदेशको प्रदेश प्रमुखलाई त्यस्तो प्रदेशको समेत कामकाज गर्ने गरी तोक्न सक्नेछ ।

\textbf{१६६. प्रदेश प्रमुखको काम, कर्तव्य र अधिकारः}

(१) प्रदेश प्रमुखले यो संविधान वा कानून बमोजिम निजलाई प्राप्त अधिकारको प्रयोग र कर्तव्यको पालन गर्नेछ ।

(२) उपधारा (१) बमोजिम अधिकारको प्रयोग गर्दा वा कर्तव्यको पालन गर्दा यो संविधान वा कानून बमोजिम कुनै निकाय वा पदाधिकारीको
सिफारिसमा गरिने भनी किटानीसाथ व्यवस्था भएको कार्य बाहेक प्रदेश प्रमुखबाट सम्पादन गरिने अन्य जुनसुकै कार्य प्रदेश मन्त्रिपरिषदको सिफारिस र सम्मतिबाट हुनेछ । त्यस्तो सिफारिस र सम्मति मुख्यमन्त्री मार्फत पेश हुनेछ ।

(३) उपधारा (२) बमोजिम प्रदेश प्रमुखको नाममा हुने निर्णय वा आदेश र तत्सम्बन्धी अधिकारपत्रको प्रमाणीकरण प्रदेश कानून बमोजिम
हुनेछ ।

\textbf{१६७. प्रदेश प्रमुखको शपथः}

प्रदेश प्रमुखले आफ्नो कार्यभार सम्हाल्नु अघि संघीय कानून बमोजिम राष्ट्रपति समक्ष पद तथा गोपनीयताको शपथ लिनु पर्नेछ ।

\textbf{१६८. प्रदेश मन्त्रिपरिषदको गठनः}

(१) प्रदेश प्रमुखले प्रदेश सभामा बहुमत प्राप्त संसदीय दलको नेतालाई मुख्यमन्त्री नियुक्त गर्नेछ र निजको अध्यक्षतामा प्रदेश मन्त्रिपरिषदको गठन हुनेछ ।

(२) उपधारा (१) बमोजिम प्रदेश सभामा कुनै पनि दलको स्पष्ट बहुमत नरहेको अवस्थामा प्रदेश सभामा प्रतिनिधित्व गर्ने दुई वा दुई भन्दा
बढी दलहरूको समर्थनमा बहुमत प्राप्त गर्न सक्ने प्रदेश सभाको सदस्यलाई प्रदेश प्रमुखले मुख्यमन्त्री नियुक्त गर्नेछ ।

(३) प्रदेश सभाको निर्वाचनको अन्तिम परिणाम घोषणा भएको मितिले तीस दिनभित्र उपधारा (२) बमोजिम मुख्यमन्त्री नियुक्ति हुन सक्ने अवस्था नभएमा वा त्यसरी नियुक्त मुख्यमन्त्रीले उपधारा (४) बमोजिम विश्वासको मत प्राप्त गर्न नसकेमा प्रदेश प्रमुखले प्रदेश सभामा सबैभन्दा बढी सदस्यहरू भएको संसदीय दलको नेतालाई मुख्यमन्त्री नियुक्त गर्नेछ ।

(४) उपधारा (२) वा (३) बमोजिम नियुक्त मुख्यमन्त्रीले त्यसरी नियुक्त भएको तीस दिनभित्र प्रदेश सभाबाट विश्वासको मत प्राप्त गर्नु पर्नेछ ।

(५) उपधारा (३) बमोजिम नियुक्त मुख्यमन्त्रीले उपधारा (४) बमोजिम विश्वासको मत प्राप्त गर्न नसकेमा उपधारा (२) बमोजिमको कुनै सदस्यले प्रदेश सभामा विश्वासको मत प्राप्त गर्न सक्ने अवस्था भएमा प्रदेश प्रमुखले त्यस्तो सदस्यलाई मुख्यमन्त्री नियुक्त गर्नेछ ।

(६) उपधारा (५) बमोजिम नियुक्त मुख्यमन्त्रीले उपधारा (४) बमोजिम विश्वासको मत प्राप्त गर्नु पर्नेछ ।

(७) उपधारा (५) बमोजिम नियुक्त मुख्यमन्त्रीले विश्वासको मत प्राप्त गर्न नसकेमा वा मुख्यमन्त्री नियुक्त हुन नसकेमा मुख्यमन्त्रीको सिफारिसमा प्रदेश प्रमुखले प्रदेश सभालाई विघटन गरी छ महीनाभित्र अर्को प्रदेश सभाको निर्वाचन सम्पन्न हुने गरी निर्वाचन मिति तोक्नेछ ।
(८) यस संविधान बमोजिम भएको प्रदेश सभाको निर्वाचनको अन्तिम परिणाम घोषणा भएको वा मुख्यमन्त्रीको पद रिक्त भएको मितिले पैंतीस दिनभित्र यस धारा बमोजिम मुख्यमन्त्री नियुक्ति सम्बन्धी प्रक्रिया सम्पन्न गर्नु पर्नेछ ।

(९) प्रदेश प्रमुखले मुख्यमन्त्रीको सिफारिसमा प्रदेश सभाका सदस्यमध्येबाट समावेशी सिद्धान्त बमोजिम मुख्यमन्त्री सहित प्रदेश सभाका
कुल सदस्य संख्याको बीस प्रतिशत भन्दा बढी नहुने गरीे प्रदेश मन्त्रिपरिषद गठन गर्नेछ ।

स्पष्टीकरण :

यस भागको प्रयोजनका लागि “मन्त्री” भन्नाले मन्त्री, राज्य मन्त्री र सहायक मन्त्री सम्झनु पर्छ ।

(१०) मुख्यमन्त्री र मन्त्री सामूहिक रूपमा प्रदेश सभाप्रति उत्तरदायी हुनेछन् र मन्त्रीहरू आफ्नो मन्त्रालयको कामका लागि व्यक्तिगत रूपमा मुख्यमन्त्री र प्रदेश सभाप्रति उत्तरदायी हुनेछन् ।

\textbf{१६९. मुख्यमन्त्री तथा मन्त्रीको पद रिक्त हुुने अवस्थाः}

(१) देहायको कुनै अवस्थामा मुख्यमन्त्रीको पद रिक्त हुनेछः–

(क) निजले प्रदेश प्रमुख समक्ष लिखित राजीनामा दिएमा,
(ख) धारा १८८ बमोजिम विश्वासको प्रस्ताव पारित हुन नसकेमा वा निजको विरुद्ध अविश्वासको प्रस्ताव पारित भएमा,
(ग) निज प्रदेश सभाको सदस्य नरहेमा,
(घ) निजको मृत्यु भएमा ।

(२) देहायको कुनै अवस्थामा मन्त्रीको पद रिक्त हुनेछः–

(क) निजले मुख्यमन्त्री समक्ष लिखित राजीनामा दिएमा,
(ख) मुख्यमन्त्रीले निजलाई पदमुक्त गरेमा,
(ग) उपधारा (१) को खण्ड (क), (ख) वा (ग) बमोजिम मुख्यमन्त्रीको पद रिक्त भएमा,
(घ) निजको मृत्यु भएमा ।
(३) उपधारा (१) बमोजिम मुख्यमन्त्रीको पद रिक्त भए तापनि अर्को प्रदेश मन्त्रिपरिषद गठन नभएसम्म सोही मन्त्रिपरिषदले कार्य सञ्चालन गर्नेछ ।

तर मुख्यमन्त्रीको मृत्यु भएमा अर्काे मुख्यमन्त्री नियुक्ति नभएसम्मका लागि वरिष्ठतम मन्त्रीले मुख्यमन्त्रीको रूपमा कार्य सञ्चालन गर्नेछ ।
\textbf{१७०. प्रदेश सभाको सदस्य नभएको व्यक्ति मन्त्री हुन सक्नेः}

(१) धारा १६८ को उपधारा (९) मा जुनसुकै कुरा लेखिएको भए तापनि प्रदेश प्रमुखले मुख्यमन्त्रीको सिफारिसमा प्रदेश सभाको सदस्य नभएको कुनै व्यक्तिलाई मन्त्री पदमा नियुक्त गर्न सक्नेछ ।

(२) उपधारा (१) बमोजिम नियुक्त मन्त्रीले शपथ ग्रहण गरेको मितिले छ महीनाभित्र प्रदेश सभाको सदस्यता प्राप्त गर्नु पर्नेछ ।

(३) उपधारा (२) बमोजिमको अवधिभित्र प्रदेश सभाको सदस्यता प्राप्त गर्न नसकेमा तत्काल कायम रहेको प्रदेश सभाको कार्यकालमा निज मन्त्री पदमा पुनः नियुक्तिका लागि योग्य हुने छैन ।

(४) उपधारा (१) मा जुनसुकै कुरा लेखिएको भए तापनि तत्काल कायम रहेको प्रदेश सभाको निर्वाचनमा पराजित भएको व्यक्ति सो प्रदेश
सभाको कार्यकालमा उपधारा (१) बमोजिम मन्त्री पदमा नियुक्तिका लागि योग्य हुने छैन ।

\textbf{१७१. मुख्यमन्त्री र मन्त्रीको पारिश्रमिक तथा अन्य सुविधाः}  मुख्यमन्त्री र मन्त्रीको पारिश्रमिक र अन्य सुविधा प्रदेश ऐन बमोजिम हुनेछ । त्यस्तो ऐन नबनेसम्म प्रदेश सरकारले तोके बमोजिम हुनेछ ।

\textbf{१७२. शपथः }मुख्यमन्त्री र मन्त्रीले प्रदेश प्रमुख समक्ष तथा राज्यमन्त्री र सहायक मन्त्रीले मुख्यमन्त्री समक्ष आफ्नो कार्यभार सम्हाल्नु अघि प्रदेश कानून बमोजिम पद तथा गोपनीयताको शपथ लिनु पर्नेछ ।

\textbf{१७३. प्रदेश प्रमुखलाई जानकारी दिनेः} मुख्यमन्त्रीले देहायका विषयमा प्रदेश प्रमुखलाई जानकारी दिनेछः–
(क) प्रदेश मन्त्रिपरिषदका निर्णय,
(ख) प्रदेश सभामा पेश गरिने विधेयक,
(ग) खण्ड (क) र (ख) मा उल्लिखित विषयमा प्रदेश प्रमुखले जानकारी मागेको अन्य आवश्यक विवरण, र
(घ) प्रदेशको समसामयिक परिस्थिति ।

\textbf{१७४. प्रदेश सरकारको कार्य सञ्चालनः}

(१) प्रदेश सरकारबाट स्वीकृत नियमावली बमोजिम प्रदेश सरकारको कार्य विभाजन र कार्य सम्पादन हुनेछ ।
(२) उपधारा (१) अन्तर्गतको नियमावलीको पालन भयो वा भएन भन्ने प्रश्न कुनै अदालतमा उठाउन सकिने छैन ।