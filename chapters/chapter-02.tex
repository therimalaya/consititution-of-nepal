\section{भाग–२ नागरिकता}

\textbf{११. नेपालको नागरिक ठहर्नेः}

(१) यो संविधान प्रारम्भ हुँदाका बखत नेपालको नागरिकता प्राप्त गरेका र यस भाग बमोजिम नागरिकता प्राप्त गर्न योग्य
व्यक्तिहरू नेपालको नागरिक हुनेछन् ।

(२) यो संविधान प्रारम्भ हुँदाका बखत नेपालमा स्थायी बसोवास भएको देहायको व्यक्ति वंशजको आधारमा नेपालको नागरिक ठहर्नेछः–

(क) यो संविधान प्रारम्भ हुनुभन्दा अघि वंशजको आधारमा नेपालको नागरिकता प्राप्त गरेको व्यक्ति,
(ख) कुनै व्यक्तिको जन्म हुँदाका बखत निजको बाबु वा आमा नेपालको नागरिक रहेछ भने त्यस्तो व्यक्ति ।

(३) यो संविधान प्रारम्भ हुनुभन्दा अघि जन्मको आधारमा नेपालको नागरिकता प्राप्त गरेको नागरिकको सन्तानले बाबु र आमा दुवै नेपालको नागरिक रहेछन् भने निज बालिग भएपछि वंशजको आधारमा नेपालको नागरिकता प्राप्त गर्नेछ ।

(४) नेपालभित्र फेला परेको पितृत्व र मातृत्वको ठेगान नभएको प्रत्येक नाबालक निजको बाबु वा आमा फेला नपरेसम्म वंशजको आधारमा
नेपालको नागरिक ठहर्नेछ ।

(५) नेपालको नागरिक आमाबाट नेपालमा जन्म भई नेपालमा नै बसोबास गरेको र बाबुको पहिचान हुन नसकेको व्यक्तिलाई वंशजको
आधारमा नेपालको नागरिकता प्रदान गरिनेछ ।

तर बाबु विदेशी नागरिक भएको ठहरेमा त्यस्तो व्यक्तिको नागरिकता संघीय कानून बमोजिम अंगीकृत नागरिकतामा परिणत हुनेछ ।

(६) नेपाली नागरिकसँग वैवाहिक सम्बन्ध कायम गरेकी विदेशी महिलाले चाहेमा संघीय कानून बमोजिम नेपालको अंगीकृत नागरिकता लिन सक्नेछ ।

(७) यस धारामा अन्यत्र जुनसुकै कुरा लेखिएको भए तापनि विदेशी नागरिकसँग विवाह गरेकी नेपाली महिला नागरिकबाट जन्मिएको व्यक्तिको हकमा निज नेपालमा नै स्थायी बसोबास गरेको र निजले विदेशी मुलुकको नागरिकता प्राप्त गरेको रहेनछ भने निजले संघीय कानून बमोजिम नेपालको अंगीकृत नागरिकता प्राप्त गर्न सक्नेछ ।

तर नागरिकता प्राप्त गर्दाका बखत निजका आमा र बाबु दुवै नेपाली नागरिक रहेछन् भने नेपालमा जन्मेको त्यस्तो व्यक्तिले वंशजको आधारमा नेपालको नागरिकता प्राप्त गर्न सक्नेछ ।

(८) यस धारामा लेखिएदेखि बाहेक नेपाल सरकारले संघीय कानून बमोजिम नेपालको अंगीकृत नागरिकता प्रदान गर्न सक्नेछ ।

(९) नेपाल सरकारले संघीय कानून बमोजिम नेपालको सम्मानार्थ नागरिकता प्रदान गर्न सक्नेछ ।

(१०) नेपालभित्र गाभिने गरी कुनै क्षेत्र प्राप्त भएमा त्यस्तो क्षेत्रभित्र बसोबास भएको व्यक्ति संघीय कानूनको अधीनमा रही नेपालको नागरिक हुनेछ ।

\textbf{१२. वंशीय आधार तथा लैंगिक पहिचान सहितको नागरिकताः}

यो संविधान बमोजिम वंशजको आधारमा नेपालको नागरिकता प्राप्त गर्ने व्यक्तिले निजको आमा वा बाबुको नामबाट लैंगिक पहिचान सहितको नेपालको नागरिकताको प्रमाणपत्र पाउन सक्नेछ ।

\textbf{१३. नागरिकताको प्राप्ति, पुनःप्राप्ति र समाप्तिः}

नागरिकताको प्राप्ति, पुनःप्राप्ति र समाप्ति सम्बन्धी अन्य व्यवस्था संघीय कानून बमोजिम हुनेछ ।

\textbf{१४. गैरआवासीय नेपाली नागरिकता प्रदान गर्न सकिनेः}

विदेशी मुलुकको नागरिकता प्राप्त गरेको दक्षिण एशियाली क्षेत्रीय सहयोग संगठनको सदस्य राष्ट्र बाहेकका देशमा बसोबास गरेको साबिकमा वंशजको वा जन्मको आधारमा निज वा निजको बाबु वा आमा, बाजे वा बज्यै नेपालको नागरिक रही पछि विदेशी मुलुकको नागरिकता प्राप्त गरेको व्यक्तिलाई संघीय कानून बमोजिम आर्थिक, सामाजिक र सांस्कृतिक अधिकार उपभोग गर्न पाउने गरी
नेपालको गैरआवासीय नागरिकता प्रदान गर्न सकिनेछ ।

\textbf{१५. नेपालको नागरिकता सम्बन्धी अन्य व्यवस्थाः}

नेपालको प्रत्येक नागरिकको परिचय खुल्ने गरी अभिलेख राख्ने तथा नेपालको नागरिकता सम्बन्धी अन्य व्यवस्था संघीय कानून बमोजिम हुनेछ ।