\section{भाग–१५ प्रदेश व्यवस्थापन कार्यविधि}

(३) “अर्थ विधेयक” भन्नाले देहायमा उल्लिखित सबै वा कुनै विषयसँग सम्बन्ध राख्ने विधेयकलाई जनाउँछः–

(क) प्रदेशमा कर लगाउने, उठाउने, खारेज गर्ने, छूट दिने, परिवर्तन गर्ने वा कर प्रणालीलाई व्यवस्थित गर्ने विषय,

(ख) प्रदेश सञ्चित कोष वा अन्य कुनै प्रदेश सरकारी कोषको संरक्षण गर्ने, त्यस्तो कोषमा रकम जम्मा गर्ने वा त्यस्तो कोषबाट कुनै रकम विनियोजन वा खर्च गर्ने वा विनियोजन वा खर्च गर्न खोजिएको रकम घटाउने, बढाउने वा खारेज गर्ने विषय,

(ग) प्रदेश सरकारले ऋण प्राप्त गर्ने वा जमानत दिने विषय व्यवस्थित गर्ने वा प्रदेशको सरकारले लिएको वा लिने आर्थिक दायित्व सम्बन्धी कानून संशोधन गर्ने विषय,

(घ) प्रदेश सरकारी कोषमा प्राप्त हुने सबै प्रकारको राजस्व, ऋण असुलीबाट प्राप्त रकम र अनुदानको रकम जिम्मा राख्ने, लगानी गर्ने वा प्रदेश सरकारको लेखा वा लेखा परीक्षण गर्ने विषय, वा

(ङ) खण्ड (क), (ख), (ग) वा (घ) सँग प्रत्यक्ष सम्बन्ध भएका प्रासंगिक विषय ।

तर कुनै अनुमतिपत्र दस्तुर, निवेदन दस्तुर, नवीकरण दस्तुर जस्ता दस्तुर, शुल्क वा महसूल लगाउने वा कुनै जरिवाना वा कैद हुने व्यवस्था
भएको कारणले मात्र कुनै विधेयक अर्थ विधेयक मानिने छैन ।

(४) कुनै विधेयक अर्थ विधेयक हो होइन भन्ने प्रश्न उठेमा प्रदेश सभाको सभामुखको निर्णय अन्तिम हुनेछ ।

\textbf{१९९. विधेयक पारित गर्ने विधिः}

(१) प्रदेश सभाले पारित गरेको विधेयक प्रमाणीकरणका लागि प्रदेश प्रमुख समक्ष पेश गरिनेछ ।

(२) कुनै विधेयक विचाराधीन रहेको अवस्थामा प्रदेश सभाको अधिवेशनको अन्त्य भए पनि सो विधेयकमाथि अर्को अधिवेशनमा कारबाही
हुन सक्नेछ ।

तर कुनै विधेयक प्रदेश सभामा विचाराधीन रहेको अवस्थामा प्रदेश सभा विघटन भएमा वा सो सभाको कार्यकाल समाप्त भएमा त्यस्तो विधेयक निष्क्रिय हुनेछ ।

\textbf{२००. विधेयक फिर्ता लिन सक्नेः} विधेयक प्रस्तुतकर्ताले प्रदेश सभाको स्वीकृति लिई विधेयक फिर्ता लिन सक्नेछ ।

\textbf{२०१. विधेयकमा प्रमाणीकरणः} (१) धारा १९९ बमोजिम प्रमाणीकरणका लागि प्रदेश प्रमुख समक्ष पेश गरिने विधेयक प्रदेश सभाको सभामुखले प्रमाणित गरी पेश गर्नु पर्नेछ ।

तर अर्थ विधेयकका हकमा अर्थ विधेयक हो भनी प्रदेश सभामुखले प्रमाणित गर्नु पर्नेछ ।

(२) यस धारा बमोजिम प्रमाणीकरणका लागि प्रदेश प्रमुख समक्ष पेश भएको विधेयक प्रदेश प्रमुखले पन्ध्र दिनभित्र प्रमाणीकरण गरी त्यसको सूचना यथासम्भव चाँडो प्रदेश सभालाई दिनेछ ।

(३) प्रमाणीकरणका लागि पेश भएको अर्थ विधेयक बाहेक अन्य विधेयकको सम्बन्धमा पुनर्विचार हुन आवश्यक छ भन्ने प्रदेश प्रमुखलाई
लागेमा निजले विधेयक पेश भएको पन्ध्र दिनभित्र सन्देश सहित प्रदेश सभामा फिर्ता पठाउन सक्नेछ ।

(४) उपधारा (३) बमोजिम प्रदेश प्रमुखले कुनै विधेयक सन्देश सहित फिर्ता गरेमा त्यस विधेयकमाथि प्रदेश सभाले पुनर्विचार गरी त्यस्तो विधेयक प्रस्तुत रूपमा वा संशोधन सहित पारित गरी पुनः पेश गरेमा त्यसरी पेश भएको पन्ध्र दिन भित्र प्रदेश प्रमुखले प्रमाणीकरण गर्नेछ ।

(५) प्रदेश प्रमुखबाट प्रमाणीकरण भएपछि विधेयक ऐन बन्नेछ ।

\textbf{२०२. अध्यादेशः}

(१) प्रदेश सभाको अधिवेशन चलिरहेको अवस्थामा बाहेक अन्य अवस्थामा तत्काल केही गर्न आवश्यक परेमा प्रदेश मन्त्रिपरिषदको
सिफारिसमा प्रदेश प्रमुखले अध्यादेश जारी गर्न सक्नेछ ।

(२) उपधारा (१) बमोजिम जारी भएको अध्यादेश ऐन सरह मान्य हुनेछ ।

तर त्यस्तो प्रत्येक अध्यादेश,–
(क) जारी भएपछि बसेको प्रदेश सभाको बैठकमा पेश गरिनेछ र प्रदेश सभाले स्वीकार नगरेमा स्वतः निष्क्रिय हुनेछ,
(ख) प्रदेश प्रमुखबाट जुनसुकै बखत खारेज हुन सक्नेछ, र
(ग) खण्ड (क) वा (ख) बमोजिम निष्क्रिय वा खारेज नभएमा प्रदेश सभाको बैठक बसेको साठी दिनपछि स्वतः निष्क्रिय हुनेछ ।