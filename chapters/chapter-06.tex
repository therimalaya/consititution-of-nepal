\section{भाग–६ राष्ट्रपति र उपराष्ट्रपति}

(२) उपधारा (१) मा जुनसुकै कुरा लेखिएको भए तापनि कुनै प्रदेशमा प्रदेश सभाको निर्वाचन नभएका कारणले मात्र राष्ट्रपतिको निर्वाचन
प्रयोजनका लागि निर्वाचक मण्डल गठन गर्न बाधा परेको मानिने छैन ।

(३) उपधारा (१) बमोजिमको निर्वाचक मण्डलको तत्काल कायम रहेको कुल मतको बहुमत प्राप्त गर्ने व्यक्ति राष्ट्रपति निर्वाचित हुनेछ ।

(४) उपधारा (३) बमोजिम कुनै उम्मेदवारले बहुमत प्राप्त गर्न नसकेमा सबैभन्दा बढी मत प्राप्त गर्ने दुई उम्मेदवारहरू बीच मतदान हुनेछ
र त्यस्तो मतदानमा कुल मतको पचास प्रतिशतभन्दा बढी मत प्राप्त गर्ने उम्मेदवार राष्ट्रपति निर्वाचित हुनेछ ।

(५) उपधारा (४) बमोजिमको मतदानबाट समेत कुनै उम्मेदवारले कुल मतको पचास प्रतिशतभन्दा बढी मत प्राप्त गर्न नसकेमा पुनः मतदान हुनेछ । त्यस्तो मतदानमा खसेको कुल सदर मतको बढी मत प्राप्त गर्ने उम्मेदवार राष्ट्रपति निर्वाचित हुनेछ ।

(६) निर्वाचन, मनोनयन वा नियुक्ति हुने राजनीतिक पदमा बहाल रहेकोे व्यक्ति यस धारा बमोजिम राष्ट्रपति निर्वाचित भएमा निजको त्यस्तो पद स्वतः रिक्त हुनेछ ।

(७) राष्ट्रपतिको निर्वाचन र तत्सम्बन्धी अन्य व्यवस्था संघीय कानून बमोजिम हुनेछ ।

\textbf{६३. राष्ट्रपतिको पदावधिः}

(१) राष्ट्रपतिको पदावधि निर्वाचित भएको मितिले पाँच वर्षको हुनेछ ।

(२) उपधारा (१) बमोजिमको पदावधि समाप्त भएको राष्ट्रपतिले अर्काे निर्वाचित राष्ट्रपतिले पदभार नसम्हालेसम्म यस संविधान बमोजिमको कार्य सम्पादन गर्नेछ ।

\textbf{६४. राष्ट्रपतिको योग्यताः}

(१) देहायको योग्यता भएको व्यक्ति राष्ट्रपति हुनका लागि योग्य हुनेछः–

(क) संघीय संसदको सदस्य हुन योग्य भएको,
(ख) कम्तीमा पैंतालिस वर्ष उमेर पूरा भएको, र
(ग) कुनै कानूनले अयोग्य नभएको ।

(२) उपधारा (१) मा जुनसुकै कुरा लेखिएको भए तापनि दुई पटक राष्ट्रपति निर्वाचित भइसकेको व्यक्ति राष्ट्रपतिको निर्वाचनमा उम्मेदवार हुन सक्ने छैन ।

\textbf{६५. राष्ट्रपतिको पद रिक्त हुने अवस्थाः} देहायको कुनै अवस्थामा राष्ट्रपतिको पद रिक्त हुनेछः–

(क) निजले उपराष्ट्रपति समक्ष लिखित राजीनामा दिएमा,
(ख) निजको विरुद्ध धारा १०१ बमोजिम महाभियोगको प्रस्ताव पारित भएमा,
(ग) निजको पदावधि समाप्त भएमा,
(घ) निजको मृत्यु भएमा ।

\textbf{६६. राष्ट्रपतिको काम, कर्तव्य र अधिकारः}

(१) राष्ट्रपतिले यो संविधान वा संघीय कानून बमोजिम निजलाई प्राप्त अधिकारको प्रयोग र कर्तव्यको पालन
गर्नेछ ।
(२) उपधारा (१) बमोजिम अधिकारको प्रयोग वा कर्तव्यको पालन गर्दा यो संविधान वा संघीय कानून बमोजिम कुनै निकाय वा पदाधिकारीको सिफारिसमा गरिने भनी किटानीसाथ व्यवस्था भएको कार्य बाहेक राष्ट्रपतिबाट सम्पादन गरिने अन्य जुनसुकै कार्य मन्त्रिपरिषदको सिफारिस र सम्मतिबाट हुनेछ । त्यस्तो सिफारिस र सम्मति प्रधानमन्त्री मार्फत पेश हुनेछ ।
(३) उपधारा (२) बमोजिम राष्ट्रपतिको नाममा हुने निर्णय वा आदेश र तत्सम्बन्धी अधिकारपत्रको प्रमाणीकरण संघीय कानून बमोजिम हुनेछ ।

\textbf{६७. उपराष्ट्रपतिः}

(१) नेपालमा एक उपराष्ट्रपति रहनेछ ।
(२) राष्ट्रपतिको अनुपस्थितिमा राष्ट्रपतिबाट गरिने कार्यहरू उपराष्ट्रपतिबाट सम्पादन गरिनेछ ।
(३) निर्वाचन, मनोनयन वा नियुक्ति हुने राजनीतिक पदमा बहाल रहेकोे कुनै व्यक्ति उपराष्ट्रपतिको पदमा निर्वाचित भएमा निजको त्यस्तो पद स्वतः रिक्त हुनेछ ।

\textbf{६८. उपराष्ट्रपतिको पद रिक्त हुने अवस्थाः} देहायको कुनै अवस्थामा उपराष्ट्रपतिको पद रिक्त हुनेछः–

(क) निजले राष्ट्रपति समक्ष लिखित राजीनामा दिएमा,
(ख) निजको विरुद्ध धारा १०१ बमोजिम महाभियोगको प्रस्ताव पारित भएमा,
(ग) निजको पदावधि समाप्त भएमा,
(घ) निजको मृत्यु भएमा ।

\textbf{६९. उपराष्ट्रपति सम्बन्धी अन्य व्यवस्थाः} उपराष्ट्रपतिको योग्यता, निर्वाचन प्रक्रिया, पदावधि सम्बन्धी व्यवस्था राष्ट्रपतिको सरह हुनेछ ।

\textbf{७०. राष्ट्रपति र उपराष्ट्रपति फरक फरक लिंग वा समुदायको हुनेः} यस संविधान बमोजिम राष्ट्रपति र उपराष्ट्रपतिको निर्वाचन फरक फरक लिंग वा समुदायको प्रतिनिधित्व हुने गरी गर्नु पर्नेछ ।

\textbf{७१. राष्ट्रपति र उपराष्ट्रपतिको शपथः} राष्ट्रपति र उपराष्ट्रपतिले आफ्नो कार्यभार सम्हाल्नु अघि संघीय कानून बमोजिम राष्ट्रपतिले प्रधान न्यायाधीश समक्ष र उपराष्ट्रपतिले राष्ट्रपति समक्ष पद तथा गोपनीयताको शपथ लिनु पर्नेछ ।

\textbf{७२. राष्ट्रपति र उपराष्ट्रपतिको पारिश्रमिक तथा सुविधाः} राष्ट्रपति र उपराष्ट्रपतिको पारिश्रमिक तथा अन्य सुविधा संघीय ऐन बमोजिम हुनेछ र त्यस्तो ऐन नबनेसम्म नेपाल सरकारले तोके बमोजिम हुनेछ ।

\textbf{७३. राष्ट्रपति र उपराष्ट्रपतिको कार्यालयः} (१) राष्ट्रपति र उपराष्ट्रपतिको कार्य सम्पादनका लागि छुट्टाछुट्टै कार्यालय रहनेछ ।
(२) उपधारा (१) बमोजिमको कार्यालयको काम कारबाही सञ्चालन गर्न आवश्यक कर्मचारी तथा अन्य व्यवस्था नेपाल सरकारले गर्नेछ ।