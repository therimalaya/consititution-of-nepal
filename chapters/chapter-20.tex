\section{भाग–२० संघ, प्रदेश र स्थानीय तह बीच अन्तरसम्बन्ध}

\textbf{२३२. संघ, प्रदेश र स्थानीय तह बीचको सम्बन्धः} (१) संघ, प्रदेश र स्थानीय तह बीचको सम्बन्ध सहकारिता, सहअस्तित्व र समन्वयको सिद्धान्तमा आधारित हुनेछ ।

(२) नेपाल सरकारले राष्ट्रिय महत्वका विषयमा र प्रदेशहरू बीच समन्वय गर्नुपर्ने विषयमा प्रदेश मन्त्रिपरिषदलाई यो संविधान र संघीय कानून बमोजिम आवश्यक निर्देशन दिन सक्नेछ र त्यस्तो निर्देशनको पालन गर्नु सम्बन्धित प्रदेश मन्त्रिपरिषदको कर्तव्य हुनेछ ।

(३) कुनै प्रदेशमा नेपालको सार्वभौमसत्ता, भौगोलिक अखण्डता, राष्ट्रियता वा स्वाधीनतामा गम्भीर असर पर्ने किसिमको कार्य भएमा
राष्ट्रपतिले त्यस्तो प्रदेश मन्त्रिपरिषदलाई आवश्यकता अनुसार सचेत गराउन, प्रदेश मन्त्रिपरिषद र प्रदेश सभालाई बढीमा छ महीनासम्म निलम्बन गर्न वाविघटन गर्न सक्नेछ ।

(४) उपधारा (३) बमोजिम कुनै प्रदेश मन्त्रिपरिषद र प्रदेश सभा निलम्बन वा विघटन गरेकोमा त्यस्तो कार्य पैंतीस दिन भित्र संघीय संसदको
तत्काल कायम रहेको सम्पूर्ण सदस्य संख्याको बहुमतबाट अनुमोदन गराउनु पर्नेछ ।

(५) उपधारा (३) बमोजिम गरिएको विघटन सम्बन्धी कार्य संघीय संसदबाट अनुमोदन भएमा त्यस्तो प्रदेशमा छ महीनाभित्र प्रदेश सभाको
निर्वाचन हुनेछ ।
तर संघीय संसदबाट अनुमोदन नभएमा त्यस्तो निलम्बन वा विघटन सम्बन्धी कार्य स्वतः निष्क्रिय हुनेछ ।

(६) उपधारा (३) बमोजिम गरिएको निलम्बन उपधारा (४) बमोजिम अनुमोदन भएमा त्यस्तो निलम्बनको अवधिभर र उपधारा (५) बमोजिम प्रदेश सभाको निर्वाचन नभएसम्मका लागि त्यस्तो प्रदेशमा संघीय शासन कायम रहनेछ ।

(७) संघीय शासन कायम रहेको अवस्थामा संघीय संसदले अनुसूची–६ बमोजिमको सूचीमा परेको विषयमा कानून बनाउन सक्नेछ । त्यस्तो कानून सम्बन्धित प्रदेश सभाले अर्काे कानून बनाई खारेज नगरेसम्म बहाल रहनेछ ।

(८) नेपाल सरकारले आफै वा प्रदेश सरकार मार्फत गाउँ कार्यपालिका वा नगर कार्यपालिकालाई यो संविधान र संघीय कानून बमोजिम आवश्यक सहयोग गर्न र निर्देशन दिन सक्नेछ । त्यस्तो निर्देशनको पालन गर्नु गाउँ कार्यपालिका वा नगर कार्यपालिकाको कर्तव्य हुनेछ ।

\textbf{२३३. प्रदेश–प्रदेश बीचको सम्बन्धः} (१) एक प्रदेशले अर्को प्रदेशको कानूनी व्यवस्था वा न्यायिक एवं प्रशासकीय निर्णय वा आदेशको कार्यान्वयनमा सहयोग गर्नु पर्नेछ ।

(२) एक प्रदेशले अर्को प्रदेशसँग साझा चासो, सरोकार र हितको विषयमा सूचना आदान प्रदान गर्न, परामर्श गर्न, आफ्नो कार्य र विधायनका
बारेमा आपसमा समन्वय गर्न र आपसी सहयोग विस्तार गर्न सक्नेछ ।

(३) एक प्रदेशले अर्को प्रदेशको बासिन्दालाई आफ्नो प्रदेशको कानून बमोजिम समान सुरक्षा, व्यवहार र सुविधा उपलब्ध गराउनु पर्नेछ ।

\textbf{२३४. अन्तर प्रदेश परिषदः} (१) संघ र प्रदेश बीच तथा प्रदेश–प्रदेश बीच उत्पन्न राजनीतिक विवाद समाधान गर्न देहाय बमोजिमको एक अन्तर प्रदेश परिषद रहनेछः–
(क) प्रधानमन्त्री – अध्यक्ष
(ख) नेपाल सरकारका गृहमन्त्री – सदस्य
(ग) नेपाल सरकारका अर्थमन्त्री – सदस्य
(घ) सम्बन्धित प्रदेशका मुख्यमन्त्री – सदस्य

(२) अन्तर प्रदेश परिषदको बैठक आवश्यकता अनुसार बस्नेछ ।

(३) अन्तर प्रदेश परिषदले आफ्नो वैठकमा विवादको विषयसँग सम्बन्धित नेपाल सरकारको मन्त्री र सम्बन्धित प्रदेशको मन्त्री तथा
विशेषज्ञलाई आमन्त्रण गर्न सक्नेछ ।

(४) अन्तर प्रदेश परिषदको बैठक सम्बन्धी कार्यविधि सो परिषद आफैले निर्धारण गरे बमोजिम हुनेछ ।

\textbf{२३५. संघ, प्रदेश र स्थानीय तह बीचको समन्वयः} (१) संघ, प्रदेश र स्थानीय तह बीच समन्वय कायम गर्न संघीय संसदले आवश्यक कानून बनाउनेछ ।

(२) प्रदेश, गाउँपालिका वा नगरपालिका बीच समन्वय कायम गर्न र कुनै राजनीतिक विवाद उत्पन्न भएमा प्रदेश सभाले सम्बन्धित गाउँपालिका, नगरपालिका र जिल्ला समन्वय समितिसँग समन्वय गरी त्यस्तो विवादको समाधान गर्न सक्नेछ ।

(३) उपधारा (२) बमोजिम विवाद समाधान गर्ने प्रक्रिया र कार्यविधि प्रदेश कानून बमोजिम हुनेछ ।

\textbf{२३६. अन्तर प्रदेश व्यापारः} यस संविधानमा अन्यत्र जुनसुकै कुरा लेखिएको भए तापनि एक प्रदेश वा स्थानीय तहबाट अर्को प्रदेश वा स्थानीय तहको क्षेत्रमा हुने वस्तुको ढुवानी वा सेवाको विस्तार वा कुनै प्रदेश वा स्थानीय तहको क्षेत्रमा हुने वस्तुको ढुवानी वा सेवाको विस्तारमा कुनै किसिमको बाधा अवरोध गर्न वा कुनै कर, शुल्क, दस्तुर वा महसूल लगाउन वा त्यस्तो सेवा वा वस्तुको ढुवानी वा विस्तारमा कुनै किसिमको भेदभाव गर्न पाइने छैन ।

\textbf{२३७. सर्वाेच्च अदालतको संवैधानिक इजलासको अधिकार क्षेत्रमा असर नपर्नेः} यस भागमा लेखिएको कुनै कुराले धारा १३७ बमोजिमको सर्वाेच्च अदालतको संवैधानिक इजलासको अधिकार क्षेत्रमा कुनै असर पर्ने छैन ।