\section{भाग–७ संघीय कार्यपालिका}

(१) नेपालको कार्यकारिणी अधिकार यो संविधान र कानून बमोजिम मन्त्रिपरिषदमा निहित हुनेछ ।
(२) यो संविधान र कानूनको अधीनमा रही नेपालको शासन व्यवस्थाको सामान्य निर्देशन, नियन्त्रण र सञ्चालन गर्ने अभिभारा मन्त्रिपरिषदमा हुनेछ ।
(३) नेपालको संघीय कार्यकारिणी सम्बन्धी सम्पूर्ण काम नेपाल सरकारको नाममा हुनेछ ।
(४) उपधारा (३) बमोजिम नेपाल सरकारको नाममा हुने निर्णय वा आदेश र तत्सम्बन्धी अधिकारपत्रको प्रमाणीकरण संघीय कानून बमोजिम हुनेछ ।

\textbf{७६. मन्त्रिपरिषदको गठनः}

(१) राष्ट्रपतिले प्रतिनिधि सभामा बहुमत प्राप्त संसदीय दलको नेतालाई प्रधानमन्त्री नियुक्त गर्नेछ र निजको अध्यक्षतामा मन्त्रिपरिषदको गठन हुनेछ ।

(२) उपधारा (१) बमोजिम प्रतिनिधि सभामा कुनै पनि दलको स्पष्ट बहुमत नरहेको अवस्थामा प्रतिनिधि सभामा प्रतिनिधित्व गर्ने दुई वा दुई भन्दा बढी दलहरूको समर्थनमा बहुमत प्राप्त गर्न सक्ने प्रतिनिधि सभाको सदस्यलाई राष्ट्रपतिले प्रधानमन्त्री नियुक्त गर्नेछ ।

(३) प्रतिनिधि सभाको निर्वाचनको अन्तिम परिणाम घोषणा भएको मितिले तीस दिनभित्र उपधारा (२) बमोजिम प्रधानमन्त्री नियुक्ति हुन सक्ने अवस्था नभएमा वा त्यसरी नियुक्त प्रधानमन्त्रीले उपधारा  (४) बमोजिम विश्वासको मत प्राप्त गर्न नसकेमा राष्ट्रपतिले प्रतिनिधि सभामा सबैभन्दा बढी सदस्यहरू भएको दलको संसदीय दलको नेतालाई प्रधानमन्त्री नियुक्त गर्नेछ ।

(४) उपधारा (२) वा (३) बमोजिम नियुक्त प्रधानमन्त्रीले त्यसरी नियुक्त भएको मितिले तीस दिनभित्र प्रतिनिधि सभाबाट विश्वासको मत
प्राप्त गर्नु पर्नेछ ।द्धज्ञ

(५) उपधारा (३) बमोजिम नियुक्त प्रधानमन्त्रीले उपधारा (४) बमोजिम विश्वासको मत प्राप्त गर्न नसकेमा उपधारा (२) बमोजिमको कुनै सदस्यले प्रतिनिधि सभामा विश्वासको मत प्राप्त गर्न सक्ने आधार प्रस्तुत गरेमा राष्ट्रपतिले त्यस्तो सदस्यलाई प्रधानमन्त्री नियुक्त गर्नेछ ।

(६) उपधारा (५) बमोजिम नियुक्त प्रधानमन्त्रीले उपधारा (४) बमोजिम विश्वासको मत प्राप्त गर्नु पर्नेछ ।

(७) उपधारा (५) बमोजिम नियुक्त प्रधानमन्त्रीले विश्वासको मत प्राप्त गर्न नसकेमा वा प्रधानमन्त्री नियुक्त हुन नसकेमा प्रधानमन्त्रीको सिफारिसमा राष्ट्रपतिले प्रतिनिधि सभा विघटन गरी छ महीनाभित्र अर्को प्रतिनिधि सभाको निर्वाचन सम्पन्न हुने गरी निर्वाचनको मिति तोक्नेछ ।

(८) यस संविधान बमोजिम भएको प्रतिनिधि सभाको निर्वाचनको अन्तिम परिणाम घोषणा भएको वा प्रधानमन्त्रीको पद रिक्त भएको मितिले पैंतीस दिनभित्र यस धारा बमोजिम प्रधानमन्त्री नियुक्ति सम्बन्धी प्रक्रिया सम्पन्न गर्नु पर्नेछ ।

(९) राष्ट्रपतिले प्रधानमन्त्रीको सिफारिसमा संघीय संसदका सदस्यमध्येबाट समावेशी सिद्धान्त बमोजिम प्रधानमन्त्री सहित बढीमा पच्चीस
जना मन्त्री रहेको मन्त्रिपरिषद गठन गर्नेछ ।

स्पष्टीकरणः यस भागको प्रयोजनका लागि “मन्त्री” भन्नाले उपप्रधानमन्त्री, मन्त्री, राज्य मन्त्री र सहायक मन्त्री सम्झनु पर्छ ।

(१०) प्रधानमन्त्री र मन्त्री सामूहिक रूपमा संघीय संसदप्रति उत्तरदायी हुनेछन् र मन्त्री आफ्नो मन्त्रालयको कामका लागि व्यक्तिगत रूपमा प्रधानमन्त्री र संघीय संसदप्रति उत्तरदायी हुनेछन् ।

\textbf{७७. प्रधानमन्त्री तथा मन्त्रीको पद रिक्त हुने अवस्थाः} (१) देहायको कुनै अवस्थामा प्रधानमन्त्रीको पद रिक्त हुनेछः–

(क) निजले राष्ट्रपति समक्ष लिखित राजीनामा दिएमा,
(ख) धारा १०० बमोजिम विश्वासको प्रस्ताव पारित हुन नसकेमा वा निजको विरुद्ध अविश्वासको प्रस्ताव पारित भएमा,
(ग) निज प्रतिनिधि सभाको सदस्य नरहेमा,
(घ) निजको मृत्यु भएमा ।

(२) देहायको कुनै अवस्थामा मन्त्रीको पद रिक्त हुनेछ :

(क) निजले प्रधानमन्त्री समक्ष लिखित राजीनामा दिएमा,
(ख) प्रधानमन्त्रीले निजलाई पदमुक्त गरेमा,
(ग) उपधारा (१) को खण्ड (क), (ख) वा (ग) बमोजिम प्रधानमन्त्रीको पद रिक्त भएमा,
(घ) निजको मृत्यु भएमा ।

(३) उपधारा (१) बमोजिम प्रधानमन्त्रीको पद रिक्त भएमा अर्को मन्त्रिपरिषद गठन नभएसम्म सोही मन्त्रिपरिषदले कार्य सञ्चालन गर्नेछ ।

तर प्रधानमन्त्रीको मृत्यु भएमा अर्काे प्रधानमन्त्री नियुक्ति नभएसम्मका लागि वरिष्ठतम मन्त्रीले प्रधानमन्त्रीको रूपमा कार्य सञ्चालन गर्नेछ ।

\textbf{७८. संघीय संसदको सदस्य नभएको व्यक्ति मन्त्री हुनेः}

(१) धारा ७६ को उपधारा (९) मा जुनसुकै कुरा लेखिएको भए तापनि राष्ट्रपतिले प्रधानमन्त्रीको सिफारिसमा संघीय संसदको सदस्य नभएको कुनै व्यक्तिलाई मन्त्री पदमा नियुक्त गर्न सक्नेछ ।

(२) उपधारा (१) बमोजिम नियुक्त मन्त्रीले शपथग्रहण गरेको मितिले छ महीनाभित्र संघीय संसदको सदस्यता प्राप्त गर्नु पर्नेछ ।

(३) उपधारा (२) बमोजिमको अवधिभित्र संघीय संसदको सदस्यता प्राप्त गर्न नसकेमा तत्काल कायम रहेको प्रतिनिधि सभाको कार्यकालभर निज मन्त्री पदमा पुनः नियुक्तिका लागि योग्य हुने छैन ।

(४) उपधारा (१) मा जुनसुकै कुरा लेखिएको भए तापनि तत्काल कायम रहेको प्रतिनिधि सभाको निर्वाचनमा पराजित भएको व्यक्ति त्यस्तो प्रतिनिधि सभाको कार्यकालमा उपधारा (१) बमोजिम मन्त्री पदमा नियुक्तिका लागि योग्य हुने छैन ।

\textbf{७९. प्रधानमन्त्री र मन्त्रीको पारिश्रमिक तथा अन्य सुविधाः}

प्रधानमन्त्री र मन्त्रीको पारिश्रमिक र अन्य सुविधा संघीय ऐन बमोजिम हुनेछ । त्यस्तो ऐन नबनेसम्म नेपाल सरकारले तोके बमोजिम हुनेछ ।

\textbf{८०. शपथः} प्रधानमन्त्री, उपप्रधानमन्त्री र मन्त्रीले राष्ट्रपति समक्ष तथा राज्यमन्त्री र सहायक मन्त्रीले प्रधानमन्त्री समक्ष आफ्नोे कार्यभार सम्हाल्नु अघि संघीय कानून बमोजिम पद तथा गोपनीयताको शपथ लिनु पर्नेछ ।

\textbf{८१. राष्ट्रपतिलाई जानकारी दिनेः}

प्रधानमन्त्रीले देहायका विषयहरूमा राष्ट्रपतिलाई जानकारी गराउनेछ –
(क) मन्त्रिपरिषदका निर्णय,
(ख) संघीय संसदमा पेश गरिने विधेयक,
(ग) खण्ड (क) र (ख) मा उल्लिखित विषयमा राष्ट्रपतिले जानकारी मागेको अन्य आवश्यक विवरण, र
(घ) नेपालको समसामयिक परिस्थिति र वैदेशिक सम्बन्धका विषय ।

\textbf{८२. नेपाल सरकारको कार्य सञ्चालनः}

(१) नेपाल सरकारबाट स्वीकृत नियमावली बमोजिम नेपाल सरकारको कार्य विभाजन र कार्य सम्पादन हुनेछ ।

(२) उपधारा (१) अन्तर्गतको नियमावलीको पालन भयो वा भएन भन्ने प्रश्न कुनै अदालतमा उठाउन सकिने छैन ।