\section{भाग–१४ प्रदेश व्यवस्थापिका}

(ख) खण्ड (क) बमोजिम कायम हुने सदस्य संख्यालाई साठी प्रतिशत मानी बाँकी चालीस प्रतिशतमा समानुपातिक निर्वाचन प्रणालीबाट निर्वाचित हुने सदस्य ।

(२) उपधारा (१) को खण्ड (क) बमोजिमको सदस्य निर्वाचनका लागि भूगोल र जनसंख्याको आधारमा संघीय कानून बमोजिम निर्वाचन क्षेत्र कायम गरिनेछ ।

(३) प्रदेश सभाका साठी प्रतिशत सदस्यहरू पहिलो हुने निर्वाचित हुने निर्वाचन प्रणाली बमोजिम र चालीस प्रतिशत सदस्यहरू समानुपातिक
निर्वाचन प्रणाली बमोजिम निर्वाचित हुनेछन् ।

(४) उपधारा (३) बमोजिम हुने प्रदेश सभा सदस्यको निर्वाचन कानून बमोजिम वालिग मताधिकारको आधारमा गोप्य मतदानद्वारा हुनेछ ।

(५) अठार वर्ष उमेर पूरा भएको प्रदेशको क्षेत्रभित्र बसोबास गर्ने प्रत्येक नेपाली नागरिकलाई कानून बमोजिम कुनै एक निर्वाचन क्षेत्रमा
मतदान गर्ने अधिकार हुनेछ ।

(६) समानुपातिक निर्वाचन प्रणाली बमोजिम हुने प्रदेश सभाको निर्वाचनका लागि राजनीतिक दलले उम्मेदवारी दिंदा जनसंख्याको आधारमा महिला, दलित, आदिवासी जनजाति, खस आर्य, मधेसी, थारू, मुस्लिम, पिछडिएको क्षेत्र, अल्पसंख्यक समुदाय समेतबाट बन्द सूचीका आधारमा प्रतिनिधित्व हुने व्यवस्था संघीय कानून बमोजिम हुनेछ । त्यसरी उम्मेदवारी दिंदा सम्बन्धित प्रदेशको भौगोलिक सन्तुलनलाई समेत ध्यान दिनु पर्नेछ ।

स्पष्टीकरण :

यस भागको प्रयोजनका लागि “खस आर्य” भन्नाले क्षेत्री, ब्राम्हण, ठकुरी, संन्यासी (दशनामी) समुदाय सम्झनु पर्छ । (७) उपधारा (६) बमोजिम राजनीतिक दलले उम्मेदवारी दिंदा अपांगता भएका व्यक्तिको समेत प्रतिनिधित्व हुने व्यवस्था गर्नु पर्नेछ ।

(८) प्रदेश सभाको कार्यकाल छ महीना भन्दा बढी अवधि बाँकी छँदै कुनै सदस्यको स्थान रिक्त भएमा त्यस्तो स्थान जुन निर्वाचन प्रणालीबाट पूर्ति भएको थियो सोही प्रक्रियाद्वारा पूर्ति गरिनेछ ।

(९) यस धारामा अन्यत्र जुनसुकै कुरा लेखिएको भए तापनि प्रदेश सभामा प्रतिनिधित्व गर्ने प्रत्येक राजनीतिक दलबाट निर्वाचित कुल सदस्य संख्याको कम्तीमा एक तिहाइ सदस्य महिला हुनु पर्नेछ । त्यसरी निर्वाचित गर्दा उपधारा (१) को खण्ड (क) बमोजिम निर्वाचित सदस्यहरू मध्ये कुनै राजनीतिक दलको एक तिहाइ सदस्य महिला निर्वाचित नभएमा त्यस्तो राजनीतिक दलले सोही उपधाराको खण्ड (ख) बमोजिम सदस्य निर्वाचित गर्दा आफ्नो दलबाट प्रदेश सभामा निर्वाचित हुने कुल सदस्यको कम्तीमा एक तिहाइ महिला सदस्य हुने गरी निर्वाचित गर्नु पर्नेछ ।

(१०) प्रदेश सभाको सदस्यका लागि हुने निर्वाचनमा मतदान गर्ने अधिकार पाएको धारा १७८ बमोजिम योग्यता पुगेको कुनै पनि व्यक्ति
कानूनको अधीनमा रही प्रदेशको कुनै पनि निर्वाचन क्षेत्रबाट उम्मेदवार हुन पाउनेछ ।

तर एउटै व्यक्ति एक भन्दा बढी निर्वाचन क्षेत्रमा एकै पटक उम्मेदवार हुन पाउने छैन ।

(११) प्रदेश सभाको निर्वाचन सम्बन्धी अन्य व्यवस्था संघीय कानून बमोजिम हुनेछ ।

\textbf{१७७. प्रदेश सभाको कार्यकालः} (१) यस संविधान बमोजिम अगावै विघटन भएकोमा बाहेक प्रदेश सभाको कार्यकाल पाँच वर्षको हुनेछ ।

(२) उपधारा (१) मा जुनसुकै कुरा लेखिएको भए तापनि सम्बन्धित प्रदेशमा संकटकालीन अवस्थाको घोषणा वा आदेश लागू रहेको अवस्थामा प्रदेश ऐन बमोजिम प्रदेश सभाको कार्यकाल एक वर्षमा नबढ्ने गरी थप गर्न सकिनेछ ।

(३) उपधारा (२) बमोजिम थप गरिएको प्रदेश सभाको कार्यकाल सम्बन्धित प्रदेशमा संकटकालीन अवस्थाको घोषणा वा आदेश खारेज भएको मितिले छ महीना पुगेपछि स्वतः समाप्त हुनेछ ।

\textbf{१७८. प्रदेश सभाको सदस्यका लागि योग्यताः} (१) देहायको योग्यता भएको व्यक्ति प्रदेश सभाकोे सदस्य हुन योग्य हुनेछः–

(क) नेपाली नागरिक,
(ख) सम्बन्धित प्रदेशको मतदाता रहेको,
(ग) पच्चीस वर्ष उमेर पूरा भएकोे,
(घ) नैतिक पतन देखिने फौजदारी कसूरमा सजाय नपाएको,
(ङ) कुनै कानूनले अयोग्य नभएको, र
(च) कुनै लाभको पदमा बहाल नरहेको ।

स्पष्टीकरण :

यस खण्डको प्रयोजनका लागि “लाभको पद” भन्नाले निर्वाचन वा मनोनयनद्वारा पूर्ति गरिने राजनीतिक पद बाहेक सरकारी कोषबाट पारिश्रमिक वा आर्थिक सुविधा पाउने अन्य पद सम्झनु पर्छ ।

(२) निर्वाचन, मनोनयन वा नियुक्ति हुने राजनीतिक पदमा बहाल रहेकोे व्यक्ति यस भाग बमोजिम प्रदेश सभाको सदस्य पदमा निर्वाचित
भएमा त्यस्तो पदको शपथ ग्रहण गरेको दिनदेखि निजको त्यस्तो पद स्वतः रिक्त हुनेछ ।

\textbf{१७९. प्रदेश सभाका सदस्यको शपथः} प्रदेश सभाका सदस्यहरूले प्रदेश सभा वा त्यसको कुनै समितिको बैठकमा पहिलो पटक भाग लिनु अघि प्रदेश कानून बमोजिम शपथ लिनु पर्नेछ ।

\textbf{१८०. प्रदेश सभा सदस्यको स्थान रिक्त हुनेः} देहायको कुनै अवस्थामा प्रदेश सभाका सदस्यको स्थान रिक्त हुनेछः–

(क) निजले प्रदेश सभाको सभामुख समक्ष लिखित राजीनामा दिएमा,
(ख) निजको धारा १७८ बमोजिमको योग्यता नभएमा वा नरहेमा,
(ग) प्रदेश सभाको कार्यकाल समाप्त भएमा वा विघटन भएमा,
(घ) निज प्रदेश सभालाई सूचना नदिई लगातार दश वटा बैठकमा अनुपस्थित रहेमा,
(ङ) जुन दलको उम्मेदवार भई सदस्य निर्वाचित भएको हो त्यस्तो दलले संघीय कानून बमोजिम निजले दल त्याग गरेको कुरा सूचित गरेमा,
(च) निजको मृत्यु भएमा ।ढण्

१८१. प्रदेश सभा सदस्यको अयोग्यता सम्बन्धी निर्णयः प्रदेश सभाको कुनै सदस्य धारा १७८ बमोजिम अयोग्य छ वा हुन गएको छ भन्ने प्रश्न उठेमा त्यसको निर्णय सर्वाेच्च अदालतको संवैधानिक इजलासले गर्नेछ ।

१८२. प्रदेश सभाको सभामुख र उपसभामुखः (१) प्रदेश सभाको पहिलो बैठक प्रारम्भ भएको मितिले पन्ध्र दिनभित्र प्रदेश सभाका सदस्यहरूले आफूमध्येबाट प्रदेश सभामुख र प्रदेश उपसभामुखको निर्वाचन गर्नेछन् ।

(२) उपधारा (१) बमोजिम निर्वाचन गर्दा प्रदेश सभामुख वा प्रदेश उपसभामुख मध्ये एक जना महिला हुने गरी गर्नु पर्नेछ र प्रदेश सभामुख वा
प्रदेश उपसभामुख फरक फरक दलको प्रतिनिधि हुनु पर्नेछ ।

तर प्रदेश सभामा एक भन्दा बढी दलको प्रतिनिधित्व नभएको वा प्रतिनिधित्व भएर पनि उम्मेदवारी नदिएको अवस्थामा एकै दलको सदस्य
प्रदेश सभामुख वा प्रदेश उपसभामुख हुन बाधा पर्ने छैन ।

(३) प्रदेश सभामुख वा प्रदेश उपसभामुखको पद रिक्त भएमा प्रदेश सभाका सदस्यहरूले आफूमध्येबाट निर्वाचन गरी प्रदेश सभामुख वा प्रदेश उपसभामुखको पद पूर्ति गर्ने छन् ।

(४) प्रदेश सभामुखको अनुपस्थितिमा प्रदेश उपसभामुखले प्रदेश सभाको अध्यक्षता गर्नेछ ।
(५) प्रदेश सभामुख र प्रदेश उपसभामुखको निर्वाचन नभएको वा दुवै पद रिक्त भएको अवस्थामा प्रदेश सभाको बैठकको अध्यक्षता उपस्थित सदस्य मध्ये उमेरको हिसाबले ज्येष्ठ सदस्यले गर्नेछ ।

(६) देहायको कुनै अवस्थामा प्रदेश सभामुख वा प्रदेश उपसभामुखको पद रिक्त हुनेछः–

(क) निज प्रदेश सभाको सदस्य नरहेमा, तर प्रदेश सभा विघटन भएको अवस्थामा आफ्नो पदमा बहाल रहेका प्रदेश सभामुख र प्रदेश उपसभामुख प्रदेश सभाका लागि हुने अर्काे निर्वाचनको उम्मेदवारी दाखिल गर्ने अघिल्लो दिनसम्म आफ्नो पदमा बहाल रहनेछन् ।

(ख) निजले लिखित राजीनामा दिएमा,

(ग) निजले पद अनुकूल आचरण नगरेको भन्ने प्रस्ताव प्रदेश सभाको तत्काल कायम रहेको सम्पूर्ण सदस्य संख्याको
दुईतिहाइ बहुमतबाट पारित भएमा ।

(७) प्रदेश सभामुखले पद अनुकूलको आचरण नगरेको भन्ने प्रस्ताव उपर छलफल हुने बैठकको अध्यक्षता प्रदेश उपसभामुखले गर्नेछ । त्यस्तो प्रस्तावको छलफलमा प्रदेश सभामुखले भाग लिन र मत दिन पाउनेछ ।

\textbf{१८३. प्रदेश सभाको अधिवेशनको आव्हान र अन्त्यः} (१) प्रदेश प्रमुखले प्रदेश सभाका लागि भएको निर्वाचनको अन्तिम परिणाम घोषणा भएको मितिले बीस दिन भित्र प्रदेश सभाको अधिवेशन आव्हान गर्नेछ । त्यसपछि यस संविधान बमोजिम प्रदेश प्रमुखले समय समयमा अन्य अधिवेशन आव्हान गर्नेछ ।

तर एउटा अधिवेशनको समाप्ति र अर्को अधिवेशनको प्रारम्भका बीचको अवधि छ महीनाभन्दा बढी हुने छैन ।

(२) प्रदेश प्रमुखले प्रदेश सभाको अधिवेशनको अन्त्य गर्न सक्नेछ ।

(३) प्रदेश प्रमुखले प्रदेश सभाको अधिवेशन चालू नरहेको वा बैठक स्थगित भएको अवस्थामा अधिवेशन वा बैठक बोलाउन वाञ्छनीय छ भनी प्रदेश सभाको सम्पूर्ण सदस्य संख्याको एक चौथाइ सदस्यहरूले लिखित अनुरोध गरेमा त्यस्तो अधिवेशन वा बैठक बस्ने मिति र समय तोक्नेछ ।

त्यसरी तोकिएको मिति र समयमा प्रदेश सभाको अधिवेशन प्रारम्भ हुने वा बैठक बस्नेछ ।

\textbf{१८४. प्रदेश प्रमुखबाट सम्बोधनः} (१) प्रदेश प्रमुखले प्रदेश सभाको बैठकलाई सम्बोधन गर्न र त्यसका लागि सदस्यहरूको उपस्थितिको आह्वान गर्न सक्नेछ ।

(२) प्रदेश प्रमुखले प्रदेश सभाका लागि भएको निर्वाचन पछिको पहिलो अधिवेशन र प्रत्येक वर्षको पहिलो अधिवेशनको प्रारम्भ भएपछि प्रदेश सभाको बैठकलाई सम्बोधन गर्नेछ ।

\textbf{१८५. प्रदेश सभाको गणपूरक संख्याः} यस संविधानमा अन्यथा लेखिएकोमा बाहेक प्रदेश सभाको बैठकमा सम्पूर्ण सदस्य संख्याको एक चौथाइ सदस्य उपस्थित नभएसम्म कुनै प्रश्न वा प्रस्ताव निर्णयका लागि प्रस्तुत गरिने छैन ।

\textbf{१८६. प्रदेश सभामा मतदानः} प्रदेश सभामा निर्णयका लागि प्रस्तुत गरिएको जुनसुकै प्रस्तावको निर्णय उपस्थित भई मतदान गर्ने सदस्यहरूको बहुमतबाट हुनेछ । अध्यक्षता गर्ने व्यक्तिलाई मत दिने अधिकार हुने छैन ।

तर मत बराबर भएमा अध्यक्षता गर्ने व्यक्तिले आफ्नो निर्णायक मत दिनेछ ।

\textbf{१८७. प्रदेश सभाको विशेषाधिकार :}

(१) यस संविधानको अधीनमा रही प्रदेश सभामा पूर्ण वाक् स्वतन्त्रता रहनेछ र प्रदेश सभामा व्यक्त गरेको कुनै कुरा वा दिएको कुनै मतलाई लिएर कुनै पनि सदस्यलाई पक्राउ गर्ने, थुनामा राख्ने वा निज उपर कुनै अदालतमा कारबाही चलाइने छैन ।

(२) यस संविधानको अधीनमा रही प्रदेश सभालाई आफ्नो काम कारबाही र निर्णय गर्ने पूर्ण अधिकार रहनेछ र प्रदेश सभाको कुनै काम
कारबाही नियमित छ वा छैन भनी निर्णय गर्ने अधिकार प्रदेश सभालाई मात्र हुनेछ । यस सम्बन्धमा कुनै अदालतमा प्रश्न उठाइने छैन ।

(३) प्रदेश सभाको कुनै कारबाही उपर त्यसको असल नियतबारे शंका उठाई कुनै टीका–टिप्पणी गरिने छैन र कुनै सदस्यले बोलेको कुनै कुराको सम्बन्धमा जानी–जानी गलत वा भ्रामक अर्थ लगाई कुनै प्रकारको प्रकाशन वा प्रसारण गर्न पाइने छैन ।

(४) उपधारा (१) र (३) को व्यवस्था प्रदेश सभा सदस्य बाहेक प्रदेश सभाको बैठकमा भाग लिन पाउने अन्य व्यक्तिका हकमा पनि लागू हुनेछ ।

(५) प्रदेश सभाको अधिकार अन्तर्गत कुनै लिखत, प्रतिवेदन, मतदान वा कारबाही प्रकाशित गरेको विषयलाई लिएर कुनै व्यक्ति उपर अदालतमा कारबाही चल्ने छैन ।

स्पष्टीकरण :

यस उपधारा र उपधारा (१), (२), (३) र (४) को प्रयोजनका लागि “प्रदेश सभा” भन्नाले प्रदेश सभाको समितिको बैठक समेतलाई
जनाउँछ ।

(६) प्रदेश सभाको सदस्यलाई अधिवेशन बोलाइएको सूचना जारी भएपछि अधिवेशन अन्त्य नभएसम्मको अवधिभर पक्राउ गरिने छैन ।
तर कुनै फौजदारी अभियोगमा कुनै सदस्यलाई कानून बमोजिम पक्राउ गर्न यस उपधाराले बाधा पुर्‍याएको मानिने छैन । त्यसरी कुनै सदस्य पक्राउ गरिएमा पक्राउ गर्ने अधिकारीले त्यसको सूचना प्रदेश सभाको अध्यक्षता गर्ने व्यक्तिलाई तुरुन्त दिनु पर्नेछ ।

(७) विशेषाधिकारको हननलाई प्रदेश सभाको अवहेलना मानिनेछ र कुनै विशेषाधिकारको हनन भएको छ वा छैन भन्ने सम्बन्धमा निर्णय गर्ने अधिकार प्रदेश सभालाई मात्र हुनेछ ।

(८) कसैले प्रदेश सभाको अवहेलना गरेमा त्यस्तो सभाको अध्यक्षता गर्ने व्यक्तिले प्रदेश सभाको निर्णयबाट त्यस्तो व्यक्तिलाई सचेत गराउन, नसीहत दिन वा तीन महीनामा नबढ्ने गरी कैद गर्न वा दश हजारढघ रुपैयाँसम्म जरिबाना गर्न सक्नेछ र त्यस्तो जरिबाना सरकारी बाँकी सरह असूल उपर गरिनेछ ।

तर प्रदेश सभालाई सन्तोष हुने गरी त्यस्तो व्यक्तिले क्षमायाचना गरेमा त्यस्तो सभाले क्षमा प्रदान गर्न वा तोकिसकेको सजायलाई माफी वा
कम गर्न सक्नेछ ।

(९) प्रदेश सभाको विशेषाधिकार सम्बन्धी अन्य व्यवस्था प्रदेश कानून बमोजिम हुनेछ ।

\textbf{१८८. विश्वासको मत र अविश्वासको प्रस्ताव सम्बन्धी व्यवस्थाः} (१) मुख्यमन्त्रीलेकुनै पनि बखत आफूमाथि प्रदेश सभाको विश्वास छ भन्ने कुरा स्पष्ट गर्न आवश्यक वा उपयुक्त ठानेमा विश्वासको मतका लागि प्रदेश सभा समक्ष प्रस्ताव राख्न सक्नेछ ।

(२) मुख्यमन्त्रीले प्रतिनिधित्व गर्ने दल विभाजित भएमा वा प्रदेश सरकारमा सहभागी दलले आफ्नो समर्थन फिर्ता लिएमा तीस दिनभित्र
मुख्यमन्त्रीले विश्वासको मतका लागि प्रदेश सभा समक्ष प्रस्ताव राख्नु पर्नेछ ।

(३) उपधारा (१) र (२) बमोजिम पेश भएको प्रस्ताव प्रदेश सभामा तत्काल कायम रहेका सम्पूर्ण सदस्य संख्याको बहुमतले पारित हुन नसकेमा मुख्यमन्त्री आफ्नो पदबाट मुक्त हुनेछ ।

(४) प्रदेश सभामा तत्काल कायम रहेका सम्पूर्ण सदस्यहरू मध्ये एक चौथाइ सदस्यले मुख्यमन्त्रीमाथि प्रदेश सभाको विश्वास छैन भनी
लिखितरूपमा अविश्वासको प्रस्ताव पेश गर्न सक्ने छन् ।

तर मुख्यमन्त्री नियुक्त भएको पहिलो दुई वर्षसम्म र एकपटक राखेको अविश्वासको प्रस्ताव असफल भएको एक वर्ष भित्र त्यस्तो अविश्वासको प्रस्ताव पेश गर्न सकिने छैन ।

(५) उपधारा (४) बमोजिम अविश्वासको प्रस्ताव पेश गर्दा मुख्यमन्त्रीका लागि प्रस्तावित सदस्यको नाम समेत उल्लेख गरेको हुनु
पर्नेछ ।

(६) उपधारा (४) बमोजिम पेश भएको अविश्वासको प्रस्ताव प्रदेश सभामा तत्काल कायम रहेका सम्पूर्ण सदस्य संख्याको बहुमतबाट पारित
भएमा मुख्यमन्त्री पदमुक्त हुनेछ ।

(७) उपधारा (६) बमोजिम अविश्वासको प्रस्ताव पारित भई मुख्यमन्त्रीको पद रिक्त भएमा उपधारा (५) बमोजिम प्रस्ताव गरेको प्रदेश सभा सदस्यलाई प्रदेश प्रमुखले धारा १६८ बमोजिम मुख्यमन्त्री नियुक्त गर्नेछ ।

\textbf{१८९. मन्त्री, राज्यमन्त्री र सहायक मन्त्रीले प्रदेश सभाको बैठकमा भाग लिन पाउनेः}

मन्त्री, राज्यमन्त्री र सहायक मन्त्रीले प्रदेश सभा र त्यसको समितिको बैठकमा उपस्थित हुन र कारबाही तथा छलफलमा भाग लिन
पाउनेछ ।

तर प्रदेश सभाको सदस्य नभएको मन्त्री, राज्यमन्त्री वा सहायक मन्त्रीले प्रदेश सभाको बैठकमा वा त्यसको समितिमा मतदान गर्न र प्रदेश
सभाको सदस्य रहेको मन्त्री, राज्यमन्त्री वा सहायक मन्त्रीले आफू सदस्य रहेको समिति बाहेकको समितिको बैठकमा मतदान गर्न पाउने छैन ।

\textbf{१९०. प्रदेश सभामा अनधिकार उपस्थित भएमा वा मतदान गरेमा सजायः}

धारा १७९ बमोजिम शपथ नलिएको वा प्रदेश सभाको सदस्य नभएको कुनै व्यक्ति सदस्यको हैसियतले प्रदेश सभा वा त्यसको समितिको बैठकमा उपस्थित भएमा वा मतदान गरेमा निजलाई त्यस्तो बैठकको अध्यक्षता गर्ने व्यक्तिको आदेशले त्यसरी उपस्थित भएको वा मतदान गरेको प्रत्येक पटकका लागि पाँच हजार रुपैयाँ जरिबाना हुनेछ र त्यस्तो जरिबाना सरकारी बाँकी सरह असुल उपर गरिनेछ ।

\textbf{१९१. बहसमा बन्देजः }नेपालको कुनै अदालतमा विचाराधीन मुद्दाका सम्बन्धमा न्याय निरूपणमा प्रतिकूल असर पार्ने विषय तथा न्यायाधीशले कर्तव्य पालनको सिलसिलामा गरेको न्यायिक कार्यको सम्बन्धमा प्रदेश सभामा कुनै छलफल गरिने छैन ।

\textbf{१९२. सदस्यको स्थान रिक्त रहेको अवस्थामा प्रदेश सभाको कार्य सञ्चालनः} प्रदेशसभाको कुनै सदस्यको स्थान रिक्त रहेको अवस्थामा समेत प्रदेश सभाले आफ्नो कार्य सञ्चालन गर्न सक्नेछ र प्रदेश सभाको कारबाहीमा भाग लिन नपाउने कुनै व्यक्तिले भाग लिएको कुरा पछि पत्ता लाग्यो भने पनि  भइसकेको कार्य अमान्य हुने छैन ।

\textbf{१९३. प्रदेश सभाले समिति गठन गर्न सक्नेः} प्रदेश सभाको कार्य प्रणालीलाई व्यवस्थित गर्न प्रदेश सभाले नियमावली बमोजिम आवश्यकता अनुसार समिति वा विशेष समिति गठन गर्न सक्नेछ ।

\textbf{१९४. प्रदेश सभाको कार्य सञ्चालन विधिः} प्रदेश सभाले आफ्नो कार्य सञ्चालन गर्न, बैठकको सुव्यवस्था कायम राख्न र समितिहरूको गठन, काम, कारबाही र समिति सम्बन्धी अन्य विषय नियमित गर्नका लागि नियमावली बनाउनेछ । त्यसरी नियमावली नबनेसम्म प्रदेश सभाले आफ्नो कार्यविधि आफैं नियमित गर्नेछ ।

\textbf{१९५. प्रदेश सभाको सचिव र सचिवालयः} (१) प्रदेश प्रमुखले प्रदेश सभामुखको सिफारिसमा प्रदेश सभाको सचिव नियुक्त गर्नेछ ।

(२) प्रदेश सभाको काम कारबाही सञ्चालन तथा व्यवस्थापन गर्नका लागि एक सचिवालय रहनेछ । त्यस्तो सचिवालयको स्थापना र तत्सम्बन्धी अन्य व्यवस्था प्रदेश कानून बमोजिम हुनेछ ।

(३) प्रदेश सभाको सचिवको योग्यता, काम, कर्तव्य, अधिकार तथा सेवाका अन्य शर्त प्रदेश कानून बमोजिम हुनेछ ।

\textbf{१९६. पारिश्रमिकः} प्रदेश सभाको सभामुख, उपसभामुख तथा सदस्यको पारिश्रमिक र सुविधा प्रदेश कानून बमोजिम हुनेछ । त्यस्तो कानून नबनेसम्म प्रदेश सरकारले तोके बमोजिम हुनेछ ।