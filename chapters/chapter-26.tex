\section{भाग–२६ राष्ट्रिय प्राकृतिक स्रोत तथा वित्त आयोग}

तर चरम आर्थिक विश्रृंखलताका कारण संकटकाल घो षणा भएको  अवस्थामा यो व्यवस्था लागू हुने छैन ।
(८) राष्ट्रिय प्राकृतिक स्रोत तथा वित्त आयोगको अध्यक्ष वा सदस्य भइसकेको व्यक्ति अन्य सरकारी सेवामा नियुक्तिका लागि ग्राह्य हुने छैन ।
तर कुनै राजनीतिक पदमा वा कुनै विषयको अनुसन्धान, जाँचबुझ वा छानबीन गर्ने वा कुनै विषयको अध्ययन वा अन्वेषण गरी राय, मन्तव्य वा सिफारिस पेश गर्ने कुनै पदमा नियुक्त भई काम गर्न यस उपधारामा लेखिएको कुनै कुराले बाधा पुर्‍याएको मानिने छैन ।

\textbf{२५१. राष्ट्रिय प्राकृतिक स्रोत तथा वित्त आयोगको काम, कर्तव्य र अधिकारः} (१) राष्ट्रिय प्राकृतिक स्रोत तथा वित्त आयोगको काम, कर्तव्य र अधिकार देहाय बमोजिम हुनेछ : –
(क) संविधान र कानून बमोजिम संघीय सञ्चित कोषबाट संघ, प्रदेश र स्थानीय सरकारबीच राजस्वको बाँडफाँड गर्ने विस्तृत आधार र ढाँचा निर्धारण गर्ने,
(ख) संघीय सञ्चित कोषबाट प्रदेश र स्थानीय सरकारलाई प्रदान गरिने समानीकरण अनुदान सम्बन्धमा सिफारिस गर्ने,
(ग) राष्ट्रिय नीति तथा कार्यक्रम, मानक, पूर्वाधारको अवस्था अनुसार प्रदेश र स्थानीय सरकारलाई प्रदान गरिने सशर्त अनुदानको सम्बन्धमा अध्ययन अनुसन्धान गरी आधार तयार गर्ने,
(घ) प्रदेश सञ्चित कोषबाट प्रदेश र स्थानीय सरकारबीच राजस्वको बाँडफाँड गर्ने विस्तृत आधार र ढाँचा निर्धारण गर्ने,
(ङ) संघ, प्रदेश र स्थानीय सरकारको खर्च जिम्मेवारी पूरा गर्ने र राजस्व असुलीमा सुधार गर्नु पर्ने उपायहरूको सिफारिस गर्ने,
(च) समष्टिगत आर्थिक सूचकहरूको विश्लेषण गरी संघ, प्रदेश र स्थानीय सरकारले लिन सक्ने आन्तरिक ऋणको सीमा सिफारिस गर्ने,
(छ) संघ र प्रदेश सरकारको राजस्व बाँडफाँड आधारको पुनरावलोकन गरी परिमार्जनको सिफारिस गर्ने,
(ज) प्राकृतिक स्रोतको परिचालन गर्दा नेपाल सरकार, प्रदेश सरकार र स्थानीय तहको लगानी तथा प्रतिफलको हिस्सा निर्धारणको आधार तय गरी सिफारिस गर्ने,
(झ) प्राकृतिक स्रोतको बाँडफाँड सम्बन्धी विषयमा संघ र प्रदेश, प्रदेश र प्रदेश, प्रदेश र स्थानीय तह तथा स्थानीय तहहरू बीच उठ्न सक्ने संभावित विवादको विषयमा अध्ययन अनुसन्धान गरी त्यसको निवारण गर्न समन्वयात्मक रूपमा काम गर्न सुझाव दिने ।
(२) राष्ट्रिय प्राकृतिक स्रोत तथा वित्त आयोगले प्राकृतिक स्रोतको बाँडफाँड गर्दा सोसँग सम्बन्धित वातावरणीय प्रभाव मूल्यांकन सम्बन्धमा आवश्यक अध्ययन र अनुसन्धान गरी नेपाल सरकारलाई सिफारिस गर्नेछ ।
(३) राष्ट्रिय प्राकृतिक स्रोत तथा वित्त आयोगको अन्य काम, कर्तव्य र अधिकार, प्राकृतिक स्रोतको परिचालन गर्दा वा राजस्वको बाँडफाँड गर्दा अपनाउनु पर्ने विस्तृत आधार, आयोगका पदाधिकारीहरूको सेवाका शर्त लगायत अन्य व्यवस्था संघीय कानून बमोजिम हुनेछ ।