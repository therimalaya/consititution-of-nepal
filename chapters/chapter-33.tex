\section{भाग–३३ संक्रमणकालीन व्यवस्था}

(४) उपधारा (३) बमोजिमको आयोगको गठन यो संविधान प्रारम्भ भएको मितिले छ महीनाभित्र गरिनेछ । त्यस्तो आयोगको कार्यावधि एक वर्षको हुनेछ ।

\textbf{२९६. संविधान सभा व्यवस्थापिका–संसदमा रूपान्तरण हुने:} (१) यो संविधान प्रारम्भ हुँदाका बखत कायम रहेको संविधान सभा यो संविधान प्रारम्भ भएपछि व्यवस्थापिका–संसदमा स्वतः रूपान्तरण हुनेछ र त्यस्तो व्यवस्थापिका–संसदको कार्यकाल संवत् २०७४ साल माघ ७ गतेसम्म कायम रहनेछ ।

तर त्यस्तो कार्यकाल पूरा हुनु अगावै यस संविधान बमोजिमको प्रतिनिधि सभाको निर्वाचन हुने भएमा त्यस्तो निर्वाचनका लागि उम्मेदवारको मनोनयनपत्र दाखिला गर्ने अघिल्लो दिनसम्म व्यवस्थापिका–संसद कायम रहनेछ ।

(२) यो संविधान जारी हुँदाका बखत व्यवस्थापिका–संसदमा विचाराधीन रहेका विधेयकहरू उपधारा (१) बमोजिमको व्यवस्थापिका–संसदमा स्वतः सर्नेछन् ।

(३) यस संविधान बमोजिम संघीय संसदले सम्पादन गर्नु पर्ने काम यस संविधान बमोजिम प्रतिनिधि सभाको निर्वाचन नभएसम्म उपधारा (१) बमोजिमको व्यवस्थापिका–संसदले गर्नेछ ।
(४) यो संविधान प्रारम्भ भएपछि प्रदेश सभा गठन नभएसम्म अनुसूची–६ बमोजिमको विषयमा कानून बनाउने प्रदेश सभाको अधिकार उपधारा (१) बमोजिमको व्यवस्थापिका–संसदमा रहनेछ । त्यसरी बनेको कानून यो संविधान बमोजिमको प्रदेश सभा गठन भएको मितिले एक वर्षपछि त्यस्तो प्रदेशको हकमा निष्क्रिय हुनेछ ।

(५) यो संविधान प्रारम्भ हुँदाका बखतको व्यवस्थापिका–संसद सचिवालय, त्यसका महासचिव, सचिव र कर्मचारीहरू निजहरूको नियुक्ति हुँदाका बखतको सेवाका शर्तहरूको अधीनमा रही यस संविधान बमोजिमको संघीय संसद सचिवालयमा कायम हुनेछन् ।

(६) यो संविधान प्रारम्भ हुँदाका बखत व्यवस्थापिका–संसदको अधिवेशन चलिरहेको रहेनछ भने राष्ट्रपतिले यो संविधान प्रारम्भ भएको मितिले सात दिनभित्र व्यवस्थापिका–संसदको अधिवेशन आह्वान गर्नेछ । त्यस पछि राष्ट्रपतिले समय समयमा व्यवस्थापिका–संसदको अधिवेशन आह्वान गर्नेछ ।

\textbf{२९७. राष्ट्रपति र उपराष्ट्रपति सम्बन्धी व्यवस्था : } (१) यो संविधान प्रारम्भ हुँदाका बखत कायम रहेको राष्ट्रपति र उपराष्ट्रपति यस धारा बमोजिम अर्काे राष्ट्रपति र उपराष्ट्रपति निर्वाचित नभएसम्म आ–आफ्नो पदमा बहाल रहनेछन् ।

(२) यो संविधान प्रारम्भ हुँदाका बखत व्यवस्थापिका–संसदको अधिवेशन चलिरहेको रहेछ भने यो संविधान प्रारम्भ भएको मितिले र अधिवेशन चलिरहेको रहेनछ भने धारा २९६ को उपधारा (६) बमोजिम अधिवेशन आह्वान भएको मितिले एक महीनाभित्र राजनीतिक सहमतिको आधारमा राष्ट्रपति र उपराष्ट्रपतिको निर्वाचन धारा २९६ को उपधारा (१) बमोजिमको व्यवस्थापिका–संसदले गर्नेछ ।

(३) उपधारा (२) बमोजिम सहमति कायम हुन नसकेमा व्यवस्थापिका–संसदमा तत्काल कायम रहेका सम्पूर्ण सदस्यको बहुमतद्वारा राष्ट्रपति र उपराष्ट्रपतिको निर्वाचन सम्पन्न गर्नु पर्नेछ ।

(४) उपधारा (२) वा (३) बमोजिम निर्वाचित राष्ट्रपति वा उपराष्ट्रपतिकोे पद कुनै कारणले रिक्त भएमा संघीय संसदको गठन नभएसम्म यसै धारा बमोजिम व्यवस्थापिका–संसदबाट राष्ट्रपति वा उपराष्ट्रपतिकोे निर्वाचन गरिनेछ ।

(५) यस धारा बमोजिम निर्वाचित राष्ट्रपति वा उपराष्ट्रपतिको पदावधि धारा ६२ बमोजिमको निर्वाचक मण्डलबाट अर्काे राष्ट्रपति वा उपराष्ट्रपति निर्वाचित भई कार्यभार नसम्हालेसम्म कायम रहनेछ ।

(६) यस धारा बमोजिम निर्वाचित राष्ट्रपति वा उपराष्ट्रपतिको पद देहायको कुनै अवस्थामा रिक्त हुनेछः–

(क) राष्ट्रपतिले उपराष्ट्रपति समक्ष र उपराष्ट्रपतिले राष्ट्रपति समक्ष लिखित राजीनामा दिएमा,
(ख) निजको विरुद्ध उपधारा (७) बमोजिम महाभियोगको प्रस्ताव पारित भएमा,
(ग) धारा ६२ बमोजिमको निर्वाचक मण्डलबाट अर्काे राष्ट्रपति वा उपराष्ट्रपति निर्वाचित भई कार्यभार सम्हालेमा,
(घ) निजको मृत्यु भएमा ।

(७) यस धारा बमोजिम निर्वाचित राष्ट्रपति वा उपराष्ट्रपतिले यो संविधान र कानूनको गम्भीर उल्लंघन गरेको आरोपमा  धारा २९६ को उपधारा (१) बमोजिमको व्यवस्थापिका–संसदमा तत्काल कायम रहेका सम्पूर्ण सदस्य संख्याको कम्तीमा एक चौथाइ सदस्यले निजको विरुद्ध महाभियोगको प्रस्ताव पेश गर्न सक्नेछन । त्यस्तो प्रस्ताव व्यवस्थापिका–संसदमा तत्काल कायम रहेका सम्पूर्ण सदस्य संख्याको कम्तीमा दुई तिहाइ बहुमतबाट पारित भएमा निज पदमुक्त हुनेछ ।

\textbf{२९८. मन्त्रिपरिषदको गठन सम्बन्धी व्यवस्था :} (१) यो संविधान प्रारम्भ हुँदाका बखत कायम रहेको मन्त्रिपरिषद उपधारा (२) बमोजिमको मन्त्रिपरिषद गठन नभए सम्म कायम रहनेछ ।

(२) यो संविधान प्रारम्भ हुँदाका बखत व्यवस्थापिका–संसदको अधिवेशन चलिरहेको रहेछ भने यो संविधान प्रारम्भ भएको मितिले र अधिवेशन चलिरहेको रहेनछ भने धारा २९६ को उपधारा (६) बमोजिम आह्वान गरिएको व्यवस्थापिका–संसदको अधिवेशन प्रारम्भ भएको मितिले सात दिनभित्र राजनीतिक सहमतिका आधारमा प्रधानमन्त्रीको निर्वाचन सम्पन्न गरी निजको अध्यक्षतामा मन्त्रिपरिषदको गठन हुनेछ ।

(३) उपधारा (२) बमोजिम सहमति कायम हुन नसकेमा व्यवस्थापिका–संसदको तत्काल कायम रहेका सम्पूर्ण सदस्य संख्याको बहुमतको आधारमा प्रधानमन्त्री निर्वाचित हुनेछ ।

(४) यस धारा बमोजिम गठन हुने मन्त्रिपरिषदको संरचना र कार्य विभाजन आपसी सहमतिबाट तय गरिनेछ ।

(५) यस धारा बमोजिम गठन हुने मन्त्रिपरिषदमा आवश्यकता अनुसार उप–प्रधानमन्त्री र अन्य मन्त्रीहरू रहनेछन् ।

(६) यस धारा बमोजिम नियुक्त प्रधानमन्त्रीले उपधारा (५) बमोजिम मन्त्री नियुक्ति गर्दा सम्बन्धित दलको सिफारिसमा व्यवस्थापिका–संसदका सदस्यहरू मध्येबाट नियुक्ति गर्नु पर्नेछ ।

(७) यस धारा बमोजिम नियुक्त प्रधानमन्त्री र अन्य मन्त्रीहरू व्यवस्थापिका–संसदप्रति सामूहिक रूपमा उत्तरदायी हुनेछन् र मन्त्रीहरू आफ्ना मन्त्रालयको कामका लागि व्यक्तिगत रूपमा प्रधानमन्त्री र व्यवस्थापिका–संसदप्रति उत्तरदायीहुनेछन्।

(८) देहायको कुनै अवस्थामा यस धारा बमोजिम नियुक्त प्रधानमन्त्री आफ्नो पदबाट मुक्त हुनेछः–

(क) निजले राष्ट्रपति समक्ष लिखित राजीनामा दिएमा,
(ख) उपधारा (१४) बमोजिम निजको विरुद्ध अविश्वासको प्रस्ताव पारित भएमा वा विश्वासको प्रस्ताव पारित हुन नसकेमा,
(ग) निज व्यवस्थापिका–संसदको सदस्य नरहेमा,
(घ) निजको मृत्यु भएमा ।

(९) यस धारा बमोजिम नियुक्त उप–प्रधानमन्त्री, मन्त्री, राज्य मन्त्री तथा सहायक मन्त्री देहायको कुनै अवस्थामा आफ्नो पदबाट मुक्त हुनेछन्ः–
(क) निजले प्रधानमन्त्री समक्ष लिखित राजीनामा दिएमा,
(ख) उपधारा (८) बमोजिम प्रधानमन्त्री आफ्नो पदबाट मुक्त भएमा,
(ग) सम्बन्धित दलको सिफारिसमा वा सम्बन्धित दलसँगको सल्लाहमा प्रधानमन्त्रीले निजलाई पदमुक्त गरेमा,
(घ) निजको मृत्यु भएमा ।

(१०) उपधारा (८) बमोजिम प्रधानमन्त्री आफ्नो पदबाट मुक्त भए पनि अर्को मन्त्रिपरिषद गठन नभएसम्म सोही मन्त्रिपरिषदले कार्य सञ्चालन गरी रहनेछ ।

(११) यस धारा बमोजिम नियुक्त प्रधानमन्त्रीको मृत्यु भएमा अर्काे प्रधानमन्त्रीको चयन नभएसम्मका लागि उप–प्रधानमन्त्री वा वरिष्ठतम मन्त्रीले प्रधानमन्त्रीको रूपमा कार्य सञ्चालन गर्नेछ ।

(१२) यस धारा बमोजिम नियुक्त प्रधानमन्त्रीले कुनै पनि बखत आफूमाथि व्यवस्थापिका–संसदको विश्वास छ भन्ने कुरा स्पष्ट गर्न आवश्यक वा उपयुक्त ठानेमा विश्वासको मतका लागि व्यवस्थापिका–संसद समक्ष प्रस्ताव राख्न सक्नेछ ।

(१३) व्यवस्थापिका–संसदका सम्पूर्ण सदस्य संख्याको कम्तीमा एक चौथाइ सदस्यले यस धारा बमोजिम नियुक्त प्रधानमन्त्री उपर व्यवस्थापिका–संसदको विश्वास छैन भनी लिखित रूपमा अविश्वासको प्रस्ताव पेश गर्न सक्नेछन् ।
तर यस धारा बमोजिम नियुक्त एउटै प्रधानमन्त्री उपर छ महीनामा एक पटकभन्दा बढी अविश्वासको प्रस्ताव पेश गर्न सकिने छैन ।

(१४) उपधारा (१२) वा (१३) बमोजिमको प्रस्तावको निर्णय व्यवस्थापिका–संसदमा तत्काल कायम रहेका सम्पूर्ण सदस्य संख्याको बहुमतबाट हुनेछ ।

(१५) यो संविधान प्रारम्भ भएपछि यस संविधान बमोजिमको प्रदेश मन्त्रिपरिषदको गठन नभएसम्म प्रदेशको कार्यकारिणी अधिकार नेपाल सरकारले प्रयोग गर्नेछ ।

\textbf{२९९. सभामुख र उपसभामुख सम्बन्धी व्यवस्था :}(१) यो संविधान प्रारम्भ हुँदाका बखत कायम रहेका व्यवस्थापिका–संसदका सभामुख र उपसभामुख यस धारा बमोजिम अर्काे सभामुख र उपसभामुख निर्वाचित नभएसम्म आ–आफ्नो पदमा बहाल रहनेछन् ।

(२) यो संविधान प्रारम्भ हुँदाका बखत व्यवस्थापिका–संसदको अधिवेशन चलिरहेको रहेछ भने यो संविधान प्रारम्भ भएको मितिले र अधिवेशन चलिरहेको रहेनछ भने धारा २९६ को उपधारा (६) बमोजिम अधिवेशन आह्वान भएको मितिले बीस दिनभित्र व्यवस्थापिका–संसदका सदस्यहरूले आफूमध्येबाट राजनीतिक सहमतिको आधारमा एकजना सभामुख र एकजना उपसभामुखको निर्वाचन गर्नेछन् ।

(३) उपधारा (२) बमोजिमको सहमति कायम हुन नसकेमा व्यवस्थापिका–संसदमा तत्काल कायम रहेका सम्पूर्ण सदस्य संख्याको बहुमत प्राप्त गर्ने व्यवस्थापिका–संसदको सदस्य व्यवस्थापिका–संसदको सभामुख वा उपसभामुख पदमा निर्वाचित भएको मानिनेछ ।

(४) उपधारा (२) वा (३) बमोजिम निर्वाचन गर्दा सभामुख र उपसभामुख व्यवस्थापिका–संसदमा प्रतिनिधित्व गर्ने अलग–अलग राजनीतिक दलबाट प्रतिनिधित्व गर्ने सदस्य हुनु पर्नेछ ।

(५) सभामुख वा उपसभामुखले यस संविधान बमोजिम आफ्नो कार्य सम्पादन गर्दा कुनै पनि राजनीतिक दलको पक्ष वा विपक्षमा नरही तटस्थ व्यक्तिको हैसियतले गर्नेछ ।

(६) देहायको कुनै अवस्थामा सभामुख वा उपसभामुखको पद रिक्त हुनेछः–

(क) निजले लिखित राजीनामा दिएमा,
(ख) व्यवस्थापिका–संसदमा निजको सदस्यता नरहेमा,
(ग) निजले पद अनुकूलको आचरण गरेको छैन भन्ने प्रस्ताव व्यवस्थापिका संसदका सम्पूर्ण सदस्य संख्याको कम्तीमा दुई तिहाइ सदस्यको बहुमतबाट पारित भएमा,
(घ) निजको मृत्यु भएमा ।

(७) व्यवस्थापिका संसदको सभामुखले पद अनुकूलको आचरण गरेको छैन भन्ने प्रस्ताव उपर छलफल हुने बैठकको अध्यक्षता उपसभामुख वा अन्य कुनै सदस्यले गर्नेछ र त्यस्तो प्रस्तावको छलफलमा सभामुखले भाग लिन र मत दिन पाउनेछ ।

(८) सभामुख र उपसभामुखको निर्वाचन सम्बन्धी अन्य प्रक्रिया र सभामुख वा उपसभामुखले पद अनुकूलको आचरण नगरेको भन्ने प्रस्ताव पेश गर्ने र पारित गर्ने प्रक्रिया तत्काल प्रचलित व्यवस्थापिका–संसदको नियमावली बमोजिम हुनेछ ।

\textbf{३००. न्यायपालिका सम्बन्धी व्यवस्था :}(१) यो संविधान प्रारम्भ हुँदाका बखत कायम रहेका सर्वोच्च अदालत, संविधान सभा अदालत, पुनरावेदन अदालत र जिल्ला अदालतहरू यस संविधान बमोजिमको न्यायपालिकाको संरचना तयार नभएसम्म कायम रहनेछन् । यो संविधान प्रारम्भ हुनु अघि त्यस्ता अदालतमा दायर भएका मुद्दाहरू र यो संविधान प्रारम्भ भएपछि दायर हुने मुद्दाहरू तत् तत् अदालतबाट निरूपण गर्न यस संविधानले बाधा पुर्‍याएको मानिने छैन ।

(२) यो संविधान प्रारम्भ हुँदाका बखत सर्वाेच्च अदालत, पुनरावेदन अदालत र जिल्ला अदालतमा बहाल रहेका सर्वाेच्च अदालतका प्रधान न्यायाधीश वा न्यायाधीश, पुनरावेदन अदालतका मुख्य न्यायाधीश, न्यायाधीश र जिल्ला अदालतका न्यायाधीश यसै संविधान बमोजिम नियुक्त भएको मानिनेछ ।

(३) यो संविधान प्रारम्भ भएको मितिले एक वर्षभित्र संघीय कानून बमोजिम धारा १३९ बमोजिमका उच्च अदालतको स्थापना गरिनेछ । त्यस्तो अदालतको स्थापना भएपछि यो संविधान प्रारम्भ हुँदाका बखत कायम रहेको पुनरावेदन अदालत विघटन हुनेछ ।

(४) उपधारा (३) बमोजिम उच्च अदालत स्थापना भएपछि पुनरावेदन अदालतमा विचाराधीन रहेका मुद्दाहरू नेपाल सरकारले न्यायपरिषदको परामर्शमा नेपाल राजपत्रमा सूचना प्रकाशन गरी तोकेको उच्च अदालतमा सर्नेछन् ।

(५) उपधारा (३) बमोजिम उच्च अदालत स्थापना भएपछि न्यायपरिषदको सिफारिसमा प्रधान न्यायाधीशले यो संविधान प्रारम्भ हुँदाका बखत बहाल रहेका पुनरावेदन अदालतका मुख्य न्यायाधीश र न्यायाधीशहरूलाई उच्च अदालतको मुख्य न्यायाधीश र न्यायाधीशमा पदस्थापन गर्नेछ ।

(६) यो संविधान प्रारम्भ हुँदाका बखत बहाल रहेका पुनरावेदन अदालतका अतिरिक्त न्यायाधीशहरू नियुक्ति हुँदा तोकिएको अवधिसम्म बहाल रहन सक्नेछन् ।

(७) यो संविधान प्रारम्भ हुँदाका बखत अदालत बाहेक अन्य निकायमा विचाराधीन रहेका एक वर्षभन्दा बढी कैद सजाय हुने फौजदारी कसूर सम्बन्धी मुद्दा यो संविधान प्रारम्भ भएपछि सम्बन्धित जिल्ला अदालतमा सर्नेछन् ।

\textbf{३०१. संवैधानिक निकाय र पदाधिकारी सम्बन्धी व्यवस्था :} (१) यो संविधान प्रारम्भ हुँदाका बखत कायम रहेका र यस संविधानमा व्यवस्था भएका संवैधानिक निकाय यसै संविधान बमोजिम गठन भएको मानिनेछ र त्यस्ता निकायमा विचाराधीन रहेका विषयलाई यो संविधानको अधीनमा रही फर्छैट गर्न बाधा पुर्‍याएको मानिने छैन ।

(२) यो संविधान प्रारम्भ हुँदाका बखत बहाल रहेका संवैधानिक निकायका प्रमुख वा पदाधिकारी यसै संविधान बमोजिम नियुक्त भएको मानिनेछ र निज नियुक्त हुँदाका बखतको सेवाको शर्तको अधीनमा रही आफ्नो पदमा बहाल रहनेछ ।

(३) यो संविधान प्रारम्भ हुँदाका बखत अख्तियार दुरुपयोग अनुसन्धान आयोग र लोक सेवा आयोगमा यस संविधानमा उल्लेख भएको संख्या भन्दा बढी संख्यामा कार्यरत पदाधिकारी निज नियुक्त हुँदाका बखतको सेवाको शर्तको अधीनमा रही आफ्नो पदमा बहाल रहनेछ ।

\textbf{३०२. प्रदेश र स्थानीय तहमा सरकारी सेवाहरूको गठन र सञ्चालन :} (१) प्रदेश र स्थानीय तहमा आवश्यक सेवा प्रवाह गर्न नेपाल सरकारले आवश्यक व्यवस्था गर्नेछ ।

(२) उपधारा (१) बमोजिमको व्यवस्था गर्दा यो संविधान प्रारम्भ हुँदाका बखत सरकारी सेवामा कार्यरत राष्ट्रसेवक कर्मचारीलाई नेपाल सरकारले कानून बमोजिम संघ, प्रदेश र स्थानीय तहमा समायोजन गरी सेवा प्रवाहको व्यवस्था मिलाउन सक्नेछ ।

\textbf{३०३. स्थानीय निकाय सम्बन्धी व्यवस्था  :} (१) यो संविधान प्रारम्भ हुँदाका बखत कायम रहेका स्थानीय निकायहरू यो संविधान बमोजिम स्थानीय तहको संख्या र क्षेत्र निर्धारण नभएसम्म कायम रहनेछन् ।

(२) उपधारा (१) बमोजिम कायम रहेका स्थानीय निकायका पदाधिकारीको निर्वाचन कानून बमोजिम हुनेछ ।

(३) उपधारा (२) बमोजिम निर्वाचित स्थानीय निकायका पदाधिकारीहरू यो संविधान बमोजिम स्थानीय तहको निर्वाचन नभएसम्म कायम रहनेछन् ।

\textbf{३०४. वर्तमान कानून लागू रहने  :} (१) यो संविधान प्रारम्भ हुँदाका बखत कायम रहेका नेपाल कानून खारेज वा संशोधन नभएसम्म लागू रहनेछन् ।

तर यो संविधानसँग बाझिएको कानून यो संविधान बमोजिमको संघीय संसदको पहिलो अधिवेशन बसेको मितिले एक वर्षपछि बाझिएको हदसम्म स्वतः अमान्य हुनेछ ।

(२) नेपालको अन्तरिम संविधान, २०६३ बमोजिमका शान्ति प्रक्रिया सम्बन्धी कार्यहरू यसै संविधान बमोजिम भए गरेको मानिनेछ ।

\textbf{३०५. बाधा अड्काउ फुकाउने अधिकार :} यो संविधान बमोजिम संघीय संसदको निर्वाचन भई त्यसको पहिलो अधिवेशन प्रारम्भ नभएसम्म यो संविधानको कार्यान्वयन गर्न कुनै बाधा अड्काउ परेमा राष्ट्रपतिले नेपाल सरकार, मन्त्रिपरिषदको सिफारिसमा त्यस्तो बाधा अड्काउ फुकाउन आवश्यक आदेश जारी गर्न सक्नेछ । त्यस्तो आदेश जारी भए पछि तत्काल बसेको व्यवस्थापिका–संसद वा संघीय संसद समक्ष अनुमोदनका लागि पेश गर्नु पर्नेछ ।

 