\section{भाग–२१ अख्तियार दुरुपयोग अनुसन्धान आयोग}

(४) उपधारा (३) मा जुनसुकै कुरा लेखिएको भए तापनि देहायको कुनै अवस्थामा प्रमुख आयुक्त वा आयुक्तको पद रिक्त हुनेछः–
(क) निजले राष्ट्रपति समक्ष लिखित राजीनामा दिएमा,
(ख) निजको उमेर पैंसठ्ठी वर्ष पूरा भएमा,
(ग) निजको विरुद्ध धारा १०१ बमोजिम महाभियोगको प्रस्ताव पारित भएमा,
(घ) शारीरिक वा मानसिक अस्वस्थताको कारण सेवामा रही कार्य सम्पादन गर्न असमर्थ रहेको भनी संवैधानिक परिषदको सिफारिसमा राष्ट्रपतिले पदमुक्त गरेमा,
(ङ) निजको मृत्यु भएमा ।

(५) उपधारा (२) बमोजिम नियुक्त प्रमुख आयुक्त तथा आयुक्तको पुनः नियुक्ति हुन सक्ने छैन ।
तर आयुक्तलाई प्रमुख आयुक्तको पदमा नियुक्ति गर्न सकिनेछ र त्यस्तो आयुक्त प्रमुख आयुक्तको पदमा नियुक्त भएमा निजको पदावधि गणना गर्दा आयुक्त भएको अवधिलाई समेत जोडी गणना गरिनेछ ।

(६) देहायको योग्यता भएको व्यक्ति अख्तियार दुरुपयोग अनुसन्धान आयोगको प्रमुख आयुक्त वा आयुक्तको पदमा नियुक्तिका लागि योग्य हुनेछः–
(क) मान्यताप्राप्त विश्वविद्यालयबाट स्नातक उपाधि प्राप्त गरेको,
(ख) नियुक्ति हुँदाका बखत कुनै राजनीतिक दलको सदस्य नरहेको,
(ग) लेखा, राजस्व, इन्जिनियरिङ, कानून, विकास वा अनुसन्धानको क्षेत्रमा कम्तीमा बीस वर्ष काम गरी अनुभव र ख्याति प्राप्त
गरेको,
(घ) पैंतालिस वर्ष उमेर पूरा भएको, र
(ङ) उच्च नैतिक चरित्र भएको ।

(७) प्रमुख आयुक्त र आयुक्तको पारिश्रमिक र सेवाका शर्त संघीय कानून बमोजिम हुनेछ । प्रमुख आयुक्त र आयुक्त आफ्नो पदमा बहाल
रहेसम्म निजलाई मर्का पर्ने गरी पारिश्रमिक र सेवाका शर्त परिवर्तन गरिने छैन ।
तर चरम आर्थिक विश्रृंखलताका कारण संकटकाल घोषणा भएको अवस्थामा यो व्यवस्था लागू हुने छैन ।

(८) प्रमुख आयुक्त वा आयुक्त भइसकेको व्यक्ति अन्य सरकारी सेवामा नियुक्तिका लागि ग्राह्य हुने छैन ।
तर कुनै राजनीतिक पदमा वा कुनै विषयको अनुसन्धान, जाँचबुझ वा छानबीन गर्ने वा कुनै विषयको अध्ययन वा अन्वेषण गरी राय, मन्तव्य वा सिफारिस पेश गर्ने कुनै पदमा नियुक्त भई काम गर्न यस उपधारामा लेखिएको कुनै कुराले बाधा पुर्‍याएको मानिने छैन ।

\textbf{२३९. अख्तियार दुरुपयोग अनुसन्धान आयोगको काम, कर्तव्य र अधिकारः} (१) कुनै सार्वजनिक पद धारण गरेको व्यक्तिले भ्रष्टाचार गरी अख्तियारको दुरुपयोग गरेको सम्बन्धमा अख्तियार दुरुपयोग अनुसन्धान आयोगले कानून बमोजिम अनुसन्धान गर्न वा गराउन सक्नेछ ।
तर यस संविधानमा छुट्टै व्यवस्था भएको पदाधिकारी र अन्य कानूनले छुट्टै विशेष व्यवस्था गरेको पदाधिकारीको हकमा यो उपधारा लागू हुने छैन ।

(२) धारा १०१ बमोजिम महाभियोग प्रस्ताव पारित भई पदमुक्त हुने व्यक्ति, न्याय परिषदबाट पदमुक्त हुने न्यायाधीश र सैनिक ऐन बमोजिम कारबाही हुने व्यक्तिका हकमा निज पदमुक्त भइसकेपछि संघीय कानून बमोजिम अनुसन्धान गर्न वा गराउन सकिनेछ ।

(३) उपधारा (१) वा (२) बमोजिम भएको अनुसन्धानबाट सार्वजनिक पद धारण गरेको कुनै व्यक्तिले कानून बमोजिम भ्रष्टाचार मानिने कुनै काम गरेको देखिएमा अख्तियार दुरुपयोग अनुसन्धान आयोगले त्यस्तो व्यक्ति र सो अपराधमा संलग्न अन्य व्यक्ति उपर कानून बमोजिम अधिकार प्राप्त अदालतमा मुद्दा दायर गर्न वा गराउन सक्नेछ ।ज्ञज्ञट

(४) उपधारा (१) वा (२) बमोजिम भएको अनुसन्धानबाट सार्वजनिक पद धारण गरेको व्यक्तिको काम कारबाही अन्य अधिकारी वा निकायको अधिकारक्षेत्र अन्तर्गत पर्ने प्रकृतिको देखिएमा अख्तियार दुरुपयोग अनुसन्धान आयोगले आवश्यक कारबाहीका लागि सम्बन्धित अधिकारी वा निकाय समक्ष लेखी पठाउन सक्नेछ ।

(५) अख्तियार दुरुपयोग अनुसन्धान आयोगले अनुसन्धान गर्ने वा मुद्दा चलाउने आफ्नो काम, कर्तव्य र अधिकार मध्ये कुनै काम, कर्तव्य र अधिकार प्रमुख आयुक्त, कुनै आयुक्त वा नेपाल सरकारको अधिकृत कर्मचारीलाई तोकिएको शर्तको अधीनमा रही प्रयोग तथा पालन गर्ने गरी प्रत्यायोजन गर्न सक्नेछ ।

(६) अख्तियार दुरुपयोग अनुसन्धान आयोगको अन्य काम, कर्तव्य र अधिकार तथा कार्यविधि संघीय कानून बमोजिम हुनेछ ।