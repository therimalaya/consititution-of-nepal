\section{भाग–२७ अन्य आयोगहरू}

(४) उपधारा (३) मा जुनसुकै कुरा लेखिएको भए तापनि देहायको कुनै अवस्थामा राष्ट्रिय महिला आयोगका अध्यक्ष वा सदस्यको पद रिक्त हुनेछ :–
(क) निजले राष्ट्रपति समक्ष लिखित राजीनामा दिएमा,
(ख) निजको उमेर पैंसठ्ठी वर्ष पूरा भएमा,
(ग) निजको विरुद्ध धारा १०१ बमोजिम महाभियोग प्रस्ताव पारित भएमा,
(घ) शारीरिक वा मानसिक अस्वस्थताको कारण सेवामा रही कार्य सम्पादन गर्न असमर्थ रहेको भनी संवैधानिक परिषदको  सिफारिसमा राष्ट्रपतिले पदमुक्त गरेमा,  गरेमा,

(ङ) निजको मृत्यु भएमा ।

(५) उपधारा (२) बमोजिम नियुक्त अध्यक्ष तथा सदस्यको पुनः नियुक्ति हुन सक्ने छैन ।
तर सदस्यलाई अध्यक्षको पदमा नियुक्ति गर्न सकिनेछ र त्यस्तो सदस्य अध्यक्षको पदमा नियुक्ति भएमा निजको पदावधि गणना गर्दा सदस्य भएको अवधिलाई समेत जोडी गणना गरिनेछ ।

(६) देहायको योग्यता भएको व्यक्ति राष्ट्रिय महिला आयोगको अध्यक्ष र सदस्यको पदमा नियुक्तिका लागि योग्य हुनेछ :–

(क) कम्तीमा दश वर्ष महिलाको हक, हित वा लैंगिक न्याय वा महिला विकास वा मानव अधिकार र कानूनको क्षेत्रमा महत्वपूर्ण योगदान पुर्‍याएको महिला,
(ख) अध्यक्षको हकमा मान्यताप्राप्त विश्वविद्यालयबाट स्नातक उपाधि हासिल गरेको,
(ग) पैंतालिस वर्ष उमेर पूरा गरेको,
(घ) नियुक्ति हुँदाको बखत कुनै राजनीतिक दलको सदस्य नरहेको, र
(ङ) उच्च नैतिक चरित्र भएको ।

(७) राष्ट्रिय महिला आयोगको अध्यक्ष र सदस्यको पारिश्रमिक र सेवाका शर्त संघीय कानून बमोजिम हुनेछ र निजहरू बहाल रहेसम्म निजलाई मर्का पर्ने गरी पारिश्रमिक र सेवाका शर्त परिवर्तन गरिने छैन ।

तर चरम आर्थिक विश्रृंखलताका कारण संकटकाल घोषणा भएको खलताका कारण भएको अवस्थामा यो व्यवस्था लागू हुने छैन ।

(८) राष्ट्रिय महिला आयोगको अध्यक्ष वा सदस्य भइसकेको व्यक्ति अन्य सरकारी सेवामा नियुक्तिका लागि ग्राह्य हुने छैन।
तर कुनै राजनीतिक पदमा वा कुनै विषयको अनुसन्धान, जाँचबुझ वा छानबीन गर्ने वा कुनै विषयको अध्ययन वा अन्वेषण गरी राय, मन्तव्य वा सिफारिस पेश गर्ने कुनै पदमा नियुक्त भई काम गर्न यस उपधारामा लेखिएको कुनै कुराले बाधा पुर्‍याएको मानिने छैन ।

\textbf{२५३. राष्ट्रिय महिला आयोगको काम, कर्तव्य र अधिकार :} (१) राष्ट्रिय महिला आयोगको काम, कर्तव्य र अधिकार देहाय बमोजिम हुनेछ :–

(क) महिलाको हक हितसँग सरोकार राख्ने नीति तथा कार्यक्रमको तर्जुमा गरी कार्यान्वयनका लागि नेपाल सरकार समक्ष पेश गर्ने,
(ख) महिलाको हक हितसँग सम्बन्धित कानून वा नेपाल पक्ष भएको अन्तर्राष्ट्रिय सन्धि वा सम्झौता अन्तर्गतको दायित्व कार्यान्वयन भए वा नभएको विषयमा अनुगमन गरी त्यसको प्रभावकारी पालन वा कार्यान्वयनको उपाय सहित नेपाल सरकारलाई सुझाव दिने,
(ग) महिलालाई राष्ट्रिय विकासको मूल प्रवाहमा समाहित गर्न तथा राज्यका सबै निकायमा समानुपातिक सहभागिता सुनिश्चित गर्न मौजूदा नीति तथा कार्यक्रमको समीक्षा, अनुगमन तथा मूल्यांकन गर्ने र त्यसको प्रभावकारी कार्यान्वयनका लागि नेपाल सरकारलाई सिफारिस गर्ने,
(घ) लैंगिक समानता, महिला सशक्तीकरण तथा महिलासँग सम्बन्धित कानूनी व्यवस्थाको अध्ययन, अनुसन्धान गरी त्यस्ता कानूनमा गर्नुपर्ने सुधारका सम्बन्धमा सम्बन्धित निकायलाई सिफारिस गर्ने र सोको अनुगमन गर्ने,
(ङ) महिला अधिकारसँग सम्बन्धित नेपाल पक्ष भएको अन्तर्राष्ट्रिय सन्धि वा सम्झौतामा भएको व्यवस्था बमोजिम नेपालले पठाउनु पर्ने प्रतिवेदन तयारीका सम्बन्धमा नेपाल सरकारलाई सुझाव दिने,
(च) महिला हिंसा वा सामाजिक कुरीतिबाट पीडित भएको वा महिला अधिकार प्रयोग गर्न नदिएको वा वञ्चित गरेको विषयमा कुनै व्यक्ति वा संस्था विरुद्ध मुद्दा दायर गर्नुपर्ने आवश्यकता देखिएमा कानून बमोजिम अदालतमा मुद्दा दायर गर्न सम्बन्धित निकाय समक्ष सिफारिस गर्ने ।

(२) राष्ट्रिय महिला आयोगले आफ्नो काम, कर्तव्य र अधिकार मध्ये कुनै काम, कर्तव्य र अधिकार आयोगको अध्यक्ष, कुनै सदस्य वा नेपाल सरकारको कुनै कर्मचारीलाई तोकिएको शर्तको अधीनमा रही प्रयोग तथा पालन गर्ने गरी प्रत्यायोजन गर्न सक्नेछ ।

(३) राष्ट्रिय महिला आयोगको अन्य काम, कर्तव्य, अधिकार तथा तत्सम्बन्धी अन्य व्यवस्था संघीय कानून बमोजिम हुनेछ।

\textbf{२५४. प्रदेशमा कार्यालय स्थापना गर्न सक्ने :} राष्ट्रिय महिला आयोगले आवश्यकता अनुसार प्रदेशमा आफ्नो कार्यालय स्थापना गर्न सक्नेछ ।

\textbf{२५५. राष्ट्रिय दलित आयोग  :} (१) नेपालमा एक राष्ट्रिय दलित आयोग रहनेछ जसमा अध्यक्ष र अन्य चार जना सदस्य रहनेछन् ।

(२) राष्ट्रपतिले संवैधानिक परिषदको सिफारिसमा राष्ट्रिय दलित आयोगका अध्यक्ष र सदस्यको नियुक्ति गर्नेछ ।

(३) राष्ट्रिय दलित आयोगका अध्यक्ष तथा सदस्यको पदावधि नियुक्ति भएको मितिले छ वर्षको हुनेछ ।

(४) उपधारा (३) मा जुनसुकै कुरा लेखिएको भए तापनि देहायको कुनै अवस्थामा राष्ट्रिय दलित आयोगका अध्यक्ष वा सदस्यको पद रिक्त हुनेछ :–

(क) निजले राष्ट्रपति समक्ष लिखित राजीनामा दिएमा,
(ख) निजको उमेर पैंसठ्ठी वर्ष पूरा भएमा,
(ग) निजको विरुद्ध धारा १०१ बमोजिम महाभियोगको प्रस्ताव पारित भएमा,
(घ) शारीरिक वा मानसिअस्वस्थताको कारण सेवामा रही कार्य  गर्न असमर्थ  रहेको भनी संवैधानिक परिषदको  संवैधानिक परिषदको सिफारिसमा राष्ट्रपतिले पदमुक्त गरेमा,

(ङ) निजको मृत्यु भएमा ।

(५) उपधारा (२) बमोजिम नियुक्त अध्यक्ष तथा सदस्यको पुनः नियुक्ति हुन सक्ने छैन ।
तर सदस्यलाई अध्यक्षको पदमा नियुक्ति गर्न सकिनेछ र त्यस्तो सदस्य अध्यक्षको पदमा नियुक्ति भएमा निजको पदावधि गणना गर्दा सदस्य भएको अवधिलाई समेत जोडी गणना गरिनेछ ।

(६) देहायको योग्यता भएको व्यक्ति राष्ट्रिय दलित आयोगको अध्यक्ष वा सदस्यको पदमा नियुक्तिका लागि योग्य हुनेछ :–

(क) कम्तीमा दश वर्ष दलित समुदायको हक हित वा मानव अधिकार र कानूनको क्षेत्रमा महत्वपूर्ण योगदान पु¥याएको दलित,
(ख) अध्यक्षको हकमा मान्यताप्राप्त विश्वविद्यालयबाट कम्तीमा स्नातक उपाधि प्राप्त गरेको,
(ग) पैँतालिस वर्ष उमेर पूरा भएको,
(घ) नियुक्ति हुँदाको बखत कुनै राजनीतिक दलको सदस्य नरहेको, र
(ङ) उच्च नैतिक चरित्र भएको ।

(७) राष्ट्रिय दलित आयोगको अध्यक्ष र सदस्यको पारिश्रमिक र सेवाका शर्त संघीय कानून बमोजिम हुनेछ र निजलाई मर्का पर्ने गरी पारिश्रमिक र सेवाका शर्त परिवर्तन गरिने छैन ।

तर चरम आर्थिक विश्रृंखलताका कारण संकटकाल घोषणा भएको यो व्यवस्था लागू हुने छैनअवस्थामा यो व्यवस्था लागू हुने छैन

(८) राष्ट्रिय दलित आयोगको अध्यक्ष वा सदस्य भइसकेको व्यक्ति अन्य सरकारी सेवामा नियुक्तिका लागि ग्राह्य हुने छैन ।
तर कुनै राजनीतिक पदमा वा कुनै विषयको अनुसन्धान, जाँचबुझ वा छानबीन गर्ने वा कुनै विषयको अध्ययन वा अन्वेषण गरी राय, मन्तव्य वा सिफारिस पेश गर्ने कुनै पदमा नियुक्त भई काम गर्न यस उपधारामा लेखिएको कुनै कुराले बाधा पु¥याएको मानिने छैन ।

\textbf{२५६. राष्ट्रिय दलित आयोगको काम, कर्तव्य र अधिकार  :} (१) राष्ट्रिय दलित आयोगको काम, कर्तव्य र अधिकार देहाय बमोजिम हुनेछ :–

(क) दलित समुदायको समग्र स्थितिको अध्ययन तथा अन्वेषण गरी तत्सम्बन्धमा गर्नु पर्ने नीतिगत, कानूनी र संस्थागत सुधारका विषय पहिचान गरी नेपाल सरकारलाई सिफारिस गर्ने,
(ख) जातीय छुवाछूत, उत्पीडन र विभेदको अन्त्य गरी दलित उत्थान र विकासका लागि दलित हितसँग सरोकार राख्ने राष्ट्रिय नीति तथा कार्यक्रमको तर्जुमा गरी कार्यान्वयनका लागि नेपाल सरकार समक्ष पेश गर्ने,
(ग) दलित समुदायको उत्थान तथा हितमा भएका विशेष व्यवस्था लगायत दलित हितसँग सम्वन्धित कानूनको प्रभावकारी रूपमा पालना भए वा नभएको विषयमा अनुगमन गरी पालना वा कार्यान्वयन नभएको भए सोको पालना वा कार्यान्वयनका लागि नेपाल सरकार समक्ष सुझाब दिने,
(घ) दलित समुदायको अधिकारसंग सम्बन्धित नेपाल पक्ष भएको अन्तर्राष्ट्रिय सन्धि वा सम्झौतामा भएको व्यवस्था बमोजिम नेपालले पठाउनु पर्ने प्रतिवेदन तयारीका सम्बन्धमा नेपाल सरकारलाई सुझाव दिने,
(ङ) दलित समुदायलाई राष्ट्रिय विकासको मूल प्रवाहमा समाहित गर्न तथा राज्यका सबै अंगहरूमा समानुपातिक सहभागिता सुनिश्चित गर्न मौजूदा नीति तथा कार्यक्रमकोे समीक्षा, अनुगमन तथा मूल्यांकन गर्ने र सोको प्रभावकारी कार्यान्वयनका लागि नेपाल सरकारलाई सिफारिस गर्ने,
(च) जा तीय भेदभाव तथा छूवाछूत वा सामाजिक कुरीतिबाट पीडित भएको वा दलितको हक प्रयोग गर्न नदिएको  कुनै व्यक्ति वा संस्था विरुद्ध मुद्दा वञ्चित गरेको विषयमा  संस्था विरुद्ध मुद्दा दायर गर्नु पर्ने आवश्यकता देखिएमा कानून बमोजिम दायर गर्नु पर्ने आवश्यकता देखिएमा अदालतमा मुद्दा दायर गर्न सम्बन्धित निकाय मक्ष सिफारिस  दायर गर्न सम्बन्धित निकाय सिफारिस गर्ने।

(२) राष्ट्रिय दलित आयोगले आफ्नो काम, कर्तव्य र अधिकार मध्ये कुनै काम, कर्तव्य र अधिकार आयोगको अध्यक्ष, कुनै सदस्य वा नेपाल सरकारको कुनै कर्मचारीलाई तोकिएको शर्तको अधीनमा रही प्रयोग तथा पालन गर्ने गरी प्रत्यायोजन गर्न सक्नेछ ।
(३) राष्ट्रिय दलित आयोगको अन्य काम, कर्तव्य, अधिकार तथा तत्सम्बन्धी अन्य व्यवस्था संघीय कानून बमोजिम हुनेछ ।

\textbf{२५७. प्रदेशमा कार्यालय स्थापना गर्न सक्नेः}  राष्ट्रिय दलित आयोगले आवश्यकता अनुसार प्रदेशमा आफ्नो कार्यालय स्थापना गर्न सक्नेछ ।
\textbf{२५८. राष्ट्रिय समावेशी आयोग :} (१) नेपालमा एक राष्ट्रिय समावेशी आयोग रहनेछ जसमा अध्यक्ष र अन्य चार जनासम्म सदस्य रहनेछन् ।

(२) राष्ट्रपतिले संवैधानिक परिषदको सिफारिसमा राष्ट्रिय समावेशी आयोगको अध्यक्ष र सदस्यको नियुक्ति गर्नेेछ ।
(३) राष्ट्रिय समावेशी आयोगका अध्यक्ष तथा सदस्यको पदावधि नियुक्ति भएको मितिले छ वर्षको हुनेछ ।
(४) उपधारा (३) मा जुनसुकै कुरा लेखिएको भए तापनि देहायको कुनै अवस्थामा राष्ट्रिय समावेशी आयोगका अध्यक्ष वा सदस्यको पद रिक्त भएको मानिनेछ :–
(क) निजले राष्ट्रपति समक्ष लिखित राजीनामा दिएमा,
(ख) निजको उमेर पैंसठ्ठी वर्ष पूरा भएमा,
(ग) निजको विरुद्ध धारा १०१ बमोजिम महाभियोगको प्रस्ताव पारित भएमा,
(घ) शारीरिक वा मानसिअस्वस्थताको कारण सेवामा रही कार्य शारीरिक वा मानसिक अस्वस्थताको कारण सेवामा रही कार्य  सम्पादन गर्न असमर्थ रहेको भनी संवैधानिक परिषदको समर्थ रहेको भनी राष्ट्रपतिले पदमुक्त गरेमा, सिफारिसमा राष्ट्रपतिले पदमुक्त गरेमा
(ङ) निजको मृत्यु भएमा ।
(५) उपधारा (२) बमोजिम नियुक्त अध्यक्ष तथा सदस्यको पुनः नियुक्ति हुन सक्ने छैन ।
तर सदस्यलाई अध्यक्षको पदमा नियुक्ति गर्न सकिनेछ र त्यस्तो सदस्य अध्यक्षको पदमा नियुक्ति भएमा निजको पदावधि गणना गर्दा सदस्य भएको अवधिलाई समेत जोडी गणना गरिनेछ ।

(६) देहायको योग्यता भएको व्यक्ति राष्ट्रिय समावेशी आयोगको अध्यक्ष वा सदस्यको पदमा नियुक्ति हुन योग्य हुनेछः –
(क) कम्तीमा दश वर्ष सामाजिक समावेशीकरण, अपांगता भएका व्यक्ति, अल्पसंख्यक एवं सीमान्तीकृत समुदाय तथा पिछडिएको क्षेत्र र वर्गको हक हित वा विकास वा मानव अधिकारको क्षेत्रमा महत्वपूर्ण योगदान पुर्‍याएको,
(ख) अध्यक्षको हकमा मान्यताप्राप्त विश्वविद्यालयबाट स्नातक उपाधि हासिल गरेको,
(ग) पैंतालिस वर्ष उमेर पूरा गरेको,
(घ) नियुक्ति हुँदाको बखत कुनै राजनीतिक दलको सदस्य नरहेको, र
(ङ) उच्च नैतिक चरित्र भएको ।
(७) राष्ट्रिय समावेशी आयोगको अध्यक्ष र सदस्यको पारिश्रमिक र सेवाका शर्त संघीय कानून बमोजिम हुनेछ र निज बहाल रहेसम्म निजलाई मर्का पर्ने गरी पारिश्रमिक र सेवाका शर्त परिवर्तन गरिने छैन ।
तर चरम आर्थिक  विश्रृंखलताका कारण संकटकाल घोषणा भएको  अवस्थामा यो व्यवस्था लागू हुने छैन ।
(८) राष्ट्रिय समावेशी आयोगको अध्यक्ष वा सदस्य भइसकेको व्यक्ति अन्य सरकारी सेवामा नियुक्तिका लागि ग्राह्य हुने छैन ।
तर कुनै राजनीतिक पदमा वा कुनै विषयको अनुसन्धान, जाँचबुझ वा छानबीन गर्ने वा कुनै विषयको अध्ययन वा अन्वेषण गरी राय, मन्तव्य वा सिफारिस पेश गर्ने कुनै पदमा नियुक्त भई काम गर्न यस उपधारामा लेखिएको कुनै कुराले बाधा पु¥याएको मानिने छैन ।

\textbf{२५९. राष्ट्रिय समावेशी आयोगको काम, कर्तव्य र अधिकार :} (१)(क) खस आर्य, पिछडा वर्ग, अपांगता भएका व्यक्ति, ज्येष्ठ नागरिक, श्रमिक, किसान, अल्पसंख्यक एवं सीमान्तीकृत समुदाय तथा पिछडिएको वर्ग र कर्णाली तथा आर्थिक रूपले विपन्न वर्ग लगायतका समुदायको हक अधिकारको संरक्षणका लागि अध्ययन तथा अनुसन्धान गर्ने,
(ख) खण्ड (क) मा उल्लिखित समुदाय, वर्ग र क्षेत्रको समावेशीकरणका लागि नेपाल सरकारले अवलम्बन गरेको
नीति तथा कानूनको कार्यान्वयन अवस्थाको अध्ययन गरी सुधारका लागि नेपाल सरकारलाई सुझाव दिने,
(ग) खण्ड (क) मा उल्लिखित समुदाय, वर्ग र क्षेत्रकोे राज्य संयन्त्रमा उचित प्रतिनिधित्व भए नभएको अध्ययन गरी त्यस्तो समुदाय, वर्ग र क्षेत्रकोे प्रतिनिधित्वका लागि गरिएको विशेष व्यवस्थाको पुनरावलोकन गर्न नेपाल सरकारलाई सुझाव दिने,
(घ) खण्ड (क) मा उल्लिखित समुदाय, वर्ग र क्षेत्रकोे संरक्षण, सशक्तीकरण र विकास सन्तोषजनक भए नभएको अध्ययन गरी भविष्यमा अवलम्बन गर्नु पर्ने नीतिको सम्बन्धमा नेपाल सरकार समक्ष सिफारिस गर्ने,
(ङ) कर्णाली र पिछडिएको क्षेत्रको विकास र समृृद्धिका लागि अबलम्बन गर्नु पर्ने नीति र कार्यक्रमको सम्बन्धमा नेपाल सरकारलाई सुझाव दिने,
(च) अल्पसंख्यक तथा सीमान्तीकृत समुदाय सम्बन्धी कानूनमा समयानुकुल परिमार्जनका लागि सिफारिस गर्ने,
(छ) अल्पसंख्यक तथा सीमान्तीकृत समुदायका लागि प्रत्याभूत हक अधिकारको कार्यान्वयन स्थिति अनुगमन गरी आवधिक रूपमा हुने राष्ट्रिय जनगणना तथा मानव विकास सूचकांक सम्बन्धी प्रतिवेदनको आधारमा आवश्यक पुनरावलोकन गरी परिमार्जनका लागि सिफारिस गर्ने ।
(२) राष्ट्रिय समावेशी आयोगले आफ्नो काम, कर्तव्य र अधिकार मध्ये कुनै काम, कर्तव्य र अधिकार सो आयोगको अध्यक्ष, कुनै सदस्य वा नेपाल सरकारको कुनै कर्मचारीलाई तोकिएको शर्तको अधीनमा रही प्रयोग तथा पालन गर्ने गरी प्रत्यायोजन गर्न सक्नेछ ।
(३) राष्ट्रिय समावेशी आयोगको अन्य काम, कर्तव्य, अधिकार तथा तत्सम्बन्धी अन्य व्यवस्था संघीय कानून बमोजिम हुनेछ ।

\textbf{२६०. प्रदेशमा कार्यालय स्थापना गर्न सक्नेः} राष्ट्रिय समावेशी आयोगले आवश्यकता अनुसार प्रदेशमा आफ्नो कार्यालय स्थापना गर्न सक्नेछ ।

\textbf{२६१. आदिवासी जनजाति आयोगः} (१) नेपालमा एक आदिवासी जनजाति आयोग रहनेछ जसमा अध्यक्ष र अन्य चार जनासम्म सदस्य रहनेछन् ।
(२) राष्ट्रपतिले संवैधानिक परिषदको सिफारिसमा आदिवासी जनजाति आयोगका अध्यक्ष र सदस्यको नियुक्ति गर्नेेछ ।

(३) आदिवासी जनजाति आयोगका अध्यक्ष तथा सदस्यको पदावधि नियुक्ति भएको मितिले छ वर्षको हुनेछ ।
(४) आदिवासी जनजाति आयोगका अध्यक्ष तथा सदस्यको योग्यता, पद रिक्त हुने अवस्था, पारिश्रमिक र सेवाका शर्तहरु र त्यस्तो आयोगको काम, कर्तव्य र अधिकार सम्बन्धी अन्य व्यवस्था संघीय कानून बमोजिम हुनेछ ।

\textbf{२६२. मधेशी आयोग :} (१) नेपालमा एक मधेशी आयोग रहनेछ जसमा अध्यक्ष र अन्य चार जनासम्म सदस्य रहनेछन् ।

(२) राष्ट्रपतिले संवैधानिक परिषदको सिफारिसमा मधेशी आयोगका अध्यक्ष र सदस्यको नियुक्ति गर्नेेछ ।
(३) मधेशी आयोगका अध्यक्ष तथा सदस्यको पदावधि नियुक्ति भएको मितिले छ वर्षको हुनेछ ।
(४) मधेशी आयोगका अध्यक्ष तथा सदस्यको योग्यता, पद रिक्त हुने अवस्था, पारिश्रमिक र सेवाका शर्तहरु र त्यस्तो आयोगको काम, कर्तव्य र अधिकार सम्बन्धी अन्य व्यवस्था संघीय कानून बमोजिम हुनेछ ।

\textbf{२६३. थारू आयोग :}(१) नेपालमा एक थारू आयोग रहनेछ जसमा अध्यक्ष र अन्य चार जनासम्म सदस्य रहनेछन् ।

(२) राष्ट्रपतिले संवैधानिक परिषदको सिफारिसमा थारू आयोगका अध्यक्ष र सदस्यको नियुक्ति गर्नेेछ ।
(३) थारु आयोगका अध्यक्ष तथा सदस्यको पदावधि नियुक्ति भएको मितिले छ वर्षको हुनेछ ।
(४) थारू आयोगका अध्यक्ष तथा सदस्यको योग्यता, पद रिक्त हुने अवस्था, पारिश्रमिक र सेवाका शर्तहरु र त्यस्तो आयोगको काम, कर्तव्य र अधिकार सम्बन्धी अन्य व्यवस्था संघीय कानून बमोजिम हुनेछ ।

\textbf{२६४. मुस्लिम आयोग :} (१) नेपालमा एक मुस्लिम आयोग रहनेछ जसमा अध्यक्ष र अन्य चार जनासम्म सदस्य रहनेछन् ।

(२) राष्ट्रपतिले संवैधानिक परिषदको सिफारिसमा मुस्लिम आयोगका अध्यक्ष र सदस्यको नियुक्ति गर्नेेछ ।
(३) मुस्लिम आयोगका अध्यक्ष तथा सदस्यको पदावधि नियुक्ति भएको मितिले छ वर्षको हुनेछ ।

(४) मुस्लिम आयोगका अध्यक्ष तथा सदस्यको योग्यता, पद रिक्त हुने अवस्था, पारिश्रमिक र सेवाका शर्तहरु र त्यस्तो आयोगको काम, कर्तव्य र अधिकार सम्बन्धी अन्य व्यवस्था संघीय कानून बमोजिम हुनेछ ।

\textbf{२६५. आयोगको पुनरावलोकन :} यस भाग बमोजिम गठन भएका आयोगहरुको संघीय संसदले यो संविधान प्रारम्भ भएको मितिले दश वर्ष पछि पुनरावलोकन गर्नेछ ।

 