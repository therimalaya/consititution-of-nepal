\section{भाग–२९ राजनीतिक दल सम्बन्धी व्यवस्था}

तर विशेष परिस्थिति उत्पन्न भई पाँच वर्षभित्रमा पदाधिकारीको निर्वाचन सम्पन्न हुन नसकेमा छ महीनाभित्र त्यस्तो निर्वाचन गर्न सकिने गरी राजनीतिक दलको विधानमा व्यवस्था गर्न बाधा पर्ने छैन ।

(ग) दलको विभिन्न तहका कार्यकारिणी समितिमा नेपालको विविधतालाई प्रतिबिम्बित गर्ने गरी समावेशी प्रतिनिधित्वको व्यवस्था गरिएको हुनु पर्छ ।

(५) कुनै राजनीतिक दलको नाम, उद्देश्य, चिन्ह वा झण्डा देशको धार्मिक वा साम्प्रदायिक एकतामा खलल पार्ने वा देशलाई विखण्डित गर्ने प्रकृतिको रहेछ भने त्यस्तो राजनीतिक दल दर्ता हुने छैन ।

\textbf{२७०. राजनीतिक दललाई प्रतिबन्ध लगाउन बन्देज :} (१) धारा २६९ बमोजिम राजनीतिक दलको गठन गरी सञ्चालन गर्न र दलको विचारधारा, दर्शन र कार्यक्रमप्रति जनसाधारणको समर्थन र सहयोग प्राप्त गर्नका लागि त्यसको प्रचार र प्रसार गर्ने कार्यमा कुनै प्रतिबन्ध लगाउने गरी बनाइएको कानून वा गरिएको कुनै व्यवस्था वा निर्णय यो संविधानको प्रतिकूल मानिनेछ र स्वतः अमान्य हुनेछ।

(२) कुनै एउटै राजनीतिक दल वा एकै किसिमको राजनीतिक विचारधारा, दर्शन वा कार्यक्रम भएका व्यक्तिहरूले मात्र निर्वाचन, देशको राजनीतिक प्रणाली वा राज्य व्यवस्था सञ्चालनमा भाग लिन पाउने वा सम्मिलित हुन पाउने गरी बनाइएको कानून वा गरिएको कुनै व्यवस्था वा निर्णय यो संविधानको प्रतिकूल मानिनेछ र स्वतः अमान्य हुनेछ ।

\textbf{२७१. राजनीतिक दलको रूपमा निर्वाचनका लागि मान्यता प्राप्त गर्न दर्ता गराउनु पर्ने :} (१) निर्वाचनको प्रयोजनका लागि निर्वाचन आयोगबाट मान्यता प्राप्त गर्न चाहने धारा २६९ बमोजिम दर्ता भएको प्रत्येक राजनीतिक दलले संघीय कानून बमोजिमको कार्यविधि पूरा गरी निर्वाचन आयोगमा दर्ता गराउनु पर्नेछ ।

(२) उपधारा (१) को प्रयोजनका लागि निवेदन दिंदा राजनीतिक दलले धारा २६९ को उपधारा (३) मा उल्लिखित विवरणको अतिरिक्त वार्षिक लेखापरीक्षण प्रतिवेदन पेश गर्नु पर्नेछ र सोही धाराको उपधारा (४) मा उल्लिखित शर्त समेत पूरा गरेको हुनु पर्नेछ ।

\textbf{२७२. राजनीतिक दल सम्बन्धी अन्य व्यवस्था :}  राजनीतिक दलको गठन, दर्ता, सञ्चालन र सुविधा तथा तत्सम्बन्धी अन्य विषय संघीय कानून बमोजिम हुनेछ ।