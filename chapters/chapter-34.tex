\section{भाग–३४ परिभाषा र व्याख्या}

(ग) “धारा” भन्नाले यस संविधानको धारा सम्झनु पर्छ ।

(घ) “नगरपालिका” भन्नाले महानगरपालिका र उपमहानगरपालिका समेत सम्झनु पर्छ ।

(ङ) “नागरिक” भन्नाले नेपालको नागरिक सम्झनु पर्छ ।

(च) “प्रदेश” भन्नाले यस संविधान बमोजिम संघीय इकाइमा विभाजन गरिएको नेपालको संघीय इकाइको क्षेत्र र स्वरूप सम्झनु पर्छ ।

(छ) “पारिश्रमिक” भन्नाले तलब, भत्ता र अन्य कुनै किसिमको पारिश्रमिक तथा सुविधा समेत सम्झनु पर्छ ।

(ज) “राज्यशक्ति” भन्नाले राज्यको कार्यपालिका, व्यवस्थापिका र न्यायपालिका सम्बन्धी अधिकार सम्झनु पर्छ र सो शब्दले अवशिष्ट अधिकार समेतलाई जनाउँछ ।

(झ) “विधेयक” भन्नाले संघीय संसद वा प्रदेश सभामा पेश भएको संविधान संशोधन वा ऐनको मस्यौदा सम्झनु पर्छ ।

(ञ) “संघ” भन्नाले संघीय संरचनाको सबैभन्दा माथिल्लो इकाइको रूपमा रहने संघीय तह सम्झनु पर्छ ।

(ट) “संघीय इकाइ” भन्नाले संघ, प्रदेश र स्थानीय तह सम्झनु पर्छ ।

(ठ) “संवैधानिक निकाय” भन्नाले यस संविधान बमोजिम गठन गरिएका अख्तियार दुरुपयोग अनुसन्धान आयोग, महालेखा  परीक्षक, लोकसेवा आयोग, निर्वाचन आयोग, राष्ट्रिय मानव अधिकार आयोग, राष्ट्रिय प्राकृतिक स्रोत तथा वित्त आयोग, राष्ट्रिय महिला आयोग, राष्ट्रिय दलित आयोग, राष्ट्रिय समावेशी आयोग, आदिवासी जनजाति आयोग, मधेशी आयोग, थारू आयोग र मुस्लिम आयोग सम्झनु पर्छ ।

(ड) “सीमान्तीकृत” भन्नाले राजनीतिक, आर्थिक र सामाजिक रूपले पछाडि पारिएका, विभेद र उत्पीडन तथा भौगोलिक विकटताको कारणले सेवा सुविधाको उपभोग गर्न नसकेका वा त्यसबाट वञ्चित रहेका संघीय कानून बमोजिमको मानव विकासको स्तर भन्दा न्यून स्थितिमा रहेका समुदाय सम्झनु पर्छ र सो शब्दले अतिसीमान्तीकृत र लोपोन्मुख समुदाय  समेतलाई जनाउँछ ।

(ढ) “स्थानीय तह” भन्नाले यस संविधान बमोजिम स्थापना हुने गाउँँपालिका, नगरपालिका र जिल्ला सभालाई सम्झनु पर्छ।

(२) विषय वा प्रसङ्गले अर्को अर्थ नलागेमा यस संविधानमा व्यक्त भएका कुराहरूको अधीनमा रही कानूनको व्याख्या सम्बन्धी कानूनी व्यवस्था नेपाल कानूनको व्याख्यामा लागू भए सरह यस संविधानको व्याख्यामा पनि लागू हुनेछ ।

 