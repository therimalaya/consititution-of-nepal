\section{भाग–२२ महालेखा परीक्षक}

(क) मान्यताप्राप्त विश्वविद्यालयबाट व्यवस्थापन, वाणिज्यशास्त्र वा लेखामा स्नातक उपाधि प्राप्त गरी वा चार्टर्ड एकाउन्टेन्सी परीक्षा उत्तीर्ण गरी नेपाल सरकारको विशिष्ट श्रेणीको पदमा काम गरेको वा लेखा परीक्षण सम्बन्धी काममा कम्तीमा बीस वर्ष अनुभव प्राप्त गरेको,

(ख) नियुक्ति हुँदाका बखत कुनै राजनीतिक दलको सदस्य नरहेको,
(ग) पैँतालिस वर्ष उमेर पूरा भएको, र
(घ) उच्च नैतिक चरित्र भएको ।

(७) महालेखा परीक्षकको पारिश्रमिक र सेवाका शर्त संघीय कानून बमोजिम हुनेछ । महालेखा परीक्षक आफ्नो पदमा बहाल रहेसम्म निजलाई मर्का पर्ने गरी पारिश्रमिक र सेवाका शर्त परिवर्तन गरिने छैन ।
तर चरम आर्थिक विश्रृंखलताका कारण संकटकाल घोषणा भएको अवस्थामा यो व्यवस्था लागू हुने छैन ।

(८) महालेखा परीक्षक भइसकेको व्यक्ति अन्य सरकारी सेवामा नियुक्तिका लागि ग्राह्य हुने छैन ।
तर कुनै राजनीतिक पदमा वा कुनै विषयको अनुसन्धान, जाँचबुझ वा छानबीन गर्ने वा कुनै विषयको अध्ययन वा अन्वेषण गरी राय, मन्तव्य वा सिफारिस पेश गर्ने कुनै पदमा नियुक्त भई काम गर्न यस उपधारामा लेखिएको कुनै कुराले बाधा पुर्‍याएको मानिने छैन ।

\textbf{२४१. महालेखा परीक्षकको काम, कर्तव्य र अधिकार : }(१) राष्ट्रपति र उपराष्ट्रपतिको कार्यालय, सर्वोच्च अदालत, संघीय संसद, प्रदेश सभा, प्रदेश सरकार, स्थानीय तह, संवैधानिक निकाय वा सोको कार्यालय, अदालत, महान्यायाधिवक्ताको कार्यालय र नेपाली सेना, नेपाल प्रहरी वा सशस्त्र प्रहरी बल, नेपाल लगायतका सबै संघीय र प्रदेश सरकारी कार्यालयको लेखा कानून बमोजिम नियमितता, मितव्ययिता, कार्यदक्षता, प्रभावकारिता र औचित्य समेतको विचार गरी महालेखा परीक्षकबाट लेखापरीक्षण हुनेछ ।

(२) पचास प्रतिशतभन्दा बढी शेयर वा जायजेथामा नेपाल सरकार वा प्रदेश सरकारको स्वामित्व भएको संगठित संस्थाको लेखापरीक्षणका लागि लेखापरीक्षक नियुक्त गर्दा महालेखा परीक्षकसँग परामर्श गरिनेछ । त्यस्तो संगठित संस्थाको लेखापरीक्षण गर्दा अपनाउनु पर्ने सिद्धान्तको सम्बन्धमा महालेखा परीक्षकले आवश्यक निर्देशन दिन सक्नेछ ।

(३) महालेखा परीक्षकलाई उपधारा (१) बमोजिमको कामका लागि लेखा सम्बन्धी कागजपत्र जुनसुकै बखत हेर्न पाउने अधिकार हुनेछ ।
महालेखा परीक्षक वा त्यसका कुनै कर्मचारीले माग गरेको जुनसुकै कागजपत्र तथा जानकारी उपलब्ध गराउनु सम्बन्धित कार्यालय प्रमुखको कर्तव्य हुनेछ ।

(४) उपधारा (१) बमोजिम लेखापरीक्षण गरिने लेखा संघीय कानून बमोजिम महालेखा परीक्षकले तोकेको ढाँचामा राखिनेछ ।

(५) उपधारा (१) मा उल्लेख भएका कार्यालयहरूको लेखाका अतिरिक्त अन्य कुनै कार्यालय वा संस्थाको महालेखा परीक्षकबाट लेखापरीक्षण गर्नु पर्ने गरी संघीय कानून बमोजिम व्यवस्था गर्न सकिनेछ ।