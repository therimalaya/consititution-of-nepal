\section{भाग–१८ स्थानीय व्यवस्थापिका}

(२) उपधारा (१) बमोजिमको गाउँ सभामा गाउँ कार्यपालिका अध्यक्ष र उपाध्यक्ष, वडा अध्यक्ष र प्रत्येक वडाबाट निर्वाचित चारजना सदस्य र धारा २१५ को उपधारा (४) बमोजिम दलित वा अल्पसंख्यक समुदायबाट निर्वाचित गाउँ कार्यपालिकाका सदस्य रहनेछन् ।

(३) उपधारा (१) बमोजिम गठन हुने गाउँ सभामा प्रत्येक वडाबाट कम्तीमा दुईजना महिलाको प्रतिनिधित्व हुनेछ ।

(४) संघीय कानून बमोजिम गाउँपालिकामा रहने प्रत्येक वडामा वडा अध्यक्ष र चारजना सदस्यहरू रहेको वडा समिति गठन हुनेछ । त्यस्तो वडा अध्यक्ष र वडा सदस्यको निर्वाचन पहिलो हुने निर्वाचित हुने निर्वाचन प्रणाली बमोजिम हुनेछ ।

(५) अठार वर्ष उमेर पूरा भएको गाउँपालिकाको मतदाता नामावलीमा नाम समावेश भएको व्यक्तिलाई संघीय कानूनमा व्यवस्था भए बमोजिम मतदान गर्ने अधिकार हुनेछ ।

(६) देहायको योग्यता भएको व्यक्ति गाउँ सभाको सदस्यको पदमा उम्मेदवार हुन योग्य हुनेछः–

(क) नेपाली नागरिक,

(ख) एक्काइस वर्ष उमेर पूरा भएको,

(ग) गाउँपालिकाको मतदाता नामावलीमा नाम समावेश भएको, र

(घ) कुनै कानूनले अयोग्य नभएको ।

(७) गाउँ सभाको निर्वाचन र तत्सम्बन्धी अन्य व्यवस्था संघीय कानून बमोजिम हुनेछ ।

\textbf{२२३. नगर सभाको गठनः}

(१) प्रत्येक नगरपालिकामा एक नगर सभा रहनेछ ।

(२) उपधारा (१) बमोजिमको नगर सभामा नगरकार्यपालिकाका प्रमुख र उपप्रमुख, वडा अध्यक्ष र प्रत्येक वडाबाट निर्वाचित चारजना सदस्य र धारा २१६ को उपधारा (४) बमोजिम दलित वा अल्पसंख्यक समुदायबाट निर्वाचित नगर कार्यपालिकाका सदस्य रहनेछन् ।

(३) उपधारा (१) बमोजिम गठन हुने नगर सभामा प्रत्येक वडाबाट कम्तीमा दुईजना महिलाको प्रतिनिधित्व हुनेछ ।

(४) संघीय कानून बमोजिम नगरपालिकामा रहने प्रत्येक वडामा वडा अध्यक्ष र चारजना सदस्यहरू रहेको वडा समिति गठन हुनेछ । त्यस्तो वडा अध्यक्ष र वडा सदस्यको निर्वाचन पहिलो हुने निर्वाचित हुने निर्वाचन प्रणाली बमोजिम हुनेछ ।

(५) अठार वर्ष उमेर पूरा भएको नगरपालिकाको मतदाता नामावलीमा नाम समावेश भएको व्यक्तिलाई संघीय कानून बमोजिम मतदान
गर्ने अधिकार हुनेछ ।

(६) देहायको योग्यता भएको व्यक्ति नगर सभाको सदस्यको पदमा उम्मेदवार हुन योग्य हुनेछः–

(क) नेपाली नागरिक,

(ख) एक्काइस वर्ष उमेर पूरा भएको,

(ग) नगरपालिकाको मतदाता नामावलीमा नाम समावेश भएको, र

(घ) कुनै कानूनले अयोग्य नभएको ।

(७) नगर सभाको निर्वाचन र तत्सम्बन्धी अन्य व्यवस्था संघीय कानून बमोजिम हुनेछ ।

\textbf{२२४. गाउँ सभा र नगर सभाको अध्यक्ष र उपाध्यक्षः} गाउँ कार्यपालिकाको अध्यक्ष र उपाध्यक्ष तथा नगर कार्यपालिकाको प्रमुख र उपप्रमुखले क्रमशः गाउँ सभा र नगर सभाको पदेन अध्यक्ष र उपाध्यक्ष भई कार्य सम्पादन गर्नेछ ।

\textbf{२२५. गाउँ सभा र नगर सभाको कार्यकालः} गाउँ सभा र नगर सभाको कार्यकाल निर्वाचन भएको मितिले पाँच वर्षको हुनेछ । त्यस्तो कार्यकाल समाप्त भएको छ महीनाभित्र अर्काे गाउँ सभा र नगर सभाको निर्वाचन सम्पन्न गर्नु पर्नेछ ।

\textbf{२२६. कानून बनाउन सक्नेः} (१) गाउँ सभा र नगर सभाले अनुसूची–८ र अनुसूची–९ बमोजिमको सूचीमा उल्लिखित विषयमा आवश्यक कानून बनाउन सक्नेछ ।

(२) उपधारा (१) बमोजिम कानून बनाउने प्रक्रिया प्रदेश कानून बमोजिम हुनेछ ।

\textbf{२२७. गाउँ सभा र नगर सभा सम्बन्धी अन्य व्यवस्थाः} गाउँ सभा र नगर सभाको सञ्चालन, बैठकको कार्यविधि, समिति गठन, सदस्यको पद रिक्त हुने अवस्था, गाउँ सभा र नगर सभाका सदस्यले पाउने सुविधा, गाउँपालिका र नगरपालिकाको कर्मचारी र कार्यालय सम्बन्धी अन्य व्यवस्था प्रदेश कानून बमोजिम हुनेछ ।