\section{भाग–२४ निर्वाचन आयोग}

(४) उपधारा (३) मा जुनसुकै कुरा लेखिएको भए तापनि देहायको कुनै अवस्थामा प्रमुख निर्वाचन आयुक्त र निर्वाचन आयुक्तको पद रिक्त हुनेछः–
(क) निजले राष्ट्रपति समक्ष लिखित राजीनामा दिएमा,

(ख) निजको उमेर पैंसठ्ठी वर्ष पूरा भएमा,

(ग) निजको विरुद्ध धारा १०१ बमोजिम महाभियोगको प्रस्ताव पारित भएमा,

(घ) शारीरिक वा मानसिक अस्वस्थताको कारण सेवामा रही कार्य सम्पादन गर्न असमर्थ रहेको भनी संवैधानिक परिषदको
सिफारिसमा राष्ट्रपतिले पदमुक्त गरेमा,

(ङ) निजको मृत्यु भएमा ।

(५) उपधारा (२) बमोजिम नियुक्त प्रमुख निर्वाचन आयुक्त तथा आयुक्तको पुनः नियुिक्त हुन सक्ने छैन ।
तर आयुक्तलाई प्रमुख निर्वाचन आयुक्तको पदमा नियुक्ति गर्न सकिनेछ र त्यस्तो आयुक्त प्रमुख निर्वाचन आयुक्तको पदमा नियुक्ति भएमा निजको पदावधि गणना गर्दा आयुक्त भएको अवधिलाई समेत जोडी गणना गरिनेछ ।

(६) देहायको योग्यता भएको व्यक्ति प्रमुख निर्वाचन आयुक्त वा निर्वाचन आयुक्त पदमा नियुक्तिका लागि योग्य हुनेछः–

(क) मान्यताप्राप्त विश्वविद्यालयबाट स्नातक उपाधि प्राप्त गरेको,

(ख) नियुक्ति हुँदाका बखत कुनै राजनीतिक दलको सदस्य नरहेको,

(ग) पैंतालिस वर्ष उमेर पूरा भएको, र

(घ) उच्च नैतिक चरित्र भएको ।

(७) प्रमुख निर्वाचन आयुक्त र आयुक्तकोे पारिश्रमिक र सेवाका शर्त संघीय कानून बमोजिम हुनेछन् । प्रमुख निर्वाचन आयुक्त र आयुक्त आफ्नो पदमा बहाल रहेसम्म निजहरूलाई मर्का पर्ने गरी पारिश्रमिक र सेवाका शर्त परिवर्तन गरिने छैन ।
तर चरम आर्थिक विश्रृंखलताका कारण संकटकाल घोषणा भएको अवस्थामा यो व्यवस्था लागू हुने छैन ।

(८) निर्वाचन आयोगको प्रमुख निर्वाचन आयुक्त र आयुक्त भइसकेको व्यक्ति अन्य सरकारी सेवामा नियुक्तिका लागि ग्राह्य हुने छैन ।
तर कुनै राजनीतिक पदमा वा कुनै विषयको अनुसन्धान, जाँचबुझ वा छानबीन गर्ने वा कुनै विषयको अध्ययन वा अन्वेषण गरी राय, मन्तव्य वा सिफारिस पेश गर्ने कुनै पदमा नियुक्त भई काम गर्न यस उपधारामा लेखिएको कुनै कुराले बाधा पुर्‍याएको मानिने छैन ।

\textbf{२४६. निर्वाचन आयोगको काम, कर्तव्य र अधिकारः} (१) निर्वाचन आयोगले यस संविधान र संघीय कानूनको अधीनमा रही राष्ट्रपति, उपराष्ट्रपति, संघीय संसदका सदस्य, प्रदेश सभाका सदस्य, स्थानीय तहका सदस्यको निर्वाचनको संचालन, रेखदेख, निर्देशन र नियन्त्रण गर्नेछ । निर्वाचनको प्रयोजनका लागि मतदाताको नामावली तयार गर्ने कार्य निर्वाचन आयोगले गर्नेछ ।

(२) निर्वाचन आयोगले यस संविधान र संघीय कानून बमोजिम राष्ट्रिय महत्वको विषयमा जनमत संग्रह गराउनेछ ।

(३) राष्ट्रपति, उपराष्ट्रपति, संघीय संसदका सदस्य, प्रदेश सभा सदस्य वा स्थानीय तहका सदस्यका लागि उम्मेदवारीको मनोनयन दर्ता भइसकेको तर निर्वाचन परिणाम घोषणा भई नसकेको अवस्थामा कुनै उम्मेदवारको योग्यता सम्बन्धमा कुनै प्रश्न उठेमा त्यसको निर्णय निर्वाचन आयोगले गर्नेछ ।

(४) निर्वाचन आयोगले आफ्नो काम, कर्तव्य र अधिकार मध्ये कुनै काम, कर्तव्य र अधिकार प्रमुख निर्वाचन आयुक्त, कुनै निर्वाचन आयुक्त वा सरकारी कर्मचारीलाई तोकिएको शर्तको अधीनमा रही प्रयोग तथा पालन गर्ने गरी प्रत्यायोजन गर्न सक्नेछ ।

(५) निर्वाचन आयोगको अन्य काम, कर्तव्य र अधिकार तथा कार्यविधि संघीय कानून बमोजिम हुनेछ ।

\textbf{२४७. आवश्यक सहयोग गर्नु पर्नेः} यस संविधान बमोजिम निर्वाचन आयोगलाई आफ्नो काम पूरा गर्न आवश्यक पर्ने कर्मचारी र अन्य सहयोग नेपाल सरकार, प्रदेश सरकार र स्थानीय सरकारले उपलब्ध गराउनेछ ।