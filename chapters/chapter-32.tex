\section{भाग–३२ विविध}

\textbf{२७७. उपाधि, सम्मान र विभूषण :} (१) राष्ट्रपतिले राज्यको तर्फबाट प्रदान गरिने उपाधि, सम्मान र विभूषण प्रदान गर्नेछ ।
(२) नेपाल सरकारबाट स्वीकृति प्राप्त नगरी नेपालको कुनै नागरिकले कुनै विदेशी सरकारबाट प्रदान गरिने उपाधि, सम्मान वा विभूषण ग्रहण गर्नु हुँदैन ।

\textbf{२७८. सन्धि सम्पन्न गर्ने अधिकार :} (१) सन्धि वा सम्झौता गर्ने अधिकार संघमा निहित रहनेछ ।
(२) प्रदेशको अधिकारको सूचीमा पर्ने विषयमा सन्धि वा सम्झौता गर्दा नेपाल सरकारले सम्बन्धित प्रदेशसँग परामर्श गर्नु पर्नेछ ।

(३) प्रदेश मन्त्रिपरिषदले नेपाल सरकारको सहमति लिई आर्थिक तथा औद्योगिक विषयका करारजन्य सम्झौता गर्न सक्नेछ।

\textbf{२७९. सन्धि वा सम्झौताको अनुमोदन, सम्मिलन, स्वीकृति वा समर्थन :}  (१) नेपाल राज्य वा नेपाल सरकार पक्ष हुने सन्धि वा सम्झौताको अनुमोदन, सम्मिलन, स्वीकृति वा समर्थन संघीय कानून बमोजिम हुनेछ ।

(२) उपधारा (१) बमोजिम कानून बनाउँदा देहायका विषयका सन्धि वा सम्झौताको अनुमोदन, सम्मिलन, स्वीकृति वा समर्थन संघीय संसदका दुवै सदनमा तत्काल कायम रहेका सम्पूर्ण सदस्य संख्याको दुई तिहाइ बहुमतले गर्नुपर्ने शर्त राखिनेछ :–
(क) शान्ति र मैत्री,
(ख) सुरक्षा एवं सामरिक सम्बन्ध,
(ग) नेपाल राज्यको सीमाना, र
(घ) प्राकृतिक स्रोत तथा त्यसको उपयोगको बाँडफाँड ।

तर खण्ड (क) र (घ) मा उल्लिखित विषयका सन्धि वा सम्झौता मध्ये राष्ट्रलाई व्यापक, गम्भीर वा दीर्घकालीन असर नपर्ने साधारण प्रकृतिका सन्धि वा सम्झौताको अनुमोदन, सम्मिलन, स्वीकृति वा समर्थन प्रतिनिधि सभाको बैठकमा उपस्थित सदस्यहरूको साधारण बहुमतबाट हुन सक्नेछ ।

(३) यो संविधान प्रारम्भ भएपछि हुने कुनै सन्धि वा सम्झौता यस धारा बमोजिम अनुमोदन, सम्मिलन, स्वीकृति वा समर्थन नभएसम्म नेपाल सरकार वा नेपालको हकमा लागू हुने छैन ।
(४) उपधारा (१) र (२) मा जुनसुकै कुरा लेखिएको भए तापनि नेपालको भौगोलिक अखण्डतामा प्रतिकूल असर पर्ने गरी कुनै सन्धि वा सम्झौता गरिने छैन ।

\textbf{२८०. राष्ट्रपतिको कार्य गर्ने सम्बन्धी विशेष व्यवस्था }: यस संविधान बमोजिम राष्ट्रपति र उपराष्ट्रपति दुवैको पद रिक्त भएमा राष्ट्रपति वा उपराष्ट्रपतिको निर्वाचन भई कार्यभार नसम्हालेसम्म यस संविधान बमोजिम राष्ट्रपतिबाट गरिने कार्य प्रतिनिधि सभाको सभामुखले सम्पादन गर्नेछ ।

\textbf{२८१. विशेष अधिकारको समीक्षा तथा पुनरावलोकन :} नेपाल सरकारले प्रत्येक दश वर्षमा हुने राष्ट्रिय जनगणनासँगै महिला तथा दलित समुदायको विशेष अधिकारको व्यवस्थाको कार्यान्वयन र त्यसको प्रभाव सम्बन्धमा मानव विकास सूचकांकको आधारमा समीक्षा तथा पुनरावलोकन गर्नेछ ।

\textbf{२८२. राजदूत र विशेष प्रतिनिधि :} (१) राष्ट्रपतिले समावेशी सिद्धान्तको आधारमा नेपाली राजदूत र कुनै खास प्रयोजनका लागि विशेष प्रतिनिधि नियुक्ति गर्न सक्नेछ ।

(२) राष्ट्रपतिले विदेशी राजदूत तथा कूटनीतिक प्रतिनिधिबाट ओहोदाको प्रमाणपत्र ग्रहण गर्नेछ ।

\textbf{२८३. समावेशी सिद्धान्त बमोजिम नियुक्ति गर्नु पर्ने :} संवैधानिक अंग र निकायका पदमा नियुक्ति गर्दा समावेशी सिद्धान्त बमोजिम गरिनेछ ।

\textbf{२८४. संवैधानिक परिषद सम्बन्धी व्यवस्था :} (१) यस संविधान बमोजिम प्रधान न्यायाधीश र संवैधानिक निकायका प्रमुख र पदाधिकारीहरूको नियुक्तिको सिफारिस गर्न देहाय बमोजिमका अध्यक्ष र सदस्य रहेको एक संवैधानिक परिषद रहनेछः–

(क) प्रधानमन्त्री –अध्यक्ष
(ख) प्रधान न्यायाधीश –सदस्य
(ग) प्रतिनिधि सभाको सभामुख –सदस्य
(घ) राष्ट्रिय सभाको अध्यक्ष –सदस्य
(ङ) प्रतिनिधि सभाको विपक्षी दलको नेता –सदस्य
(च) प्रतिनिधि सभाको उपसभामुख –सदस्य

(२) प्रधान न्यायाधीशको पद रिक्त भएको अवस्थामा प्रधान न्यायाधीशको नियुक्तिको सिफारिस गर्दा संवैधानिक परिषदमा नेपाल सरकारको कानून तथा न्याय मन्त्री सदस्यको रूपमा रहनेछ ।

(३) संवैधानिक परिषदले प्रधान न्यायाधीश वा संवैधानिक निकायका कुनै प्रमुख वा पदाधिकारीको पद रिक्त हुनुभन्दा एक महीना अगावै यस संविधान बमोजिम नियुक्तिका लागि सिफारिस गर्नु पर्नेछ ।
तर मृत्यु भई वा राजीनामा दिई त्यस्तो पद रिक्त भएको अवस्थामा रिक्त भएको मितिले एक महीनाभित्र पदपूर्ति हुने गरी नियुक्तिका लागि सिफारिस गर्न सक्नेछ ।

(४) संवैधानिक परिषदको अन्य काम, कर्तव्य र अधिकार तथा प्रधान न्यायाधीश वा संवैधानिक निकायका प्रमुख वा पदाधिकारीको नियुक्ति सम्बन्धी कार्यविधि संघीय कानून बमोजिम हुनेछ ।

(५) नेपाल सरकारको मुख्य सचिवले संवैधानिक परिषदको सचिव भई काम गर्नेछ ।

\textbf{२८५. सरकारी सेवाको गठन :} (१) नेपाल सरकारले देशको प्रशासन सञ्चालन गर्न संघीय निजामती सेवा र आवश्यकता अनुसार अन्य संघीय सरकारी सेवाहरूको गठन गर्न सक्नेछ । त्यस्ता सेवाहरूको गठन, सञ्चालन र सेवाका शर्त संघीय ऐन बमोजिम हुनेछ ।

(२) संघीय निजामती सेवा लगायत सवै संघीय सरकारी सेवामा प्रतियोगितात्मक परीक्षाद्वारा पदपूर्ति गर्दा संघीय कानून बमोजिम खुला र समानुपातिक समावेशी सिद्धान्तका आधारमा हुनेछ ।

(३) प्रदेश मन्त्रिपरिषद, गाउँ कार्यपालिका र नगर कार्यपालिकाले आफ्नो प्रशासन सञ्चालन गर्न आवश्यकता अनुसार कानून बमोजिम विभिन्न सरकारी सेवाहरूको गठन र सञ्चालन गर्न सक्नेछन् ।

\textbf{२८६. निर्वाचन क्षेत्र निर्धारण आयोग :} (१) यस संविधान बमोजिम संघीय संसदका सदस्य र प्रदेश सभाका सदस्यको निर्वाचन गर्ने प्रयोजनका लागि निर्वाचन क्षेत्र निर्धारण गर्न नेपाल सरकारले देहायका अध्यक्ष र सदस्य रहेको एक निर्वाचन क्षेत्र निर्धारण आयोगको गठन गर्न सक्नेछः–
(क) सर्वाेच्च अदालतको सेवानिवृत्त न्यायाधीश –अध्यक्ष
(ख) भूगोलविद एक जना – सदस्य
(ग) समाजशास्त्री वा मानवशास्त्री एक जना – सदस्य
(घ) प्रशासनविद वा कानूनविद एक जना – सदस्य
(ङ) नेपाल सरकारको विशिष्ट श्रेणीको अधिकृत –सदस्य सचिव

(२) निर्वाचन क्षेत्र निर्धारण आयोगको कार्यावधि त्यस्तो आयोग गठन गर्दाका बखत तोके बमोजिम हुनेछ ।
(३) देहायको योग्यता भएको व्यक्ति निर्वाचन क्षेत्र निर्धारण आयोगको अध्यक्ष वा सदस्यको पदमा नियुक्त हुन योग्य हुनेछ :–

(क) मान्यताप्राप्त विश्वविद्यालयबाट सम्बन्धित विषयमा कम्तीमा स्नातक उपाधि प्राप्त गरेको,
(ख) पैंतालिस वर्ष उमेर पूरा भएको, र
(ग) उच्च नैतिक चरित्र भएको ।
(४) देहायको कुनै अवस्थामा निर्वाचन क्षेत्र निर्धारण आयोगका अध्यक्ष वा सदस्यको पद रिक्त हुनेछ : –
(क) निजले लिखित राजीनामा दिएमा,
(ख) निजलाई नेपाल सरकार, मन्त्रिपरिषदले हटाएमा,
(ग) निजको मृत्यु भएमा ।

(५) निर्वाचन क्षेत्र निर्धारण आयोगबाट यस धारा बमोजिम निर्वाचन क्षेत्र निर्धारण गर्दा धारा ८४ को उपधारा (१) को खण्ड (क) को अधीनमा रही प्रतिनिधित्वको लागि जनसंख्यालाई मुख्य र भूगोललाई दोस्रो आधार मानी संघीय कानून बमोजिम प्रदेशमा निर्वाचन क्षेत्र निर्धारण गरिनेछ र प्रदेशभित्र रहेका प्रत्येक जिल्लामा कम्तीमा एक निर्वाचन क्षेत्र रहने छन् ।

(६) उपधारा (५) बमोजिम निर्वाचन क्षेत्र निर्धारण गर्दा जनसंख्या र भौगोलिक अनुकुलता, सो क्षेत्रको जनसंख्याको घनत्व, भौगोलिक विशिष्टता, प्रशासनिक एवं यातायातको सुगमता, सामुदायिक तथा सांस्कृतिक पक्षलाई समेत ध्यान दिनु पर्नेछ ।

(७) निर्वाचन क्षेत्र निर्धारण आयोगद्वारा निर्वाचन क्षेत्र निर्धारण गरिएको र पुनरावलोकन गरिएको विषयमा कुनै अदालतमा प्रश्न उठाउन पाइने छैन ।

(८) निर्वाचन क्षेत्र निर्धारण आयोगले आफूले सम्पादन गरेको कामको प्रतिवेदन नेपाल सरकार समक्ष पेश गर्नेछ ।

(९) नेपाल सरकार, मन्त्रिपरिषदले उपधारा (८) बमोजिमको प्रतिवेदन संघीय संसद समक्ष पेश गर्नुको अतिरिक्त कार्यान्वयनका लागि निर्वाचन आयोगमा पठाउनेछ ।

(१०) निर्वाचन क्षेत्र निर्धारण आयोगले आफ्नो कार्यविधि आफैं निर्धारण गर्नेछ ।

(११) निर्वाचन क्षेत्र निर्धारण आयोगका अध्यक्ष र सदस्यको पारिश्रमिक तथा सुविधा क्रमशः निर्वाचन आयोगका प्रमुख निर्वाचन आयुक्त र निर्वाचन आयुक्त सरह हुनेछ ।

(१२) उपधारा (५) बमोजिम निर्धारण भएको निर्वाचन क्षेत्रको प्रत्येक बीस वर्षमा पुनरावलोकन गर्नु पर्नेछ ।

(१३) निर्वाचन क्षेत्र निर्धारण आयोगलाई आवश्यक पर्ने कर्मचारी नेपाल सरकारले उपलब्ध गराउनेछ ।

\textbf{२८७. भाषा आयोग :} (१) यो संविधान प्रारम्भ भएको मितिले एक वर्षभित्र नेपाल सरकारले प्रदेशहरूको प्रतिनिधित्व हुने गरी एक भाषा आयोगको गठन गर्नेछ ।

(२) भाषा आयोगमा अध्यक्षका अतिरिक्त आवश्यक संख्यामा सदस्यहरू रहनेछन् ।

(३) भाषा आयोगका अध्यक्ष र सदस्यको पदावधि नियुक्तिको मितिले छ वर्षको हुनेछ । निजहरुको पुनः नियुक्ति हुन सक्ने छैन ।

(४) देहायको योग्यता भएको व्यक्ति भाषा आयोगको अध्यक्ष वा सदस्यको पदमा नियुक्त हुन योग्य हुनेछः–

(क) मान्यताप्राप्त विश्वविद्यालयबाट सम्बन्धित विषयमा स्नातकोत्तर उपाधि प्राप्त गरेको,
(ख) नेपालका विभिन्न भाषाहरूको सम्बन्धमा अध्ययन, अध्यापन, अनुसन्धान र अन्वेषणको क्षेत्रमा कम्तीमा बीस वर्षको कार्य अनुभव भएको,
(ग) पैंतालिस वर्ष उमेर पूरा भएको, र
(घ) उच्च नैतिक चरित्र भएको ।
(५) देहायको कुनै अवस्थामा भाषा आयोगका अध्यक्ष वा सदस्यको पद रिक्त हुनेछ :–

(क) निजले लिखित राजीनामा दिएमा,
(ख) निजलाई नेपाल सरकार, मन्त्रिपरिषदले हटाएमा,
(ग) निजको उमेर पैंसठ्ठी वर्ष पूरा भएमा,
(घ) निजको मृत्यु भएमा ।
(६) भाषा आयोगको काम, कर्तव्य र अधिकार देहाय बमोजिम हुनेछ :–
(क) सरकारी कामकाजको भाषाका रूपमा मान्यता पाउन पूरा गर्नुपर्ने आधारहरूको निर्धारण गरी नेपाल सरकार समक्ष भाषाको सिफारिस गर्ने,

(ख) भाषाहरूको संरक्षण, संवर्धन र विकासका लागि अवलम्बन गर्नुपर्ने उपायहरूको नेपाल सरकार समक्ष सिफारिस गर्ने,

(ग) मातृभाषाहरूको विकासको स्तर मापन गरी शिक्षामा प्रयोगको सम्भाव्यताका बारेमा नेपाल सरकार समक्ष सुझाव पेश गर्ने,

(घ) भाषाहरूको अध्ययन, अनुसन्धान र अनुगमन गर्ने ।

(७) भाषा आयोगले उपधारा (६) को खण्ड (क) बमोजिमको कार्य आयोग गठन भएको मितिले पाँच वर्ष भित्र सम्पन्न गर्नेछ ।

(८) नेपाल सरकारले प्रदेश सरकारसँग समन्वय गरी प्रदेशमा भाषा आयोगको शाखा स्थापना गर्न सक्नेछ ।

(९) भाषा आयोगको अन्य काम, कर्तव्य र अधिकार तथा कार्यविधि संघीय कानून बमोजिम हुनेछ ।

\textbf{२८८. राजधानी :} (१) नेपालको राजधानी काठमाडौंमा रहनेछ ।

(२) यस संविधान बमोजिमका प्रदेशको राजधानी सम्बन्धित प्रदेश सभामा तत्काल कायम रहेका सदस्य संख्याको दुई तिहाइ बहुमतबाट निर्णय भए बमोजिम हुनेछ ।

(३) उपधारा (२) बमोजिम निर्णय नभएसम्म नेपाल सरकारले तोके बमोजिमको स्थानबाट प्रदेशको कार्य सञ्चालन हुनेछ ।

\textbf{२८९. पदाधिकारीको नागरिकता सम्बन्धी विशेष व्यवस्था :} (१) राष्ट्रपति, उपराष्ट्रपति, प्रधानमन्त्री, प्रधान न्यायाधीश, प्रतिनिधि सभाका सभामुख, राष्ट्रिय सभाका अध्यक्ष, प्रदेश प्रमुख, मुख्यमन्त्री, प्रदेश सभाको सभामुख र सुरक्षा निकायका प्रमुखको पदमा निर्वाचित, मनोनीत वा नियुक्ति हुन वंशजको आधारमा नेपालको नागरिकता प्राप्त गरेको हुनु पर्नेछ ।

(२) उपधारा (१) मा उल्लिखित पद बाहेक अन्य संवैधानिक निकायको पदमा यस संविधान बमोजिम नियुक्तिको लागि वंशजको आधारमा नेपालको नागरिकता प्राप्त गरेको व्यक्ति, नेपालको अंगीकृत नागरिकता प्राप्त गरेको व्यक्ति वा जन्मको आधारमा नेपालको नागरिकता प्राप्त गरेको व्यक्ति समेत योग्य हुनेछ ।

तर नेपालको अंगीकृत नागरिकता प्राप्त गरेको व्यक्तिको हकमा कम्तीमा दश वर्ष, जन्मको आधारमा नेपालको नागरिकता प्राप्त गरेको व्यक्ति र धारा ११ को उपधारा (६) बमोजिम नेपालको अंगीकृत नागरिकता प्राप्त गरेको व्यक्तिको हकमा कम्तीमा पाँच वर्ष नेपालमा बसोबास गरेको हुनु पर्नेछ ।

\textbf{२९०. गुठी सम्बन्धी व्यवस्था :} (१) गुठीको मूलभूत मान्यतामा प्रतिकूल असर नपर्ने गरी गुठी जग्गामा भोगाधिकार भैरहेका किसान एवं गुठीको अधिकारका सम्बन्धमा संघीय संसदले आवश्यक कानून बनाउनेछ ।

(२) गुठी सम्बन्धी अन्य व्यवस्था संघीय कानून बमोजिम हुनेछ ।

\textbf{२९१. नियुक्तिका लागि योग्य नहुने :} (१) यस संविधानमा अन्यत्र जुनसुकै कुरा लेखिएको भए तापनि विदेशको स्थायी आवासीय अनुमतिपत्र लिएको नेपालको नागरिक यस संविधान बमोजिम निर्वाचन, मनोनयन वा नियुक्ति हुने पदमा निर्वाचित, मनोनीत वा नियुक्तिको लागि योग्य हुने छैन ।

तर त्यस्तो विदेशको स्थायी आवासीय अनुमतिपत्र त्यागेको व्यक्तिलाई कम्तीमा तीन महीनाको अवधि व्यतित भए पछि त्यस्तो पदमा निर्वाचित, मनोनीत वा नियुक्त गर्न बाधा पर्ने छैन ।

(२) उपधारा (१) बमोजिमको विदेशको स्थायी आवासीय अनुमतिपत्र लिएको नेपालको नागरिक सम्बन्धी अन्य व्यवस्था संघीय कानून बमोजिम हुनेछ ।

\textbf{२९२. संसदीय सुनुवाई सम्बन्धी व्यवस्था :} (१) यस संविधान बमोजिम संवैधानिक परिषदको सिफारिसमा नियुक्त हुने प्रधान न्यायाधीश, सर्वाेच्च अदालतका न्यायाधीश, न्याय परिषदका सदस्य, संवैधानिक निकायको प्रमुख वा पदाधिकारी र राजदूतको पदमा नियुक्ति हुनु अघि संघीय कानून बमोजिम संसदीय सुनुवाई हुनेछ ।

(२) उपधारा (१) को प्रयोजनका लागि संघीय संसदका दुवै सदनका सदस्यहरू रहने गरी संघीय कानून बमोजिम पन्ध्र सदस्यीय एक संयुक्त समिति गठन गरिनेछ ।
(३) उपधारा (२) बमोजिमको संयुक्त समितिमा रहने सदस्यले संघीय संसदको उक्त कार्यकालभर सर्वाेच्च अदालतमा उपस्थित भई बहस पैरबी गर्न पाउने छैन ।

\textbf{२९३. संवैधानिक निकायको काम कारबाहीको अनुगमनः} संवैधानिक निकायका प्रमुख र पदाधिकारी संघीय संसदप्रति उत्तरदायी र जवाफदेही रहनु पर्नेछ । प्रतिनिधि सभाका समितिले राष्ट्रिय मानव अधिकार आयोग बाहेकका अन्य संवैधानिक निकायको प्रतिवेदन लगायतका काम कारबाहीको अनुगमन र मूल्यांकन गरी आवश्यक निर्देशन वा राय सल्लाह दिन सक्नेछ ।

\textbf{२९४. संवैधानिक निकायको वार्षिक प्रतिवेदन :} (१) यस संविधान बमोजिमका संवैधानिक निकायले आफूले गरेको काम कारबाहीको वार्षिक प्रतिवेदन राष्ट्रपति समक्ष पेश गर्नेछ र राष्ट्रपतिले प्रधानमन्त्री मार्फत त्यस्तो प्रतिवेदन संघीय संसद समक्ष पेश गर्न लगाउनेछ ।

(२) उपधारा (१) बमोजिमको वार्षिक प्रतिवेदनमा खुलाउनु पर्ने कुराहरू संघीय कानून बमोजिम हुनेछ ।

(३) उपधारा (१) मा जुनसुकै कुरा लेखिएको भए तापनि संवैधानिक निकायले प्रत्येक प्रदेशको काम कारबाहीको सम्बन्धमा अलग अलग प्रतिवेदन तयार गरी प्रदेश प्रमुख समक्ष पेश गर्न सक्नेछ ।