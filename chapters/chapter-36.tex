\section{अनुसूची-१ नेपालको राष्ट्रिय झण्डा}

\includegraphics{images/Nepal-Flag.png}

(३) ख घ रेखामा ख बाट क ख जति लिई ङ चिनो लाउने ।

(४) ङ हुँदै क ख को समानान्तर पारेर क ग मा पर्ने बिन्दु च बाट शुरुगरी दाहिनेतिर छ सम्म क ख को लम्बाइ जति रेखा खिच्ने ।

(५) ग र छ लाई जोड्ने ।

(ख) चन्द्र बनाउने तरीका

(६) क ख को चतुर्थांश जति क बाट दाहिनेमा ज चिनो लाउने रत्यहाँबाट माथि क ग को समानान्तर पारेर ग छ लाई झ मा छुनेरेखा खिच्ने ।

(७) ग च को आधा ञ बाट क ख को समानान्तर पारेर रेखा दायाँतिरखिची ग छ लाई ट मा छुने ।

(८) ञ ट र ज झ रेखा काटिएको ठाउँमा ठ चिनो राख्ने ।

(९) ञ र छ जोड्ने ।

(१०) ञ छ र ज झ काटिएको बिन्दुमा ड चिनो लाउने ।

(११) ड लाई केन्द्र मानी ख घ रेखालाई न्यूनतम अन्तर पर्ने गरी स्पर्शगर्दा हुने जति दूरी पर्ने गरी ज झ रेखाको तल्लो भागमा ढ चिनोलगाउने।

(१२) ड मा छोई क ख को समानान्तर रेखा बायाँबाट दायाँतिर खिच्ने रयसले क ग लाई छोएको बिन्दुको नाम ण राख्ने ।

(१३) ठ केन्द्र लिएर ठ ढ व्यासाद्र्धले तल्लो भागमा वृत्त खण्ड खिच्ने र णड बाट गएको रेखालाई यसले छोएको दुवै ठाउँमा क्रमशः त र थनाम राख्ने ।

(१४) ड लाई केन्द्र मानी ड थ व्यासाद्र्धले तल्लो भागमा अर्ध वृत्ताकार तथ लाई छुने गरी खिच्ने ।

(१५) ढ केन्द्र मानी ढ ड को व्यासाद्र्धले त ढ थ वृत्त खण्डको दुवैतर्फछुने गरी वृत्त खण्ड खिच्ने र यसले त ढ थ लाई छोएकोविन्दुहरूको नाम क्रमशः द र ध राख्ने । द ध लाई जोड्ने । द ढ रज झ काटिएको विन्दुको नाम न राख्ने ।

(१६) न लाई केन्द्र मानेर व्यासाद्र्ध न ध ले त ढ थ को माथिल्लो भागमादुवै ठाउँमा छुने गरी अर्ध वृत्ताकार खिच्ने ।

(१७) न लाई केन्द्र मानेर व्यासाद्र्ध न ड ले त ढ थ को माथिल्लोभागमा दुवै ठाउँमा छुने गरी वृत्त खण्ड खिच्ने ।

(१८) यस अनुसूचीको नं. (१६) को अर्ध वृत्ताकार भित्र र नं. (१७) कोवृत्त खण्ड बाहिर चन्द्रमाको आठवटा बराबरका कोण बनाउने ।

(ग) सूर्य बनाउने तरीका

(१९) क च को आधा प बाट क ख को समानान्तर पारेर ख ङ मा छुने गरीप फ रेखा खिच्ने ।

(२०) ज झ र प फ काटिएको विन्दु ब केन्द्र मानेर ड ढ को व्यासाद्र्धलेवृत्ताकार पूरा खिच्ने ।

(२१) ब लाई केन्द्र मानेर ठ ढ व्यासाद्र्धले वृत्ताकार पूरा खिच्ने ।

(२२) यो अनुसूचीको नं. (२०) को वृत्ताकार बाहिर र यो अनुसूचीको नं. (२१) को वृत्ताकारभित्र परेको गोल घेराको बीच भागमा सूर्यकोबाह्रवटा बराबरका कोणहरू दुई चुच्चाहरूले ज झ रेखामा छुने गरीबनाउने ।

(घ) किनारा बनाउने तरीका

(२३) न ढ को चौडाइ जति गाढा नीलो रंगको किनारा झण्डाको आकारकोबाहिरी सबैतिरको सीमामा थप्ने, तर झण्डाको पाँच कोणहरूमा चाहिंबाहिरी कोणहरू पनि भित्रै सरहका बनाउने ।

(२४) झण्डा डोरी लगाई प्रयोग गरेमा माथि बताइएकै पट्टी राख्ने । झण्डालट्ठीमा घुसार्ने हो भने क ग पट्टि आवश्यक परे जति किनाराचौड्याउने । डोरी वा लट्ठीको प्रयोगमा क ग को पट्टीमा प्वाल राख्ने।

स्पष्टीकरणः झण्डा बनाउँदा खिचिएका ज झ द ध, च ङ, ङ घ, ञछ, ण थ, ञ ट र प फ रेखाहरू कल्पित हुन् । त्यस्तै सूर्यका बाहिरी रभित्री वृत्ताकारहरू तथा खुर्पे चन्द्र बाहेक अरु वृत्त खण्ड पनि कल्पितहुन् । यिनलाई झण्डामा देखाइँदैन ।

द्रष्टव्यः राष्ट्रिय झण्डाको आकार नेपाल सरकारले निर्धारण गरेबमोजिम हुनेछ ।

 