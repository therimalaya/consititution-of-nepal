\section{प्रस्तावना}

हामी सार्वभौमसत्तासम्पन्न नेपाली जनता;

नेपालको स्वतन्त्रता, सार्वभौमिकता, भौगोलिक अखण्डता, राष्ट्रिय एकता, स्वाधीनता र स्वाभिमानलाई अक्षुण्ण राखी जनताको सार्वभौम अधिकार, स्वायत्तता र स्वशासनको अधिकारलाई आत्मसात् गर्दै;

राष्ट्रहित, लोकतन्त्र र अग्रगामी परिवर्तनका लागि नेपाली जनताले पटक– पटक गर्दै आएका ऐतिहासिक जन आन्दोलन, सशस्त्र संघर्ष, त्याग र बलिदानको गौरवपूर्ण इतिहासलाई स्मरण एवं शहीदहरू तथा बेपत्ता र पीडित नागरिकहरूलाई सम्मान गर्दै;

सामन्ती, निरंकुश, केन्द्रीकृत र एकात्मक राज्यव्यवस्थाले सृजना गरेका सबै प्रकारका विभेद र उत्पीडनको अन्त्य गर्दै;

बहुजातीय, बहुभाषिक, बहुधार्मिक, बहुसांस्कृतिक तथा भौगोलिक विविधतायुक्त विशेषतालाई आत्मसात् गरी विविधताबीचको एकता, सामाजिक सांस्कृतिक ऐक्यबद्धता, सहिष्णुता र सद्भावलाई संरक्षण एवं प्रवर्धन गर्दै; वर्गीय, जातीय, क्षेत्रीय, भाषिक, धार्मिक, लैंगिक विभेद र सबै प्रकारका जातीय छुवाछूतको अन्त्य गरी आर्थिक समानता, समृद्धि र सामाजिक न्याय सुनिश्चित गर्न समानुपातिक समावेशी र सहभागितामूलक सिद्धान्तका आधारमा समतामूलक समाजको निर्माण गर्ने संकल्प गर्दै;

जनताको प्रतिस्पर्धात्मक बहुदलीय लोकतान्त्रिक शासन प्रणाली, नागरिक स्वतन्त्रता, मौलिक अधिकार, मानव अधिकार, बालिग मताधिकार, आवधिक निर्वाचन, पूर्ण प्रेस स्वतन्त्रता तथा स्वतन्त्र, निष्पक्ष र सक्षम न्यायपालिका र कानूनी राज्यको अवधारणा लगायतका लोकतान्त्रिक मूल्य र मान्यतामा आधारित समाजवादप्रति प्रतिबद्ध रही समृद्ध राष्ट्र निर्माण गर्न;

संघीय लोकतान्त्रिक गणतन्त्रात्मक शासन व्यवस्थाको माध्यमद्वारा दिगो शान्ति, सुशासन, विकास र समृद्धिको आकांक्षा पूरा गर्न संविधान सभाबाट पारित गरी यो संविधान जारी गर्दछौं ।