\section{भाग–२३ लोक सेवा आयोग}

(४) लोक सेवा आयोगको अध्यक्ष र सदस्यको पदावधि नियुक्ति भएको मितिले छ वर्षको हुनेछ ।

(५) उपधारा (२) बमोजिम नियुक्त अध्यक्ष तथा सदस्यको पुनः नियुक्ति हुन सक्ने छैन ।
तर सदस्यलाई अध्यक्षको पदमा नियुक्ति गर्न सकिनेछ र त्यस्तो सदस्य अध्यक्षको पदमा नियुक्ति भएमा निजको पदावधि गणना गर्दा सदस्य भएको अवधिलाई समेत जोडी गणना गरिनेछ ।

(६) उपधारा (४) मा जुनसुकै कुरा लेखिएको भए तापनि देहायको कुनै अवस्थामा लोक सेवा आयोगको अध्यक्ष वा सदस्यको पद रिक्त हुनेछः–

(क) निजले राष्ट्रपति समक्ष लिखित राजीनामा दिएमा,

(ख) निजको उमेर पैंसठ्ठी वर्ष पूरा भएमा,

(ग) निजको विरुद्ध धारा १०१ बमोजिम महाभियोगको प्रस्ताव पारित भएमा,

(घ) शारीरिक वा मानसिक अस्वस्थताको कारण सेवामा रही कार्य सम्पादन गर्न असमर्थ रहेको भनी संवैधानिक परिषदको सिफारिसमा राष्ट्रपतिले पदमुक्त गरेमा,

(ङ) निजको मृत्यु भएमा ।

(७) देहायको योग्यता भएको व्यक्ति लोक सेवा आयोगको अध्यक्ष वा सदस्यको पदमा नियुक्तिका लागि योग्य हुनेछः–

(क) मान्यताप्राप्त विश्वविद्यालयबाट स्नातकोत्तर उपाधि प्राप्त गरेको,

(ख) नियुक्ति हुँदाका बखत कुनै राजनीतिक दलको सदस्य नरहेको,

(ग) पैंतालिस वर्ष उमेर पूरा भएको, र

(घ) उच्च नैतिक चरित्र भएको ।

(८) लोक सेवा आयोगका अध्यक्ष र सदस्यको पारिश्रमिक र सेवाका शर्त संघीय कानून बमोजिम हुनेछ । लोक सेवा आयोगका अध्यक्ष र सदस्य आफ्नो पदमा बहाल रहेसम्म निजहरूलाई मर्का पर्ने गरी पारिश्रमिक र सेवाका शर्त परिवर्तन गरिने छैन ।
तर चरम आर्थिक विश्रृंखलताका कारण संकटकाल घोषणा भएको अवस्थामा यो व्यवस्था लागू हुने छैन ।

(९) लोक सेवा आयोगको अध्यक्ष र सदस्य भइसकेको व्यक्ति अन्य सरकारी सेवामा नियुक्तिका लागि ग्राह्य हुने छैन ।
तर कुनै राजनीतिक पदमा वा कुनै विषयको अनुसन्धान, जाँचबुझ वा छानबीन गर्ने वा कुनै विषयको अध्ययन वा अन्वेषण गरी राय, मन्तव्य वा सिफारिस पेश गर्ने कुनै पदमा नियुक्त भई काम गर्न यस उपधारामा लेखिएको कुनै कुराले बाधा पुर्‍याएको मानिने छैन ।

\textbf{२४३. लोक सेवा आयोगको काम, कर्तव्य र अधिकारः} (१) निजामती सेवाको पदमा नियुक्तिका लागि उपयुक्त उम्मेदवार छनौट गर्न परीक्षा सञ्चालन गर्नु लोक सेवा आयोगको कर्तव्य हुनेछ ।

स्पष्टीकरणः यस धाराको प्रयोजनका लागि “निजामती सेवाको पद” भन्नाले सैनिक वा नेपाल प्रहरी वा सशस्त्र प्रहरी बल, नेपालको कर्मचारीको सेवाको पद तथा निजामती सेवाको पद होइन भनी ऐन बमोजिम तोकिएको अन्य सेवाको पद बाहेक नेपाल सरकारका अरु सबै सेवाको पद सम्झनु पर्छ ।

(२) निजामती सेवाको पद बाहेक नेपाली सेना, नेपाल प्रहरी, सशस्त्र प्रहरी बल, नेपाल, अन्य संघीय सरकारी सेवा र संगठित संस्थाको पदमा
पदपूर्तिका लागि लिईने लिखित परीक्षा लोकसेवा आयोगले सञ्चालन गर्नेछ ।
स्पष्टीकरणः यस धाराको प्रयोजनका लागि “संगठित संस्था” भन्नाले विश्वविद्यालय र शिक्षक सेवा आयोग बाहेकका पचास प्रतिशत वा सो भन्दा बढी शेयर वा जायजेथामा नेपाल सरकारको स्वामित्व वा नियन्त्रण भएको संस्थान, कम्पनी, बैंक, समिति वा संघीय कानून बमोजिम स्थापित वा नेपाल सरकारद्वारा गठित आयोग, संस्थान, प्राधिकरण, निगम, प्रतिष्ठान, बोर्ड, केन्द्र, परिषद र यस्तै प्रकृतिका अन्य संगठित संस्था सम्झनु पर्छ ।

(३) नेपाली सेना, नेपाल प्रहरी, सशस्त्र प्रहरी बल, नेपाल र अन्य संघीय सरकारी सेवाका पदमा बढुवा गर्दा अपनाउनु पर्ने सामान्य सिद्धान्तको विषयमा लोक सेवा आयोगको परामर्श लिनु पर्नेछ ।

(४) कुनै संगठित संस्थाको सेवाका कर्मचारीको सेवाका शर्त सम्बन्धी कानून र त्यस्तो सेवाका पदमा बढुवा र विभागीय कारबाही गर्दा अपनाउनु पर्ने सामान्य सिद्धान्तको विषयमा लोक सेवा आयोगको परामर्श लिनु पर्नेछ ।

(५) नेपाल सरकारबाट निवृत्तिभरण पाउने पदमा लोक सेवा आयोगको परामर्श विना स्थायी नियुक्ति गरिने छैन ।

(६) देहायका विषयमा लोक सेवा आयोगको परामर्श लिनु पर्नेछः–

(क) संघीय निजामती सेवाको शर्त सम्बन्धी कानूनको विषयमा,

(ख) संघीय निजामती सेवा वा पदमा नियुक्ति, बढुवा र विभागीय कारबाही गर्दा अपनाउनु पर्ने सिद्धान्तको विषयमा,

(ग) संघीय निजामती सेवाको पदमा छ महीनाभन्दा बढी समयका लागि नियुक्ति गर्दा उम्मेदवारको उपयुक्तताको विषयमा,

(घ) कुनै एक प्रकारको संघीय निजामती सेवाको पदबाट अर्को प्रकारको संघीय निजामती सेवाको पदमा वा अन्य सरकारी सेवाबाट संघीय निजामती सेवामा सरुवा वा बढुवा गर्दा वा कुनै प्रदेशको निजामती सेवाको पदबाट संघीय निजामती सेवाको पदमा वा संघीय निजामती सेवाको पदबाट प्रदेश निजामती सेवाको पदमा सेवा परिवर्तन वा स्थानान्तरण गर्दा उम्मेदवारको उपयुक्तताको विषयमा,

(ङ) लोक सेवा आयोगको परामर्श लिनु नपर्ने अवस्थाको पदमा बहाल रहेको कर्मचारीलाई लोक सेवा आयोगको परामर्श लिनु
पर्ने अवस्थाको पदमा स्थायी सरुवा वा बढुवा गर्ने विषयमा, र

(च) संघीय निजामती सेवाको कर्मचारीलाई दिइने विभागीय सजायको विषयमा ।ज्ञद्दद्द

(७) उपधारा (६) मा जुनसुकै कुरा लेखिएको भए तापनि धारा १५४ बमोजिमको न्याय सेवा आयोगको अधिकार क्षेत्रभित्र पर्ने विषयमा सोही
बमोजिम हुनेछ ।

(८) लोक सेवा आयोगले आफ्नो काम, कर्तव्य र अधिकार मध्ये कुनै काम, कर्तव्य र अधिकार आयोगको अध्यक्ष वा कुनै सदस्य वा नेपाल
सरकारको कर्मचारीलाई तोकिएको शर्तको अधीनमा रही प्रयोग तथा पालन गर्ने गरी प्रत्यायोजन गर्न सक्नेछ ।

(९) लोक सेवा आयोगको अन्य काम, कर्तव्य र अधिकार संघीय कानून बमोजिम हुनेछ ।

\textbf{२४४. प्रदेश लोक सेवा आयोग सम्बन्धी व्यवस्थाः} (१) प्रत्येक प्रदेशमा प्रदेश लोक सेवा आयोग रहनेछ ।

(२) प्रदेश लोक सेवा आयोगको गठन, काम, कर्तव्य र अधिकार प्रदेश कानून बमोजिम हुनेछ ।

(३) उपधारा (२) को प्रयोजनका लागि संघीय संसदले कानून बनाई आधार र मापदण्ड निर्धारण गर्नेछ ।